% Copyright (c) 2011-2020 Khronos Group, Inc.
%
% This work is licensed under a Creative Commons Attribution 4.0
% International License.
% http://creativecommons.org/licenses/by/4.0/

% !TEX root = sycl-1.2.1.tex

%*******************************************************************************
% Queue class
%*******************************************************************************

The SYCL \codeinline{queue} class encapsulates a single SYCL queue which
schedules kernels on a SYCL device. A SYCL queue may be an OpenCL queue in
which case it must encapsulate at least one valid underlying OpenCL
\codeinline{cl_command_queue}, or it may be a SYCL host queue in which case it
must not. The underlying OpenCL \codeinline{cl_command_queue}(s) may execute 
either in-order or out-of-order, however the SYCL \codeinline{queue}
must behave as an out-of-order queue.

A SYCL \codeinline{queue} can be used to submit \glspl{command-group} to be
executed by the \gls{sycl-runtime} using the \codeinline{submit} member
function.

All member functions of the \codeinline{queue} class are synchronous and errors
are handled by throwing synchronous SYCL exceptions. The \codeinline{submit}
member function schedules \glspl{command-group} asynchronously, so any errors
in the submission of a \gls{command-group} are handled by throwing synchronous
SYCL exceptions. Any exceptions from the \gls{command-group} after it has
been submitted are handled by throwing asynchronous SYCL exceptions to an
\gls{async-handler} on calling \codeinline{throw_asynchronous} or \codeinline{
wait_and_throw}, or on destruction of the SYCL \codeinline{queue}, if one was
provided when the SYCL \codeinline{queue} was constructed.

A SYCL \codeinline{queue} can wait for all \glspl{command-group} that it has
submitted by calling \codeinline{wait} or \codeinline{wait_and_throw}.

The default constructor of the SYCL \codeinline{queue} class will construct a
queue based on the SYCL \codeinline{device} returned from the \codeinline{default_selector} (see Section~\ref{sec:device-selector}), therefore
the constructed SYCL \codeinline{queue} could be either a host queue or a device
queue. All other constructors construct a host or device queue, determined by the
parameters provided. All constructors will implicitly construct a SYCL
\codeinline{platform}, \codeinline{device} and \codeinline{context} in order to
facilitate the construction of the queue.

With the exception of the interoperability constructor, each constructor takes as the last
parameter an optional SYCL \codeinline{property_list} to provide properties to
the SYCL \codeinline{queue}.

The SYCL \codeinline{queue} class provides the common reference semantics
(see Section~\ref{sec:reference-semantics}).

%*******************************************************************************
% Queue interface
%*******************************************************************************
\subsubsection{Queue interface}

A synopsis of the SYCL \codeinline{queue} class is provided below. The
constructors and member functions of the SYCL \codeinline{queue} class are
listed in Tables~\ref{table.constructors.queue} and \ref{table.members.queue}
respectively. The additional common special member functions and common member
functions are listed in \ref{sec:reference-semantics} in
Tables~\ref{table.specialmembers.common.reference} and
\ref{table.hiddenfriends.common.reference}, respectively.

%Interface for class: queue
\lstinputlistingSkipLicense{headers/queue.h}

%-------------------------------------------------------------------------------
\startTable{Constructor}
  \addFootNotes{Constructors of the \codeinline{queue} class}
  {table.constructors.queue}
  \addRow
    { explicit queue(const property_list \&propList = \{\}) }
    {
      Constructs a SYCL \codeinline{queue} instance using the device returned by
      an instance of \codeinline{default_selector}. Zero or more properties can
      be provided to the constructed SYCL \codeinline{queue} via an instance of
      \codeinline{property_list}.
    }
  \addRowTwoL
    { explicit queue(const async_handler \&asyncHandler, }
    { const property_list \&propList = \{\}) }
    {
      Constructs a SYCL \codeinline{queue} instance with an \codeinline{
      async_handler} using the device returned by an instance of \codeinline{
      default_selector}. Zero or more properties can be provided to the
      constructed SYCL \codeinline{queue} via an instance of \codeinline{
      property_list}.
    }
  \addRowTwoL
    { explicit queue(const device_selector \&deviceSelector, }
    { const property_list \&propList = \{\}) }
    {
      Constructs a SYCL \codeinline{queue} instance using the device returned by
      the \codeinline{deviceSelector} provided. Zero or more properties can be
      provided to the constructed SYCL \codeinline{queue} via an instance of
      \codeinline{property_list}.
    }
  \addRowThreeL
    { queue(const device_selector \&deviceSelector, }
    { const async_handler \&asyncHandler, }
    { const property_list \&propList = \{\}) }
    {
      Constructs a SYCL \codeinline{queue} instance with an \codeinline{
      async_handler} using the device returned by the \codeinline{deviceSelector
      } provided. Zero or more properties can be provided to the constructed
      SYCL \codeinline{queue} via an instance of \codeinline{property_list}.
    }
  \addRowTwoL
    { explicit queue(const device \&syclDevice, }
    { const property_list \&propList = \{\}) }
    {
      Constructs a SYCL \codeinline{queue} instance using the \codeinline{
      syclDevice} provided. Zero or more properties can be provided to the
      constructed SYCL \codeinline{queue} via an instance of \codeinline{
      property_list}.
    }
  \addRowThreeL
    { explicit queue(const device \&syclDevice, }
    { const async_handler \&asyncHandler, }
    { const property_list \&propList = \{\}) }
    {
      Constructs a SYCL \codeinline{queue} instance with an \codeinline{
      async_handler} using the \codeinline{syclDevice} provided. Zero or more
      properties can be provided to the constructed SYCL \codeinline{queue}
      via an instance of \codeinline{property_list}.
    }
  \addRowThreeL
    { explicit queue(const context \&syclContext, }
    { const device_selector \&deviceSelector, }
    { const property_list \&propList = \{\}) }
    {
      Constructs a SYCL \codeinline{queue} instance that is associated with the
      \codeinline{syclContext} provided, using the device returned by the
      \codeinline{deviceSelector} provided. Must throw an
      \codeinline{invalid_object_error} SYCL exception if
      \codeinline{syclContext} does not encapsulate the SYCL
      \codeinline{device} returned by \codeinline{deviceSelector}. Zero or more
      properties can be provided to the constructed SYCL \codeinline{queue}
      via an instance of \codeinline{property_list}.
    }
  \addRowFourL
    { explicit queue(const context \&syclContext, }
    { const device_selector \&deviceSelector, }
    { const async_handler \&asyncHandler, }
    { const property_list \&propList = \{\}) }
    {
      Constructs a SYCL \codeinline{queue} instance with an \codeinline{
      async_handler} that is associated with the \codeinline{syclContext}
      provided, using the device returned by the \codeinline{deviceSelector}
      provided. Must throw an \codeinline{invalid_object_error} SYCL exception
      if \codeinline{syclContext} does not encapsulate the SYCL
      \codeinline{device} returned by \codeinline{deviceSelector}. Zero or more
      properties can be provided to the constructed SYCL \codeinline{queue} via
      an instance of \codeinline{property_list}.
    }
  \addRowThreeL
    { explicit queue(const context \&syclContext, }
    { const device \&syclDevice, }
    { const property_list \&propList = \{\}) }
    {
      Constructs a SYCL \codeinline{queue} instance using the \codeinline{
      syclDevice} provided, and associated with the
      \codeinline{syclContext} provided. Must throw an
      \codeinline{invalid_object_error} SYCL exception if
      \codeinline{syclContext} does not encapsulate the SYCL
      \codeinline{device} \codeinline{syclDevice}. Zero or more
      properties can be provided to the constructed SYCL \codeinline{queue}
      via an instance of \codeinline{property_list}.
    }
  \addRowFourL
    { explicit queue(const context \&syclContext, }
    { const device \&syclDevice, }
    { const async_handler \&asyncHandler, }
    { const property_list \&propList = \{\}) }
    {
      Constructs a SYCL \codeinline{queue} instance with an \codeinline{
      async_handler} using the \codeinline{syclDevice} provided, and
      associated with the \codeinline{syclContext} provided. Must throw an
      \codeinline{invalid_object_error} SYCL exception if
      \codeinline{syclContext} does not encapsulate the SYCL
      \codeinline{device} \codeinline{syclDevice}.  Zero or more
      properties can be provided to the constructed SYCL \codeinline{queue} via
      an instance of \codeinline{property_list}.
    }
  \addRowThreeL
    { explicit queue(cl_command_queue clQueue,}
    { const context \&syclContext, }
    { const async_handler \&asyncHandler = \{\}) }
    {
      Constructs a SYCL \codeinline{queue} instance with an optional
      \codeinline{async_handler} from an OpenCL \codeinline{cl_command_queue}
      in accordance with the requirements described
      in~\ref{sec:opencl-interoperability}.
    }
\completeInfoTable
%-------------------------------------------------------------------------------

%-------------------------------------------------------------------------------
\startTable{Member function}
  \addFootNotes{Member functions for class queue}{table.members.queue}
  \addRow
    {cl_command_queue get() const}
    {   
      Returns a valid \codeinline{cl_command_queue} instance in accordance with the requirements described in \ref{sec:opencl-interoperability}.    
    }
  \addRow
    {context get_context () const}
    {
      Returns the SYCL queue's context.
      Reports errors using SYCL exception classes.
      The value returned must be equal to that returned by \codeinline{get_info<info::queue::context>()}.
    }
  \addRow
    {device get_device () const}
    {
      Returns the SYCL device the queue is associated with.
      Reports errors using SYCL exception classes.
      The value returned must be equal to that returned by \codeinline{get_info<info::queue::devices>()}.
    }
  \addRow
   {bool is_host () const}
   {
      Returns whether the queue is executing on a SYCL host device.
   }
  \addRow
    {void wait() }
    {
      Performs a blocking wait for the completion of all enqueued tasks
      in the queue.  Synchronous errors will be reported through SYCL
      exceptions.
    }
  \addRow
    {void wait_and_throw () }
    {
      Performs a blocking wait for the completion of all enqueued tasks
      in the queue.  Synchronous errors will be reported via SYCL
      exceptions. Asynchronous errors will be passed to the
      \gls{async-handler} passed to the queue on
      construction. If no \codeinline{async_handler} was provided then
      asynchronous exceptions will be lost.
    }
  \addRow
    {void throw_asynchronous () }
    {
      Checks to see if any asynchronous errors have been produced by
      the queue and if so reports them by passing them to the
      \gls{async-handler} passed to the queue on
      construction. If no \codeinline{async_handler} was provided then
      asynchronous exceptions will be lost.
    }
  \addRowFourL
    { template <info::queue param> }
    {  typename info::param_traits}
    {  <info::queue, param>::return_type}
    {  get_info ()  const}
    {Queries the platform for \codeinline{cl_command_queue_info}}
  \addRowTwoL
    {template <typename T>}
    {event submit(T cgf)}
    {Submit a \gls{command-group-function-object} to the queue, in order to be scheduled
    for execution on the device.}
  \addRowThreeL
    {template <typename T>}
    {event submit(T cgf,}
    {             queue \& secondaryQueue)}
    {Submit a \gls{command-group-function-object} to the queue, in order to be scheduled
    for execution on the device. On a kernel error, this \gls{command-group-function-object},
    is then scheduled for execution on the secondary queue. Returns an
    event, which corresponds to the queue the \gls{command-group-function-object}
    is being enqueued on.}
\completeTable
%-------------------------------------------------------------------------------

\subsubsection{Queue information descriptors}

A SYCL \gls{queue} can be queried for all of the following information using the
\codeinline{get_info} function. All SYCL queues have valid queries for them,
including the SYCL host queue. The available information is in
Table~\ref{table.queue.info}. The interface of all available device descriptors is
in appendix~\ref{appendix.queue.descriptors}.

\startInfoTable{Queue Descriptors}
\addInfoFootNotes{Queue information descriptors}{table.queue.info}
\addInfoRow
{info::queue::context}
{context}
{
  Returns the SYCL \codeinline{context} associated with this SYCL \codeinline{queue}.
}
\addInfoRow
{info::queue::device}
{device}
{
  Returns the SYCL \gls{device} associated with this SYCL \codeinline{queue}.
}
\addInfoRow
{info::queue::reference_count}
{cl_uint}
{Return the command-queue reference count.}
\completeTable

\subsubsection{Queue Properties}
\label{sec:queue-properties}

The properties that can be provided when constructing the SYCL
\codeinline{queue} class are describe in
Table~\ref{table.properties.queue}.

%---------------------------------------------------------------------
\startTable{Property}
\addFootNotes{Properties supported by the SYCL \codeinline{queue} class} {table.properties.queue}

\addRow
  { property::queue::enable_profiling }
  {
    The \codeinline{enable_profiling} property adds the requirement
    that the \gls{sycl-runtime} must capture profiling information for
    the \glspl{command-group} that are submitted from this SYCL
    \codeinline{queue} and provide said information via the SYCL
    \codeinline{event} class \codeinline{get_profiling_info} member
    function, if the associated SYCL \codeinline{device} supports
    queue profiling (i.e. the \codeinline{
    info::device::queue_profiling} info parameter returns \codeinline{
    true}).
  }
\completeTable
%---------------------------------------------------------------------

The constructors of the queue property classes are listed in Table~\ref{table.constructors.properties.queue}.

%---------------------------------------------------------------------
\startTable{Constructor}
\addFootNotes{Constructors of the queue property classes}
{table.constructors.properties.queue}
\addRow
{property::queue::enable_profiling::enable_profiling()}
{
  Constructs a SYCL \codeinline{enable_profiling} property instance.
}
\completeTable
%---------------------------------------------------------------------

\subsubsection{Queue error handling}
\label{sec:interface.queue.errors}

Queue errors come in two forms:
\begin{itemize}

  \item
    \textbf{Synchronous Errors} are those that we would expect to be
    reported directly at the point of waiting on an event, and hence
    waiting for a queue to complete, as well as any immediate errors
    reported by enqueuing work onto a queue. Such errors are returned
    through exceptions.

  \item
    \textbf{Asynchronous errors} are those that are produced through
    callback functions only. These will be stored within the queue's
    context until they are dispatched to the context's asynchronous
    error handler, the \gls{async-handler}. If a queue is constructed 
    with a user-supplied context, then it is this context's asynchronous error
    handler to which asynchronous errors are reported. If a queue is constructed
    without a user-supplied context, then the queue's constructor
    can be supplied with a queue-specific asynchronous error handler
    which will be used to construct the queue's context.
    To ensure that such errors are
    processed predictably in a known host thread, these errors are only
    passed to the asynchronous error handler on request when either
    \codeinline{wait_and_throw} is called or when
    \codeinline{throw_asynchronous} is called. If no
    asynchronous error handler is passed to the queue or its context
    on construction, then such errors go unhandled, much as they would
    if no callback were passed to an OpenCL context.

\end{itemize}

Note that if there are exceptions to be processed when a queue
using an asynchronous handler is destructed, the handler is called and
this might delay or block the destruction, according to the behavior
of the handler.


%%% Local Variables:
%%% mode: latex
%%% TeX-master: "sycl-1.2.1"
%%% TeX-auto-untabify: t
%%% TeX-PDF-mode: t
%%% ispell-local-dictionary: "american"
%%% End:
