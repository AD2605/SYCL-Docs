% !TEX root = sycl-1.2.1.tex

\chapter{Introduction}

SYCL (pronounced ``sickle'') is a
royalty-free, cross-platform abstraction C++ programming model for
OpenCL. SYCL builds on the underlying concepts, portability and
efficiency of OpenCL while adding much of the ease of use and
flexibility of single-source C++.
Developers using SYCL are able to write standard
C++ code, with many of the techniques they are accustomed to, such as
inheritance and templating. At the same time developers have access to
the full range of capabilities of OpenCL both through the features of
the SYCL libraries and, where necessary, through interoperation with
code written directly to the OpenCL APIs.

SYCL implements a \gls{smcp}  design which offers the power of source
integration while allowing toolchains to remain flexible. The \gls{smcp}
design supports embedding of code intended to be compiled for an OpenCL device,
for example a GPU, inline with host code. This embedding of code offers three
primary benefits:
\begin{description}
\item[Simplicity]
  For novice programmers, the separation of host and device source code in
  OpenCL can become complicated to deal with, particularly when similar kernel
  code is used for multiple different operations. A single compiler flow and
  integrated tool chain combined with libraries that perform a lot of simple
  tasks simplifies initial OpenCL programs to a minimum complexity. This reduces
  the learning curve for programmers new to OpenCL and allows them to
  concentrate on parallelization techniques rather than syntax.
\item[Reuse]
  C++'s type system allows for complex interactions between different code units
  and supports efficient abstract interface design and reuse of library code.
  For example, a \keyword{transform} or \keyword{map} operation applied to an
  array of data may allow specialization on both the operation applied to each
  element of the array and on the type of the data. The \gls{smcp} design of
  SYCL enables this interaction to bridge the host code/device code boundary
  such that the device code to be specialized on both of these factors directly
  from the host code.
\item[Efficiency]
  Tight integration with the type system and reuse of library code enables a
  compiler to perform inlining of code and to produce efficient specialized
  device code based on decisions made in the host code without having to
  generate kernel source strings dynamically.
\end{description}

SYCL is designed to allow a compilation flow where the source file is passed
through multiple different compilers, including a standard C++ host compiler of
the developer's choice, and where the resulting application combines the results
of these compilation passes. This is distinct from a single-source flow that
might use language extensions that preclude the use of a standard host compiler.
The SYCL standard does not preclude the use of a single compiler flow, but is
designed to not require it.

The advantages of this design are two-fold. First, it offers better integration
with existing tool chains. An application that already builds using a chosen
compiler can continue to do so when SYCL code is added. Using the SYCL tools on
a source file within a project will both compile for an OpenCL device and let
the same source file be compiled using the same host compiler that the rest of
the project is compiled with. Linking and library relationships are unaffected.
This design simplifies porting of pre-existing applications to SYCL. Second, the
design allows the optimal compiler to be chosen for each device where different
vendors may provide optimized tool-chains.

SYCL is designed to be as close to standard C++ as possible. In practice, this
means that as long as no dependence is created on SYCL's integration with
OpenCL, a standard C++ compiler can compile the SYCL programs and they will run
correctly on host CPU. Any use of specialized low-level features can be masked
using the C pre-processor in the same way that compiler-specific intrinsics may
be hidden to ensure portability between different host compilers.

SYCL retains the execution model, runtime feature set and device capabilities of
the underlying OpenCL standard. The OpenCL C specification imposes some
limitations on the full range of C++ features that SYCL is able to support. This
ensures portability of device code across as wide a range of devices as
possible. As a result, while the code can be written in standard C++ syntax with
interoperability with standard C++ programs, the entire set of C++ features is
not available in SYCL device code. In particular, SYCL  device code, as defined
by this specification, does not support virtual function calls, function
pointers in general, exceptions, runtime type information or the full set of C++
libraries that may depend on these features or on features of a particular host
compiler.

The use of C++ features such as templates and inheritance on top of the OpenCL
execution model opens a wide scope for innovation in software design for
heterogeneous systems. Clean integration of device and host code within a single
C++ type system enables the development of modern, templated libraries that
build simple, yet efficient, interfaces to offer more developers access to
OpenCL capabilities and devices. SYCL is intended to serve as a foundation for
innovation in programming models for heterogeneous systems, that builds on an
open and widely implemented standard foundation in the form of OpenCL.


To reduce programming effort and increase the flexibility with which developers
can write code, SYCL extends the underlying OpenCL model in two ways beyond the
general use of C++ features:
\begin{itemize}
  \item
    The hierarchical parallelism syntax offers a way of expressing the
    data-parallel OpenCL execution model in an easy-to-understand C++ form.
    It more cleanly layers parallel loops and synchronization points to avoid
    fragmentation of code and to more efficiently map to CPU-style
    architectures.
  \item
    Data access in SYCL is separated from data storage. By relying on the
    C++-style resource acquisition is initialization (RAII) idiom to capture
    data dependencies between device code blocks, the runtime library can track
    data movement and provide correct behavior without the complexity of
    manually managing event dependencies between kernel instances and without
    the programming having to explicitly move data. This approach enables the
    data-parallel task-graphs that are already part of the OpenCL execution
    model to be built up easily and safely by SYCL programmers.
\end{itemize}

To summarize, SYCL enables OpenCL kernels to be written inside C++
source files. This means that software developers can develop and use generic
algorithms and data structures using standard C++ template techniques, while
still supporting the multi-platform, multi-device heterogeneous execution of
OpenCL. The specification has been designed to enable implementation across as
wide a variety of platforms as possible as well as ease of integration with
other platform-specific technologies, thereby letting both users and
implementers build on top of SYCL as an open platform for heterogeneous
processing innovation.


%%% Local Variables:
%%% mode: latex
%%% TeX-master: "sycl-1.2.1"
%%% TeX-auto-untabify: t
%%% TeX-PDF-mode: t
%%% ispell-local-dictionary: "american"
%%% End:
