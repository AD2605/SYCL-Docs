% Copyright (c) 2011-2020 Khronos Group, Inc.
%
% This work is licensed under a Creative Commons Attribution 4.0
% International License.
% http://creativecommons.org/licenses/by/4.0/

% !TEX root = sycl-1.2.1.tex

%\usepackage[usenames,dvipsnames]{xcolor}
\usepackage[left=1.25in, right=1.0in, top=1.0in, bottom=1.5in]{geometry}
\usepackage{verbatim}

\def\SYCLVERSION{1.2.1}
\def\SYCLLANGVERSION{121}
\def\DOCUMENTVERSION{7}
\def\SYCLNAME{SYCL}
\def\OPENCLVERSION{1.2}

\usepackage[pdftex,bookmarksnumbered=true,plainpages=false,
  pdftitle={Khronos Group \SYCLNAME{} \SYCLVERSION{} Specification},
  colorlinks=true,linktocpage,linkcolor=blue,citecolor=blue,urlcolor=blue]{hyperref}
\usepackage[pdftex]{graphicx} \usepackage{titletoc}
\usepackage{titlesec}

\usepackage{glossaries}

\usepackage{enumerate,framed,multicol,longtable,url}
\usepackage{verbatim,wasysym,fancyhdr,pdfcolmk,multirow}
\usepackage[breakwords,fit]{truncate}
\usepackage[normalem]{ulem}
\def\func#1{{\bf #1}}
\usepackage{layout}
\usepackage{marginnote}
\usepackage{tabularx}
% Use colortbl package mode
% svgnames https://www.w3.org/TR/SVG11/types.html#ColorKeywords
% But everything is described in
% /usr/share/doc/texlive-doc/latex/xcolor/xcolor.pdf
\usepackage[dvipsnames,svgnames,x11names,table]{xcolor}
\usepackage{indentfirst}
\usepackage{cite}
%\usepackage{titlesec}
%\usepackage{hhline}
\usepackage{parskip}
\usepackage{ulem}
\usepackage{array}

% For the \textregistered, \texttrademark... symbols
\usepackage{textcomp}

% https://tex.stackexchange.com/questions/27661/how-to-prevent-last-period-in-captions-from-appearing-in-the-list-of-figures
% To have a "." added at the end of the captions but not in the caption
% lists of the table-of-content
\usepackage[textformat=period]{caption}

\usepackage{pgf}
\usepackage{tikz}
\usetikzlibrary{arrows}
%\usetikzlibrary{backgrounds}
\usetikzlibrary{calc}
%\usetikzlibrary{decorations.markings}
%\usetikzlibrary{decorations.pathmorphing}
\usetikzlibrary{fit}
%\usetikzlibrary{mindmap}
\usetikzlibrary{patterns}
\usetikzlibrary{positioning}
\usetikzlibrary{scopes}
\usetikzlibrary{shapes}

% Problems with this package on local machine
\usepackage{txfonts}

% Tweak to match OpenCL spec
\renewcommand*\rmdefault{ptm}
\titleformat{\chapter}
  {\normalfont\rmfamily\LARGE\bfseries\color{black}}
  {\thechapter.}{40pt}{\LARGE}
\titleformat{\section}
  {\normalfont\sffamily\LARGE\bfseries\color{black}}
  {\thesection}{40pt}{\Large}
\titleformat{\subsection}
  {\normalfont\sffamily\Large\bfseries\color{black}}
  {\thesubsection}{40pt}{}
\titlespacing*{\chapter}{0pt}{0pt}{40pt}


%\hypersetup{
%    colorlinks,
%    citecolor=black,
%    filecolor=black,
%    linkcolor=red,
%    urlcolor=black
%}

\setcounter{secnumdepth}{3}
\setcounter{tocdepth}{3}

\usepackage{listings}
\lstloadlanguages{C++}

\lstnewenvironment{code}
    {\lstset{}%
       \csname lst@SetFirstLabel\endcsname}
     {\csname lst@SaveFirstLabel\endcsname}
     \lstset{
       xleftmargin=9pt,
       basicstyle=\ttfamily,
       flexiblecolumns=true,
       basewidth={0.5em,0.45em},
       literate={<~}{{$\leftsquigarrow$}}2,
       numbers=left
     }

% \begin{lstlisting}[style=nonumbers,basicstyle=\ttfamily\small,backgroundcolor=\color{white},frame=]

\lstset{language=C++}

% Some missing C++ keywords
\lstset{
morekeywords=[1]{
 constexpr,
 mutex,
 size_t,
 shared_ptr,
 std,
 unique_ptr,
}
}

% terminology (NDRange, Compute Unit, ...)
\lstset{
morekeywords=[2]{
 accelerator_selector,
 accessor,
 access,
 async_exception,
 atomic,
 buffer,
 constant_ptr,
 context,
 cpu_selector,
 default_selector,
 device,
 device_event,
 device_selector,
 event,
 exception,
 exception_list,
 exception_ptr,
 exception_ptr_class,
 function_class,
 global_ptr,
 gpu_selector,
 group,
 handler,
 hash_class,
 h_item,
 host_selector,
 id,
 image,
 item,
 kernel,
 local_ptr,
 multi_ptr,
 mutex_class,
 nd_item,
 nd_range,
 platform,
 private_memory,
 private_ptr,
 program,
 property,
 property_list,
 queue,
 range,
 sampler,
 shared_ptr_class,
 stream,
 string_class,
 sycl,
 vec,
 vector_class,
 weak_ptr_class,
}
}

% structural keywords and functions (parallel_for)
\lstset{
 morekeywords=[3]{
 barrier,
 get,
 get_access,
 parallel_for,
 parallel_for_work_group,
 parallel_for_work_item,
 single_task,
 submit,
}
}

% data types (int4)
\lstset{
 morekeywords=[4]{
 char2, char3, char4, char8, char16,
 double2,double3,double4,double8,double16,
 float2,float3,float4,float8,float16,
 half,half2,half3,half4,half8,half16,
 int2,int3,int4,int8,int16,
 long2,long3,long4,long8,long16,
 longlong2,longlong3,longlong4,longlong8,longlong16,
 schar2, schar3, schar4, schar8, schar16,
 short2, short3, short4, short8, short16,
 uchar2, uchar3, uchar4, uchar8, uchar16,
 uint2,uint3,uint4,uint8,uint16,
 ulong2,ulong3,ulong4,ulong8,ulong16,
 ulonglong2,ulonglong3,ulonglong4,ulonglong8,ulonglong16,
 ushort2, ushort3, ushort4, ushort8, ushort16,
}
}

% Generic types (Gen, SGen, GenVec, ...)
\lstset{
 morekeywords=[5]{
 genchar, charn,
 genfloatd, doublen,
 genfloatf, floatn,
 genfloath, halfn,
 genfloat, sgenfloat,
 gengeofloat, gengeodouble,
 geninteger32bit, geninteger64bit,
 geninteger8bit, geninteger16bit,
 geninteger, sgeninteger, genintegerNbit,
 genint, intn,
 genlonglong, longlongn,
 genlong, longn,
 genshort, shortn,
 gentype, genhalf, genvector,
 igenchar, scharn,
 igeninteger32bit, igeninteger64bit,
 igeninteger8bit, igeninteger16bit,
 igeninteger, igenintegerNbit,
 igenlonginteger, ugenlonginteger,
 ugenchar, ucharn,
 ugeninteger32bit, ugeninteger64bit,
 ugeninteger8bit, ugeninteger16bit,
 ugeninteger, ugenintegerNbit,
 ugenint, uintn,
 ugenlonglong, ulonglongn,
 ugenlong, ulongn,
 ugenshort, ushortn,
 genfloatptr, genintptr
}
}

% OpenCL compatibility (cl_bool...)
\lstset{
 morekeywords=[6]{
 __kernel,
 __read_only,
 __write_only,
 __read_write,
 cl,
 cl_bool,
 cl_char,
 cl_char16,
 cl_char2,
 cl_char3,
 cl_char4,
 cl_char8,
 cl_command_queue,
 cl_double,
 cl_double16,
 cl_double2,
 cl_double3,
 cl_double4,
 cl_double8,
 cl_event,
 cl_exception,
 cl_float,
 cl_float16,
 cl_float2,
 cl_float3,
 cl_float4,
 cl_float8,
 cl_half,
 cl_half16,
 cl_half2,
 cl_half3,
 cl_half4,
 cl_half8,
 cl_int,
 cl_int16,
 cl_int2,
 cl_int3,
 cl_int4,
 cl_int8,
 cl_kernel,
 cl_long,
 cl_long16,
 cl_long2,
 cl_long3,
 cl_long4,
 cl_long8,
 cl_mem,
 cl_pipe,
 cl_program,
 cl_sampler,
 cl_short,
 cl_short16,
 cl_short2,
 cl_short3,
 cl_short4,
 cl_short8,
 cl_uchar,
 cl_uchar16,
 cl_uchar2,
 cl_uchar3,
 cl_uchar4,
 cl_uchar8,
 cl_uint,
 cl_uint16,
 cl_uint2,
 cl_uint3,
 cl_uint4,
 cl_uint8,
 cl_ulong,
 cl_ulong16,
 cl_ulong2,
 cl_ulong4,
 cl_ulong8,
 cl_ushort,
 cl_ushort16,
 cl_ushort2,
 cl_ushort3,
 cl_ushort4,
 cl_ushort8,
 event_t
}
}


% Macros
\lstset{
morekeywords=[7]{
 SYCL_EXTERNAL,
}
}


\lstset{
literate={~} {$\sim$}{1}
}

\lstset{
backgroundcolor=\color[rgb]{0.95, 0.95, 0.95},
tabsize=2,
rulecolor=,
%basicstyle=\scriptsize,
upquote=true,
aboveskip={1.5\baselineskip},
columns=fixed,
showstringspaces=false,
extendedchars=true,
breaklines=true,
%prebreak = \raisebox{0ex}[0ex][0ex]{\ensuremath{\hookleftarrow}},
frame=single,
showtabs=false,
showspaces=false,
showstringspaces=false,
identifierstyle=\ttfamily,
% Normal C++ keywords
keywordstyle=[1]\color[rgb]{0,0,0.75},
% terminology (NDRange, Compute Unit, ...)
keywordstyle=[2]\color{VioletRed1},
% structural keywords and functions (parallel_for)
keywordstyle=[3]\color{Turquoise3},
% data types (int4)
keywordstyle=[4]\color{DarkOrchid3},
% Generic types (Gen, SGen, GenVec, ...)
keywordstyle=[5]\color{Magenta1},
% OpenCL compatibility (cl_bool...)
keywordstyle=[6]\color{OrangeRed1},
% Macros
keywordstyle=[7]\color{OliveDrab2},
commentstyle=\color[rgb]{0.133,0.545,0.133},
stringstyle=\color[rgb]{0.639,0.082,0.082},
}


% We should use \gls and other functions from glossaries instead now.
% This can be seen as a \todo inside the document about adding new
% glossary entries...
\newcommand{\keyword}[1]{\textit{#1}}



\def \showtodos{false}
\def \true{true}
\def \showdiffs{false}

%% Attention:
%% Uncomment this line in order to enable comments, i.e. show the todos and
%% comments in the produced pdf file.
%% \def\showtodos{true}

\ifx\showtodos\true
\usepackage[draft]{todonotes}
\else
\usepackage[disable]{todonotes}
\fi

%% Uncomment to show diffs from the version of the specification voted on at the Phoenix F2F 2014
%% differences shown as red in the pdf file
%\def\showdiffs{true}



\usepackage{setspace}
\newcommand{\fixme}[1]{\todo[inline]{#1}}


\newcounter{mycomment}
\newcommand{\mycomment}[2][]{%
% initials of the author (optional) + note in the margin
\refstepcounter{mycomment}%
{%
\setstretch{0.7}% spacing
\todo[color={cyan!100!green!33},size=\scriptsize]{%
Comment [#1\themycomment]:~\newline #2}%
}}


% Define the page style with header & footer
\pagestyle{fancy}
% Reorganize the chapter and section mark to avoid collisions
\fancyhead{}
\fancyhead[LO,RE]{SYCL \SYCLVERSION{}}
\fancyhead[RO,LE]{\rightmark}% The section name
\fancyfoot{}
\fancyfoot[LE,RO]{\thepage}
\fancyfoot[LO,RE]{\leftmark}% The chapter name
% Since the \chapter page uses always the plain style, redefine plain style
\fancypagestyle{plain}{%
  \fancyhf{}% Clear all header and footer fields
  \fancyfoot[LE,RO]{\thepage}% The page as usual
}

\def\tm{\texttrademark}
% Commands for printing out templated classes and functions and keywords
\newcommand{\tf}{\texttt}
\newcommand{\tbf}{\textbf}
\newcommand{\tfb}[1]{\textbf{\texttt{#1}}}
\newcommand{\tclass}[2]{\texttt{#1}$<$#2$>$}
\newcommand{\tbclass}[2]{\texttt{\textbf{#1}}$<$#2$>$}
\newcommand{\nline}[1]{#1 \newline }
\newcommand{\nlineII}[2]{#1 \newline #2 \newline }
\newcommand{\nlineIII}[3]{#1 \newline #2 \newline #3 \newline }
\newcommand{\nlineIV}[4]{#1 \newline #2 \newline #3 \newline #4 \newline }
\newcommand{\nlineV}[5]{#1 \newline #2 \newline #3 \newline #4 \newline #5 \newline }
\newcommand{\nlineVI}[6]{#1 \newline #2 \newline #3 \newline #4 \newline #5 \newline #6 \newline }
% The problem of this implementation is that #1 is parsed with the
% wrong catcode, which means that if it starts with a _, it has to be
% escaped \_
\newcommand{\codeinline}[1]{\lstinline[basicstyle=\ttfamily\small]$#1$}
\newcommand{\scriptinline}[1]{\lstinline[basicstyle=\ttfamily\scriptsize]$#1$}
\lstdefinestyle{nonumbers}{numbers=none}

\newcommand{\twoLinesOfCode}[2]{\codeinline{#1} \newline \hspace*{2em}\codeinline{#2}}
\newcommand{\threeLinesOfCode}[3]{\twoLinesOfCode{#1}{#2} \newline \hspace*{2em}\codeinline{#3}}
\newcommand{\fourLinesOfCode}[4]{\threeLinesOfCode{#1}{#2}{#3} \newline \hspace*{2em}\codeinline{#4}}
\newcommand{\fiveLinesOfCode}[5]{\fourLinesOfCode{#1}{#2}{#3}{#4} \newline \hspace*{2em}\codeinline{#5}}
\newcommand{\sixLinesOfCode}[6]{\fiveLinesOfCode{#1}{#2}{#3}{#4}{#5} \newline \hspace*{2em}\codeinline{#6}}
\newcommand{\sevenLinesOfCode}[7]{\sixLinesOfCode{#1}{#2}{#3}{#4}{#5}{#6} \newline \hspace*{2em}\codeinline{#7}}
\newcommand{\eightLinesOfCode}[8]{\sevenLinesOfCode{#1}{#2}{#3}{#4}{#5}{#6}{#7} \newline \hspace*{2em}\codeinline{#8}}

\newcommand{\twoLinesSCode}[2]{\codeinline{#1} \newline \codeinline{#2}}
\newcommand{\threeLinesSCode}[3]{\twoLinesSCode{#1}{#2} \newline \hspace*{2em}\codeinline{#3}}
\newcommand{\fourLinesSCode}[4]{\threeLinesSCode{#1}{#2}{#3} \newline \hspace*{2em} \codeinline{#4}}
\newcommand{\fiveLinesSCode}[5]{\fourLinesSCode{#1}{#2}{#3}{#4} \newline \hspace*{2em} \codeinline{#5}}
\newcommand{\sixLinesSCode}[6]{\fiveLinesSCode{#1}{#2}{#3}{#4}{#5} \newline \hspace*{2em} \codeinline{#6}}
\newcommand{\sevenLinesSCode}[7]{\sixLinesSCode{#1}{#2}{#3}{#4}{#5}{#6} \newline \hspace*{2em} \codeinline{#7}}
\newcommand{\eightLinesSCode}[8]{\sevenLinesSCode{#1}{#2}{#3}{#4}{#5}{#6}{#7} \newline \hspace*{2em} \codeinline{#8}}

 \lstnewenvironment{interface}
    {\lstset{}%
       \csname lst@SetFirstLabel\endcsname}
     {\csname lst@SaveFirstLabel\endcsname}
     \lstset{
       style=nonumbers,
       basicstyle=\ttfamily\small,
       backgroundcolor=\color{white},
       frame=none,
     }


% Commands for describing tables of enums
% Used for error tables etc throughout the text
\newcommand{\startEnumTable}[1]{\begingroup\begin{table}[!h]\setlength{\extrarowheight}{5pt}\small\centering\begin{tabular}[b]{|p{8cm} | p{6cm} |}\hline\cellcolor{lightgray}\textbf{#1} & \cellcolor{lightgray}\textbf{Description}\\ \hline}
\newcommand{\enumTableRow}[2]{\lstinline$#1$ & #2\\ \hline}
\newcommand{\enumTableRowEn}[3]{\lstinline$#1$ & #3 \\ \lstinline$#2$ & \\ \hline}
\newcommand{\enumTableRowEnhanced}[4]{\lstinline$#1$ & #4 \\ \lstinline$#2$ & \\ \lstinline$#3$ & \\\hline}
\newcommand{\completeEnumTabular}[0]{\end{tabular}}
\newcommand{\captionEnumTable}[1]{\caption{#1}}
\newcommand{\labelEnumTable}[1]{\label{#1}}
\newcommand{\completeEnumTable}[0]{\end{table}\endgroup}


%\newcommand{\startTable}[1]{\begingroup\begin{table}[!h]\setlength{\extrarowheight}{5pt}\small\centering\begin{tabular}[b]{|p{8.8cm} | p{6.2cm} |}\hline\cellcolor{lightgray}\textbf{#1} & \cellcolor{lightgray}\textbf{Description}\\ \hline}
\newcommand{\addRowNC}[2]{{#1} & {#2}\\ \hline} 
\newcommand{\addRow}[2]{\codeinline{#1} & {#2}\\ \hline}
\newcommand{\addRowB}[2]{\codeinlineB{#1} & {#2}\\ \hline}
\newcommand{\addRowTwoL}[3]{\twoLinesOfCode{#1}{#2} & #3 \\ \hline}
\newcommand{\addRowThreeL}[4]{\threeLinesOfCode{#1}{#2}{#3} & #4 \\ \hline}
\newcommand{\addRowFourL}[5]{\fourLinesOfCode{#1}{#2}{#3}{#4} & #5 \\ \hline}
\newcommand{\addRowFiveL}[6]{\fiveLinesOfCode{#1}{#2}{#3}{#4}{#5} & #6 \\ \hline}
\newcommand{\addRowSixL}[7]{\sixLinesOfCode{#1}{#2}{#3}{#4}{#5}{#6} & #7 \\ \hline}
\newcommand{\addRowSevenL}[8]{\sevenLinesOfCode{#1}{#2}{#3}{#4}{#5}{#6}{#7} & #8\\ \hline}
\newcommand{\addRowEightL}[9]{\eightLinesOfCode{#1}{#2}{#3}{#4}{#5}{#6}{#7}{#8} & #9\\ \hline}

\newcommand{\addRowTwoSL}[3]{\twoLinesSCode{#1}{#2} & #3 \\ \hline}
\newcommand{\addRowThreeSL}[4]{\threeLinesSCode{#1}{#2}{#3} & #4 \\ \hline}
\newcommand{\addRowFourSL}[5]{\fourLinesSCode{#1}{#2}{#3}{#4} & #5 \\ \hline}
\newcommand{\addRowFiveSL}[6]{\fiveLinesSCode{#1}{#2}{#3}{#4}{#5} & #6 \\ \hline}
\newcommand{\addRowSixSL}[7]{\sixLinesSCode{#1}{#2}{#3}{#4}{#5}{#6} & #7 \\ \hline}
\newcommand{\addRowSevenSL}[8]{\sevenLinesSCode{#1}{#2}{#3}{#4}{#5}{#6}{#7} & #8\\ \hline}
\newcommand{\addRowEightSL}[9]{\eightLinesSCode{#1}{#2}{#3}{#4}{#5}{#6}{#7}{#8} & #9\\ \hline}

\newcommand{\completeTabular}[0]{\end{tabular}}
\newcommand{\captionTable}[1]{\caption{#1}}
\newcommand{\labelTable}[1]{\label{#1}}
%\newcommand{\completeTable}[0]{\end{table}\endgroup}

\newcommand{\startTable}[1]{\begingroup\begin{longtable}[!h]{|>{\raggedright\arraybackslash}p{8.8cm} | p{6.2cm} |}\hline\cellcolor{lightgray}\textbf{#1} & \cellcolor{lightgray}\textbf{Description}\\ \hline \endhead}
\newcommand{\startTableDim}[3]{\begingroup\begin{longtable}[!h]{|>{\raggedright\arraybackslash}p{#2} | p{#3} |}\hline\cellcolor{lightgray}\textbf{#1} & \cellcolor{lightgray}\textbf{Description}\\ \hline \endhead}
\newcommand{\addFootNotes}[2]{\hline\multicolumn{2}{|r|}{{Continued on next page}} \\ \hline
\caption{#1} \\ \endfoot \hline\multicolumn{2}{|r|}{{End of table}}\\\hline\caption{#1}  \labelTable{#2}\endlastfoot}
\newcommand{\completeTable}[0]{\end{longtable}\endgroup}

\newcommand{\startInfoTable}[1]{\begingroup\begin{longtable}{| p{5.6cm} | p{3cm} | p{6.6cm} |}
\hline\cellcolor{lightgray}\textbf{#1} & \cellcolor{lightgray}\textbf{Return type} & \cellcolor{lightgray}\textbf{Description}\\ \hline \endhead}

\newcommand{\startInfoTableDims}[4]{\begingroup\begin{longtable}{| p{#2} | p{#3} | p{#4} |}
\hline\cellcolor{lightgray}\textbf{#1} & \cellcolor{lightgray}\textbf{Return type} & \cellcolor{lightgray}\textbf{Description}\\ \hline \endhead}

\newcommand{\addInfoRow}[3]{\codeinline{#1} & \codeinline{#2} &{#3}\\ \hline}
\newcommand{\captionInfoTable}[1]{\caption{#1}}
\newcommand{\labelInfoTable}[1]{\label{#1}}

\newcommand{\addInfoFootNotes}[2]{\hline\multicolumn{3}{|r|}{{Continued on next page}} \\ \hline
\caption{#1} \\ \endfoot \hline\multicolumn{3}{|r|}{{End of table}}\\\hline\caption{#1}  \labelTable{#2}\endlastfoot}

\newcommand{\completeInfoTable}[0]{\end{longtable}\endgroup}

\newcommand{\startGenericTable}[2]{\begingroup\begin{table}[!h]\setlength{\extrarowheight}{5pt}\small\centering\begin{tabular}[b]{|p{8.8cm} | p{6.2cm} |}\hline\cellcolor{lightgray}\textbf{#1} & \cellcolor{lightgray}\textbf{#2}\\ \hline}
\newcommand{\genericRow}[2]{\codeinline{#1} & #2\\ \hline}
\newcommand{\genericRowTwoL}[3]{\twoLinesOfCode{#1}{#2} & #3 \\ \hline}
\newcommand{\genericRowThreeL}[4]{\threeLinesOfCode{#1}{#2}{#3} & #4 \\ \hline}
\newcommand{\completeGenericTabular}[0]{\end{tabular}}
\newcommand{\captionGenericTable}[1]{\caption{#1}}
\newcommand{\labelGenericTable}[1]{\label{#1}}
\newcommand{\completeGenericTable}[0]{\end{table}\endgroup}

\newcommand{\startGenericThreeColTable}[6]{\begingroup\begin{table}[!h]\setlength{\extrarowheight}{5pt}\small\centering\begin{tabular}[b]{|p{#1} | p{#2} | p{#3} |}\hline\cellcolor{lightgray}\textbf{#4} & \cellcolor{lightgray}\textbf{#5} & \cellcolor{lightgray}\textbf{#6}\\ \hline}
\newcommand{\genericThreeColRow}[3]{\codeinline{#1} & \codeinline{#2} & #3\\ \hline}


\newcommand{\startMethodTable}[0]{\begingroup\scriptsize\centering\begin{tabular}[c]{|p{1.5cm}|p{2cm}|p{3cm}|p{4cm}|p{4cm}|}\hline\cellcolor{gray}\textbf{Access} & \cellcolor{gray}\textbf{Return} & \cellcolor{gray}\textbf{Name} & \cellcolor{gray}\textbf{Parameters} & \cellcolor{gray}\textbf{Description}\\ \hline}
\newcommand{\methodTableRow}[5]{\lstinline$#1$ & \lstinline$#2$ & \lstinline$#3$ & \lstinline$#4$ & #5\\ \hline}
\newcommand{\completeMethodTable}[0]{\end{tabular}\endgroup}


\newcommand{\startInterfaceTable}[0]{\begingroup
        \begin{tabular}[c]{|c|p{8cm}|c|c|}
        \hline
        \cellcolor{gray}\textbf{Object} & \multicolumn{1}{c}{\cellcolor{gray}\textbf{Description}} & \multicolumn{2}{|c|}{\cellcolor{gray}\textbf{Visibility}}\\\hhline{|>{\arrayrulecolor{gray}}-->{\arrayrulecolor{black}}--}
        \cellcolor{gray}& \cellcolor{gray}& \cellcolor{gray}\textbf{Host} & \cellcolor{gray}\textbf{Device}\\ \hline}
%\newcommand{\startInterfaceTable}[0]{\begingroup
%        \begin{tabular}[c]{|c|p{8cm}|c|c|}
%        \hline
%        \cellcolor{gray}\textbf{Object} & \multicolumn{1}{c}{\cellcolor{gray}\textbf{Description}} & \multicolumn{2}{|c|}{\cellcolor{gray}\textbf{Visibility}}\\\arrayrulecolor{gray}\hline\arrayrulecolor{black}
%        \cellcolor{gray}& \cellcolor{gray}& \cellcolor{gray}\textbf{Host} & \cellcolor{gray}\textbf{Device}\\ \hline}
\newcommand{\interfaceTableRow}[4]{\lstinline$#1$ & #2 & #3 & #4\\ \hline}
\newcommand{\completeInterfaceTable}[0]{\end{tabular}\endgroup}


%\setlength{\parindent}{15pt}
%\setlength{\parindent}{0pt}
\setlength{\parskip}{10pt}



\usepackage{framed,color}
\definecolor{shadecolor}{gray}{0.9}
\definecolor{Gray}{gray}{0.9}


% Use a special tabbing environment that does not add too much
% vertical spacing before and after.
% I looked for example at
% http://tex.stackexchange.com/questions/16383/removing-vertical-space-before-and-after-tabbing-environment
% but it does not work in the tabular environments...

% So start back from the source
% ftp://ftp.dante.de/pub/tex/macros/latex/unpacked/lttab.dtx
% /usr/share/texlive/texmf-dist/tex/latex/base/latex.ltx
% and apply some modifications:
\makeatletter
\gdef\nstabbing{\lineskip \z@skip\let\>\@rtab\let\<\@ltab\let\=\@settab
     \let\+\@tabplus\let\-\@tabminus\let\`\@tabrj\let\'\@tablab
     \let\\=\@tabcr
     \@hightab\@firsttab
     \global\@nxttabmar\@firsttab
     \dimen\@firsttab\@totalleftmargin
     \global\@tabpush\z@ \global\@rjfieldfalse
     % Remove the trivlist and the \vskip that may insert vertical space
     %\trivlist \item\relax
     %\if@minipage\else\vskip\parskip\fi
     %\setbox\@tabfbox\hbox{%
     %  \rlap{\hskip\@totalleftmargin\indent\the\everypar}}%
     \def\@itemfudge{\box\@tabfbox}%
     \@startline\ignorespaces}
\gdef\endnstabbing{%
  \@stopline\ifnum\@tabpush >\z@ \@badpoptabs \fi
  % \endtrivlist
}

% To skip the Apache 2.0 license added in front of C++ files
\usepackage{xparse}
% Declare \lstinputlistingSkipLicense as a command accepting optional
% [parameters] to keep them for the real \lstinputlisting
\NewDocumentCommand{\lstinputlistingSkipLicense}
                   {o}
                   {\IfNoValueTF{#1}
                    {%There was no optional [parameter]
                     \lstinputlisting[% Skip the license text which is
                                      % 14 line long:
                                      firstline=15]}
                    {%There were some optional [parameters]
                     \lstinputlisting[% Skip the license text which is
                                      % 14 line long:
                                      firstline=15,
                                      % Concatenate the provided options:
                                      #1]}}

% Do not cross the right margin
\sloppy

% Consider '_' as a (almost) normal character
\usepackage[strings]{underscore}


%%% SYCL Algebraic definitions
\DeclareMathOperator{\SYCLeval}{Satisfied}
\DeclareMathOperator{\SYCLperform}{Perform}

% Read the git description file if any
% (Use a LaTeX package such as gitinfo2 if something more complex is required)
\newcommand{\GitDescribe}{\InputIfFileExists{git_description}{}{}}

%%% Local Variables:
%%% mode: latex
%%% TeX-master: "sycl-1.2.1"
%%% End:
