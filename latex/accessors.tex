% Copyright (c) 2012-2019 Khronos Group.
%
% This work is licensed under a Creative Commons Attribution 4.0
% International License.
% http://creativecommons.org/licenses/by/4.0/

%***********************************************************************************
% Accessors
%***********************************************************************************

An \gls{accessor} is defined by the SYCL \codeinline{accessor} class template.
An \codeinline{accessor} provides access to the data managed by a \gls{buffer}
or \gls{image}, or to shared \gls{local-memory} allocated by the runtime. An
\gls{accessor} allows users to define \textbf{requirements} to memory objects
(see Section~\ref{sub.section.memmodel.app}).  Note that construction of an
\gls{accessor} is what defines a memory object requirement, and these
requirements are independent of whether there are any uses of an \gls{accessor}.

The SYCL \codeinline{accessor} class template takes five template parameters:

\begin{itemize}
\item A typename specifying the data type that the \codeinline{accessor} is
providing access to.
\item An integer specifying the dimensionality of the accessor.
\item A value of \codeinline{access::mode} specifying the mode of access the
\codeinline{accessor} is providing.
\item A value of \codeinline{access::target} specifying the target of access
the \codeinline{accessor} is providing.
\item A value of \codeinline{access::placeholder} specifying whether the
\codeinline{accessor} is a placeholder accessor.
\end{itemize}

The parameters described above determine the data an \codeinline{accessor}
provides access to and the way in which that access is provided. This separation
allows a \gls{sycl-runtime} implementation to choose an efficient way to provide
access to the data within an execution schedule.

Because of this the interface of the \codeinline{accessor} will
be different depending on the possible combinations of those parameters. There
are three main categories of accessor; buffer accesors (see Section%
~\ref{sub.section.accessors.buffer}), local accessors (see Section%
~\ref{sub.section.accessors.local}) and image accessors (see Section%
~\ref{sub.section.accessors.image}).

%*******************************************************************************
% Access targets
%*******************************************************************************

\subsubsection{Access targets}
\label{sub.section.access.targets}

The access target of an \codeinline{accessor} specifies what the accessor is
providing access to.

The \codeinline{access::target} enumeration, shown in
Table~\ref{interfaces.accesstarget.description}, describes the potential targets
of an \codeinline{accessor}.

\lstinputlistingSkipLicense{headers/accessTarget.h}

%-------------------------------------------------------------------------------
\startTable{access::target}
\addFootNotes{Enumeration of access modes available to accessors}
{interfaces.accesstarget.description}
\addRow
    {access::target::global_buffer}{Access \codeinline{buffer} via \gls{global-memory}.}
  \addRow
    {access::target::constant_buffer}{Access \codeinline{buffer} via \gls{constant-memory}.}
  \addRow
    {access::target::local}{Access work-group \gls{local-memory}.}
  \addRow
    {access::target::image}{Access an \codeinline{image}.}
  \addRow
    {access::target::host_buffer}{Access a \codeinline{buffer} immediately in host code.}
  \addRow
    {access::target::host_image}{Access an \codeinline{image} immediately in host code.}
  \addRow
    {access::target::image_array}{Access an array of \codeinline{image}s on a device.}
\completeTable
%-------------------------------------------------------------------------------

%*******************************************************************************
% Access modes
%*******************************************************************************

\subsubsection{Access modes}
\label{sub.section.access.mode}

The access mode of an \codeinline{accessor} specifies the kind of access that is
being provided. This information is used by the runtime to ensure that any
data dependencies are resolved by enqueuing any data transfers before
or after the execution of a kernel. If a command group contains only
\keyword{discard write mode} accesses to a buffer, then the previous contents
of the buffer (or sub-range of the buffer, if provided) are not
preserved. If a user wants to modify only certain parts of a buffer,
preserving other parts of the buffer, then the user should specify the
exact sub-range of modification of the buffer.
Atomic access is only valid to \codeinline{local}, \codeinline{global_buffer}
and \codeinline{host_buffer} targets (see next section).

The \codeinline{access::mode} enumeration, shown in Table~\ref{interfaces.accessmode.description}, describes the potential modes of an \codeinline{accessor}.

\lstinputlistingSkipLicense{headers/accessMode.h}

%-------------------------------------------------------------------------------
\startTable{access::mode}
\addFootNotes{Enumeration of access modes available to accessors}
{interfaces.accessmode.description}
 \addRow
    {access::mode::read}
    {%
      Read-only access.
    }
  \addRow
    {access::mode::write}
    {%
     Write-only access.
     Previous contents not discarded.
    }
  \addRow
    {access::mode::read_write}
    {%
      Read and write access.
    }
  \addRow
    {access::mode::discard_write}
    {%
      Write-only access.
      Previous contents discarded.
    }
  \addRow
    {access::mode::discard_read_write}
    {%
      Read and write access.
      Previous contents discarded.
    }
  \addRow
    {access::mode::atomic}
    {%
      Read and write atomic access.
    }
\completeTable
%-------------------------------------------------------------------------------

%*******************************************************************************
% Device and host accessors
%*******************************************************************************

\subsubsection{Device and host accessors}

A SYCL \codeinline{accessor} can be a device accessor in which case it provides
access to data within a SYCL kernel function, or a host accessor in which case
it provides immediate access on the host.

If an \codeinline{accessor} has the access target \codeinline{
access::target::global_buffer}, \codeinline{access::target::constant_buffer},
\codeinline{access::target::local}, \codeinline{access::target::image} or
\codeinline{access::target::image_array} then it is considered a device
accessor, and therefore can only be used within a SYCL kernel function and must
be associated with a \gls{command-group}. Creating a device accessor is a
non-blocking operation which defines a requirement on the device and adds the
requirement to the queue.

If an \codeinline{accessor} has the access target \codeinline{
access::target::host_buffer} or \codeinline{access::target::host_image} then it
is considered a host accessor and can only be used on the \gls{host}. Creating a
host accessor is a blocking operation which defines a requirement on the host
and blocks the caller until the requirement is satisfied.

A host accessor provides immediate access and continues to provide access until
it is destroyed.

%*******************************************************************************
% Placeholder accessor
%*******************************************************************************

\subsubsection{Placeholder accessor}
\label{sub.section.access.placeholder}

A placeholder accessor can be constructed outside of a command group and then
later bound to a command group. A SYCL \codeinline{accessor} is considered a
placeholder accessor if it has the access placeholder \codeinline{
access::placeholder::true_t}.

Accessors can optionally be defined as \keyword{placeholder} accessors.
A \keyword{placeholder} accessor defines an accessor instance that is
not bound to a specific \gls{command-group}. The accessor defines only
the type of the accessor (target memory, access mode, base type, \ldots).
When associated with a a command group using the appropriate
handler interface, it defines a \textbf{requirement} for the command group.
The same placeholder accessor can be required by multiple command groups.

The \codeinline{access::placeholder} enumeration, shown in
Table~\ref{interfaces.placeholder.description}, describes the potential
placeholder values of an \codeinline{accessor}.

%-------------------------------------------------------------------------------
\startTable{placeholder::mode}
 \addFootNotes{Enumeration of placeholder values available to accessors}
 {interfaces.placeholder.description}
 \addRow
    {access::placeholder::false_t}
    {
      Non-placeholder accessor.
    }
  \addRow
    {access::placeholder::true_t}
    {
      Placeholder accessor.
    }
\completeTable
%-------------------------------------------------------------------------------

%*******************************************************************************
% Buffer accessor
%*******************************************************************************

\subsubsection{Buffer accessor}
\label{sub.section.accessors.buffer}

A buffer accessor provides access to a SYCL \codeinline{buffer} instance. A SYCL \codeinline{accessor} is considered a buffer accessor if it has the access
target \codeinline{access::target::global_buffer}, \codeinline{
access::target::constant_buffer} or \codeinline{access::target::host_buffer}.

A buffer accessor can provide access to memory managed by a SYCL \codeinline{
buffer} class via either \gls{global-memory} or \gls{constant-memory},
corresponding to the access targets \codeinline{access::target::global_buffer}
and \codeinline{access::target::constant_buffer} respectively. A buffer accessor
accessing a SYCL \codeinline{buffer} via \gls{constant-memory} is restricted by
the available \gls{constant-memory} available on the SYCL \codeinline{device}
being executed on.

Alternatively a buffer accessor can provide access to memory managed by a SYCL
\codeinline{buffer} immediately on the \gls{host}, using the access target
\codeinline{access::target::host_buffer}. If the SYCL \codeinline{buffer} this
SYCL \codeinline{accessor} is accessing was constructed with the property 
\codeinline{property::buffer::use_host_ptr} the address of the memory accessed
on the \gls{host} must be the address the SYCL \codeinline{buffer} was
constructed with, otherwise the \gls{sycl-runtime} is free to allocate temporary
memory to provide access on the \gls{host}.

The data type of a buffer accessor must match that of the SYCL \codeinline{
buffer} which it is accessing.

The dimensionality of a buffer accessor must match that of the SYCL \codeinline{
buffer} which it is accessing, with the exception of \codeinline{0} in which
case the dimensionality of the SYCL \codeinline{buffer} must be \codeinline{1}.

There are three ways a SYCL \codeinline{accessor} can provide access to the
elements of a SYCL \codeinline{buffer}. Firstly by passing a SYCL \codeinline{
id} instance of the same dimensionality as the SYCL \codeinline{accessor}
subscript operator. Secondly by passing a single \codeinline{size_t} value to
multiple consecutive subscript operators (one for each dimension of the SYCL
\codeinline{accessor}, for example \codeinline{acc[id0][id1][id2]}). Finally, in the
case of the SYCL \codeinline{accessor} being \codeinline{0} dimensions, by
triggering the implicit conversion operator. Whenever a multi-dimensional index
is passed to a SYCL \codeinline{accessor} the linear index is calculated based
on the index \codeinline{\{id0, id1, id2\}} provided and the range of the SYCL
\codeinline{accessor} \codeinline{\{r0, r1, r2\}} according to row-major
ordering as follows:

\begin{equation}
\label{row-major-equation-buffer}
 id2 + (id1 \cdot r2) + (id0 \cdot r2 \cdot r1)
\end{equation}

A buffer accessor can optionally provide access to a sub range of a SYCL
\codeinline{buffer} by providing a range and offset on construction. In this
case the \gls{sycl-runtime} will only guarantee the latest copy of the data is
available in that given range and any modifications outside that range are
considered undefined behavior. This allows the \gls{sycl-runtime} to perform
optimizations such as reducing copies between devices. The indexing performed
when a SYCL \codeinline{accessor} provides access to the elements of a SYCL
\codeinline{buffer} is unaffected, i.e, the accessor will continue to index
from \codeinline{\{0,0,0\}}. This allows the offset to be provided either
manually or via the \codeinline{parallel_for} as in
\ref{listing.accessors.range}.

\begin{lstlisting}[label=listing.accessors.range]
    myQueue.submit([&](handler &cgh) {
      auto singleRange = range<3>(8, 16, 16);
      auto offset = id<3>(8, 0, 0);
      // We define the subset of the accessor we require for the kernel
      accessor<int, 1, access::mode::read_write, access::target::global_buffer>
          ptr(syclBuffer, cgh, singleRange, offset);
      // We offset the kernel by the same value to match indexes
      cgh.parallel_for<kernel>(singleRange, offset, [=](item<3> itemID) {
        ptr[itemID.get_linear_id()] = 2;
      });
    });
\end{lstlisting}

A buffer accessor with access target \codeinline{access::target::global_buffer}
can optionally provide atomic access to a SYCL \codeinline{buffer}, using the
access mode \codeinline{access::mode::atomic}, in which case all operators which
return an element of the SYCL \codeinline{buffer} return an instance of the SYCL
\codeinline{atomic} class.

The full list of capabilities that buffer accessors can support is described
in~\ref{table.accessors.buffer.capabilities}.

%-------------------------------------------------------------------------------
\begin{table}[!h]
    \setlength{\extrarowheight}{5pt}\scriptsize
    \begin{tabular}{| p{0.8 in} || p{0.4 in} | p{1.0 in} | p{1.4 in} | p{0.9 in} | p{0.6 in} |}
      \hline
        \cellcolor{lightgray} \textbf{Access target}
        & \cellcolor{lightgray} \textbf{Accessor type}
        & \cellcolor{lightgray} \textbf{Access modes}
        & \cellcolor{lightgray} \textbf{Data types}
        & \cellcolor{lightgray} \textbf{Dimensionalities}        
        & \cellcolor{lightgray} \textbf{Placeholder modes} \\
      \hline
        \tf{global_buffer}
        & device
        & \nlineVI{\tf{read}}{\tf{write}}{\tf{read_write}}{\tf{discard_write}} {\tf{discard_read_write}}{\tf{atomic}}
        & The data type of the SYCL buffer being accessed.
        & Between \tf{0} and \tf{3} (inclusive).
        & \nlineII{\tf{false_t}}{\tf{true_t}} \\
      \hline
        \tf{constant_buffer}
        & device
        & \tf{read}
        & The data type of the SYCL buffer being accessed.
        & Between \tf{0} and \tf{3} (inclusive).
        & \nlineII{\tf{false_t}}{\tf{true_t}} \\
      \hline
        \tf{host_buffer}
        & host
        & \nlineV{\tf{read}}{\tf{write}}{\tf{read_write}}{\tf{discard_write}} {\tf{discard_read_write}}
        & The data type of the SYCL buffer being accessed.
        & Between \tf{0} and \tf{3} (inclusive).
        & \nline{\tf{false_t}} \\
      \hline
    \end{tabular}
    \caption{Description of all the buffer accessor capabilities}
    \label{table.accessors.buffer.capabilities}
\end{table}
%-------------------------------------------------------------------------------

\subsubsection{Buffer accessor interface}

A synopsis of the SYCL \codeinline{accessor} class template buffer
specialization is provided below. The constructors and member functions of the
SYCL \codeinline{accessor} class template buffer specialization are listed in
Tables~\ref{table.constructors.accessor.buffer} and
\ref{table.members.accessor.buffer} respectively. The additional common special
member functions and common member functions are listed in
\ref{sec:reference-semantics} in
Tables~\ref{table.specialmembers.common.reference} and
\ref{table.hiddenfriends.common.reference}, respectively.

%Interface for class: accessor
\lstinputlistingSkipLicense[captionpos=b,caption=Accessor class for buffers, label=accessor.buffer.interface]
{headers/accessorBuffer.h}

%------------------------------------------------------------------------------
\startTable{Constructor}
\addFootNotes{Constructors of the \codeinline{accessor} class template buffer
  specialization}
{table.constructors.accessor.buffer}
  \addRow
    { accessor(buffer<dataT, 1, AllocatorT> \&bufferRef) }
    {
      Available only when: \codeinline{((isPlaceholder ==
      access::placeholder::false_t \&\& accessTarget ==
      access::target::host_buffer) || (isPlaceholder ==
      access::placeholder::true_t \&\& (accessTarget ==
      access::target::global_buffer || accessTarget ==
      access::target::constant_buffer))) \&\& dimensions == 0}.
      \newline
      If \codeinline{isPlaceholder == access::placeholder::false_t}, constructs
      a SYCL \codeinline{accessor} instance for accessing a single element of a
      SYCL \codeinline{buffer} immediately on the host. If \codeinline{
      isPlaceholder == access::placeholder::true_t}, constructs a SYCL
      placeholder \codeinline{accessor}.
    }
  \addRowTwoL
    { accessor(buffer<dataT, 1, AllocatorT> \&bufferRef, }
    { handler \&commandGroupHandlerRef) }
    {
      Available only when: \codeinline{(isPlaceholder ==
      access::placeholder::false_t \&\& (accessTarget ==
      access::target::global_buffer || accessTarget ==
      access::target::constant_buffer)) \&\& dimensions == 0}.
      \newline
      Constructs a SYCL \codeinline{accessor} instance for accessing a single
      element of a SYCL \codeinline{buffer} within a SYCL kernel function on
      the SYCL \codeinline{queue} associated with \codeinline{
      commandGroupHandlerRef}.
    }
  \addRow
    { accessor(buffer<dataT, dimensions, AllocatorT> \&bufferRef) }
    {
      Available only when: \codeinline{((isPlaceholder ==
      access::placeholder::false_t \&\& accessTarget ==
      access::target::host_buffer) || (isPlaceholder ==
      access::placeholder::true_t \&\& (accessTarget ==
      access::target::global_buffer || accessTarget ==
      access::target::constant_buffer))) \&\& dimensions > 0}.
      \newline
      If \codeinline{isPlaceholder == access::placeholder::false_t}, constructs
      a SYCL \codeinline{accessor} instance for accessing a SYCL \codeinline{
      buffer} immediately on the host. If \codeinline{isPlaceholder ==
      access::placeholder::true_t}, constructs a SYCL placeholder \codeinline{
      accessor}.
    }
  \addRowTwoL
    { accessor(buffer<dataT, dimensions, AllocatorT> \&bufferRef, }
    { handler \&commandGroupHandlerRef) }
    {
      Available only when: \codeinline{(isPlaceholder ==
      access::placeholder::false_t \&\& (accessTarget ==
      access::target::global_buffer || accessTarget ==
      access::target::constant_buffer)) \&\& dimensions > 0}.
      \newline
      Constructs a SYCL \codeinline{accessor} instance for accessing a SYCL
      \codeinline{buffer} within a SYCL kernel function on the SYCL
      \codeinline{queue} associated with \codeinline{commandGroupHandlerRef}.
    }
  \addRowThreeL
    { accessor( buffer<dataT, dimensions, AllocatorT> \&bufferRef, }
    { range<dimensions> accessRange, id<dimensions> }
    { accessOffset = \{\}) }
    {
      Available only when: \codeinline{(isPlaceholder ==
      access::placeholder::false_t \&\& accessTarget ==
      access::target::host_buffer) ||  (isPlaceholder ==
      access::placeholder::true_t \&\& (accessTarget ==
      access::target::global_buffer || accessTarget ==
      access::target::constant_buffer)) \&\& dimensions > 0}.
      \newline
      If \codeinline{isPlaceholder == access::placeholder::false_t}, constructs
      a SYCL \codeinline{accessor} instance for accessing a range of a SYCL
      \codeinline{buffer} immediately on the host. If \codeinline{
      isPlaceholder == access::placeholder::true_t}, constructs a SYCL
      placeholder \codeinline{accessor}.
    }
  \addRowFourL
    { accessor( buffer<dataT, dimensions, AllocatorT> \&bufferRef, }
    { handler \&commandGroupHandlerRef, }
    { range<dimensions> accessRange, }
    { id<dimensions> accessOffset = \{\}) }
    {
      Available only when: \codeinline{(isPlaceholder ==
      access::placeholder::false_t \&\& (accessTarget ==
      access::target::global_buffer || accessTarget ==
      access::target::constant_buffer)) \&\& dimensions > 0}.
      \newline
      Constructs a SYCL \codeinline{accessor} instance for accessing a range of
      SYCL \codeinline{buffer} within a SYCL kernel function on the SYCL
      \codeinline{queue} associated with \codeinline{commandGroupHandlerRef},
      specified by \codeinline{accessRange} and \codeinline{accessOffset}.
    }
\completeTable
%-------------------------------------------------------------------------------

%-------------------------------------------------------------------------------
\startTable{Member function}
\addFootNotes{Member functions of the \codeinline{accessor} class template
  buffer specialization}
{table.members.accessor.buffer}
  \addRow
    { constexpr bool is_placeholder() const }
    {
      Returns \codeinline{true} if \codeinline{isPlaceholder ==
      access::placeholder::true_t} otherwise returns \codeinline{false}.
    }
  \addRow
    { size_t get_size() const }
    {
      Returns the size in bytes of the region of a SYCL \codeinline{buffer}
      that this SYCL \codeinline{accessor} is accessing.
    }
  \addRow
    { size_t get_count() const }
    {
      Returns the number of elements in the region of a SYCL \codeinline{buffer}
      that this SYCL \codeinline{accessor} is accessing.
    }
  \addRow
    { range<dimensions> get_range() const }
    {
      Available only when: \codeinline{dimensions > 0}.
      \newline
      Returns the range of this SYCL \codeinline{accessor}.
    }
  \addRow
    { id<dimensions> get_offset() const }
    {
      Available only when: \codeinline{dimensions > 0}.
      \newline
      Returns the offset of this SYCL \codeinline{accessor}.
    }
  \addRow
    { operator dataT \&() const }
    {
      Available only when: \codeinline{(accessMode == access::mode::write ||
      accessMode == access::mode::read_write || accessMode ==
      access::mode::discard_write || accessMode ==
      access::mode::discard_read_write) \&\& dimensions == 0)}.
      \newline
      Returns a reference to the element stored within the SYCL \codeinline{
      buffer} this SYCL \codeinline{accessor} is accessing.
    }
  \addRow
    { dataT \&operator[](id<dimensions> index) const }
    {
      Available only when: \codeinline{(accessMode == access::mode::write ||
      accessMode == access::mode::read_write || accessMode ==
      access::mode::discard_write || accessMode ==
      access::mode::discard_read_write) \&\& dimensions > 0)}.
      \newline
      Returns a reference to the element stored within the SYCL \codeinline{
      buffer} this SYCL \codeinline{accessor} is accessing at the index
      specified by \codeinline{index}.
    }
  \addRow
    { dataT \&operator[](size_t index) const }
    {
      Available only when: \codeinline{(accessMode == access::mode::write ||
      accessMode == access::mode::read_write || accessMode ==
      access::mode::discard_write || accessMode ==
      access::mode::discard_read_write) \&\& dimensions == 1)}.
      \newline
      Returns a reference to the element stored within the SYCL \codeinline{
      buffer} this SYCL \codeinline{accessor} is accessing at the index
      specified by \codeinline{index}.
    }
  \addRow
    { operator dataT() const }
    {
      Available only when: \codeinline{accessMode == access::mode::read \&\&
      dimensions == 0)}.
      \newline
      Returns the value of the element stored within the SYCL \codeinline{
      buffer} this SYCL \codeinline{accessor} is accessing.
    }
  \addRow
    { dataT operator[](id<dimensions> index) const }
    {
      Available only when: \codeinline{accessMode == access::mode::read \&\&
      dimensions > 0)}.
      \newline
      Returns the value of the element stored within the SYCL \codeinline{
      buffer} this SYCL \codeinline{accessor} is accessing at the index
      specified by \codeinline{index}.
    }
  \addRow
    { dataT operator[](size_t index) const }
    {
      Available only when: \codeinline{accessMode == access::mode::read \&\&
      dimensions == 1)}.
      \newline
      Returns the value of the element stored within the SYCL \codeinline{
      buffer} this SYCL \codeinline{accessor} is accessing at the index
      specified by \codeinline{index}.
    }    
  \addRowTwoL
    { operator atomic<dataT, }
    { access::address_space::global_space> () const }
    {
      Available only when: \codeinline{accessMode == access::mode::atomic \&\&
      dimensions == 0)}.
      \newline
      Returns an instance of SYCL \codeinline{atomic} of type \codeinline{dataT}
      providing atomic access to the element stored within the SYCL \codeinline{
      buffer} this SYCL \codeinline{accessor} is accessing.
    }
  \addRowTwoL
    { atomic<dataT, access::address_space::global_space> }
    { operator[](id<dimensions> index) const }
    {
      Available only when: \codeinline{accessMode == access::mode::atomic \&\&
      dimensions > 0)}.
      \newline
      Returns an instance of SYCL \codeinline{atomic} of type \codeinline{dataT}
      providing atomic access to the element stored within the SYCL \codeinline{
      buffer} this SYCL \codeinline{accessor} is accessing at the index
      specified by \codeinline{index}.
    }
  \addRowTwoL
    { atomic<dataT, access::address_space::global_space> }
    { operator[](size_t index) const }
    {
      Available only when: \codeinline{accessMode == access::mode::atomic \&\&
      dimensions == 1)}.
      \newline
      Returns an instance of SYCL \codeinline{atomic} of type \codeinline{dataT}
      providing atomic access to the element stored within the SYCL \codeinline{
      buffer} this SYCL \codeinline{accessor} is accessing at the index
      specified by \codeinline{index}.
    }
  \addRow
    { \__unspecified__ \&operator[](size_t index) const }
    {
      Available only when: \codeinline{dimensions > 1}.
      \newline
      Returns an instance of an undefined intermediate type representing a
      a SYCL \codeinline{accessor} of the same type as this SYCL \codeinline{
      accessor}, with the dimensionality \codeinline{dimensions-1} and
      containing an implicit SYCL \codeinline{id} with index \codeinline{
      dimensions} set to \codeinline{index}. The intermediate type returned
      must provide all available subscript operators which take a \codeinline{
      size_t} parameter defined by the SYCL \codeinline{accessor} class that
      are appropriate for the type it represents (including this subscript
      operator).
    }
  \addRow
    { dataT *get_pointer() const}
    {
      Available only when: \codeinline{accessTarget ==
      access::target::host_buffer}.
      Returns a pointer to the memory this SYCL \codeinline{accessor} memory is
      accessing.
    }
  \addRow
    { global_ptr<dataT> get_pointer() const}
    {
      Available only when: \codeinline{accessTarget ==
      access::target::global_buffer}.
      Returns a pointer to the memory this SYCL \codeinline{accessor} memory is
      accessing.
    }
  \addRow
    { constant_ptr<dataT> get_pointer() const}
    {
      Available only when: \codeinline{accessTarget ==
      access::target::global_buffer}.
      Returns a pointer to the memory this SYCL \codeinline{accessor} memory is
      accessing.
    }
\completeTable
%-------------------------------------------------------------------------------

%*******************************************************************************
% Local accessor
%*******************************************************************************

\subsubsection{Local accessor}
\label{sub.section.accessors.local}

A local accessor provides access to \gls{sycl-runtime} allocated shared memory via
\gls{local-memory}. A SYCL \codeinline{accessor} is considered a local accessor
if it has the access target \codeinline{access::target::local}.
The memory allocated by a local accessor is non-initialised
so it is the user's responsibility to construct and destroy objects explicitly
if required. The \gls{local-memory} that is allocated is shared between
all \glspl{work-item} of a \gls{work-group}.

A local accessor does not provide access on the \gls{host} and the
memory can not be copied back to the \gls{host}.

The data type of a local accessor can be any valid SYCL kernel argument (see
Section~\ref{sec:kernel.parameter.passing}.

The size of memory allocated by the \gls{sycl-runtime} is specified by a SYCL
\codeinline{range} provided on construction. The dimensionality of the SYCL
\codeinline{range} provided must match the SYCL \codeinline{accessor}, with the
exception of \codeinline{0} in which case the dimensionality of the SYCL
\codeinline{range} must be \codeinline{0}.

There are three ways that a SYCL \codeinline{accessor} can provide access to the
elements of the allocated memory. Firstly by passing a SYCL \codeinline{
id} instance of the same dimensionality as the SYCL \codeinline{accessor}
subscript operator. Secondly by passing a single \codeinline{size_t} value to
multiple consecutive subscript operators (one for each dimension of the SYCL
\codeinline{accessor}, for example \codeinline{acc[z][y][x]}). Finally, in the
case of the SYCL \codeinline{accessor} having \codeinline{0} dimensions, by
triggering the implicit conversion operator. Whenever a multi-dimensional index
is passed to a SYCL \codeinline{accessor}, the linear index is calculated based
on the index \codeinline{\{id0, id1, id2\}} provided and the range of the SYCL 
\codeinline{accessor} \codeinline{\{r0, r1, r2\}} according to row-major
ordering as follows:

\begin{equation}
\label{row-major-equation-local}
 id2 + (id1 \cdot r2) + (id0 \cdot r2 \cdot r1)
\end{equation}

A local accessor can optionally provide atomic access to allocated memory,
using the access mode \codeinline{access::mode::atomic}, in which case all
operators which return an element of the allocated memory return an instance of
the SYCL \codeinline{atomic} class.

Local accessors are not valid in the \codeinline{single_task} or basic 
\codeinline{parallel_for} SYCL kernel function invocations, due the fact that
local \glspl{work-group} are implicitly created, and the implementation is free
to choose any size.

The full list of capabilities that local accessors can support is described
in~\ref{table.accessors.local.capabilities}.

%-------------------------------------------------------------------------------
\begin{table}[!h]
    \setlength{\extrarowheight}{5pt}\scriptsize
    \begin{tabular}{| p{0.8 in} || p{0.4 in} | p{1.0 in} | p{1.4 in} | p{0.9 in} | p{0.6 in} |}
      \hline
        \cellcolor{lightgray} \textbf{Access target}
        & \cellcolor{lightgray} \textbf{Accessor type}
        & \cellcolor{lightgray} \textbf{Access modes}
        & \cellcolor{lightgray} \textbf{Data types}
        & \cellcolor{lightgray} \textbf{Dimensionalities}        
        & \cellcolor{lightgray} \textbf{Placeholder modes} \\
      \hline
        \tf{local}
        & device
        & \nlineII{\tf{read_write}}{\tf{atomic}}
        & All available data types supported in a SYCL kernel function.
        & Between \tf{0} and \tf{3} (inclusive).
        & \tf{false_t} \\
      \hline
    \end{tabular}
    \caption{Description of all the local accessor capabilities}
    \label{table.accessors.local.capabilities}
\end{table}
%-------------------------------------------------------------------------------

\subsubsection{Local accessor interface}

A synopsis of the SYCL \codeinline{accessor} class template local
specialization is provided below. The constructors and member functions of the
SYCL \codeinline{accessor} class template local specialization are listed in
Tables~\ref{table.constructors.accessor.local} and
\ref{table.members.accessor.local} respectively. The additional common special
member functions and common member functions are listed in
\ref{sec:reference-semantics} in Tables~\ref{table.specialmembers.common.reference} and
\ref{table.hiddenfriends.common.reference}, respectively.

%Interface for class: accessor
\lstinputlistingSkipLicense[captionpos=b,caption=Accessor class for locals, label=accessor.local.interface]
{headers/accessorLocal.h}

%-------------------------------------------------------------------------------
\startTable{Constructor}
\addFootNotes {Constructors of the \codeinline{accessor} class template local
  specialization}
{table.constructors.accessor.local}
  \addRow
    { accessor(handler \&commandGroupHandlerRef) }
    {
      Available only when: \codeinline{dimensions == 0}.
      \newline
      Constructs a SYCL \codeinline{accessor} instance for accessing runtime
      allocated shared local memory of a single element (of type \codeinline{dataT}) within
      a SYCL kernel function on the SYCL \codeinline{queue} associated with \codeinline{
      commandGroupHandlerRef}.  The allocation is per work-group.
    }
  \addRowTwoL
    { accessor(range<dimensions> allocationSize, }
    { handler \&commandGroupHandlerRef) }
    {
      Available only when: \codeinline{dimensions > 0}.
      \newline      
      Constructs a SYCL \codeinline{accessor} instance for accessing runtime
      allocated shared local memory of size specified by \codeinline{
      allocationSize} within a SYCL kernel function on the SYCL \codeinline{
      queue} associated with \codeinline{commandGroupHandlerRef}.  \codeinline{allocationSize}
      defines the number of elements of type \codeinline{dataT} to be allocated.  The allocation
      is per work-group, and if multiple work-groups execute simultaneously in an
      implementation, each work-group will receive its own functionally independent
      allocation of size \codeinline{allocationSize} elements of type \codeinline{dataT}.
    }
\completeTable
%-------------------------------------------------------------------------------

%-------------------------------------------------------------------------------
\startTable{Member function}
\addFootNotes{Member functions of the \codeinline{accessor} class template
  local specialization}
{table.members.accessor.local}
  \addRow
    { size_t get_size() const }
    {
      Returns the size in bytes of the local memory allocation, per work-group,
      that this SYCL \codeinline{accessor} is accessing.
    }
  \addRow
    { size_t get_count() const }
    {
      Returns the number of \codeinline{dataT} elements in the local memory allocation, per work-group,
      that this SYCL \codeinline{accessor} is accessing.
    }
  \addRow
    { operator dataT \&() const }
    {
      Available only when: \codeinline{accessMode == access::mode::read_write
      \&\& dimensions == 0)}.
      \newline
      Returns a reference to the single element stored within the work-group's local memory
      allocation that this \codeinline{accessor} is accessing.
    }
  \addRow
    { dataT \&operator[](id<dimensions> index) const }
    {
      Available only when: \codeinline{accessMode == access::mode::read_write
      \&\& dimensions > 0)}.
      \newline
      Returns a reference to the element stored within the work-group's local memory
      allocation that this SYCL \codeinline{accessor} is accessing, at the index
      specified by \codeinline{index}.
    }
  \addRow
    { dataT \&operator[](size_t index) const }
    {
      Available only when: \codeinline{accessMode == access::mode::read_write
      \&\& dimensions == 1)}.
      \newline
      Returns a reference to the element stored within the work-group's local memory
      allocation that this SYCL \codeinline{accessor} is accessing at the index
      specified by \codeinline{index}.
    }
  \addRowTwoL
    { operator atomic<dataT, }
    { access::address_space::local_space> \&() const }
    {
      Available only when: \codeinline{accessMode == access::mode::atomic \&\&
      dimensions == 0)}.
      \newline
      Returns a reference to an instance of SYCL \codeinline{atomic} of type
      \codeinline{dataT} providing atomic access to the element stored within
      the work-group's local memory allocation that this SYCL \codeinline{accessor}
      is accessing.
    }
  \addRowTwoL
    { atomic<dataT, access::address_space::local_space> \& }
    { operator[](id<dimensions> index) const }
    {
      Available only when: \codeinline{accessMode == access::mode::atomic \&\&
      dimensions > 0)}.
      \newline
      Returns a reference to an instance of SYCL \codeinline{atomic} of type
      \codeinline{dataT} providing atomic access to the element stored within
      the work-group's local memory allocation that this SYCL \codeinline{accessor}
      is accessing, at the index specified by \codeinline{index}.
    }
  \addRowTwoL
    { atomic<dataT, access::address_space::local_space> \& }
    { operator[](size_t index) const }
    {
      Available only when: \codeinline{accessMode == access::mode::atomic \&\&
      dimensions == 1)}.
      \newline
      Returns a reference to an instance of SYCL \codeinline{atomic} of type
      \codeinline{dataT} providing atomic access to the element stored within
      the work-group's local memory allocation that this SYCL \codeinline{accessor}
      is accessing, at the index specified by \codeinline{index}.
    }
  \addRow
    { \__unspecified__ \&operator[](size_t index) const }
    {
      Available only when: \codeinline{dimensions > 1}.
      \newline
      Returns an instance of an undefined intermediate type representing a
      a SYCL \codeinline{accessor} of the same type as this SYCL \codeinline{
      accessor}, with the dimensionality \codeinline{dimensions-1} and
      containing an implicit SYCL \codeinline{id} with index \codeinline{
      dimensions} set to \codeinline{index}. The intermediate type returned
      must provide all available subscript operators which take a \codeinline{
      size_t} parameter defined by the SYCL \codeinline{accessor} class that
      are appropriate for the type it represents (including this subscript
      operator).
    }
  \addRow
    { local_ptr<dataT> get_pointer() const}
    {
      Available only when: \codeinline{accessTarget ==
      access::target::local}.
      Returns a pointer to the work-group's local memory allocation that this
      SYCL \codeinline{accessor} is accessing.
    }
\completeTable
%------------------------------------------------------------------------------

%******************************************************************************
% Image accessor
%******************************************************************************

\subsubsection{Image accessor}
\label{sub.section.accessors.image}

An image accessor provides access to a SYCL \codeinline{image} instance. A SYCL \codeinline{accessor} is considered an image accessor if it has the access
target \codeinline{access::target::image}, \codeinline{
access::target::image_array} or \codeinline{access::target::host_image}.
      
An image accessor can provide access to memory managed by a SYCL \codeinline{
image} class, using the access target \codeinline{access::target::image} or
\codeinline{access::target::image_array}.

Alternatively an image accessor can provide access to memory managed by a SYCL
\codeinline{image} immediately on the \gls{host}, using the access target
\codeinline{access::target::host_image}. If the SYCL \codeinline{image} this
SYCL \codeinline{accessor} is accessing was constructed with the property 
\codeinline{property::image::use_host_ptr} the address of the memory accessed
on the \gls{host} must be the address the SYCL \codeinline{image} was
constructed with, otherwise the \gls{sycl-runtime} is free to allocate
temporary memory to provide access on the \gls{host}.

The data type of an image accessor must be either \codeinline{cl_int4}, \codeinline{
cl_uint4}, \codeinline{cl_float4} or \codeinline{cl_half4}.

The dimensionality of an image accessor must match that of the SYCL
\codeinline{image} which it is providing access to, with the exception of when
the access target is \codeinline{access::target::image_array}, in which case
the dimensionality of the SYCL \codeinline{accessor} must be \codeinline{1}
less.

An image accessor with the access target \codeinline{access::target::image} or
\codeinline{access::target::host_image} can provide access to the elements of a
SYCL \codeinline{image} by passing a SYCL \codeinline{cl_int4} or \codeinline{
cl_float4} instance to the \codeinline{read} or \codeinline{write} member
functions. The \codeinline{read} member function optionally takes a SYCL
\codeinline{sampler} instance to perform a sampled read of the image. For
example \codeinline{acc.read(coords, sampler)}.

An image accessor with the access target \codeinline{
access::target::image_array} can provide access to a slice of an image array by
passing a \codeinline{size_t} value to the subscript operator. This returns an
instance of \codeinline{\__image_array_slice__}, an unspecified type providing
the interface of \codeinline{accessor<dataT, dimensions, mode,
access::target::image, access::placeholder::false_t>} which will provide access
to a slice of the image array specified by \codeinline{index}. The
\codeinline{\__image_array_slice__} returned can then provide access via the
\codeinline{read} or \codeinline{write} member functions as described above. For
example \codeinline{acc[arrayIndex].read(coords, sampler)}.

The full list of capabilities that image accessors can support is described
in~\ref{table.accessors.image.capabilities}.

%-------------------------------------------------------------------------------
\begin{table}[!h]
    \setlength{\extrarowheight}{5pt}\scriptsize
    \begin{tabular}{| p{0.8 in} || p{0.4 in} | p{1.0 in} | p{1.4 in} | p{0.9 in} | p{0.6 in} |}
      \hline
        \cellcolor{lightgray} \textbf{Access target}
        & \cellcolor{lightgray} \textbf{Accessor type}
        & \cellcolor{lightgray} \textbf{Access modes}
        & \cellcolor{lightgray} \textbf{Data types}
        & \cellcolor{lightgray} \textbf{Dimensionalities}        
        & \cellcolor{lightgray} \textbf{Placeholder modes} \\
      \hline
        \tf{image}
        & device
        & \tf{\nlineIII{read}{write}{discard_write}}
        & \nlineIV{\tf{cl_int4}}{\tf{cl_uint4}}{\tf{cl_float4}}{\tf{cl_half4}}
        & Between \tf{1} and \tf{3} (inclusive).
        & \tf{false_t} \\
      \hline
        \tf{image_array}
        & device
        & \tf{\nlineIII{read}{write}{discard_write}}
        & \nlineIV{\tf{cl_int4}}{\tf{cl_uint4}}{\tf{cl_float4}}{\tf{cl_half4}}
        & Between \tf{1} and \tf{2} (inclusive).
        & \tf{false_t} \\
      \hline
        \tf{host_image}
        & host
        & \tf{\nlineIV{read}{write}{read_write}{discard_write}}
        & \nlineIV{\tf{cl_int4}}{\tf{cl_uint4}}{\tf{cl_float4}}{\tf{cl_half4}}
        & Between \tf{1} and \tf{3} (inclusive).
        & \tf{false_t} \\
      \hline
    \end{tabular}
    \caption{Description of all the image accessor capabilities}
    \label{table.accessors.image.capabilities}
\end{table}
%-------------------------------------------------------------------------------

\subsubsection{Image accessor interface}

A synopsis of the SYCL \codeinline{accessor} class template image
specialization is provided below. The constructors and member functions of the
SYCL \codeinline{accessor} class template image specialization are listed in
Tables~\ref{table.constructors.accessor.image} and
\ref{table.members.accessor.image} respectively. The additional common special
member functions and common member functions are listed in
\ref{sec:reference-semantics} in
Tables~\ref{table.specialmembers.common.reference} and
\ref{table.hiddenfriends.common.reference}, respectively.

\lstinputlistingSkipLicense[caption=Accessor interface for images,label=images.listing1]{headers/accessorImage.h}

%-------------------------------------------------------------------------------
\startTable{Constructor}
\addFootNotes {Constructors of the \codeinline{accessor} class template
  image specialization}
{table.constructors.accessor.image}
  \addRowThreeL
    { template <typename AllocatorT> }
    { accessor(image<dimensions, AllocatorT> }
    { \&imageRef) }
    {
      Available only when: \codeinline{accessTarget ==
      access::target::host_image}.
      \newline
      Constructs a SYCL \codeinline{accessor} instance for accessing a SYCL
      \codeinline{image} immediately on the host.
    }
  \addRowThreeL
    { template <typename AllocatorT> }
    { accessor(image<dimensions, AllocatorT> }
    { \&imageRef, handler \&commandGroupHandlerRef) }
    {
      Available only when: \codeinline{accessTarget ==
      access::target::image}.
      \newline
      Constructs a SYCL \codeinline{accessor} instance for accessing a SYCL
      \codeinline{image} within a SYCL kernel function on the SYCL
      \codeinline{queue} associated with \codeinline{commandGroupHandlerRef}.
    }
  \addRowThreeL
    { template <typename AllocatorT> }
    { accessor(image<dimensions + 1, AllocatorT> }
    { \&imageRef, handler \&commandGroupHandlerRef) }
    {
      Available only when: \codeinline{accessTarget ==
      access::target::image_array \&\& dimensions < 3}.
      \newline
      Constructs a SYCL \codeinline{accessor} instance for accessing a SYCL
      \codeinline{image} as an array of images, within a SYCL kernel function on
      the SYCL \codeinline{queue} associated with \codeinline{
      commandGroupHandlerRef}.
    }
\completeTable
%-------------------------------------------------------------------------------

%-------------------------------------------------------------------------------
\startTable{Member function}
  \addFootNotes{Member functions of the \codeinline{accessor} class template
    image specialization}
  {table.members.accessor.image}
  \addRow
    { size_t get_size() const }
    {
      Returns the size in bytes of the SYCL \codeinline{image} this SYCL
      \codeinline{accessor} is accessing.
    }
  \addRow
    { size_t get_count() const }
    {
      Returns the number of elements of the SYCL \codeinline{image} this SYCL \codeinline{accessor} is accessing.
    }
  \addRowTwoSL
    { template <typename coordT> }
    { dataT read(const coordT \&coords) const }
    {
      Available only when: \codeinline{(accessTarget == access::target::image \&\&
      accessMode == access::mode::read) || (accessTarget == access::target::host_image \&\&
      (accessMode == access::mode::read || accessMode == access::mode::read_write))}.
      \newline
      Reads and returns an element of the image at the coordinates specified by \codeinline{coords}. Permitted types for \codeinline{coordT} are \codeinline{
      cl_int} when \codeinline{dimensions == 1}, \codeinline{cl_int2} when \codeinline{dimensions ==
      2} and \codeinline{cl_int4} when \codeinline{dimensions == 3}.
    }
  \addRowTwoSL
    { template <typename coordT> }
    { dataT read(const coordT \&coords, const sampler \&smpl) const }
    {
      Available only when: \codeinline{(accessTarget == access::target::image \&\&
      accessMode == access::mode::read) || (accessTarget == access::target::host_image \&\&
      (accessMode == access::mode::read || accessMode == access::mode::read_write))}.
      \newline
      Reads and returns a sampled element of the image at the coordinates
      specified by \codeinline{coords} using the sampler specified by
      \codeinline{smpl}. Permitted types for \codeinline{coordT} are
      \codeinline{cl_int} and \codeinline{cl_float} when \codeinline{dimensions
      == 1}, \codeinline{cl_int2} and \codeinline{cl_float2} when \codeinline{
      dimensions == 2} and \codeinline{cl_int4} and \codeinline{cl_float4} when
      \codeinline{dimensions == 3}.
    }
  \addRowTwoSL
    { template <typename coordT> }    
    { void write(const coordT \&coords, const dataT \&color) const }
    {
      Available only when: \codeinline{(accessTarget == access::target::image \&\&
      (accessMode == access::mode::write || accessMode == access::mode::discard_write)) || 
      (accessTarget == access::target::host_image \&\& (accessMode ==
      access::mode::write || accessMode == access::mode::discard_write ||
      accessMode == access::mode::read_write))}.
      \newline
      Writes the value specified by \codeinline{color} to the element of the
      image at the coordinates specified by \codeinline{coords}. Permitted types
      for \codeinline{coordT} are \codeinline{cl_int} when \codeinline{
      dimensions == 1}, \codeinline{cl_int2} when \codeinline{dimensions == 2}
      and \codeinline{cl_int4} when \codeinline{dimensions == 3}.
    }
  \addRowTwoL
    { \__image_array_slice__ }
    { operator[](size_t index) const }
    {
      Available only when: \codeinline{accessTarget ==
      access::target::image_array \&\& dimensions < 3}.
      \newline
      Returns an instance of \codeinline{\__image_array_slice__}, an
      unspecified type which provides the interface of \codeinline{accessor<
      dataT, dimensions, mode, access::target::image, access::placeholder::
      false_>} which will provide access to a slice of the image array specified
      by \codeinline{index}.
    }
\completeTable

\clearpage


%%% Local Variables:
%%% mode: latex
%%% TeX-master: "sycl-1.2.1"
%%% TeX-auto-untabify: t
%%% TeX-PDF-mode: t
%%% ispell-local-dictionary: "american"
%%% End:
