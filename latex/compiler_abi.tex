% !TEX root = sycl-1.2.1.tex

%*******************************************************************************
% SYCL Device Compiler
%*******************************************************************************
\chapter{SYCL Device Compiler}
\label{chapter.device.compiler}

This section specifies the requirements of the SYCL device compiler.
Most features described in this section relate to underlying OpenCL
capabilities of target devices and limiting the requirements of device
code to ensure portability.

%*******************************************************************************
% Offline compilation of SYCL source files
%*******************************************************************************
\section{Offline compilation of SYCL source files}

There are two alternatives for a SYCL \gls{device-compiler}: a 
\keyword{single-source device compiler} and a device compiler that supports
the technique of \gls{smcp}.

A SYCL device compiler takes in a C++ source file, extracts only
the SYCL kernels and outputs the device code in a form that can be enqueued
from host code by the associated \gls{sycl-runtime}. How the \gls{sycl-runtime} invokes the
kernels is implementation defined, but a typical approach is for a device
compiler to produce a header file with the compiled kernel contained within it.
By providing a command-line option to the host compiler, it would cause the
implementation's SYCL header files to \tf{\#include} the generated header file.
The SYCL specification has been written to allow this as an implementation
approach in order to allow \gls{smcp}. However, any of the
mechanisms needed from the SYCL compiler, the \gls{sycl-runtime} and build system are
implementation defined, as they can vary depending on the platform and approach.

A SYCL single-source device compiler takes in a C++ source file and compiles
both host and device code at the same time. This specification specifies how
a SYCL single-source device compiler sees and outputs device code for kernels,
but does not specify the host compilation.

\section{Naming of kernels}
\label{sec:naming.kernels}
SYCL kernels are extracted from C++ source files and stored in an implementation-
defined format. In the case of the shared-source compilation model, the kernels
have to be uniquely identified by both host and device compiler. This is
required in order for the host runtime to be able to load the kernel by using
the OpenCL host runtime interface.

From this requirement the following rules apply for naming the kernels:

\begin{itemize}
\item The kernel name is a \keyword{C++ typename}.
\item The kernel needs to have a \emph{globally-visible} name. In the case of a
named function object type, the name can be the typename of the function object, as long as it is
globally-visible. In the case where it isn't, a globally-visible name has to be
provided, as template parameter to the kernel invoking interface, as described
in~\ref{subsec:invokingkernels}. In C++11, lambdas\footnote{C++14 lambdas have
the same naming rules as C++11 lambdas.} do not have a globally-visible name, so
a globally-visible typename has to be provided in the kernel invoking interface,
as described in~\ref{subsec:invokingkernels}.
\item The kernel name has to be a unique identifier in the program.
\end{itemize}

In both single-source and shared-source implementations, a device compiler should
detect the kernel invocations (e.g. \codeinline{parallel_for<kernelname>})
in the source code and compile the enclosed kernels, storing them with their
associated type name.

The format of the kernel and the compilation techniques are implementation
defined. The interface between the compiler and the runtime for extracting and
executing SYCL kernels on the device is implementation defined.

%*******************************************************************************
% Language restrictions for kernels
%*******************************************************************************
\section{Language restrictions for kernels}
\label{sec:language.restrictions.kernels}

The extracted SYCL kernels need to be compiled by an OpenCL online or offline
compiler and be executed by the OpenCL 1.2 runtime. The extracted kernels
need to be OpenCL 1.2 compliant kernels and as such there are certain
restrictions that apply to them.

The following restrictions are applied to device functions and kernels:

\begin{itemize}
  \item
    Structures containing pointers may be shared but the value of any pointer
    passed between SYCL devices or between the host and a SYCL device is
    undefined.
  \item
    Memory storage allocation is not allowed in kernels. All memory allocation
    for the device is done on the host using accessor classes.
    Consequently, the default allocation \codeinline{operator new} overloads that
    allocate storage are disallowed in a SYCL kernel. The placement
    \codeinline{new} operator and any user-defined overloads that do not
    allocate storage are permitted.
  \item
    The odr-use of polymorphic classes and classes with virtual inheritance is allowed.
    However, no virtual member functions are allowed to be called in a SYCL kernel or any functions called by the kernel.
  \item
    No function pointers or references are allowed to be called in a SYCL kernel
    or any functions called by the kernel.
  \item
    RTTI is disabled inside kernels.
  \item
    No variadic functions are allowed to be called in
    a SYCL kernel or any functions called by the kernel.
  \item
    Exception-handling cannot be used inside a SYCL kernel or any
    code called from the kernel. But of course \lstinline{noexcept} is
    allowed.
  \item
    Recursion is not allowed in a SYCL kernel or any
    code called from the kernel.
  \item
    Variables with thread storage duration are not allowed to be odr-used in
    kernel code
  \item
    Variables with static storage duration that are odr-used inside a kernel
    must be \lstinline{const} or \lstinline{constexpr} and zero-initialized or
    constant-initialized.
  \item
    The rules for kernels apply to both the kernel function objects themselves
    and all functions, operators, member functions, constructors and destructors
    called by the kernel. This means that kernels can only use library functions
    that have been adapted to work with SYCL. Implementations are not required
    to support any library routines in kernels beyond those explicitly mentioned
    as usable in kernels in this spec. Developers should refer to the SYCL
    built-in functions in~\ref{sycl:builtins} to find functions that are
    specified to be usable in kernels.
  \item
     Interacting with a special \gls{sycl-runtime} class (i.e. SYCL \codeinline{accessor}, \codeinline{sampler} or \codeinline{stream}) that is stored within a C++ union is undefined behavior.
\end{itemize}

%*******************************************************************************
% Compilation of functions
%*******************************************************************************
\section{Compilation of functions}

The SYCL device compiler parses an entire C++ source file supplied by the user.
This also includes C++ header files, using \tf{\#include} directives. From this
source file, the SYCL device compiler must compile kernels for the device, as
well as any functions that the kernels call.

In SYCL, kernels are invoked using a kernel invoke function (e.g.
\codeinline{parallel_for}). The kernel invoke functions are templated by their
kernel parameter, which is a function object.
The code inside the function object that is invoked as a kernel is called
the ``kernel function''. The ``kernel function'' must always return void. Any
function called by the kernel function is compiled for device and called a
``device function''. Recursively, any function called by a device function is
itself compiled as a device function.

For example, this source code shows three functions and a kernel invoke with
comments explaining which functions need to be compiled for device.

\begin{code}
void f ()
{
    // function "f" is not compiled for device

    single_task<class kernel_name>([=] ()
        {
            // This code compiled for device
            g (); // this line forces "g" to be compiled for device
        });
}

void g ()
{
    // called from kernel, so "g" is compiled for device
}

void h ()
{
    // not called from a device function, so not compiled for device
}
\end{code}

In order for the SYCL device compiler to correctly compile device functions, all
functions in the source file, whether device functions or not, must be
syntactically correct functions according to this specification. A syntactically
correct function is a function that matches at least the C++11 specification,
plus any extensions from the C++14 specification.

%*******************************************************************************
% Compilation of functions
%*******************************************************************************
\section{Built-in scalar data types}
\label{subsection:scalartypes}
In a SYCL device compiler, the standard C++ fundamental types, including
\codeinline{int}, \codeinline{short}, \codeinline{long},
\codeinline{long long int} need to be configured so that the device
definitions of those types match the host definitions of those types. A device
compiler may have this preconfigured so that it can match them based on the
definitions of those types on the platform. Or there may be a necessity for a
device compiler command-line option to ensure the types are the same.

The standard C++ fixed width types, e.g. \codeinline{int8_t},
\codeinline{int16_t}, \codeinline{int32_t},\codeinline{int64_t},
should have the same size as defined by the C++ standard for host and
device.
\ifx\showtodos\true
To ensure that pointer types and \codeinline{size_t} use the same amount of
storage on host and device when inside structures, SYCL also requires that
device compilers use the host sizes for pointers or \codeinline{size_t}. Devices
may be using a smaller size internally for pointers and \codeinline{size_t},
but this should not impact the programming model for users. In the case where
devices use larger pointer size internally than the host, then
\codeinline{size_t} will match the \gls{host-pointer} size. For structures shared
between host and device it is recommended to use fixed width types for pointers.
Scalar data types for a SYCL device compiler are described in
Table~\ref{table.types.fundamental}.
\fi

%-------------------------------------------------------------------------------
\startTable{Fundamental data type}
\addFootNotes{Fundamental data types supported by SYCL}{table.types.fundamental}
\addRow
{
  bool
}
{
  A conditional data type which can be either true or false. The value
  true expands to the integer constant 1 and the value false expands to the
  integer constant 0.
}
\addRow
{
  char
}
{
  A signed or unsigned 8-bit integer, as defined by the C++11 ISO Standard
}
\addRow
{
  signed char
}
{
  A signed 8-bit integer, as defined by the C++11 ISO Standard
}
\addRow
{
  unsigned char
}
{
  An unsigned 8-bit integer, as defined by the C++11 ISO Standard
}
\addRow
{
  short int
}
{
  A signed integer of at least 16-bits, as defined by the C++11 ISO Standard
}
\addRow
{
  unsigned short int
}
{
  An unsigned integer of at least 16-bits, as defined by the C++11 ISO Standard
}
\addRow
{
  int
}
{
  A signed integer of at least 16-bits, as defined by the C++11 ISO Standard
}
\addRow
{
  unsigned int
}
{
  An unsigned integer of at least 16-bits, as defined by the C++11 ISO Standard
}
\addRow
{
  long int
}
{
  A signed integer of at least 32-bits, as defined by the C++11 ISO Standard
}
\addRow
{
  unsigned long int
}
{
  An unsigned integer of at least 32-bits, as defined by the C++11 ISO Standard
}
\addRow
{
  long long int
}
{
  An integer of at least 64-bits, as defined by the C++11 ISO Standard
}
\addRow
{
  unsigned long long int
}
{
  An unsigned integer of at least 64-bits, as defined by the C++11 ISO Standard
}
\addRow
{
  size_t
}
{
  An unsigned integer type which is the result of the \codeinline{sizeof}
  operator on host.
}
\addRow
{
  float
}
{
  A 32-bit floating-point. The float data type must conform to the IEEE 754
  single precision storage format.
}
\addRow
{
 double
}
{
  A 64-bit floating-point. The double data type must conform to the IEEE 754
  double precision storage format.
}
\addRow
{
  half
}
{
  A 16-bit floating-point. The half data type must conform to the IEEE 754-2008
  half precision storage format. A SYCL \codeinline{feature_not_supported}
  exception must be thrown if the \codeinline{half} type is used in a SYCL
  kernel fucntion which executes on a SYCL \codeinline{device} that does not
  support the extension \codeinline{khr_fp16}.
}
\completeTable
%-------------------------------------------------------------------------------

%*******************************************************************************
% Preprocessor directives and macros
%*******************************************************************************
\section{Preprocessor directives and macros}

The standard C++ preprocessing directives and macros are supported.

\begin{itemize}
\item \texttt{CL_SYCL_LANGUAGE_VERSION} substitutes an integer reflecting
  the version number of the SYCL language being supported by the device
  compiler. The version of SYCL defined in this document will have
  \texttt{CL_SYCL_LANGUAGE_VERSION} substitute the integer \SYCLLANGVERSION;

\item \texttt{__FAST_RELAXED_MATH__} is used to determine if the
  \tf{-cl-fast-relaxed-math} optimization option is specified in the build
  options given to the SYCL device compiler. This is an integer constant
  of 1 if the option is specified and unspecified otherwise;

\item \texttt{__SYCL_DEVICE_ONLY__} is defined to 1 if the source file is
  being compiled with a SYCL device compiler which does not produce host
  binary;

\item \texttt{__SYCL_SINGLE_SOURCE__} is defined to 1 if the source file
  is being compiled with a SYCL single-source compiler which produces host
  as well as device binary;

\item \texttt{SYCL_EXTERNAL} is an optional macro which enables external
  linkage of SYCL functions and methods to be included in a SYCL kernel.
  The macro is only defined if the implementation supports external linkage.
  For more details see~\ref{subsec:syclexternal}

\end{itemize}

%*******************************************************************************
% Attributes
%*******************************************************************************
\section{Attributes}
The \tf{attribute} syntax defined in the OpenCL C specification is supported by the SYCL device compiler.
The C++11 attribute specifier can be used in which case these attributes are available in the \tf{cl} namespace.
For instance \codeinline{\__attribute__((vec_type_hint(int2)))} and \codeinline{[[cl::vec_type_hint(int2)]]} are equivalent.

The \tf{vec_type_hint}, \tf{work_group_size_hint} and \tf{reqd_work_group_size}
kernel attributes in OpenCL C apply to kernel functions, but this is not
syntactically possible in SYCL. In SYCL, these attributes are legal on device
functions and their specification is propagated down to any caller of those
device functions, such that the kernel attributes are the sum of all the kernel
attributes of all device functions called. If there are any conflicts between
different kernel attributes, then the behavior is undefined.

%*******************************************************************************
% Address-space deduction
%*******************************************************************************
\section{Address-space deduction}

In SYCL, there are several different types of pointer, or reference:
\begin{itemize}
\item Accessors give access to shared data. They can be bound to a memory object
 in a command group and passed into a kernel. Accessors are used in scheduling
 of kernels to define ordering. Accessors to buffers have a compile-time OpenCL
 address space based on their access mode.
\item Explicit pointer classes (e.g. \codeinline{global_ptr}) contain an OpenCL
 address space. This allows the compiler to determine whether the pointer
 references global, local, constant or private memory.
\item C++ pointer and reference types (e.g. \codeinline{int*}) are allowed
 within SYCL kernels. They can be constructed from the address of local
 variables, from explicit pointer classes, or from accessors. In all cases, a
 SYCL device compiler will need to auto-deduce the address space.
\end{itemize}

Inside kernels, explicit pointer classes and C++ pointers are allowed as long as
they reference the same data type and have compatible qualifiers and address spaces.

If a kernel function or device function contains a pointer or reference type,
then address-space deduction must be attempted using the following rules:

\begin{itemize}
\item If an explicit pointer class is converted into a C++ pointer value, then
 the C++ pointer value will have the address space of the explicit pointer
 class.
\item If a variable is declared as a pointer type, but initialized in its
 declaration to a pointer value with an already-deduced address space, then that
 variable will have the same address space as its initializer.
\item If a function parameter is declared as a pointer type, and the argument is
 a pointer value with a deduced address space, then the function will be
 compiled as if the parameter had the same address space as its argument. It is
 legal for a function to be called in different places with different address
 spaces for its arguments: in this case the function is said to be
 ``duplicated'' and compiled multiple times. Each duplicated instance of the
 function must compile legally in order to have defined behavior.
\item If a function return type is declared as a pointer type and return statements use
 address space deduced expressions, then the function will be compiled as if the return type
 had the same address space. To compile legally, all return expressions must deduce to the same
 address space.
\item The rules for pointer types also apply to reference types. i.e. a
 reference variable takes its address space from its initializer. A function
 with a reference parameter takes its address space from its argument.
\item If no other rule above can be applied to a declaration of a pointer, then
 it is assumed to be in the \tf{private} address space. This default assumption
 is expected to change to be the \codeinline{generic} address space for OpenCL
 versions that support the generic address space.
\end{itemize}

It is illegal to assign a pointer value of one address space to a pointer
variable of a different address space.

%*******************************************************************************
% SYCL offline linking
%*******************************************************************************
\section{SYCL offline linking}
\fixme{Update add offline external linking}
%*******************************************************************************
% SYCL functions and methods linkage
%*******************************************************************************
\subsection{SYCL functions and methods linkage}
\label{subsec:syclexternal}
The default behavior in SYCL applications is that all the definitions and
declarations of the functions and methods are available to the SYCL compiler,
in the same translation unit. When this is not the case, all the symbols that
need to be exported to a SYCL library or from a C++ library to a SYCL
application need to be defined using the macro: \texttt{SYCL_EXTERNAL}.

The \texttt{SYCL_EXTERNAL} macro will only be defined if the implementation
supports offline linking. The macro is implementation-defined, but the following
restrictions apply:
\begin{itemize}
\item \texttt{SYCL_EXTERNAL} can only be used on functions;
\item the function cannot use raw pointers as parameter or return types.
  Explicit pointer classes must be used instead;
\item externally defined functions cannot call a
  \codeinline{cl::sycl::parallel_for_work_item} method;
\item externally defined functions cannot be called from a
  \codeinline{cl::sycl::parallel_for_work_group} scope.
\end{itemize}

The SYCL linkage mechanism is optional and implementation defined.

%*******************************************************************************
% Linking with OpenCL C libraries
%*******************************************************************************
\subsection{Offline linking with OpenCL C libraries}

SYCL supports linking \glspl{sycl-kernel-function} with OpenCL C libraries
during offline compilation or during online compilation by the
\gls{sycl-runtime} within a SYCL application.

Linking with OpenCL C kernel functions offline is an optional feature and is
unspecified. Linking with OpenCL C kernel functions online is performed by
using the SYCL \codeinline{program} class to compile and link an OpenCL C source;
using the \codeinline{compile_with_source} or \codeinline{build_with_source}
member functions.

OpenCL C functions that are linked with, using either offline or online
compilation must be declared as an extern ``C'' function declarations. The function
parameters of these function declarations must be defined as the OpenCL C
interoperability aliases; \codeinline{pointer_t} and \codeinline{
const_pointer_t}, of the \codeinline{multi_ptr} class template, \codeinline{
vector_t} of the \codeinline{vec} class template and scalar data type aliases described in Table~\ref{table.types.aliases}.

For example:

\lstinputlisting{headers/openclcInterop.h}

%%% Local Variables:
%%% mode: latex
%%% TeX-master: "sycl-1.2.1"
%%% TeX-auto-untabify: t
%%% TeX-PDF-mode: t
%%% ispell-local-dictionary: "american"
%%% End:
