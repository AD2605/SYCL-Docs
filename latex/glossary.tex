% Copyright (c) 2012-2019 Khronos Group.
%
% This work is licensed under a Creative Commons Attribution 4.0
% International License.
% http://creativecommons.org/licenses/by/4.0/

%\chapter{Glossary}


% \todo 2019/06/10
% Look at the remaining \keyword macro in the document and add the
% lacking entries here


%The purpose of this glossary is to define the key concepts involved in
%specifying OpenCL SYCL. This section includes definitions of terminology used
%throughout the specification document.
\makenoidxglossaries
\glstoctrue
%\makeglossaries
\newglossaryentry{accessor}
{
  name=accessor,
  description={An accessor is a class which allows a
               \gls{sycl-kernel-function} to access data managed by a \gls{buffer} or
               \gls{image} class. Accessors are used to express the dependencies
               among the different \glspl{command-group}. For the full
               description please refer to section [\ref{subsec:accessors}]}
}

\newglossaryentry{application-scope}
{
  name={application scope},
  description={The application scope starts with the
               construction first \gls{sycl-runtime} class object and
               finishes with the destruction of the last one.
               Application refers to the C++ \gls{sycl-application} and not
               the \gls{sycl-runtime}}
}

\newglossaryentry{async-handler}
{
  % Do not use this because it looks like MacOS X Preview 10.1
  % (944.6.16.1) cannot search it
  % name={\texttt{async\_handler}},
  %
  % This one seems better on Mac. But Safari 12.1.1 (14607.2.6.1.1)
  % has still some problems
  % in both cases to find all the 43 occurrences. Actually 33 are found
  % forward and 10 found backward... I find this... awkward :-(
  name={\texttt{async\string_handler}},
  % Add a sorting key to counter-act the LaTeX formatting
  sort=async-handler,
  description={An asynchronous error handler object is a function class instance
               providing necessary code for handling all the asynchronous
               exceptions triggered from the execution of command groups on a
               queue, within a context or an associated event. For the full
               description please refer to section [\ref{subsec:exception.class}]}
}

\newglossaryentry{buffer}
{
  name=buffer,
  description={The buffer class manages data for the SYCL C++ host
               application and the SYCL device kernels. The buffer class
               may acquire ownership of some host pointers passed to its
               constructors according to the constructor kind.
               %
               \glspar
               %
               The buffer class, together with the accessor class,
               is responsible for tracking memory transfers and guaranteeing
               data consistency among the different kernels.
               The \gls{sycl-runtime} manages the memory allocations
               on both the host and the \gls{device} within the
               lifetime of the buffer object. For the full
               description please refer to section [\ref{subsec:buffers}]}
}

\newglossaryentry{work-group-mem-fence}
{
  name={work-group mem-fence},
  description={The work-group mem-fence guarantees that any access on the
               corresponding memory address space before the barrier, must
               complete before continuing to process any data from that memory
               space after the barrier}
}

\newglossaryentry{barrier}
{
  name=barrier,
  description={A barrier is either a \gls{queue-barrier} or a
               kernel execution \gls{work-group-barrier} depending on whether it is
               a synchronization point on the command queue or on
               the work-groups of a kernel execution}
}

\newglossaryentry{queue-barrier}
{
  name={command queue barrier},
  description={The SYCL API provides two variants for functions that force
               synchronization on a SYCL command queue. The
               \codeinline{cl::sycl::queue::wait()} and
               \codeinline{cl::sycl::queue::wait_and_throw()}
               functions force the SYCL command queue to wait for the
               execution of the \gls{command-group-function-object} before it is able to
               continue executing}
}

\newglossaryentry{work-group-barrier}
{
  name={work-group barrier},
  description={The work-group barrier, cl::sycl::nd_item::barrier, is a 
               synchronization function within a \gls{work-group}. All the
               \glspl{work-item} of a work-group must execute the barrier construct before any \gls{work-item}
               continues execution beyond the barrier. Additionally the work-
               group barrier performs a \gls{work-group-mem-fence}. Note: In
               OpenCL 1.2 there is no synchronization between different work-
               groups}
}

\newglossaryentry{command-group}
{
  name={command group},
  description={In SYCL, the operations required to process data on a
    \gls{device} are represented using a \gls{command-group-function-object}.
    Each \gls{command-group-function-object} is given a unique \gls{handler} object
    to perform all the necessary work required to correctly process data
    on a \gls{device} using a kernel. In this way, the group of commands for
               transferring and processing data is enqueued as a command
               group on a \gls{device} for execution. A command group is submitted
               atomically to a SYCL queue}
}

\newglossaryentry{command-group-function-object}
{
  name={command group function object},
  description={A type which is callable with `operator()` that takes a reference
  to a\gls{handler}, that defines a \gls{command-group} which can be submitted
  by a \gls{queue}. The function object can be a named type, lambda function or
  \codeinline{function_class}}
}

\newglossaryentry{command-group-scope}
{
  name={command group scope},
  description={The command group scope is the function scope defined by the
               \gls{command-group-function-object}. The command group \gls{handler} object lifetime
               is restricted to the command group scope. For more details
               please see [\ref{sec:command.group.scope}]}
}

\newglossaryentry{queue}
{
  name={command queue},
  description={A SYCL command queue is an object that holds command
                groups to be executed on a SYCL \gls{device}. SYCL provides a
                heterogeneous platform integration using device queue, which is
                the minimum requirement for a SYCL application to run on a
                SYCL \gls{device}. For the full description please refer to
            section [\ref{sec:interface.queue.class}]}
}

\newglossaryentry{handler}
{
  name={command group handler},
  description={The command group handler class provides the interface for the
               commands that can be executed inside the
               \gls{command-group-scope}. It is provided as a scoped object
               to all of the data access requests within the command group scope.
               For the full description please refer to section
         [\ref{sec:handlerClass}]}
}

\newglossaryentry{constant-memory}
{
  name={constant memory},
  description={A region of global memory that remains constant during the
               execution of a kernel. The \gls{sycl-runtime}
               allocates and initializes memory objects placed into constant
               memory}
}

\newglossaryentry{device}
{
  name=device,
  description={A SYCL device encapsulates an OpenCL device or the SYCL
               host device, which can run SYCL kernels on host}
}

\newglossaryentry{device-compiler}
{
  name={device compiler},
  description={A SYCL device compiler is a compiler that produces OpenCL
               \gls{device} binaries from a valid \gls{sycl-application}.
               For the full description please refer to section
         [\ref{chapter.device.compiler}]}
}

\newglossaryentry{global-id}
{
  name={global id},
  description={As in OpenCL, a global ID is used to uniquely identify a
               \gls{work-item} and is derived from the number of
               global \glspl{work-item}
               specified when executing a kernel. A global ID is a one, two or
               three-dimensional value that starts at 0 per dimension}
}

\newglossaryentry{global-memory}
{
  name={global memory},
  description={Global memory is a memory region accessible to all \glspl{work-item}
               executing on a \gls{device}}
}

\newglossaryentry{group}
{
  name={group},
  description={A unique identifier representing a single \gls{work-group}
  within the index space of a SYCL kernel execution. Can be one, two or three
  dimensional. In the SYCL interface a \gls{group} is represented by the
  \codeinline{group} class (see Section~\ref{group-class})}
}

\newglossaryentry{work-group-id}
{
  name={work-group id},
  description={As in OpenCL, SYCL kernels execute in work groups. The group
               ID is the ID of the work group that a work item is executing within.
               A group ID is an one, two or three dimensional value that starts
               at 0 per dimension}
}

\newglossaryentry{work-group-range}
{
  name={work-group range},
  description={a group range is the size of the work-group for every dimension}
}

\newglossaryentry{h-item}
{
  name={h-item},
  description={A unique identifier representing a single \gls{work-item} within
  the index space of a SYCL kernel hierarchical execution. Can be one, two or
  three dimensional. In the SYCL interface a \gls{h-item} is represented by the
  \codeinline{h_item} class (see Section~\ref{hitem-class})}
}

\newglossaryentry{host}
{
  name=host,
  description={Host is the system that executes the C++ application including
               the SYCL API}
}

\newglossaryentry{host-pointer}
{
  name={host pointer},
  description={A pointer to memory on the host.
        Cannot be accessed directly from a \gls{device}}
}

\newglossaryentry{id}
{
  name=id,
  description={It is a unique identifier of an item in an index space. It can
               be one, two or three dimensional index space, since the SYCL
               Kernel execution model is an nd-range. It is one of the index
               space classes. For the full description please refer to section
         \ref{id-class}}
}

\newglossaryentry{image}
{
  name=image,
  description={Images in SYCL, like buffers, are abstractions of the OpenCL
               equivalent. As in OpenCL, an image stores a two- or
               three-dimensional structured array. The \gls{sycl-runtime} will make
               available images in OpenCL contexts in order to execute
               semantically correct kernels in different OpenCL contexts.
               For the full description please refer to section [\ref{subsec:images}]}
}

\newglossaryentry{implementation-defined}
{
  name={implementation defined},
  description={Behavior that is explicitly allowed to vary between
               conforming implementations of SYCL. A SYCL
               implementer is required to document the
               implementation defined behavior}
}

\newglossaryentry{index-space-classes}
{
  name={index space classes},
  description={The OpenCL Kernel Execution Model defines an nd-range index
              space. The \gls{sycl-runtime} class that defines an nd-range is the 
              \codeinline{cl::sycl::nd_range}, which takes as input the sizes
              of global and local work-items, represented using the 
              \codeinline{cl::sycl::range} class. The kernel library classes
              for indexing in the defined nd-range are the following classes: 
              \begin{itemize}
              \item \codeinline{cl::sycl::id} : The basic index class 
                 representing a \gls{id}. 
              \item \codeinline{cl::sycl::item} : The index class that contains
                 the \gls{global-id} and \gls{local-id}. 
              \item \codeinline{cl::sycl::nd_item} : The index class that
                 contains the \gls{global-id}, \gls{local-id} and the
                 \gls{work-group-id}.
              \item \codeinline{cl::sycl::group} : The group class that contains
                 the \gls{work-group-id} and the methods on a work-group.
              \end{itemize}
              % Since there is already a dot at the end of the
              % previous item, do not output a final dot again
              \nopostdesc
             }
}

\newglossaryentry{item}
{
  name={item},
  description={An item id is an interface used to retrieve the \gls{global-id},
        \gls{work-group-id} and \gls{local-id}. For further details see
              [\ref{subsec:item.class}]}
}

\newglossaryentry{kernel}
{
  name=kernel,
  description={A SYCL kernel which can be executed on a \gls{device},
  including the \gls{host-device}. Is created implicitly when defining a
  \gls{sycl-kernel-function} (see~\ref{sec:expr-parall-thro}) but can also be
  created manually in order to pre-compile \glspl{sycl-kernel-function}}
}

\newglossaryentry{kernel-name}
{
  name={kernel name},
  description={A kernel name is a class type that is used to assign a name to
               the kernel function, used to link the host system with the
               kernel object output by the device compiler. For details on naming
               kernels please see [\ref{sec:naming.kernels}]}
}

\newglossaryentry{sycl-kernel-function}
{
  name={SYCL kernel function},
  description={A type which is callable with `operator()` that takes a \gls{id},
  \gls{item}, \gls{nd-item} or \gls{group} which can be passed to kernel
  enqueue member functions of the \gls{handler}. A \gls{sycl-kernel-function}
  defines an entry point to a \gls{kernel}. The function object can be a named
  standard layout type or lambda function}
}

\newglossaryentry{kernel-scope}
{
  name={kernel scope},
  description={The function scope of the \tf{operator()} on a \gls{sycl-kernel-function}. Note
               that any function or method called from the kernel is also
               compiled in kernel scope. The kernel scope allows C++ language
               extensions as well as restrictions to reflect the capabilities of
               OpenCL devices. The extensions and restrictions are defined in
               the SYCL device compiler specification}
}

\newglossaryentry{local-id}
{
  name={local id},
  description={A unique identifier of a work-item among other work-items of a
               work-group}
}

\newglossaryentry{local-memory}
{
  name={local memory},
  description={Local memory is a memory region associated with a work-group
               and accessible only by work-items in that work-group}
}

\newglossaryentry{nd-item}
{
  name={nd-item},
  description={A unique identifier representing a single \gls{work-item} and
  \gls{work-group} within the index space of a SYCL kernel execution. Can be
  one, two or three dimensional. A \gls{nd-item} is capable of performing
  \glspl{work-group-barrier} and \glspl{work-group-mem-fence}. In the SYCL
  interface a \gls{nd-item} is represented by the \codeinline{nd_item} class
  (seeSection~\ref{nditem-class})}
}

\newglossaryentry{nd-range}
{
  name={nd-range},
  description={A representation of the index space of a SYCL kernel execution,
  the distribution of \glspl{work-item} within into \glspl{work-group}.
  Contains a \gls{range} specifying the number of global \glspl{work-item}, a
  \gls{range} specifying the number of local \glspl{work-item} and a \gls{id}
  specifying the global offset. Can be one, two or three dimensional. The
  minimum size of each \gls{range} within the \gls{nd-range} is 1 per
  dimension. In the SYCL interface an \gls{nd-range} is represented by the
  \codeinline{nd_range} class (see Section~\ref{subsubsec:nd-range-class})}
}

\newglossaryentry{platform}
{
  name=platform,
  description={The host together or a collection of \glspl{device} managed by the OpenCL
               framework that allow an application to share resources and
               execute kernels on \glspl{device} in the platform. A SYCL application
               can target one or multiple OpenCL platforms provided by
               OpenCL \gls{device} vendors~\cite{opencl12}}
}

\newglossaryentry{private-memory}
{
  name={private memory},
  description={A region of memory private to a work-item. Variables defined in
               one workitem’s private memory are not visible to another
               work-item. \cite{opencl12}.
               The \codeinline{cl::sycl::private_memory} class provides access to
               the work-item's private memory for the hierarchical API as it is
               described at [\ref{paragraph.private.memory}]}
}

\newglossaryentry{program-object}
{
  name={program object},
  description={A program object in SYCL is an OpenCL program object 
               encapsulated in A SYCL class. It contains OpenCL kernels and 
               functions compiled to execute on OpenCL \glspl{device}. A program object
               can be generated from SYCL C++ kernels by the \gls{sycl-runtime}, or 
               obtained from an OpenCL implementation. For furthen details
               on the \codeinline{cl::sycl::program} class see 
         [\ref{sec:interfaces.program.class}]}
}

\newglossaryentry{range}
{
  name={range},
  description={A representation of a number of \glspl{work-item} or
  \gls{work-group} within the index space of a SYCL kernel execution. Can be
  one, two or three dimensional. In the SYCL interface a \gls{group} is
  represented by the \codeinline{group} class (see Section~\ref{group-class}
  )}
}

\newglossaryentry{smcp}
{
  name={SMCP},
  description={The single-source multiple compiler-passes (SMCP) technique allows
         a single source file to be parsed by multiple compilers for
         building native programs per compilation target.
         For example, a standard C++ CPU compiler
               for targeting \gls{host} will parse the  \gls{sycl-file} to create the
               C++ \gls{sycl-application} which offloads parts of the computation to
               other \glspl{device}. A SYCL device compiler will parse the same
               source file and target only SYCL kernels},
  first={single-source multiple compiler-passes (SMCP)}
}

\newglossaryentry{host-device}
{
  name={SYCL host device},
  description={The SYCL host device is a native C++
               implementation of a \gls{device}. It does not have an OpenCL
               \codeinline{cl_device_id} and it will only appear in the 
               available SYCL devices, as it is not an OpenCL device. It has
               full SYCL capabilities and reports them through the SYCL 
               information retrieval interface. The SYCL host device is 
               mandatory for every SYCL implementation and is always available,
               but may not achieve the same performance as an OpenCL CPU device.
               Any C++ application debugger can be used for debugging SYCL
               kernels executing on a SYCL host device}
}

\newglossaryentry{sycl-library}
{
  name={SYCL C++ Template Library},
  description={The template library is a set of C++ templated classes which provide
               the programming interface to the SYCL developer}
}

\newglossaryentry{sycl-runtime}
{
  name={SYCL runtime},
  description={A SYCL runtime is an implementation of the SYCL API
               specification. The SYCL runtime manages the different OpenCL
               platforms, \glspl{device}, contexts as well as memory handling of data
               between host and OpenCL contexts to enable semantically
               correct execution of SYCL programs}
}

\newglossaryentry{sycl-file}
{
  name={SYCL File},
  description={A SYCL C++ source file that contains SYCL API calls}
}

\newglossaryentry{work-group}
{
  name={work-group},
  description={The SYCL work-group (\codeinline{cl::sycl::group} class) is a
               representation of an OpenCL work group. A collection of related
               work-items that execute on a single compute unit. The
               work-items in the group execute the same kernel-instance and
               share local memory and workgroup functions \cite{opencl12}. For
               further details for the \codeinline{cl::sycl::group} class see
               [\ref{group-class}]}
}

\newglossaryentry{work-item}
{
  name={work-item},
  description={The SYCL work-item (\codeinline{cl::sycl::nd_item} class) is a
               representation of an OpenCL work item. One of a collection of
               parallel executions of a kernel invoked on a \gls{device} by a
               command. A work-item is executed by one or more processing
               elements as part of a work-group executing on a compute unit. A
               work-item is distinguished from other work-items by its global ID
               or the combination of its work-group ID and its local ID within a
               work-group \cite{opencl12}}
}

\newglossaryentry{context}
{
  name={context},
  description={A SYCL context is an encapsulation of an OpenCL context. In OpenCL,
         any OpenCL resource is attached to a context. A context contains a
         collection of \glspl{device} of the same \gls{platform}. The context is
         defined as the \codeinline{cl::sycl::context} class, for further
               details please see [\ref{sec:interface.context.class}]}
}

\newglossaryentry{sycl-application}
{
  name={SYCL application},
  description={A SYCL application is a C++ application which uses the SYCL
         programming model in order to execute \glspl{kernel} on
         \glspl{device}}
}


%%% Local Variables:
%%% mode: latex
%%% TeX-master: "sycl-1.2.1"
%%% TeX-auto-untabify: t
%%% TeX-PDF-mode: t
%%% ispell-local-dictionary: "american"
%%% End:
