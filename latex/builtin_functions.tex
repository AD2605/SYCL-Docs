% Copyright (c) 2012-2019 Khronos Group.
%
% This work is licensed under a Creative Commons Attribution 4.0
% International License.
% http://creativecommons.org/licenses/by/4.0/

% !TEX root = sycl-1.2.1.tex

%***********************************************************************************
% SYCL built-in functions for SYCL host and device
%***********************************************************************************
%\section{SYCL built-in functions for SYCL host and device}
%\label{sycl:builtins}

SYCL kernels may execute on any SYCL device, specifically an OpenCL device or SYCL host, which requires the functions used in the kernels to be compiled and linked for both device and host. In the SYCL system the OpenCL built-ins are available for the SYCL host and device within the \codeinline{cl::sycl} namespace, although, their semantics may be different. This section follows the OpenCL 1.2 specification document \cite[ch. 6.12]{opencl12} and describes the behavior of these functions for SYCL host and device.

The SYCL built-in functions are available throughout the SYCL application, and depending on where they execute, they are either implemented using their host implementation or the device implementation. The SYCL system guarantees that all of the built-in functions fulfill the same requirements for both host and device.

%***********************************************************************************
% Description of the built-in types available for SYCL host and device
%***********************************************************************************
\subsection{Description of the built-in types available for SYCL host
  and device }

All of the OpenCL built-in types are available in the namespace
\codeinline{cl::sycl}. For the purposes of this document we use
generic type names for describing sets of valid SYCL types. The
generic type names themselves are not valid SYCL types, but they
represent a set of valid types, as defined in
Tables~\ref{table.gentypes}. Each generic type within a section is
comprised of a combination of scalar and/or SYCL \codeinline{vec}
class specializations. Note that any reference to the base type refers
to the type of a scalar or the element type of a SYCL \codeinline{vec}
specialization.

In the OpenCL 1.2 specification document\cite[ch. 6.12.1]{opencl12} in Table 6.7 the work-item
functions are defined where they provide the size of the enqueued kernel NDRange. These
functions are
available in SYCL through the item and group classes see
sections~\ref{subsec:item.class}, \ref{nditem-class} and
\ref{group-class}.

---------------------------------------------------------------------------------

%------------------------------------------------------------------------------------------------------
\startTable{Generic type name } \addFootNotes{Generic type name
  description, which serves as a description for all valid types of
  parameters to kernel functions \cite{opencl12}}
{table.gentypes}

\addRow{floatn}
{\codeinline{float2, float3, float4, float8, float16}
}

\addRow{genfloatf}
{\codeinline{float, floatn}
}

\addRow{doublen}
{\codeinline{double2, double3, double4, double8, double16}
}

\addRow{genfloatd}
{\codeinline{double, doublen}
}

\addRow{halfn}
{\codeinline{half2, half3, half4, half8, half16}
}

\addRow{genfloath}
{\codeinline{half, halfn}
}

\addRow{genfloat}
{\codeinline{genfloatf, genfloatd, genfloath}
}

\addRow{sgenfloat}
{\codeinline{float, double, half}
}

\addRow{gengeofloat}
{\codeinline{float, float2, float3, float4}
}

\addRow{gengeodouble}
{\codeinline{double, double2, double3, double4}
}

\addRow {charn}
{\codeinline{char2, char3, char4, char8, char16}
}

\addRow {scharn}
{\codeinline{schar2, schar3, schar4, schar8, schar16}
}

\addRow{ucharn}
{\codeinline{uchar2, uchar3, uchar4, uchar8, uchar16}
}

\addRow{igenchar}
{\codeinline{signed char, scharn}
}

\addRow{ugenchar}
{\codeinline{unsigned char, ucharn}
}

\addRow{genchar}
{\codeinline{char, charn, igenchar, ugenchar}
}

\addRow{shortn}
{\codeinline{short2, short3, short4, short8, short16}
}

\addRow{genshort}
{\codeinline{short, shortn}
}

\addRow{ushortn}
{\codeinline{ushort2, ushort3, ushort4, ushort8, ushort16}
}

\addRow{ugenshort}
{\codeinline{unsigned short, ushortn}
}


\addRow{uintn}
{\codeinline{uint2, uint3, uint4, uint8, uint16}
}

\addRow{ugenint}
{\codeinline{unsigned int, uintn}
}

\addRow{intn}
{\codeinline{int2, int3, int4, int8, int16}
}

\addRow{genint}
{\codeinline{int, intn}
}

\addRow{ulongn}
{\codeinline{ulong2, ulong3, ulong4, ulong8, ulong16}
}

\addRow{ugenlong}
{\codeinline{unsigned long int, ulongn}
}

\addRow{longn}
{\codeinline{long2, long3, long4, long8, long16}
}

\addRow{genlong}
{\codeinline{long int, longn}
}

\addRow{ulonglongn}
{\codeinline{ulonglong2, ulonglong3, ulonglong4, ulonglong8, ulonglong16}
}

\addRow{ugenlonglong}
{\codeinline{unsigned long long int, ulonglongn}
}

\addRow{longlongn}
{\codeinline{longlong2, longlong3, longlong4, longlong8, longlong16}
}
\addRow{genlonglong}
{\codeinline{long long int, longlongn}
}

\addRow{igenlonginteger}
{\codeinline{genlong, genlonglong}
}

\addRow{ugenlonginteger}
{\codeinline{ugenlong, ugenlonglong}
}

\addRow{geninteger}
{\codeinline{genchar, genshort, ugenshort, genint, ugenint, igenlonginteger, ugenlonginteger}
}

\addRow{genintegerNbit}
{\codeinline{All types within geninteger whose base type are N bits in size, where N = 8, 16, 32, 64.}
}

\addRow{igeninteger}
{\codeinline{igenchar, genshort, genint, igenlonginteger}
}

\addRow{igenintegerNbit}
{\codeinline{All types within igeninteger whose base type are N bits in size, where N = 8, 16, 32, 64.}
}

\addRow{ugeninteger}
{\codeinline{ugenchar, ugenshort, ugenint, ugenlonginteger}
}

\addRow{ugenintegerNbit}
{\codeinline{All types within ugeninteger whose base type are N bits in size, where N = 8, 16, 32, 64.}
}

\addRow{sgeninteger}
{\codeinline{char, signed char, unsigned char, short, unsigned short, int, unsigned int, long int, unsigned long int, long long int, unsigned long long int}
}

\addRow{gentype}
{\codeinline{genfloat, geninteger}
}
\addRow{genfloatptr}
{
  All permutations of \codeinline{multi_ptr<dataT, addressSpace>} where \codeinline{dataT} is all types within \codeinline{genfloat} and \codeinline{addressSpace} is \codeinline{access::address_space::global_space}, \codeinline{access::address_space::local_space} and \codeinline{access::address_space::private_space}.
}
\addRow{genintptr}
{
    All permutations of \codeinline{multi_ptr<dataT, addressSpace>} where \codeinline{dataT} is all types within \codeinline{genint} and \codeinline{addressSpace} is \codeinline{access::address_space::global_space}, \codeinline{access::address_space::local_space} and \codeinline{access::address_space::private_space}.
}
\completeTable
%-------------------------------------------------------------------------------

%*******************************************************************************
% Work-item functions
%*******************************************************************************
\subsection{Work-item functions}

In the OpenCL 1.2 specification document\cite[ch. 6.12.1]{opencl12} in Table 6.7
the work-item functions are defined where they provide the size of the enqueued
kernel NDRange. These functions are available in SYCL through the
\codeinline{nd_item} and \codeinline{group} classes see
section~\ref{nditem-class} and \ref{group-class}.

%***********************************************************************************
% Math functions
%***********************************************************************************
\subsection{Math functions}

In SYCL the OpenCL math functions are available in the namespace
\codeinline{cl::sycl} on host and device with the same precision
guarantees as defined in the OpenCL 1.2 specification document
\cite[ch. 7]{opencl12} for host and device. For a SYCL platform the
numerical requirements for host need to match the numerical
requirements of the OpenCL math built-in functions. The built-in
functions can take as input float or optionally double and their
\codeinline{vec} counterparts, for dimensions 1, 2, 3, 4, 8 and 16. On
the host the vector types use the \codeinline{vec} class and on an
OpenCL device use the corresponding OpenCL vector types.

The built-in functions available for SYCL host and device, with the same
precision requirements for both host and device, are described in
Table~\ref{table.math.functions}.

%-------------------------------------------------------------------------------
\startTable{Math Function}
\addFootNotes{Math functions which work on SYCL Host and device. They correspond
to Table 6.7 of the OpenCL 1.2 specification\cite{opencl12}}
{table.math.functions}
\addRow{ genfloat acos (genfloat x)} {Inverse cosine function.}
\addRow{ genfloat acosh (genfloat x)} {Inverse hyperbolic cosine.}
\addRow{ genfloat acospi (genfloat x)} {Compute $acos{x}/\pi$}
\addRow{ genfloat asin (genfloat x)} {Inverse sine function.}
\addRow{ genfloat asinh (genfloat x)} {Inverse hyperbolic sine.}
\addRow{ genfloat asinpi (genfloat x)} {Compute $asin{x}/\pi$}
\addRow{ genfloat atan (genfloat y_over_x)} {Inverse tangent function.}
\addRow{ genfloat atan2 (genfloat y, genfloat x)} {Compute atan( y / x).}
\addRow{ genfloat atanh (genfloat x)} { Hyperbolic inverse tangent.}
\addRow{ genfloat atanpi (genfloat x)} {Compute atan (x) / $\pi$.}
\addRow{ genfloat atan2pi (genfloat y, genfloat x)} {Compute atan2 (y, x) / $\pi$.}
\addRow{ genfloat cbrt (genfloat x)} { Compute cube-root.}
\addRow{ genfloat ceil (genfloat x)} {Round to integral value using the round to positive infinity
rounding mode.}
\addRow{ genfloat copysign (genfloat x, genfloat y)} {Returns x with its sign changed to match
the sign of y.}

\addRow{ genfloat cos (genfloat x)} {Compute cosine.}
\addRow{ genfloat cosh (genfloat x)} {Compute hyperbolic cosine.}
\addRow{ genfloat cospi (genfloat x)} { Compute cos ($\pi  x$).}
\addRow{ genfloat erfc (genfloat x)} {Complementary error function.}
\addRow{ genfloat erf (genfloat x)} {Error function encountered in integrating the normal
distribution.}
\addRow{ genfloat exp (genfloat x )} { Compute the base-e exponential of x.}
\addRow{ genfloat exp2 (genfloat x)} {Exponential base 2 function.}
\addRow{ genfloat exp10 (genfloat x)} {Exponential base 10 function.}
\addRow{ genfloat expm1 (genfloat x)} { Compute $\exp{(x)}-1.0$.}
\addRow{ genfloat fabs (genfloat x)} {Compute absolute value of a floating-point number.}
\addRow{ genfloat fdim (genfloat x, genfloat y)} {$ x - y$  if $ x > y$,$ +0$ if x is less than or
equal to y.}
\addRow{ genfloat floor (genfloat x)} {Round to integral value using the round to negative infinity
rounding mode.}
\addRow{ genfloat fma (genfloat a, genfloat b, genfloat c)}
{
Returns the correctly rounded floating-point
representation of the sum of c with the infinitely
precise product of a and b. Rounding of
intermediate products shall not occur. Edge case
behavior is per the IEEE 754-2008 standard.
}
\addRowTwoSL
{  genfloat fmax (genfloat x, genfloat y)}
{  genfloat fmax (genfloat x, sgenfloat y)}
{
Returns y if $x < y$, otherwise it returns x. If one
argument is a NaN, fmax() returns the other
argument. If both arguments are NaNs, fmax()
returns a NaN.
}
\addRowTwoSL
{ genfloat fmin (genfloat x, genfloat y)}
{ genfloat fmin (genfloat x, sgenfloat y)}
{
Returns y if $y < x$, otherwise it returns x. If one
argument is a NaN, fmin() returns the other
argument. If both arguments are NaNs, fmin()
returns a NaN.
}
\addRow{
   genfloat fmod (genfloat x, genfloat y)
 }
 {
  Modulus. Returns $x – y * trunc (x/y)$.
  }
\addRow
{ genfloat fract (genfloat x, genfloatptr iptr)}
{
  Returns fmin( x $-$ floor (x), nextafter(genfloat(1.0), genfloat(0.0)) ). floor(x) is returned in iptr.
}
\addRow
{ genfloat frexp (genfloat x,  genintptr exp)}
{
Extract mantissa and exponent from x. For each
component the mantissa returned is a float with
magnitude in the interval [1/2, 1) or 0. Each
component of x equals mantissa returned * 2exp.
}
\addRow{
  genfloat hypot (genfloat x, genfloat y)
}
{
  Compute the value of the square root of x2+ y2 without undue overflow or underflow.
}
\addRow
{genint ilogb (genfloat x)}
{
  Return the exponent as an integer value.
}
\addRowTwoSL
{  genfloat ldexp (genfloat x, genint k) }
{  genfloat ldexp (genfloat x, int k) }
{
  Multiply x by 2 to the power k.
}

\addRow
{
  genfloat lgamma (genfloat x)
}
{
  Log gamma function. Returns the natural
logarithm of the absolute value of the gamma
function. The sign of the gamma function is
returned in the signp argument of \codeinline{lgamma_r}.
}
\addRow
{
  genfloat lgamma_r (genfloat x, genintptr signp)
}
{
  Log gamma function. Returns the natural
logarithm of the absolute value of the gamma
function. The sign of the gamma function is
returned in the signp argument of \codeinline{lgamma_r}.
}


\addRow{
genfloat log (genfloat x)
}
{
 Compute natural logarithm.
 }
\addRow{
genfloat log2 (genfloat x)
}
{
 Compute a base 2 logarithm.
 }
\addRow{
genfloat log10 (genfloat x)
}
{
Compute a base 10 logarithm.
}
\addRow{
genfloat log1p (genfloat x)
 }
 {
 Compute $loge(1.0 + x)$.
 }
\addRow{
genfloat logb (genfloat x)
}
{
Compute the exponent of x, which is the integral
part of logr ($| x |$).
}
\addRow{
genfloat mad (genfloat a,genfloat b, genfloat c)
}
{
mad approximates a * b + c. Whether or how the
product of a * b is rounded and how supernormal or
subnormal intermediate products are handled is not
defined. mad is intended to be used where speed is
preferred over accuracy.
}
\addRow{
 genfloat maxmag (genfloat x, genfloat y)
 }
{
Returns x if $| x |  >  | y |$, y if $| y | > | x |$, otherwise
fmax(x, y).
}
\addRow{genfloat minmag (genfloat x, genfloat y)}
{
Returns x if $| x | < | y |$, y if $| y | < | x |$, otherwise
fmin(x, y).
}
\addRow{
genfloat modf (genfloat x, genfloatptr iptr)
}
{
Decompose a floating-point number. The modf
function breaks the argument x into integral and
fractional parts, each of which has the same sign as
the argument. It stores the integral part in the object
pointed to by iptr.
}
\addRowTwoSL
{ genfloatf nan (ugenint nancode)}
{ genfloatd nan (ugenlonginteger nancode)}
{
Returns a quiet NaN. The nancode may be placed
in the significand of the resulting NaN.
}
\addRow{genfloat nextafter (genfloat x,
genfloat y)
}
{
Computes the next representable single-precision
floating-point value following x in the direction of
y. Thus, if y is less than x, nextafter() returns the
largest representable floating-point number less
than x.}

\addRow{
genfloat pow (genfloat x, genfloat y)
}
{
 Compute x to the power y.
}
\addRow
{
genfloat pown (genfloat x, genint y)
}
{
Compute x to the power y, where y is an integer.
}
\addRow
{
genfloat powr (genfloat x, genfloat y)
}
{
Compute x to the power y, where $ x > = 0 $.
}
\addRow{
genfloat remainder (genfloat x, genfloat y)
}
{
Compute the value r such that r = x - n*y, where n
is the integer nearest the exact value of x/y. If there
are two integers closest to x/y, n shall be the even
one. If r is zero, it is given the same sign as x.
}
\addRow{
genfloat remquo (genfloat x, genfloat y, genintptr quo)
}
{
The remquo function computes the value r such
that r = x - k*y, where k is the integer nearest the
exact value of x/y. If there are two integers closest
to x/y, k shall be the even one. If r is zero, it is
given the same sign as x. This is the same value
that is returned by the remainder function.
remquo also calculates the lower seven bits of the
integral quotient x/y, and gives that value the same
sign as x/y. It stores this signed value in the object
pointed to by quo.
}
\addRow{
genfloat rint (genfloat x)
}
{
Round to integral value (using round to nearest
even rounding mode) in floating-point format.
Refer to section 7.1 of the OpenCL 1.2 specification
document\cite{opencl12} for description of rounding
modes.
}
\addRow{
genfloat rootn (genfloat x, genint y)
}
{
Compute x to the power 1/y.
}
\addRow{
genfloat round (genfloat x)
}
{
 Return the integral value nearest to x rounding
halfway cases away from zero, regardless of the
current rounding direction.
}
\addRow{
genfloat rsqrt (genfloat x)
}
{
Compute inverse square root.
}
\addRow{
genfloat sin (genfloat x)
}
{
 Compute sine.
}
\addRow{
genfloat sincos (genfloat x, genfloatptr cosval)
}
{
Compute sine and cosine of x. The computed sine
is the return value and computed cosine is returned
in cosval.
}
\addRow{genfloat sinh (genfloat x)}{ Compute hyperbolic sine.}
\addRow{genfloat sinpi (genfloat x)}{ Compute sin ($\pi$ x).}
\addRow{genfloat sqrt (genfloat x)}{ Compute square root.}
\addRow{genfloat tan (genfloat x)}{ Compute tangent.}
\addRow{genfloat tanh (genfloat x)}{Compute hyperbolic tangent.}
\addRow{genfloat tanpi (genfloat x)}{ Compute tan ($\pi$ x).}
\addRow{genfloat tgamma (genfloat x)}{Compute the gamma function.}
\addRow{genfloat trunc (genfloat x)}{Round to integral value using the round to zero
rounding mode.}
\completeTable
%-------------------------------------------------------------------------------

In SYCL the implementation defined precision math functions are
defined in the namespace \codeinline{cl::sycl::native}. The functions
that are available within this namespace are specified in
Tables~\ref{table.native.math.functions}.

%-------------------------------------------------------------------------------

\startTable{Native Math Function}
\addFootNotes{Native math functions}{table.native.math.functions}

\addRow
{
genfloatf cos (genfloatf x)
}
{
Compute cosine over an implementation-defined range.
The maximum error is implementation-defined.
}
\addRow
{
genfloatf divide (genfloatf x, genfloatf y)
}
{
Compute x / y over an implementation-defined range.
The maximum error is implementation-defined.
}
\addRow
{
genfloatf exp (genfloatf x)
}
{
 Compute the base- e exponential of x over an
implementation-defined range. The maximum error is
implementation-defined.
}
\addRow
{
genfloatf exp2 (genfloatf x)
}
{
Compute the base- 2 exponential of x over an
implementation-defined range. The maximum error is
implementation-defined.
}
\addRow
{
genfloatf exp10 (genfloatf x)
}
{
 Compute the base- 10 exponential of x over an
implementation-defined range. The maximum error is
implementation-defined.
}
\addRow
{
genfloatf log (genfloatf x)
}
{
 Compute natural logarithm over an implementation defined range.
 The maximum error is implementation-defined.
 }
\addRow
{
 genfloatf log2 (genfloatf x)
 }
 {
 Compute a base 2 logarithm over an implementation-defined
range. The maximum error is implementation-defined.
}
\addRow
{
genfloatf log10 (genfloatf x)
}
{
Compute a base 10 logarithm over an implementation-defined
range. The maximum error is implementation-defined.
}
\addRow
{
genfloatf powr (genfloatf x, genfloatf y)
}
{
Compute x to the power y, where $ x > = 0$. The range of
x and y are implementation-defined. The maximum error
is implementation-defined.
}
\addRow
{
genfloatf recip (genfloatf x)
}
{
Compute reciprocal over an implementation-defined
range. The maximum error is implementation-defined.
}

\addRow
{
genfloatf rsqrt (genfloatf x)
}
{
Compute inverse square root over an implementation-defined
range. The maximum error is implementation-defined.
}
\addRow
{
genfloatf sin (genfloatf x)
}
{
 Compute sine over an implementation-defined range.
The maximum error is implementation-defined.
}
\addRow
{
genfloatf sqrt (genfloatf x)
}
{
Compute square root over an implementation-defined
range. The maximum error is implementation-defined.
}
\addRow
{
genfloatf tan (genfloatf x)
}
{
Compute tangent over an implementation-defined range.
The maximum error is implementation-defined.
}
\completeTable
%-------------------------------------------------------------------------------

In SYCL the half precision math functions are defined in
\codeinline{cl::sycl::half_precision}. The functions that are
available within this namespace are specified in
Tables~\ref{table.half.math.functions}. These functions are
implemented with a minimum of 10-bits of accuracy i.e. an ULP value is
less than or equal to 8192 ulp.

%-------------------------------------------------------------------------------
\startTable{Half Math function}
\addFootNotes{Half precision math functions}
{table.half.math.functions}

\addRow{
genfloatf cos (genfloatf x)
}
{
Compute cosine. x must be in the range -216 to +216.
}
\addRow{
genfloatf divide (genfloatf x, genfloatf y)
}
{
Compute x / y.
}
\addRow{genfloatf exp (genfloatf x) }{Compute the base- e exponential of x.}
\addRow{genfloatf exp2 (genfloatf x) }{Compute the base- 2 exponential of x.}
\addRow{genfloatf exp10 (genfloatf x) }{Compute the base- 10 exponential of x.}
\addRow{genfloatf log (genfloatf x) }{Compute natural logarithm.}
\addRow{genfloatf log2 (genfloatf x) }{Compute a base 2 logarithm.}
\addRow{genfloatf log10 (genfloatf x) }{Compute a base 10 logarithm.}
\addRow{genfloatf powr (genfloatf x, genfloatf y)}
{
Compute x to the power y, where $x >= 0$.
}
\addRow{genfloatf recip (genfloatf x)}{Compute reciprocal.}
\addRow{genfloatf rsqrt (genfloatf x)}{ Compute inverse square root.}
\addRow{genfloatf sin (genfloatf x) }{Compute sine. x must be in the range -216
to +216.}
\addRow{genfloatf sqrt (genfloatf x)}{ Compute square root.}
\addRow{genfloatf tan (genfloatf x) }{Compute tangent. x must be in the range
-216 to +216.}
\completeTable
%-------------------------------------------------------------------------------

%*******************************************************************************
% Integer functions
%*******************************************************************************
\subsection{Integer functions}

In SYCL the OpenCL integer math functions are available in the
namespace \codeinline{cl::sycl} on host and device as defined in the
OpenCL 1.2 specification document\cite[par. 6.12.3] {opencl12}. The
built-in functions can take as input \codeinline{char},
\codeinline{unsigned char}, \codeinline{short}, \codeinline{unsigned
  short}, \codeinline{int}, \codeinline{unsigned int},
\codeinline{long long int}, \codeinline{unsigned long long int} and
their \codeinline{vec} counterparts, for dimensions 2, 3, 4, 8 and
16. On the host the vector types use the \codeinline{vec} class and on
an OpenCL device use the corresponding OpenCL vector types. The
supported integer math functions are described in
Table~\ref{table.integer.functions}.

%-------------------------------------------------------------------------------
\startTable{Integer Function}
\addFootNotes{Integer functions which work on SYCL Host and device, are
available in the \codeinline{cl::sycl} namespace. They correspond to Table 6.10 of
the OpenCL 1.2 specification \cite{opencl12}}{table.integer.functions}
\addRow
{
ugeninteger abs (geninteger x)
}
{
Returns $| x |$.
}
\addRow { ugeninteger abs_diff (geninteger x, geninteger y)}
{ Returns $| x - y |$  without modulo overflow.}

\addRow { geninteger add_sat (geninteger x, geninteger y)}
{ Returns $x + y$ and saturates the result.}

\addRow { geninteger hadd (geninteger x, geninteger y)}
{
 Returns $(x + y) >> 1$. The intermediate sum does
not modulo overflow.
}
\addRow { geninteger rhadd (geninteger x, geninteger y)}
 {
 Returns $(x + y + 1) >> 1$. The intermediate sum
does not modulo overflow.
}

\addRowTwoL
{geninteger clamp (geninteger x, geninteger minval, geninteger maxval)}
{geninteger clamp (geninteger x, sgeninteger minval, sgeninteger maxval)}
{
 Returns min(max(x, minval), maxval).
 Results are undefined if minval $>$ maxval.
}

\addRow { geninteger clz (geninteger x) }
{
 Returns the number of leading 0-bits in x, starting
 at the most significant bit position.
}

\addRowTwoL {  geninteger mad_hi (}
{geninteger a, geninteger b, geninteger c)}
 {Returns \codeinline{mul_hi(a, b)+c}. }

\addRowTwoL { geninteger mad_sat (geninteger a,}
{geninteger b, geninteger c)}
{Returns \codeinline{a * b + c} and saturates the result.}

\addRowTwoSL
{geninteger max (geninteger x, geninteger y)}
{geninteger max (geninteger x, sgeninteger y)}
{Returns y if $x < y$, otherwise it returns x.}


\addRowTwoSL
{geninteger min (geninteger x, geninteger y)}
{geninteger min (geninteger x, sgeninteger y)}
{Returns y if $y < x$, otherwise it returns x.}

\addRow {geninteger mul_hi (geninteger x, geninteger y) }
{
Computes \codeinline{x * y} and returns the high half of the
product of x and y.
}

\addRow {geninteger rotate (geninteger v, geninteger i) }
{
For each element in v, the bits are shifted left by
the number of bits given by the corresponding
element in i (subject to usual shift modulo rules
described in section 6.3). Bits shifted off the left
side of the element are shifted back in from the
right.
}

\addRow {geninteger sub_sat (geninteger x, geninteger y)}
{ Returns $x - y$ and saturates the result.}

\addRow{ ugeninteger16bit upsample (ugeninteger8bit hi, ugeninteger8bit lo)}
{\codeinline{result[i] = ((ushort)hi[i] << 8) | lo[i]}}

\addRow {igeninteger16bit upsample (igeninteger8bit hi, ugeninteger8bit lo)}
{\codeinline{result[i] = ((short)hi[i] << 8) | lo[i]}}

\addRow {ugeninteger32bit upsample (ugeninteger16bit hi, ugeninteger16bit lo)}
{\codeinline{result[i] = ((uint)hi[i] << 16) | lo[i]}}

\addRow {igeninteger32bit upsample (igeninteger16bit hi, ugeninteger16bit lo)}
{\codeinline{result[i] = ((int)hi[i] << 16) | lo[i]}}

\addRow {ugeninteger64bit upsample (ugeninteger32bit hi, ugeninteger32bit lo)}
{\codeinline{result[i] = ((ulonglong)hi[i] << 32) | lo[i]}}

\addRow {igeninteger64bit upsample (igeninteger32bit hi, ugeninteger32bit lo)}
{\codeinline{result[i] = ((longlong)hi[i] << 32) | lo[i]}}

\addRow {geninteger popcount (geninteger x)}
{ Returns the number of non-zero bits in x.}

\addRow
{geninteger32bit mad24 (geninteger32bit x, geninteger32bit y, geninteger32bit z)}
{
Multipy two 24-bit integer values x and y and add
the 32-bit integer result to the 32-bit integer z.
Refer to definition of mul24 to see how the 24-bit
integer multiplication is performed.
}
\addRow
{geninteger32bit mul24 (geninteger32bit x, geninteger32bit y)}
{
Multiply two 24-bit integer values x and y. x and y
are 32-bit integers but only the low 24-bits are used
to perform the multiplication. mul24 should only
be used when values in x and y are in the range [-
223, 223-1] if x and y are signed integers and in the
range [0, 224-1] if x and y are unsigned integers. If
x and y are not in this range, the multiplication
result is implementation-defined.
}
\completeTable
%-------------------------------------------------------------------------------

%*******************************************************************************
% Common functions
%*******************************************************************************
\subsection{Common functions}
In SYCL the OpenCL \keyword{common} functions are available in the
namespace \codeinline{cl::sycl} on host and device as defined in the
OpenCL 1.2 specification document\cite[par. 6.12.4]{opencl12}. They
are described here in Table~\ref{table.common.functions}. The built-in
functions can take as input \codeinline{float} or optionally
\codeinline{double} and their \codeinline{vec} counterparts, for
dimensions 2, 3, 4, 8 and 16. On the host the vector types use the
\codeinline{vec} class and on an OpenCL device use the corresponding
OpenCL vector types.

%-------------------------------------------------------------------------------
\startTable{Common Function} \addFootNotes{Common functions which work
  on SYCL Host and device, are available in the \codeinline{cl::sycl}
  namespace. They correspond to Table 6.12 of the OpenCL 1.2
  specification \cite{opencl12}}{table.common.functions}
\addRowThreeSL
{genfloat clamp (genfloat x, genfloat minval, genfloat maxval)}
{genfloatf clamp (genfloatf x, float minval, float maxval)}
{genfloatd clamp (genfloatd x, double minval, double maxval)}
{
Returns fmin(fmax(x, minval), maxval).
Results are undefined if $minval > maxval$.
}

\addRow
{genfloat degrees (genfloat radians)}
{
 Converts radians to degrees,
i.e.$ (180 / \pi) *radians$.
}

\addRowThreeSL
 {genfloat max (genfloat x, genfloat y)}
 {genfloatf max (genfloatf x, float y)}
 {genfloatd max (genfloatd x, double y)}
{
Returns y if $x < y$, otherwise it returns x. If x or y
are infinite or NaN, the return values are undefined.
}

\addRowThreeSL
{genfloat min (genfloat x, genfloat y)}
{genfloatf min (genfloatf x, float y)}
{genfloatd min (genfloatd x, double y)}
{
Returns y if $y < x$, otherwise it returns x. If x or y
are infinite or NaN, the return values are undefined.
}

\addRowThreeSL
{genfloat mix (genfloat x, genfloat y, genfloat a)}
{genfloatf mix (genfloatf x, genfloatf y, float a)}
{genfloatd mix (genfloatd x, genfloatd y, double a)}
{
Returns the linear blend of $x \& y$ implemented as:
$x + (y - x) * a$.
\keyword{a} must be a value in the range 0.0 ... 1.0. If a is not
in the range 0.0 ... 1.0, the return values are
undefined.
}
\addRow
{genfloat radians (genfloat degrees) }
{
Converts degrees to radians, i.e. $(\pi / 180) * degrees$.
}
\addRowThreeSL
{genfloat step (genfloat edge, genfloat x)}
{genfloatf step (float edge, genfloatf x)}
{genfloatd step (double edge, genfloatd x)}
{
Returns 0.0 if $x < edge$, otherwise it returns 1.0.
}

\addRowThreeSL
{genfloat smoothstep (genfloat edge0,
genfloat edge1, genfloat x)}
{genfloatf smoothstep (float edge0,
float edge1, genfloatf x)}
{genfloatd smoothstep (double edge0,
double edge1, genfloatd x)}
{
Returns $0.0$ if $x <= edge0$ and $1.0$ if $x >= edge1$ and
performs smooth Hermite interpolation between 0
and 1 when $edge0 < x < edge1$. This is useful in
cases where you would want a threshold function
with a smooth transition.

This is equivalent to:\newline
\codeinline{gentype t;}\newline
\codeinline{t = clamp ((x <= edge0) / (edge1 >= edge0), 0, 1);}\newline
\codeinline{return t * t * (3 - 2 * t);}\newline

Results are undefined if $edge0 >= edge1$ or if x,
edge0 or edge1 is a NaN.
}
\addRow {genfloat sign (genfloat x) }
{
Returns $1.0$ if $x > 0$, $-0.0$ if $x = -0.0$, $+0.0$ if $x =+0.0$, or $-1.0$
if $x < 0$. Returns $0.0$ if x is a NaN.
}
\completeTable
%-------------------------------------------------------------------------------

%*******************************************************************************
% Geometric functions
%*******************************************************************************
\subsection{Geometric Functions}

In SYCL the OpenCL \keyword{geometric} functions are available in the
namespace \codeinline{cl::sycl} on host and device as defined in the
OpenCL 1.2 specification document \cite[par. 6.12.5]{opencl12}. The
built-in functions can take as input float or optionally double and
their \codeinline{vec} counterparts, for dimensions 2, 3 and 4. On the
host the vector types use the \codeinline{vec} class and on an OpenCL
device use the corresponding OpenCL vector types. All of the geometric
functions use round-to-nearest-even rounding
mode. Table~\ref{table.geometric.functions} contains the definitions
of supported geometric functions.

%-------------------------------------------------------------------------------
\startTable{Geometric Function} \addFootNotes{Geometric functions
  which work on SYCL Host and device, are available in the
  \codeinline{cl::sycl} namespace. They correspond to Table 6.13 of
  the OpenCL 1.2 specification \cite{opencl12}}
{table.geometric.functions}

\addRowFourSL
{float4 cross (float4 p0, float4 p1)}
{float3 cross (float3 p0, float3 p1)}
{double4 cross (double4 p0, double4 p1)}
{double3 cross (double3 p0, double3 p1)}
{
Returns the cross product of p0.xyz and p1.xyz. The
$w$ component of float4 result returned will be 0.0.
}
\addRowTwoSL
{ float dot (gengeofloat p0, gengeofloat p1)}
{ double dot (gengeodouble p0, gengeodouble p1) }
{
Compute dot product.
}
\addRowTwoSL
{float distance (gengeofloat p0, gengeofloat p1)}
{double distance (gengeodouble p0, gengeodouble p1)}
{
Returns the distance between p0 and p1. This is
calculated as \codeinline{length(p0 - p1)}.
}
\addRowTwoSL
{float length (gengeofloat p)}
{double length (gengeodouble p)}
{
Return the length of vector p, i.e.,
$\sqrt{ p.x^2 + p.y^2 + ...}$
}
\addRowTwoSL
{gengeofloat normalize (gengeofloat p)}
{gengeodouble normalize (gengeodouble p)}
{
Returns a vector in the same direction as p but with a
length of 1.
}

\addRow{float fast_distance (gengeofloat p0, gengeofloat p1)}
{
Returns \codeinline{fast_length(p0 - p1)}.
}
\addRow{float fast_length (gengeofloat p) }
{
Returns the length of vector p computed as:
\codeinline{sqrt((half)(pow(p.x,2) + pow(p.y,2) + ...))}
}
\addRow {gengeofloat fast_normalize (gengeofloat p) }
{
Returns a vector in the same direction as p but with a
length of 1. fast_normalize is computed as:
\codeinline{p * rsqrt((half) (pow(p.x,2) + pow(p.y,2) + ... ))}\newline
The result shall be within 8192 ulps error from the
infinitely precise result of
\codeinline{if (all(p == 0.0f))}\newline
\codeinline{result = p;}\newline
\codeinline{else}\newline
\codeinline{result = p / sqrt (pow(p.x,2) + pow(p.y,2) + ... );}\newline
with the following exceptions:
\begin{enumerate}
\item If the sum of squares is greater than \codeinline{FLT_MAX}
then the value of the floating-point values in the
result vector are undefined.

\item If the sum of squares is less than \codeinline{FLT_MIN} then
the implementation may return back p.

\item If the device is in ``denorms are flushed to zero''
mode, individual operand elements with magnitude
less than \codeinline{sqrt(FLT_MIN)} may be flushed to zero
before proceeding with the calculation.
\end{enumerate}
}
\completeTable
%-------------------------------------------------------------------------------

%*******************************************************************************
% Relational functions
%*******************************************************************************
\subsection{Relational functions}

In SYCL the OpenCL \keyword{relational} functions are available in the
namespace \codeinline{cl::sycl} on host and device as defined in the
OpenCL 1.2 specification document \cite[par. 6.12.6]{opencl12}. The
built-in functions can take as input \codeinline{char},
\codeinline{unsigned char}, \codeinline{short}, \codeinline{unsigned
  short}, \codeinline{int}, \codeinline{unsigned int},
\codeinline{long}, \codeinline{unsigned long}, \codeinline{float} or
optionally \codeinline{double} and their \codeinline{vec}
counterparts, for dimensions 2,3,4,8, and 16. On the host the vector
types use the \codeinline{vec} class and on an OpenCL device use the
corresponding OpenCL vector types.  The relational operators are
available on both host and device.  The relational functions are
provided in addition to the the operators and will return 0 if the
conditional is \codeinline{false} and 1 otherwise.  The available
built-in functions are described in
Tables~\ref{table.relational.functions}

\fixme{fixed typo originating from the OpenCL 1.2 spec: when those functions
take double as a parameter they should return longlong. }


%-------------------------------------------------------------------------------
\startTable{Relational Function}
\addFootNotes{ Relational functions which work on SYCL Host and device, are
available in the \codeinline{cl::sycl} namespace. They correspond to Table 6.14
of the OpenCL 1.2 specification \cite{opencl12}}{table.relational.functions}
\addRowTwoSL
{igeninteger32bit isequal (genfloatf x, genfloatf y)}
{igeninteger64bit isequal (genfloatd x, genfloatd y)}
{
Returns the component-wise compare of $x == y$.
}
\addRowTwoSL
{igeninteger32bit isnotequal (genfloatf x, genfloatf y)}
{igeninteger64bit isnotequal (genfloatd x, genfloatd y)}
{
Returns the component-wise compare of $x != y$.
}
\addRowTwoSL
{igeninteger32bit isgreater (genfloatf x, genfloatf y)}
{igeninteger64bit isgreater (genfloatd x, genfloatd y)}
{
Returns the component-wise compare of $x > y$.
}
\addRowTwoSL
{igeninteger32bit isgreaterequal (genfloatf x, genfloatf y)}
{igeninteger64bit isgreaterequal (genfloatd x, genfloatd y)}
{
Returns the component-wise compare of $x >= y$.
}
\addRowTwoSL
{igeninteger32bit isless (genfloatf x, genfloatf y)}
{igeninteger64bit isless (genfloatd x, genfloatd y)}
{
Returns the component-wise compare of $x < y$.
}

\addRowTwoSL
{igeninteger32bit islessequal (genfloatf x, genfloatf y)}
{igeninteger64bit islessequal (genfloatd x, genfloatd y)}
{
Returns the component-wise compare of $x <= y$.
}
\addRowTwoSL
{igeninteger32bit islessgreater (genfloatf x, genfloatf y)}
{igeninteger64bit islessgreater (genfloatd x, genfloatd y)}
{
Returns the component-wise compare of
$(x < y) || (x > y) $.
}

\addRowTwoSL
{igeninteger32bit isfinite (genfloatf x)}
{igeninteger64bit isfinite (genfloatd x)}
{
Test for finite value.
}
\addRowTwoSL
{igeninteger32bit isinf (genfloatf x)}
{igeninteger64bit isinf (genfloatd x)}
{
Test for infinity value (positive or negative) .
}
\addRowTwoSL
{igeninteger32bit isnan (genfloatf x)}
{igeninteger64bit isnan (genfloatd x)}
{
Test for a NaN.
}
\addRowTwoSL
{igeninteger32bit isnormal (genfloatf x)}
{igeninteger64bit isnormal (genfloatd x)}
{
Test for a normal value.
}
\addRowTwoSL
{igeninteger32bit isordered (genfloatf x, genfloatf y)}
{igeninteger64bit isordered (genfloatd x, genfloatd y)}
{
Test if arguments are ordered. isordered() takes
arguments x and y, and returns the result
isequal(x, x) $\&\&$ isequal(y, y).
}
\addRowTwoSL
{igeninteger32bit isunordered (genfloatf x, genfloatf y)}
{igeninteger64bit isunordered (genfloatd x, genfloatd y)}
{
Test if arguments are unordered. isunordered()
takes arguments x and y, returning non-zero if x or
y is NaN, and zero otherwise.
}

\addRowTwoSL
{igeninteger32bit signbit (genfloatf x)}
{igeninteger64bit signbit (genfloatd x)}
{
Test for sign bit. The scalar version of the
function returns a 1 if the sign bit in the float is set
else returns 0.

The vector version of the function
returns the following for each component in \keyword{floatn}:

-1 (i.e all bits set) if the sign bit in the float is set
else returns 0.
}

\addRow{int any (igeninteger x)}
{
Returns 1 if the most significant bit in any
component of x is set; otherwise returns 0.
}
\addRow {int all (igeninteger x) }
{
Returns 1 if the most significant bit in all
components of x is set; otherwise returns 0.
}
\addRow
{gentype bitselect (gentype a, gentype b, gentype c) }
{
Each bit of the result is the corresponding bit of a
if the corresponding bit of c is 0. Otherwise it is
the corresponding bit of b.
}
\addRowSixSL
{geninteger select (geninteger a, geninteger b, igeninteger c)}
{geninteger select (geninteger a, geninteger b, ugeninteger c)}
{genfloatf select (genfloatf a, genfloatf b, genint c)}
{genfloatf select (genfloatf a, genfloatf b, ugenint c)}
{genfloatd select (genfloatd a, genfloatd b, igeninteger64 c)}
{genfloatd select (genfloatd a, genfloatd b, ugeninteger64 c)}
{
For each component of a vector type:\newline
\codeinline{result[i] =  (MSB of c[i] is set) ? b[i] : a[i].}\newline
For a scalar type:\newline
\codeinline{result = c ? b : a}.\newline
\codeinline{geninteger} must have the same number
of elements and bits as \codeinline{gentype}.
}
\completeTable
%-------------------------------------------------------------------------------

%*******************************************************************************
% Vector data and store functions
%*******************************************************************************
\subsection{Vector data load and store functions}

The functionality from the OpenCL functions as defined in the OpenCL 1.2
specification document\cite[par. 6.12.7]{opencl12} is available in SYCL through
the \codeinline{vec} class in section \ref{sec:vector.type}.

\subsection{Synchronization Functions}

In SYCL the OpenCL \keyword{synchronization} functions are available through the
\codeinline{nd_item} class~\ref{nditem-class}, as they are applied to work-items for local
or global address spaces. Please see~\ref{table.members.nditem}.

%*******************************************************************************
% printf function
%*******************************************************************************
\subsection{\texttt{printf} function}

The functionality of the \texttt{printf} function is covered by the
\codeinline{stream} class~\ref{subsection:stream}, which has the
capability to print to standard output all of the SYCL classes and primitives,
and covers the capabilities defined in the OpenCL 1.2 specification
document\cite[par. 6.12.13]{opencl12}.


%%% Local Variables:
%%% mode: latex
%%% TeX-master: "sycl-1.2.1"
%%% TeX-auto-untabify: t
%%% TeX-PDF-mode: t
%%% ispell-local-dictionary: "american"
%%% End:
