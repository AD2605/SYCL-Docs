% !TEX root = sycl-1.2.1.tex
\def \debugging{false}

%*********************************************************************
% SYCL Programming Interface
%*********************************************************************
\chapter{SYCL Programming Interface}
\label{chapter:sycl-programming-interface}

The SYCL programming interface provides a C++ abstraction to OpenCL
1.2 functionality and feature set. This section describes all the
available classes and interfaces of SYCL, focusing on the C++
interface of the underlying runtime. In this section, we are defining
all the classes and member functions for the SYCL API, which are available for
SYCL host and OpenCL devices. This section also describes the
synchronization rules and OpenCL API interoperability rules which
guarantee that all the member functions and special member functions of the SYCL
classes are thread safe.

It is assumed that the OpenCL API is also available to the developer
at the same time as SYCL.

%*********************************************************************
% Header files and namespaces
%*********************************************************************
\section{Header files and namespaces}
\label{sec:headers-and-namespaces}

SYCL provides one standard header file: \codeinline{"CL/sycl.hpp"},
which needs to be included in every SYCL program.

All SYCL classes, constants, types and functions defined by this
specification should exist within the \codeinline{cl::sycl} namespace.

Any SYCL classes, constants, types and functions defined as an extension
to this specification should exist within the
\codeinline{cl::sycl::<vendor_name>} namespace.

The \codeinline{cl::sycl::detail} namespace is reserved for
implementation details.

\section{Class availability}

In SYCL some \gls{sycl-runtime} classes are available to the host application, some are 
available within a \gls{sycl-kernel-function} and some available on both and can
be passed as parameters to a \gls{sycl-kernel-function}.

Each of the following \gls{sycl-runtime} classes: \codeinline{device_selector},
\codeinline{platform}, \codeinline{device}, \codeinline{context}, \codeinline{
queue}, \codeinline{program}, \codeinline{kernel}, \codeinline{event},
\codeinline{buffer}, \codeinline{image}, \codeinline{sampler}, \codeinline{
stream}, \codeinline{handler}, \codeinline{nd_range}, \codeinline{range},
\codeinline{id}, \codeinline{vec}, \codeinline{buffer_allocator}, \codeinline{
image_allocator} and \codeinline{exception} must be available to the host application.

Each of the following \gls{sycl-runtime} classes: \codeinline{accessor}, \codeinline{sampler}, \codeinline{stream}, \codeinline{vec}, \codeinline{multi_ptr} \codeinline{device_event}, \codeinline{id}, \codeinline{range}, \codeinline{item}, \codeinline{nd_item}, \codeinline{h_item}, \codeinline{group} and \codeinline{atomic}
must be available within a \gls{sycl-kernel-function}.

Each of the following \gls{sycl-runtime} classes: \codeinline{accessor}, \codeinline{sampler}, \codeinline{stream}, \codeinline{vec}, \codeinline{id} and \codeinline{range} are permitted as parameters to a \gls{sycl-kernel-function}.

\section{Common interface}

When a dimension template parameter is used in SYCL classes, it is
defaulted as 1 in most cases.

\subsection{OpenCL interoperability}
\label{sec:opencl-interoperability}

Many of the \gls{sycl-runtime} classes encapsulate an associated OpenCL opaque type and provide facilities for interoperating between the SYCL classes and the OpenCL opaque types they encapsulate in order to allow interoperability between SYCL and OpenCL applications.

Each of the following \gls{sycl-runtime} classes: \codeinline{platform},
\codeinline{device}, \codeinline{context}, \codeinline{queue}, \codeinline{program}, \codeinline{kernel}, \codeinline{event}, \codeinline{buffer}, \codeinline{image}, \codeinline{sampler} and \codeinline{stream} must obey the following statements, where \codeinline{T} is the runtime class type:

\begin{itemize}

\item \codeinline{T} on the host application must encapsulate at least one valid instance of the associated OpenCL opaque type if the SYCL class instance is an OpenCL instance and must not encapsulate any instance of the associated OpenCL opaque type if this instance of T is a host instance.

\item \codeinline{T} must provide an interoperability constructor on the host application which takes as a parameter, a valid instance of the associated OpenCL opaque type, which must be retained on construction of an instance of \codeinline{T}, where applicable. The constructed instance of T must be an OpenCL instance and must encapsulate only the single instance of the associated OpenCL opaque type provided during construction.

\item \codeinline{T} must provide a \codeinline{get()} member function on the host application which returns an encapsulated instance of the associated OpenCL opaque type if this instance of \codeinline{T} is an OpenCL instance, and must throw an \codeinline{invalid_object_error} SYCL exception if this instance of \codeinline{T} is a host instance. The instance of the associated OpenCL type must be retained before returning and must always be the same instance, where applicable.

\item \codeinline{T} must release each encapsulated instance of the associated OpenCL opaque type on destruction on the host application, where applicable.

\end{itemize}

The only exceptions to these rules are the SYCL \codeinline{buffer}, \codeinline{image} and \codeinline{sampler} classes which do not require a \codeinline{get()} member function.

For more details regarding these facilities and considerations for their use see section \ref{sec:interfacing-with-opencl}.

\subsection{Common reference semantics}
\label{sec:reference-semantics}

Each of the following \gls{sycl-runtime} classes: \codeinline{device}, \codeinline{context}, \codeinline{queue}, \codeinline{program}, \codeinline{kernel}, \codeinline{event}, \codeinline{buffer}, \codeinline{image}, \codeinline{sampler}, \codeinline{accessor} and \codeinline{stream} must obey the following statements, where \codeinline{T} is the runtime class type:

\begin{itemize}

\item \codeinline{T} must be copy constructible and copy assignable on the host application and within SYCL kernel functions in the case that \codeinline{T} is a valid kernel argument. Any instance of \codeinline{T} that is constructed as a copy of another instance, via either the copy constructor or copy assignment operator, must behave as-if it were the original instance and as-if any action performed on it were also performed on the original instance and if said instance is not a host object must represent and continue to represent the same underlying OpenCL objects as the original instance where applicable.

\item \codeinline{T} must be destructible on the host application and within SYCL kernel functions in the case that \codeinline{T} is a valid kernel argument. When any instance of \codeinline{T} is destroyed, including as a result of the copy assignment operator, any behavior specific to \codeinline{T} that is specified as performed on destruction is only performed if this instance is the last remaining host copy, in accordance with the above definition of a copy and the destructor requirements described in \ref{sec:opencl-interoperability} where applicable.

\item \codeinline{T} must be move constructible and move assignable on the host application and within SYCL kernel functions in the case that T is a valid kernel argument. Any instance of T that is constructed as a move of another instance, via either the move constructor or move assignment operator, must replace the original instance rendering said instance invalid and if said instance is not a host object must represent and continue to represent the same underlying OpenCL objects as the original instance where applicable.

\item \codeinline{T} must be equality comparable on the host application. Equality between two instances of \codeinline{T} (i.e. \codeinline{a == b}) must be true if one instance is a copy of the other and non-equality between two instances of \codeinline{T} (i.e. \codeinline{a != b}) must be true if neither instance is a copy of the other, in accordance with the above definition of a copy, unless either instance has become invalidated by a move operation. By extension of the requirements above, equality on \codeinline{T} must guarantee to be reflexive (i.e. \codeinline{a == a}), symmetric (i.e. \codeinline{a == b} implies \codeinline{b == a} and \codeinline{a != b} implies \codeinline{b != a}) and transitive (i.e. \codeinline{a == b && b == c} implies \codeinline{c == a}).

\item A specialization of \codeinline{hash_class} for \codeinline{T} must exist on the host application that returns a unique value such that if two instances of T are equal, in accordance with the above definition, then their resulting hash values are also equal and subsequently if two hash values are not equal, then their corresponding instances are also not equal, in accordance with the above definition.

\end{itemize}

Some \gls{sycl-runtime} classes will have additional behavior associated with copy, movement, assignment or destruction semantics. If these are specified they are in addition to those specified above unless stated otherwise.

Each of the runtime classes mentioned above must provide a common interface of special member functions and member functions in order to fulfil the copy, move, destruction and equality requirements.

These common special member functions and member functions are described in Tables~\ref{table.specialmembers.common.reference} and \ref{table.members.common.reference} respectively.

\lstinputlisting{headers/common-reference.h}

\startTable{Special member function}
\addFootNotes{Common special member functions for reference semantics}
{table.specialmembers.common.reference}
  \addRow
    {T(const T \&rhs)}
    {
      Constructs a \codeinline{T} instance as a copy of the RHS SYCL \codeinline{T} in accordance with the requirements set out above.
    }
  \addRow
    {T(T \&\&rhs)}
    {
      Constructs a SYCL \codeinline{T} instance as a move of the RHS SYCL \codeinline{T} in accordance with the requirements set out above.
    }  
   \addRow
   {T \&operator=(const T \&rhs)}
   {
     Assigns this SYCL \codeinline{T} instance with a copy of the RHS SYCL \codeinline{T} in accordance with the requirements set out above.
   }
   \addRow
   {T \&operator=(T \&\&rhs)}
   {
     Assigns this SYCL \codeinline{T} instance with a move of the RHS SYCL \codeinline{T} in accordance with the requirements set out above.
   }
   \addRow
   {\~T()}
   {
     Destroys this SYCL \codeinline{T} instance in accordance with the requirements set out in \ref{sec:reference-semantics}.
     Must release a reference to the associated OpenCL object if this SYCL \codeinline{T} instance was constructed with the interoperability constructor.
   }
\completeTable

\startTable{Member function}
\addFootNotes{Common member functions for reference semantics}
{table.members.common.reference}
   \addRow
   {bool operator==(const T \&rhs) const}
   {
     Returns true if this SYCL \codeinline{T} is equal to the RHS SYCL \codeinline{T} in accordance with the requirements set out above, otherwise returns false.
   }
   \addRow
   {bool operator!=(const T \&rhs) const}
   {
     Returns true if this SYCL \codeinline{T} is not equal to the RHS SYCL \codeinline{T} in accordance with the requirements set out above, otherwise returns false.
   }
\completeTable

\subsection{Common by-value semantics}
\label{sec:byval-semantics}

Each of the following \gls{sycl-runtime} classes: \codeinline{id}, \codeinline{range}, \codeinline{item}, \codeinline{nd_item}, \codeinline{h_item}, \codeinline{group} and \codeinline{nd_range} must follow the following statements, where \codeinline{T} is the runtime class type:

\begin{itemize}

\item \codeinline{T} must be default copy constructible and copy assignable on the host application and within SYCL kernel functions.

\item \codeinline{T} must be default destructible on the host application and within SYCL kernel functions.

\item \codeinline{T} must be default move constructible and default move assignable on the host application and within SYCL kernel functions.

\item \codeinline{T} must be equality comparable on the host application and within SYCL kernel functions. Equality between two instances of \codeinline{T} (i.e. \codeinline{a == b}) must be true if the value of all members are equal and non-equality between two instances of \codeinline{T} (i.e. \codeinline{a != b}) must be true if the value of any members are not equal, unless either instance has become invalidated by a move operation. By extension of the requirements above, equality on \codeinline{T} must guarantee to be reflexive (i.e. \codeinline{a == a}), symmetric (i.e. \codeinline{a == b} implies \codeinline{b == a} and \codeinline{a != b} implies \codeinline{b != a}) and transitive (i.e. \codeinline{a == b && b == c} implies \codeinline{c == a}).

\end{itemize}

Some \gls{sycl-runtime} classes will have additional behavior associated with copy, movement, assignment or destruction semantics. If these are specified they are in addition to those specified above unless stated otherwise.

Each of the runtime classes mentioned above must provide a common interface of special member functions and member functions in order to fulfil the copy, move, destruction and equality requirements.

These common special member functions and member functions are described in Tables~\ref{table.specialmembers.common.byval} and \ref{table.members.common.byval} respectively.

\lstinputlisting{headers/common-byval.h}

\startTable{Special member function}
\addFootNotes{Common special member functions for by-value semantics}
{table.specialmembers.common.byval}
  \addRow
    {T(const T \&rhs) = default}
    {
      Default copy constructor.  
    }
  \addRow
    {T(T \&\&rhs) = default}
    {
      Default move constructor.
    }  
   \addRow
   {T \&operator=(const T \&rhs) = default}
   {
     Default copy assignment operator.
   }
   \addRow
   {T \&operator=(T \&\&rhs) = default}
   {
     Default move assignment operator.
   }
   \addRow
   {\~T() = default}
   {
     Default destructor.
   }
\completeTable

\startTable{Member function}
\addFootNotes{Common member functions for by-value semantics}
{table.members.common.byval}
   \addRow
   {bool operator==(const T \&rhs) const}
   {
     Returns true if this SYCL \codeinline{T} is equal to the RHS SYCL \codeinline{T} in accordance with the requirements set out above, otherwise returns false.
   }
   \addRow
   {bool operator!=(const T \&rhs) const}
   {
     Returns true if this SYCL \codeinline{T} is not equal to the RHS SYCL \codeinline{T} in accordance with the requirements set out above, otherwise returns false.
   }
\completeTable

%*********************************************************************
% Properties
%*********************************************************************
\subsection{Properties}

Each of the following \gls{sycl-runtime} classes: \codeinline{buffer}, \codeinline{image} and \codeinline{queue} provide an optional parameter in each of their constructors to provide a \codeinline{property_list} which contains zero or more properties. Each of those properties augments the semantics of the class with a particular feature. Each of those classes must also provide \codeinline{has_property} and \codeinline{get_property} member functions for querying for a particular property.

The listing below illustrates the usage of various buffer properties, described in \ref{sec:buffer-properties}.

The example illustrates how using properties does not affect the type
of the object, thus, does not prevent the usage of SYCL objects in
containers.

\lstinputlisting{code/propertyExample.cpp}

Each property is represented by a unique class and an instance of a property is an instance of that type. Some properties can be default constructed while other will require an argument on construction. A property may be applicable to more than one class, however some properties may not be compatible with each other. See the requirements for the properties of the SYCL \codeinline{buffer} class and SYCL \codeinline{image} class in Table~\ref{table.properties.buffer} and Table~\ref{table.properties.image} respectively.

Any property that is provided to a \gls{sycl-runtime} class via an instance of the SYCL \codeinline{property_list} class must become encapsulated by that class and therefore shared between copies of that class. As a result properties must inherit the copy and move semantics of that class as described in \ref{sec:reference-semantics}.

A SYCL implementation may provide additional properties other than those defined here, provided they are defined in accordance with the requirements described in \ref{sec:headers-and-namespaces}.

\subsubsection{Properties interface}

Each of the runtime classes mentioned above must provide a common interface of member functions in order to fulfil the property interface requirements.

A synopsis of the common properties interface, the SYCL \codeinline{property_list} class and the SYCL property classes is provided below. The member functions of the common properties interface are listed in Table~\ref{table.members.propertyinterface}. The constructors of the SYCL \codeinline{property_list} class are listed in Table~\ref{table.constructors.propertylist}.

\lstinputlisting{headers/properties.h}

%---------------------------------------------------------------------
\startTable{Member function}
\addFootNotes{Common member functions of the SYCL property interface}
{table.members.propertyinterface}
  \addRowTwoSL
    { template <typename propertyT> }  
    { bool has_property() const }
    {
      Returns true if \codeinline{T} was constructed with the property
      specified by \codeinline{propertyT}. Returns false if it was
      not.
    }
  \addRowTwoSL
    { template <typename propertyT> }  
    { propertyT get_property() const }
    {
      Returns a copy of the property of type \codeinline{propertyT}
      that \codeinline{T} was constructed with. Must throw an
      \codeinline{invalid_object_error} SYCL exception if \codeinline{
      T} was not constructed with the \codeinline{propertyT} property.
    }
\completeTable
%---------------------------------------------------------------------

%---------------------------------------------------------------------
\startTable{Constructor}
\addFootNotes{Constructors of the SYCL \codeinline{property_list} class}
{table.constructors.propertylist}
\addRowTwoSL
{ template <typename... propertyTN> }
{ property_list(propertyTN... props) }
{
  Construct a SYCL \codeinline{property_list} with zero or more properties.
}
\completeTable
%---------------------------------------------------------------------

%*********************************************************************
% Param traits class
%*********************************************************************
\section{Param traits class}
\label{sec:param-traits}

The class \codeinline{param_traits} is a C++ type trait for providing an alias to the return type associated with each info parameter. An implementation must provide a specialization of the \codeinline{param_traits} class for every info
parameter with the associated return type as defined in the info
parameter tables.

\lstinputlisting{headers/paramTraits.h}

%*********************************************************************
% C++ Standard library classes required for the interface
%*********************************************************************
\section{C++ Standard library classes required for the interface}
\label{sec:stlclasses}

The SYCL programming interfaces make extensive use of vectors, strings
and function objects to carry information. Moreover, smart pointer
and mutex classes allow extending the SYCL programming interface in
terms of host data management. SYCL will default to using the STL
string, vector, function, mutex and smart pointer classes, unless
defined otherwise.

A SYCL implementation must provide aliases for the STL types that 
are used on the interface. 
These types are exposed internally as
\codeinline{vector_class},
\codeinline{string_class},
\codeinline{function_class},
\codeinline{mutex_class},
\codeinline{shared_ptr_class},
\codeinline{weak_ptr_class},
\codeinline{hash_class} and
\codeinline{exception_ptr_class}.

Typically, the SYCL types will be aliases to the system 
STL library, as shown in the listing below:

\lstinputlisting{headers/stlclasses.h}

However, a SYCL implementation may provide a custom implementation 
of any of these objects. 
This enables SYCL implementations to use optimized classes for specific
platforms.
To guarantee interoperability with the implementation types, users should
use the aliases on the SYCL namespace instead of the standard types.
Implementations must provide their own implicit conversion operations from the 
standard types into the custom defined types if they are not the same
as the ones provided by the default standard template library of the system.

%***********************************************************************************
% SYCL runtime classes
%***********************************************************************************
\section{SYCL runtime classes}


%***********************************************************************************
% Device selection class
%***********************************************************************************
\subsection{Device selection class}
\label{sec:device-selector}

The SYCL \codeinline{device_selector} class is an
object which enables the \gls{sycl-runtime} to choose the best device based
on heuristics specified by the user, or by one of the built-in device
selectors. The built-in device selectors are listed in Table
~\ref{table.device.selectors}.

The constructors and member functions of the SYCL \codeinline{device_selector} class are described in Tables~\ref{table.constructors.deviceSelector} and \ref{table.members.deviceSelector} respectively.

All member functions of the \codeinline{device_selector} class are synchronous and errors are handled by throwing synchronous SYCL exceptions.

\subsubsection{Device selector interface}

The function call operator; \codeinline{operator()} of the SYCL \codeinline{device_selector} is an abstract member function which takes a reference to a SYCL \codeinline{device} and returns an integer score.  This abstract member function can be implemented in a derived class in order to provide a logic for selecting a SYCL \codeinline{device}.

At any point where the \gls{sycl-runtime} needs to select a SYCL \codeinline{device}, the system will call the \codeinline{select_device()} member functions, which will query all available SYCL \codeinline{device}s in the system, pass each to this function call operator and select the one which returns the highest highest score. If a negative score is returned the the corresponding SYCL \codeinline{device} will never be chosen. The SYCL \codeinline{device}s that are provided to the SYCL \codeinline{device_selector} can be any number of OpenCL devices but must contain a single host device.

\lstinputlisting{headers/deviceSelectorInterface.h}
%---------------------------------------------------------------------
\startTable{Constructor}
\addFootNotes{Constructors of the \codeinline{device_selector} class}
{table.constructors.deviceSelector}
\addRow
{device_selector()}
{
  Constructs a SYCL\codeinline{device_selector} instance.
}
\addRow
{device_selector(const device_selector \&rhs)}
{
  Constructs a SYCL \codeinline{device_selector} instance from another instance.
}
\addRow
{device_selector \&operator=(const device_selector \&rhs)}
{
  Assigns this SYCL \codeinline{device_selector} instance with another instance.
}
\addRow
{virtual ~device_selector()}
{
  Destroys this SYCL \codeinline{device_selector} instance.
}
\completeTable

%---------------------------------------------------------------------

\startTable{Member function}
\addFootNotes{Member functions for the \codeinline{device_selector} class}
{table.members.deviceSelector}
\addRow
{device select_device() const}
{
  Returns a SYCL \codeinline{device} that has been selected based on the highest score returned by the function call operator for all available SYCL \codeinline{device}s in the system.
}
\addRow
{virtual int operator()(const device \&device) const}
{
  Pure virtual member function, required to be implemented in a derived class to provide a logic for selecting a SYCL \codeinline{device}.
  Returns an integer score for the \codeinline{device} parameter based on the logic defined within it.
}
\completeTable
%---------------------------------------------------------------------

\subsubsection{Derived device selector classes}

As the SYCL \codeinline{device_selector} is an abstract class, it must be derived from with a valid implementation of the function call operator in order to be used by the \gls{sycl-runtime}.

Any class which derives from the \codeinline{device_selector}, in order to
be used polymorphically, must have a valid copy constructor,
copy assignment operator and destructor and it must implement the
abstract function call operator.

The system provides a number of built-in derived \codeinline{device_selector} types, including a selectors type which chooses a SYCL \codeinline{device} based on the default behavior of the \gls{sycl-runtime}, known as the \codeinline{default_selector}. It is important to note that the behavior of the \codeinline{default_selector} may be restricted by the platforms that the implementation chooses to target, and it must select a host device if no other suitable OpenCL device can be found. The SYCL \codeinline{default_selector} is used in some cases as the default SYCL \codeinline{device_selector} if one if not provided.

%---------------------------------------------------------------------
\startTable{SYCL device selectors}
\addFootNotes{Standard device selectors included with all SYCL
  implementations}{table.device.selectors}
  \addRow
    {default_selector}
    {
      Derived SYCL \codeinline{device_selector} which selects a SYCL \codeinline{device} based on an implementation defined heuristic. Must select a host device if no other suitable OpenCL device can be found.
    }
  \addRow
    {gpu_selector}
    {
      Derived SYCL \codeinline{device_selector} which selects a SYCL \codeinline{device} for which the device type is \codeinline{info::device::device_type::gpu}. Must throw a \codeinline{runtime_error} SYCL exception if no OpenCL device matching this requirement can be found.
    }
  \addRow
    {accelerator_selector}
    {
      Derived SYCL \codeinline{device_selector} which selects a SYCL \codeinline{device} for which the device type is \codeinline{info::device::device_type::accelerator}. Must throw a \codeinline{runtime_error} SYCL exception if no OpenCL device matching this requirement can be found.
    }
  \addRow
    {cpu_selector}
    {
      Derived SYCL \codeinline{device_selector} which selects a SYCL \codeinline{device} for which the device type is \codeinline{info::device::device_type::cpu}. Must throw a \codeinline{runtime_error} SYCL exception if no OpenCL device matching this requirement can be found.    
    }
  \addRow
    {host_selector}
    {
      Derived SYCL \codeinline{device_selector} which selects a SYCL \codeinline{device} that is a host device. Must always return a valid SYCL \codeinline{device}.
    }

\completeTable
%---------------------------------------------------------------------

%*********************************************************************
% Platform class
%*********************************************************************
\subsection{Platform class}

\label{sec:platform-class}

The SYCL \codeinline{platform} class encapsulates a single SYCL platform on which SYCL kernel functions may be executed. A SYCL platform may be an OpenCL platform in which case it must encapsulate a valid underlying OpenCL \codeinline{cl_platform_id}, or it may be a SYCL host platform in which case it must not.

A SYCL \codeinline{platform} is also associated with one or more SYCL \codeinline{device}s. These can be any number of OpenCL devices or exactly one host device.

All member functions of the \codeinline{platform} class are synchronous and errors are handled by throwing synchronous SYCL exceptions.

The default constructor of the SYCL \codeinline{platform} class will construct
a host platform. The explicit constructor of the SYCL \codeinline{platform}
class which takes a \codeinline{device_selector} will construct a host platform
if \codeinline{select_device} returns a host device, otherwise will construct
an OpenCL platform. The OpenCL interop constructor of the SYCL
\codeinline{platform} class will construct an OpenCL platform.

The SYCL \codeinline{platform} class provides the common reference semantics
(see Section~\ref{sec:reference-semantics}).

\subsubsection{Platform interface}

A synopsis of the SYCL \codeinline{platform} class is provided below. The constructors, member functions and static member functions of the SYCL \codeinline{platform} class are listed in Tables~\ref{table.constructors.platform}, \ref{table.members.platform} and \ref{table.staticmembers.platform} respectively. The additional common special member functions and common member functions are listed in \ref{sec:reference-semantics} in Tables~\ref{table.specialmembers.common.reference} and \ref{table.members.common.reference} respectively.

%Interface of platform class
\lstinputlisting{headers/platform.h}

%-------------------------------------------------------------------------------
\startTableDim{Constructor}{6.7cm}{8.3cm}
\addFootNotes{Constructors of the SYCL \codeinline{platform} class}
{table.constructors.platform}
  \addRow
    {platform()}
    {
      Constructs a SYCL \codeinline{platform} instance as a host platform.
    }
  \addRow
    {explicit platform(cl_platform_id platformID)}
    {    
     Constructs a SYCL \codeinline{platform} instance from an OpenCL \codeinline{cl_platform_id} in accordance with the requirements described in \ref{sec:opencl-interoperability}.
    }
  \addRow
    {explicit platform(const device_selector \&deviceSelector)}
    {
      Constructs a SYCL \codeinline{platform} instance using the the \codeinline{deviceSelector} parameter. One of the SYCL \codeinline{device}s that is associated with the constructed SYCL \codeinline{platform} instance must be the SYCL \codeinline{device} that is produced from the \codeinline{deviceSelector} parameter.
    }
\completeTable
%-------------------------------------------------------------------------------

%-------------------------------------------------------------------------------
\startTable{Member function}
\addFootNotes{Member functions of the SYCL \codeinline{platform} class}{table.members.platform}
  \addRow
    {cl_platform_id get() const}
    {  
      Returns a valid \codeinline{cl_platform_id} instance in accordance with the requirements described in \ref{sec:opencl-interoperability}.    
    }
  \addRowThreeL
    {template <info::platform param>}
    {  typename info::param_traits<info::platform, param>::return_type}
    {  get_info() const}
    {
      Queries this SYCL \codeinline{platform} for information requested by the template parameter \codeinline{param}.
      Specializations of \codeinline{info::param_traits} must be defined in accordance with the info parameters in Table~\ref{table.device.info} to facilitate returning the type associated with the \codeinline{param} parameter.
    }
  \addRow
    {bool has_extension(const string_class \& extension) const}
    {
      Returns true if this SYCL \codeinline{platform} supports the extension queried by the \codeinline{extension} parameter. A SYCL \codeinline{platform} can only support an extension if all associated SYCL \codeinline{device}s support that extension.
    }
  \addRow
    {bool is_host() const}
    {
      Returns true if this SYCL \codeinline{platform} is a host platform.
    }
  \addRowTwoL
    {vector_class<device> get_devices(}
    {  info::device_type = info::device_type::all) const}
    {
      Returns a vector_class containing all SYCL \codeinline{device}s associated with this SYCL \codeinline{platform}. The returned vector_class must contain only a single SYCL \codeinline{device} that is a host device if this SYCL \codeinline{platform} is a host platform.
      Must return an empty \codeinline{vector_class} instance if there are no
      devices that match the given \codeinline{info::device_type}.
    }
\completeTable
%-------------------------------------------------------------------------------

%-------------------------------------------------------------------------------
\startTable{Static member function}
\addFootNotes{Static member functions of the SYCL \codeinline{platform} class}{table.staticmembers.platform}
  \addRow
    {static vector_class<platform> get_platforms()}
    {
      Returns a \codeinline{vector_class} containing all SYCL \codeinline{platform}s available in the system. The returned vector_class must contain a single SYCL \codeinline{platform} that is a host platform.
    }
\completeTable
%-------------------------------------------------------------------------------

\subsubsection{Platform information descriptors}

A SYCL \codeinline{platform} can be queried for all of the following information using the \codeinline{get_info} member function. All SYCL \codeinline{platform}s must have valid values for every query, including a host platform. The information that can be queried is described in Table~\ref{table.platform.info}. The interface for all information types and enumerations are described in appendix~\ref{appendix.platform.descriptors}.

%-------------------------------------------------------------------------------

\startInfoTableDims{Platform descriptors}{5cm}{2cm}{7cm}
\addInfoFootNotes{Platform information descriptors}{table.platform.info}

  \addInfoRow
    {info::platform::profile}
    {string_class}
    {
      Returns the OpenCL profile as a \codeinline{string_class}, if this SYCL \codeinline{platform} is an OpenCL platform. The value returned can be one of the following strings:
      \begin{itemize}
        \item \codeinline{"FULL_PROFILE"} --- if the platform supports the OpenCL specification (functionality defined as part of the core specification and does not require any extensions to be supported).
        \item \codeinline{"EMBEDDED_PROFILE"} --- if the platform supports the OpenCL embedded profile.
      \end{itemize}
    Must return a \codeinline{string_class} with the value \codeinline{"FULL PROFILE"} if this is a host platform.
    }
    \addInfoRow
    {info::platform::version}
    {string_class}
    {
      Returns the OpenCL software driver version as a \codeinline{string_class} in the form: major_number.minor_number, if this SYCL \codeinline{platform} is an OpenCL platform. Must return a \codeinline{string_class} with the value \codeinline{"1.2"} if this SYCL \codeinline{platform} is a host platform.
    }
  \addInfoRow
    {info::platform::name}
    {string_class}
    {
      Returns the device name of this SYCL \codeinline{platform}.
    }
  \addInfoRow
    {info::platform::vendor}
    {string_class}
    {
      Returns the vendor of this SYCL \codeinline{platform}.
    }
  \addInfoRow
    {info::platform::extensions}
    {vector_class<string_class>}
    {
      Returns a \codeinline{vector_class} of extension names (the extension names do not contain any spaces) supported by this SYCL \codeinline{platform}. An extension can only be returned here if it is supported by all associated SYCL \codeinline{device}s.
    }
\completeTable
%------------------------------------------------------------------------------------------------------


%***********************************************************************************
% Context class
%***********************************************************************************
\subsection{Context class}
\label{sec:interface.context.class}

The \gls{context} class represents a SYCL \gls{context} on which SYCL kernel functions may be executed. A SYCL \gls{context} may be an OpenCL context, in which case it must encapsulate a valid underlying OpenCL \codeinline{cl_context}, or it may be a SYCL host context, in which case it must not. A SYCL \codeinline{context} must encapsulate a single SYCL \gls{platform} and a collection of SYCL \gls{device}s all of which are associated with said \gls{platform}.

The SYCL \codeinline{context} class provides the common reference semantics
(see Section~\ref{sec:reference-semantics}).

%***********************************************************************************
% Context interface
%***********************************************************************************

\subsubsection{Context interface}

The constructors and member functions of the SYCL \codeinline{context} class are listed in Tables~\ref{table.constructors.context} and \ref{table.members.context}, respectively. The additional common special member functions and common member functions are listed in \ref{sec:reference-semantics} in Tables~\ref{table.specialmembers.common.reference} and \ref{table.members.common.reference}, respectively.

All member functions of the \gls{context} class are synchronous and errors are handled by throwing synchronous SYCL exceptions.

All constructors of the SYCL \gls{context} class, excluding the interoperability constructor, will construct either an OpenCL context or a host context, determined by the constructor parameters or, in the case of the default constructor, the SYCL \codeinline{device} produced by the \codeinline{default_selector}. If the SYCL \codeinline{platform} or SYCL \codeinline{device} is a host platform or host device respectively then the constructed SYCL \codeinline{context} is a host context. Subsequently if the constructed SYCL \codeinline{context} is a host context, then the associated SYCL \codeinline{platform} must be a host platform and the constructed SYCL \codeinline{context} must have a single associated SYCL \codeinline{device} that is a host device.

A SYCL \codeinline{context} can optionally be constructed with an \codeinline{async_handler} parameter. In this case the \codeinline{async_handler} provided is passed on to a SYCL \codeinline{queue} to be used to report asynchronous SYCL exceptions.

Information about a SYCL \gls{context} may be queried through the \codeinline{get_info()}
member function.

%Interface for class: context
\lstinputlisting{headers/context.h}

%------------------------------------------------------------------------------------------------------
\startTable{Constructor}
\addFootNotes{Constructors of the SYCL \codeinline{context} class} {table.constructors.context}
  \addRow
  {explicit context(async_handler asyncHandler = \{\})}
  {
    Constructs a SYCL \codeinline{context} instance using an instance of \codeinline{default_selector} to select the associated SYCL \codeinline{platform} and \codeinline{device}(s). One of the SYCL \codeinline{device}s that is associated with this SYCL \codeinline{context} must be the SYCL \codeinline{device} that is produced from the \codeinline{default_selector} instance. The constructed SYCL \codeinline{context} will use the \codeinline{asyncHandler} parameter to handle exceptions.
  }
  \addRowTwoL
  {context(const device \&dev,}
  {async_handler asyncHandler = \{\})}
  {
     Constructs a SYCL \codeinline{context} instance using the \codeinline{dev} parameter as the associated SYCL \codeinline{device} and the SYCL \codeinline{platform} associated with the \codeinline{dev} parameter as the associated SYCL \codeinline{platform}. The constructed SYCL \codeinline{context} will use the \codeinline{asyncHandler} parameter to handle exceptions.
  }
  \addRowTwoL
  {context(const platform \&plt,}
  {async_handler asyncHandler = \{\})}
  {
     Constructs a SYCL \codeinline{context} instance using the \codeinline{plt} parameter as the associated SYCL \codeinline{platform} and the SYCL \codeinline{device}(s) associated with the \codeinline{plt} parameter as the associated SYCL \codeinline{device}(s). The constructed SYCL \codeinline{context} will use the \codeinline{asyncHandler} parameter to handle exceptions.
  }   
  \addRowTwoL
  {context(const vector_class<device> \& deviceList,}
  {async_handler asyncHandler = \{\})}
  {
     Constructs a SYCL \codeinline{context} instance using the SYCL \codeinline{device}(s) in the \codeinline{deviceList} parameter as the associated SYCL \codeinline{device}(s) and the SYCL \codeinline{platform} associated with each SYCL \codeinline{device} in the \codeinline{deviceList} parameter as the associated SYCL \codeinline{platform}. This requires that all SYCL \codeinline{device}s in the \codeinline{deviceList} parameter have the same associated SYCL \codeinline{platform}. The constructed SYCL \codeinline{context} will use the \codeinline{asyncHandler} parameter to handle exceptions.
  } 
  \addRowTwoL
    {context (cl_context clContext,}
    {async_handler asyncHandler = \{\})}
    {    
      Constructs a SYCL \codeinline{context} instance from an OpenCL \codeinline{cl_context} in accordance with the requirements described in \ref{sec:opencl-interoperability}.
    }
\completeTable
%------------------------------------------------------------------------------------------------------

%------------------------------------------------------------------------------------------------------
\startTable{Member function}
\addFootNotes{Member functions of the \codeinline{context} class}
{table.members.context}
  \addRow
    {cl_context get () const}
    { 
      Returns a valid \codeinline{cl_context} instance in accordance with the requirements described in \ref{sec:opencl-interoperability}.  
    }
  \addRow
    {bool is_host () const}
    {
      Returns true if this SYCL \codeinline{context} is a host context.
    }
  \addRowTwoL
    {template <info::context param> typename info::param_traits<info::context, param>::return_type}
    {  get_info() const}
    {
      Queries this SYCL \codeinline{context} for information requested by the template parameter \codeinline{param} using the \codeinline{param_traits} class template to facilitate returning the appropriate type associated with the \codeinline{param} parameter.
    }
  \addRow
    {platform get_platform() const}
    {
      Returns the SYCL \codeinline{platform} that is associated with this SYCL \codeinline{context}. The value returned must be equal to that returned by \codeinline{get_info<info::context::platform>()}.
    }
  \addRowTwoL
    {vector_class<device>}
    {get_devices() const}
    {
      Returns a \codeinline{vector_class} containing all SYCL \codeinline{device}s that are associated with this SYCL \codeinline{context}. The value returned must be equal to that returned by \codeinline{get_info<info::context::devices>()}.
    }
\completeTable
%-------------------------------------------------------------------------------

%*******************************************************************************
% Context information descriptors
%*******************************************************************************

\subsubsection{Context information descriptors}

A SYCL \codeinline{context} can be queried for all of the following information using the
\codeinline{get_info} member function. All SYCL \codeinline{context}s have valid devices for them, including the SYCL host context. The available information is in Table~\ref{table.context.info}. The interface of all available context descriptors in the appendix~\ref{appendix.context.descriptors}.

%-------------------------------------------------------------------------------

\startInfoTableDims{Context Descriptors}{5cm}{4.4cm}{5cm}
\addInfoFootNotes{Context information descriptors}{table.context.info}
\addInfoRow
{info::context::reference_count}
{cl_uint}
{
  Returns the reference count of the underlying OpenCL \codeinline{cl_context} if this SYCL \codeinline{context} is an OpenCL context. Returns \codeinline{0} if this SYCL \codeinline{context} is a host context.
}
\addInfoRow
{info::context::platform}
{platform}
{
  Returns the SYCL \codeinline{platform} associated with this SYCL \codeinline{context}.
}
\addInfoRow
{info::context::devices}
{vector_class<device>}
{
  Returns a \codeinline{vector_class} containing the SYCL \codeinline{device}s associated with this SYCL \codeinline{context}.
}
\completeInfoTable

%------------------------------------------------------------------------------------------------------

%***********************************************************************************
% Device class
%***********************************************************************************
% \subsection{Device class}
% \label{sec:interface.device.class}
% Copyright (c) 2011-2020 Khronos Group, Inc.
%
% This work is licensed under a Creative Commons Attribution 4.0
% International License.
% http://creativecommons.org/licenses/by/4.0/

% !TEX root = sycl-1.2.1.tex

%***********************************************************************************
% Device class
%***********************************************************************************

\subsection{Device class}

The SYCL \codeinline{device} class encapsulates a single SYCL device on which kernels may be executed. A SYCL device may be an OpenCL device in which case it must encapsulate a valid underlying OpenCL \codeinline{cl_device_id}, or it may be a SYCL host device in which case it must not.

All member functions of the \codeinline{device} class are synchronous and errors are handled by throwing synchronous SYCL exceptions.

The default constructor of the SYCL \codeinline{device} class will construct a
host device. The explicit constructor of the SYCL \codeinline{device} class
which takes a \codeinline{device_selector} will construct a host device if
\codeinline{select_device} returns a host device, otherwise will construct an
OpenCL device. The OpenCL interop constructor of the SYCL \codeinline{device}
class will construct an OpenCL device.

A SYCL \codeinline{device} can be partitioned into multiple SYCL devices, by calling the \codeinline{create_sub_devices()} member function template. The resulting SYCL \codeinline{device}s are considered sub devices, and it is valid to partition these sub devices further. The range of support for this feature is implementation defined and can be queried for through \codeinline{get_info()}.

For convenience there are member functions that check the device type. The member function \codeinline{is_host()} returns true if the SYCL \codeinline{device} is a host device and the member functions \codeinline{is_cpu()}, \codeinline{is_gpu()} and \codeinline{is_accelerator()} return true if the device type is \codeinline{info::device_type::cpu}, \codeinline{info::device_type::gpu} or \codeinline{info::device_type::accelerator} respectively.

The SYCL \codeinline{device} class provides the common reference semantics
(see Section~\ref{sec:reference-semantics}).

\subsubsection{Device interface}

A synopsis of the SYCL \codeinline{device} class is provided below. The constructors, member functions and static member functions of the SYCL \codeinline{device} class are listed in Tables~\ref{table.constructors.device}, \ref{table.members.device} and \ref{table.staticmembers.device} respectively. The additional common special member functions and common member functions are listed in \ref{sec:reference-semantics} in Tables~\ref{table.specialmembers.common.reference} and \ref{table.hiddenfriends.common.reference}, respectively.

% Interface of the device class
\lstinputlistingSkipLicense{headers/device.h}

%-------------------------------------------------------------------------------
\startTable{Constructor}
    \addFootNotes{Constructors of the SYCL \codeinline{device} class}{table.constructors.device}
  \addRow
    {device()}
    {
      Constructs a SYCL \codeinline{device} instance as a host device.
    }
  \addRow
    {explicit device(const device_selector \&deviceSelector)}
    {
      Constructs a SYCL \codeinline{device} instance using the device selected by the \codeinline{deviceSelector} provided.
    }
  \addRow
    {explicit device(cl_device_id deviceId)}
    {    
     Constructs a SYCL \codeinline{device} instance from an OpenCL \codeinline{cl_device_id} in accordance with the requirements described in \ref{sec:opencl-interoperability}.
    }
\completeTable
%-------------------------------------------------------------------------------

%-------------------------------------------------------------------------------
\startTable{Member function}
  \addFootNotes{Member functions of the SYCL \codeinline{device} class}{table.members.device}
  \addRow
    {cl_device_id get() const}
    { 
      Returns a valid \codeinline{cl_device_id} instance in accordance with the requirements described in \ref{sec:opencl-interoperability}.
    }
  \addRow
   {platform get_platform() const}
   {
     Returns the associated SYCL \codeinline{platform}. If this SYCL \codeinline{device} is an OpenCL device then the SYCL \codeinline{platform} must encapsulate the OpenCL \codeinline{cl_plaform_id} associated with the underlying OpenCL \codeinline{cl_device_id} of this SYCL \codeinline{device}. If this SYCL \codeinline{device} is a host device then the SYCL \codeinline{platform} must be a host platform. The value returned must be equal to that returned by \codeinline{get_info<info::device::platform>()}.
   }
  \addRow
    {bool is_host() const}
    {
      Returns true if this SYCL \codeinline{device} is a host device.
    }
   \addRow
    {bool is_cpu() const}
    {
      Returns true if this SYCL \codeinline{device} is an OpenCL device and the device type is \codeinline{info::device_type::cpu}.
    }
   \addRow
    {bool is_gpu() const}
    {
      Returns true if this SYCL \codeinline{device} is an OpenCL device and the device type is \codeinline{info::device_type::gpu}.
    }
   \addRow
    {bool is_accelerator() const}
    {
      Returns true if this SYCL \codeinline{device} is an OpenCL device and the device type is \codeinline{info::device_type::accelerator}.
    }
  \addRowTwoL
    {template <info::device param> typename info::param_traits<info::device, param>::return_type}
    {  get_info() const}
    {
      Queries this SYCL \codeinline{device} for information requested by the template parameter \codeinline{param}.
      Specializations of \codeinline{info::param_traits} must be defined in accordance with the info parameters in Table~\ref{table.device.info} to facilitate returning the type associated with the \codeinline{param} parameter.
    }
  \addRow
    {bool has_extension (const string_class \&extension) const}
    {
      Returns true if this SYCL \codeinline{device} supports the extension queried by the \codeinline{extension} parameter.     
    }
  \addRowThreeSL
    {template <info::partition_property prop>}
    {vector_class<device> create_sub_devices(}
    { size_t nbSubDev) const}
    {
      Available only when prop is \codeinline{info::partition_property::partition_equally}.
      Returns a vector_class of sub devices partitioned from this SYCL \codeinline{device} equally based on the \codeinline{nbSubDev} parameter.   
      If this SYCL \codeinline{device} does not support \codeinline{info::partition_property::partition_equally} a \codeinline{feature_not_supported} exception must be thrown.
    }
  \addRowThreeSL
    {template <info::partition_property prop>}
    {vector_class<device> create_sub_devices(}
    { const vector_class<size_t> \&counts) const}
    {
      Available only when prop is
      \codeinline{info::partition_property::partition_by_count}.
      Returns a vector_class of sub devices partitioned from this SYCL \codeinline{device} by count sizes based on the \codeinline{counts} parameter.      
      If the SYCL \codeinline{device} does not support \codeinline{info::partition_property::partition_by_count} a \codeinline{feature_not_supported} exception must be thrown.
    }
  \addRowThreeSL
    {template <info::partition_property prop>}
    {vector_class<device> create_sub_devices(}
    { info::affinity_domain affinityDomain) const}
    {
      Available only when prop is
      \codeinline{info::partition_property::partition_by_affinity_domain}.
      Returns a vector_class of sub devices partitioned from this SYCL \codeinline{device} by affinity domain based on the \codeinline{affinityDomain} parameter.
      Partitions the device into sub devices based upon the affinity domain.
      If the SYCL \codeinline{device} does not support \codeinline{info::partition_property::partition_by_affinity_domain} or the SYCL \codeinline{device} does not support \codeinline{info::affinity_domain} provided a \codeinline{feature_not_supported} exception must be thrown.
    }
\completeTable
%-------------------------------------------------------------------------------

%-------------------------------------------------------------------------------
\startTable{Static member function}
  \addFootNotes{Static member functions of the SYCL \codeinline{device} class}{table.staticmembers.device}
  \addRowFourL
   {static vector_class<device>}
   {  get_devices(}
   {  info::device_type deviceType = }
   {  info::device_type::all)}
   {
     Returns a \codeinline{vector_class} containing all SYCL \codeinline{device}s available in the system of the device type specified by the parameter \codeinline{deviceType}. The returned \codeinline{vector_class} must contain at least a SYCL \codeinline{device} that is a host device if the \codeinline{deviceType} is \codeinline{info::device_type::all}, or a single host device if the \codeinline{deviceType} is \codeinline{info::device_type::host}.
   }
\completeTable
%-------------------------------------------------------------------------------

%***********************************************************************************
% Device information descriptors
%***********************************************************************************
\subsubsection{Device information descriptors}

A SYCL \codeinline{device} can be queried for all of the following information using the \codeinline{get_info} member function. All SYCL \codeinline{device}s must have valid values for every query, including a host device. The information that can be queried is described in Table~\ref{table.device.info}. The interface for all information types and enumerations are described in appendix~\ref{appendix.device.descriptors}.

\fixme{UPDATE: info::device::address_bits changed from unsigned int to cl_uint.}
\fixme{info tables consistency changes. Table description changed to Device
information descriptors.}

%-------------------------------------------------------------------------------
\startInfoTableDims{Device descriptors}{5cm}{2.5cm}{6.5cm}
\addInfoFootNotes{Device information descriptors}{table.device.info}
    \addInfoRow
      {info::device::device_type} {info::device_type}
      {
        Returns the device type. Must not return \codeinline{info::device_type::all}.
        }

    \addInfoRow
      {info::device::vendor_id}
      {cl_uint}
      {
        Returns a unique vendor device identifier. An example of a unique
        device identifier could be the PCIe ID.
     }

    \addInfoRow
      {info::device::max_compute_units} {cl_uint}
      {
         Returns the number of parallel compute units available. 
         The minimum value is 1.
         }

    \addInfoRow
       {info::device::max_work_item_dimensions } {cl_uint}
       {
         Returns the maximum dimensions that specify the global and local work-item IDs used by the data parallel execution model.
         The minimum value is 3 if this SYCL \codeinline{device} is not of device type \codeinline{info::device_type::custom}.
         }

     \addInfoRow
       {info::device::max_work_item_sizes} {id<3>}
       {
         Returns the maximum number of work-items that are permitted in each
           dimension of the work-group of the \codeinline{nd_range}. The minimum value
           is $(1, 1, 1)$ for \codeinline{device}s that are not of device type
           \codeinline{info::device_type::custom}.
       }

     \addInfoRow
       {info::device::max_work_group_size}
       {size_t }
       {
         Returns the maximum number of work-items that are permitted in a work-group executing a kernel on a single compute unit.
         The minimum value is 1.
       }

     \addInfoRow
       {
         info::device::preferred_vector_width_char
         info::device::preferred_vector_width_short
         info::device::preferred_vector_width_int
         info::device::preferred_vector_width_long
         info::device::preferred_vector_width_float
         info::device::preferred_vector_width_double
         info::device::preferred_vector_width_half
       }
       {cl_uint}
       {
         Returns the preferred native vector width size for built-in scalar types that can be put into vectors. The vector width is defined as the number of scalar elements that can be stored in the vector. Must return 0 for \codeinline{info::device::preferred_vector_width_double} if the \codeinline{cl_khr_fp64} extension is not supported by this SYCL \codeinline{device} and must return 0 for  \codeinline{info::device::preferred_vector_width_half} if the \codeinline{cl_khr_fp16} extension is not supported by this SYCL \codeinline{device}.
       }
     \addInfoRow
    {
         info::device::native_vector_width_char
         info::device::native_vector_width_short
         info::device::native_vector_width_int
         info::device::native_vector_width_long
         info::device::native_vector_width_float
         info::device::native_vector_width_double
         info::device::native_vector_width_half
       }
       {cl_uint}
       {
    Returns the native ISA vector width. The vector width is defined as the number of scalar elements that can be stored in the vector. Must return 0 for \codeinline{info::device::preferred_vector_width_double} if the \codeinline{cl_khr_fp64} extension is not supported by this SYCL \codeinline{device} and must return 0 for  \codeinline{info::device::preferred_vector_width_half} if the \codeinline{cl_khr_fp16} extension is not supported by this SYCL \codeinline{device}.
      }
   \addInfoRow
     {info::device::max_clock_frequency}
     {cl_uint}
     {
         Returns the maximum configured clock frequency of this SYCL \codeinline{device} in MHz.
     }

   \addInfoRow
     {info::device::address_bits} {cl_uint}
     {
   Returns the default compute device address space size specified as an unsigned integer value in bits. Must return either 32 or 64.
     }

    \addInfoRow
     {info::device::max_mem_alloc_size}
     {cl_ulong}
     {
      Returns the maximum size of memory object allocation in  bytes. The minimum value is max (1/4th of \codeinline{info::device::global_mem_size},128*1024*1024) if this SYCL \codeinline{device} is not of device type \codeinline{info::device_type::custom}.
     }
   \addInfoRow
     {info::device::image_support}
     {bool}
     {
       Returns true if images are supported by this SYCL \codeinline{device} and
       false if they are not. If this SYCL \codeinline{device} is a
       \gls{host-device}, images must be supported, and therefore this query
       must always return true.
     }
   \addInfoRow
     {info::device::max_read_image_args}
     {cl_uint}
     {
      Returns the maximum number of simultaneous image objects that can be read from by a kernel. The minimum value is 128 if \codeinline{info::device::image_support} returns true for this SYCL \codeinline{device}.
        }

   \addInfoRow
     {info::device::max_write_image_args}
     {cl_uint}
     {
    Returns the maximum number of simultaneous image objects that can be written to by a kernel. The minimum value is 8 if \codeinline{info::device::image_support} returns true for this SYCL \codeinline{device}.
     }

   \addInfoRow
     {info::device::image2d_max_width}
     {size_t}
     {
    Returns the maximum width of a 2D image or 1D image in pixels. The minimum value is 8192 if \codeinline{info::device::image_support} returns true for this SYCL \codeinline{device}.
     }

   \addInfoRow
     {info::device::image2d_max_height}
     {size_t}
     {
    Returns the maximum height of a 2D image in pixels. The minimum value is 8192 if \codeinline{info::device::image_support} returns true for this SYCL \codeinline{device}.
     }

   \addInfoRow
     {info::device::image3d_max_width}
     {size_t}
     {
     Returns the maximum width of a 3D image in pixels. The minimum value is 2048 if \codeinline{info::device::image_support} returns true for this SYCL \codeinline{device}.
     }

   \addInfoRow
     {info::device::image3d_max_height}
     {size_t}
     {
     Returns the maximum height of a 3D image in pixels. The minimum value is 2048 if \codeinline{info::device::image_support} returns true for this SYCL \codeinline{device}.
     }

  \addInfoRow
  {info::device::image3d_max_depth}
  {size_t}
  {
    Returns the maximum depth of a 3D image in pixels. The minimum value is 2048 if \codeinline{info::device::image_support} returns true for this SYCL \codeinline{device}.
  }
  
  \addInfoRow
  {info::device::image_max_buffer_size}
  {size_t}
  {
      Returns the number of pixels for a 1D image created from a buffer object. The minimum value is 65536 if \codeinline{info::device::image_support} if \codeinline{info::device::image_support} returns true for this SYCL \codeinline{device}. Note that this information is intended for OpenCL interoperability only as this feature is not supported in SYCL.
  }

  \addInfoRow
  {info::device::image_max_array_size}
  {size_t}
  {
   Returns the maximum number of images in a 1D or 2D image array. The minimum value is 2048 if \codeinline{info::device::image_support} returns true for this SYCL \codeinline{device}.
   }

  \addInfoRow
  {info::device::max_samplers}
  {cl_uint}
  {
   Returns the maximum number of samplers that can be used in a kernel.  The minimum value is 16 if \codeinline{info::device::image_support} returns true for this SYCL \codeinline{device}.
    }

  \addInfoRow
  {info::device::max_parameter_size}
  {size_t}
  {
   Returns the maximum size in bytes of the arguments that can be passed to a kernel. The minimum value is 1024 if this SYCL \codeinline{device} is not of device type \codeinline{info::device_type::custom}. For this minimum value, only a maximum of 128 arguments can be passed to a kernel.
    }

  \addInfoRow
  {info::device::mem_base_addr_align}
  {cl_uint}
  {
   Returns the minimum value in bits of the largest supported SYCL built-in
    data type if this SYCL \codeinline{device} is not of device type \codeinline{info::device_type::custom}.
    }

  \addInfoRow
  {info::device::half_fp_config}
  {vector_class<info::fp_config>}
  {
    Returns a \codeinline{vector_class} of \codeinline{info::fp_config} describing the half precision floating-point capability of this SYCL \codeinline{device}. The \codeinline{vector_class} may contain zero or more of the following values:
  \begin{itemize}
  \item \codeinline{info::fp_config::denorm}: denorms are supported.
  \item \codeinline{info::fp_config::inf_nan}: INF and quiet NaNs are
  supported.
  \item \codeinline{info::fp_config::round_to_nearest}: round to
  nearest even rounding mode is supported.
  \item \codeinline{info::fp_config::round_to_zero}: round to zero
  rounding mode is supported.
  \item \codeinline{info::fp_config::round_to_inf}: round to positive
  and negative infinity rounding modes are supported.
  \item \codeinline{info::fp_config::fma}: IEEE754-2008 fused multiply add
  is supported.
  \item \codeinline{info::fp_config::correctly_rounded_divide_sqrt}: divide and sqrt
        are correctly rounded as defined by the IEEE754 specification.
    \item \codeinline{info::fp_config::soft_float}: basic floating-point operations
        (such as addition, subtraction, multiplication) are implemented in software.
   \end{itemize}
    If half precision is supported by this SYCL \codeinline{device} (i.e. the \codeinline{cl_khr_fp16} extension is supported) there is no minimum floating-point capability. If half support is not supported the returned \codeinline{vector_class} must be empty.
  }

  \addInfoRow
  {info::device::single_fp_config}
  {vector_class<info::fp_config>}
  {
    Returns a vector_class of \codeinline{info::fp_config} describing the single precision floating-point capability of this SYCL \codeinline{device}. The \codeinline{vector_class} must contain one or more of the following values:
  \begin{itemize}
  \item \codeinline{info::fp_config::denorm}: denorms are supported.
  \item \codeinline{info::fp_config::inf_nan}: INF and quiet NaNs are
  supported.
  \item \codeinline{info::fp_config::round_to_nearest}: round to
  nearest even rounding mode is supported.
  \item \codeinline{info::fp_config::round_to_zero}: round to zero
  rounding mode is supported.
  \item \codeinline{info::fp_config::round_to_inf}: round to positive
  and negative infinity rounding modes are supported.
  \item \codeinline{info::fp_config::fma}: IEEE754-2008 fused multiply add
  is supported.
  \item \codeinline{info::fp_config::correctly_rounded_divide_sqrt}: divide and sqrt
        are correctly rounded as defined by the IEEE754 specification.
    \item \codeinline{info::fp_config::soft_float}: basic floating-point operations
        (such as addition, subtraction, multiplication) are implemented in software.
   \end{itemize}
    If this SYCL \codeinline{device} is not of type \codeinline{info::device_type::custom} then the minimum floating-point capability must be:
    \codeinline{info::fp_config::round_to_nearest} and \codeinline{info::fp_config::inf_nan}.
  }

  \addInfoRow
  {info::device::double_fp_config}
  {vector_class<info::fp_config>}
  {
  Returns a vector_class of \codeinline{info::fp_config} describing the double precision floating-point capability of this SYCL \codeinline{device}. The \codeinline{vector_class} may contain zero or more of the following values:
  \begin{itemize}
    \item \codeinline{info::fp_config::denorm}: denorms are
    supported.
    \item \codeinline{info::fp_config::inf_nan}: INF and NaNs are
    supported.
    \item \codeinline{info::fp_config::round_to_nearest}: round to
    nearest even rounding mode is supported.
    \item \codeinline{info::fp_config::round_to_zero}: round to
    zero rounding mode is supported.
    \item \codeinline{info::fp_config::round_to_inf}: round to
    positive and negative infinity rounding modes are supported.
    \item \codeinline{info::fp_config::fma}: IEEE754-2008 fused
    multiply-add is supported.
    \item \codeinline{info::fp_config::soft_float}: basic
    floating-point operations (such as addition, subtraction, multiplication) are implemented in software.
  \end{itemize}
  If double precision is supported by this SYCL \codeinline{device} (i.e. the \codeinline{cl_khr_fp64} extension is supported) and this SYCL \codeinline{device} is not of type \codeinline{info::device_type::custom} then the minimum floating-point capability must be:
  \codeinline{info::fp_config::fma}, \codeinline{info::fp_config::round_to_nearest}, \codeinline{info::fp_config::round_to_zero}, \codeinline{info::fp_config::round_to_inf}, \codeinline{info::fp_config::inf_nan} and \codeinline{info::fp_config::denorm}. If double support is not supported the returned \codeinline{vector_class} must be empty.
  }

  \addInfoRow
  {info::device::global_mem_cache_type}
  {info::global_mem_cache\-_type}
  { 
    Returns the type of global memory cache supported.
  }

  \addInfoRow
  {info::device::global_mem_cache_line_size}
  {cl_uint}
  {
    Returns the size of global memory cache line in bytes.
  }

  \addInfoRow
  {info::device::global_mem_cache_size}
  {cl_ulong}
  {Returns the size of global memory cache in bytes.}

  \addInfoRow
  {info::device::global_mem_size}
  {cl_ulong}
  {Returns the size of global device memory in bytes.}

  \addInfoRow
  {info::device::max_constant_buffer_size}
  {cl_ulong}
  {
  Returns the maximum size in bytes of a constant buffer allocation. The minimum value is 64 KB if this SYCL \codeinline{device} is not of type \codeinline{info::device_type::custom}.
  }

  \addInfoRow
  {info::device::max_constant_args}
  {cl_uint}
  {
  Returns the maximum number of constant arguments that can be declared in a kernel. The minimum value is 8 if this SYCL \codeinline{device} is not of type \codeinline{info::device_type::custom}.
  }

  \addInfoRow
  {info::device::local_mem_type}
  {info::local_mem_type}
  {
   Returns the type of local memory supported. This can
  be \codeinline{info::local_mem_type::local} implying dedicated
  local memory storage such as SRAM, or \codeinline{info::local_mem_type::global}.
  If this SYCL \codeinline{device} is of type \codeinline{info::device_type::custom} this can also be \codeinline{info::local_mem_type::none}, indicating local memory is not supported.
  }

  \addInfoRow
  {info::device::local_mem_size}
  {cl_ulong}
  {
  Returns the size of local memory arena in bytes. The minimum value is 32 KB if this SYCL \codeinline{device} is not of type \codeinline{info::device_type::custom}.
  }

  \addInfoRow
  {info::device::error_correction_support}
  {bool}
  {
   Returns true if the device implements error correction for all accesses to
  compute device memory (global and constant). Returns false if the device does
  not implement such error correction.
  }

  \addInfoRow
  {info::device::host_unified_memory}
  {bool}
  {
   Returns true if the device and the host have a unified memory subsystem and
  returns false otherwise.
  }

  \addInfoRow
  {info::device::profiling_timer_resolution}
  {size_t}
  {
  Returns the resolution of device timer in nanoseconds.
  }

  \addInfoRow
  {info::device::is_endian_little}
  {bool}
  {
  Returns true if this SYCL \codeinline{device} is a little endian device and returns false otherwise.
  }

  \addInfoRow
  {info::device::is_available}
  {bool}
  {
  Returns true if the SYCL \codeinline{device} is available and returns false if the device is not
  available.
  }

  \addInfoRow
  {info::device::is_compiler_available}
  {bool}
  {
  Returns false if the implementation does not have a compiler available to
  compile the program source. An OpenCL device that conforms to the OpenCL Embedded Profile may not have an online compiler available.
  }

  \addInfoRow
  {info::device::is_linker_available}
  {bool}
  {
  Returns false if the implementation does not have a linker available.  An
  OpenCL device that conforms to the OpenCL Embedded Profile may not have a linker
  available. However, it needs to be true if \codeinline{info::device::is_compiler_available} returns true for this SYCL \codeinline{device}.
  }
  
    \addInfoRow
  {info::device::execution_capabilities}
  {vector_class<info::execution_\-capability>}
  {
    Returns a \codeinline{vector_class} of the \codeinline{info::execution_capability} describing the supported execution capabilities.
    Note that this information is intended for OpenCL interoperability only as  SYCL only supports \codeinline{info::execution_capability::exec_kernel}.
    }
    
    \addInfoRow
    {info::device::queue_profiling}
    {bool}
    {
      Returns true if this \codeinline{device} supports queue profiling.
    }

  \addInfoRow
  {info::device::built_in_kernels}
  {vector_class<string_class>}
  { Returns a vector_class of built-in OpenCL
  kernels supported by this SYCL \codeinline{device}.
  }

  \addInfoRow
  {info::device::platform}
  {platform}
  {Returns the SYCL \codeinline{platform} associated with this SYCL \codeinline{device}.}

  \addInfoRow
  {info::device::name}
  {string_class}
  {Returns the device name of this SYCL \codeinline{device}.}

  \addInfoRow
  {info::device::vendor}
  {string_class}
  {Returns the vendor of this SYCL \codeinline{device}.}

  \addInfoRow
  {info::device::driver_version}
  {string_class}
  { Returns the OpenCL software driver version as a \codeinline{string_class} in the form: major_number.minor_number, if this SYCL \codeinline{device} is an OpenCL device. Must return a string_class with the value \codeinline{"1.2"} if this SYCL \codeinline{device} is a host device.
  }

  \addInfoRow
  {info::device::profile}
  {string_class}
  {
  Returns the OpenCL profile as a \codeinline{string_class}, if this SYCL \codeinline{device} is an OpenCL device. The value returned can be one of the following strings:
  \begin{itemize}
    \item FULL_PROFILE - if the device supports
    the OpenCL specification (functionality
    defined as part of the core specification and
    does not require any extensions to be
    supported).
    \item EMBEDDED_PROFILE - if the device
    supports the OpenCL embedded profile.
  \end{itemize}
  Must return a \codeinline{string_class} with the value "FULL PROFILE" if this is a host device.
  }

  \addInfoRow
  {info::device::version}
  {string_class}
  {
   Returns the SYCL version as a \codeinline{string_class} in the form:
  \codeinline{<major_version>.<minor_version>}.
  If this SYCL \codeinline{device} is a host device, the <major_version>.<minor_version> value returned must be \codeinline{"1.2"}.
  }
  
  \addInfoRow
  {info::device::opencl_c_version}
  {string_class}
  {
      Returns a \codeinline{string_class} describing the OpenCL C version that is supported by the OpenCL C compiler of this \codeinline{device}.
      Note that this information is intended for OpenCL interoperability only as SYCL kernel functions are compiled offline.
    }

  \addInfoRow
  {info::device::extensions}
  {vector_class<string_class>}
  {
         Returns a \codeinline{vector_class} of extension names (the extension names
         do not contain any spaces) supported by this SYCL \codeinline{device}. The extension names returned can be vendor supported extension names and one or more of the following Khronos approved extension names:
  \begin{itemize}
    \item \codeinline{cl_khr_int64_base_atomics}
    \item \codeinline{cl_khr_int64_extended_atomics}
    \item \codeinline{cl_khr_3d_image_writes}
    \item \codeinline{cl_khr_fp16}
    \item \codeinline{cl_khr_gl_sharing}
    \item \codeinline{cl_khr_gl_event}
    \item \codeinline{cl_khr_d3d10_sharing}
    \item \codeinline{cl_khr_dx9_media_sharing}
    \item \codeinline{cl_khr_d3d11_sharing}
    \item \codeinline{cl_khr_depth_images}
    \item \codeinline{cl_khr_gl_depth_images}
    \item \codeinline{cl_khr_gl_msaa_sharing}
    \item \codeinline{cl_khr_image2d_from_buffer}
    \item \codeinline{cl_khr_initialize_memory}
    \item \codeinline{cl_khr_context_abort}
    \item \codeinline{cl_khr_spir}
  \end{itemize}
  If this SYCL \codeinline{device} is an OpenCL device then following approved Khronos extension names must be returned by all device that support OpenCL C 1.2:
  \begin{itemize}
    \item \codeinline{cl_khr_global_int32_base_atomics}
    \item \codeinline{cl_khr_global_int32_extended_atomics}
    \item \codeinline{cl_khr_local_int32_base_atomics}
    \item \codeinline{cl_khr_local_int32_extended_atomics}
    \item \codeinline{cl_khr_byte_addressable_store}
    \item \codeinline{cl_khr_fp64} (for backward compatibility if
    double precision is supported)
  \end{itemize}
  Please refer to the OpenCL 1.2 Extension
  Specification for a detailed description of
  these extensions.
  }

  \addInfoRow
  {info::device::printf_buffer_size}
  {size_t}
  {
         Returns the maximum size of the internal buffer that holds the output of printf calls from a kernel. The minimum value is 1 MB if \codeinline{info::device::profile} returns true for this SYCL \codeinline{device}.
  }

  \addInfoRow
  {info::device::preferred_interop_user_sync}
  {bool}
  {
  Returns true if the preference for this SYCL \codeinline{device} is for the user to be responsible for synchronization, when sharing memory objects between OpenCL and other APIs  such as DirectX, false if the device/implementation has a performant path for performing synchronization of memory object shared between OpenCL and other APIs such as DirectX.
  }

  \addInfoRow
  {info::device::parent_device}
  {device}
  {
  Returns the parent SYCL \codeinline{device} to which this sub-device is a child if this is a sub-device.
  Must throw a \codeinline{invalid_object_error} SYCL exception if this SYCL \codeinline{device} is not a sub device.
  }

  \addInfoRow
  {info::device::partition_max_sub_devices}
  {cl_uint}
  {
  Returns the maximum number of subdevices that can be created when this SYCL \codeinline{device} is partitioned. The value returned cannot exceed the value returned by \codeinline{info::device::device_max_compute_units}.
  }

  \addInfoRow
  {info::device::partition_properties}
  {vector_class<info::partition_prop\-erty>}
  {
  Returns the partition properties supported by this SYCL \codeinline{device}; a vector of \codeinline{info::partition_property}. If this SYCL \codeinline{device} cannot be partitioned into at least two sub devices then the returned vector must be empty.
  }

  \addInfoRow
  {info::device::partition_affinity_domains}
  {vector_class<info::parition_affini\-ty_domain>}
  {
  Returns a vector_class of the partition affinity domains supported by this SYCL \codeinline{device} when partitioning with \codeinline{info::partition_property::parition_by_affinity_domain}.
  }

  \addInfoRow
  {info::device::partition_type_property}
  {info::partition_prop\-erty}
  {
  Returns the partition property of this SYCL \codeinline{device}. If this SYCL \codeinline{device} is not a sub device then the the return value must be \codeinline{info::partition_property::no_partition}, otherwise it must be one of the following values:
  \begin{itemize}
    \item info::partition_property::partition_equally
    \item info::partition_property::partition_by_counts
    \item info::partition_property::partition_by_affinity_domain
  \end{itemize}
  }
  
  \addInfoRow
  {info::device::partition_type_affinity_domain}
  {info::partition_affi-nity_domain}
  {
  Returns the partition affinity domain of this SYCL \codeinline{device}. If this SYCL \codeinline{device} is not a sub device or the sub device was not partitioned with \codeinline{info::partition_type::partition_by_affinity_domain} then the the return value must be \codeinline{info::partition_affinity_domain::not_applicable}, otherwise it must be one of the following values:
  \begin{itemize}
    \item info::partition_affinity_domain::numa
    \item info::partition_affinity_domain::L4_cache
    \item info::partition_affinity_domain::L3_cache
    \item info::partition_affinity_domain::L2_cache
    \item info::partition_affinity_domain::L1_cache
    \item info::partition_affinity_domain::next_partitionable
  \end{itemize}
  }

  \addInfoRow
  {info::device::reference_count}
  {cl_uint}
  {
  Returns the device reference count. If the device is not a sub-device the value returned must be 1.
  }
\completeTable


%%% Local Variables:
%%% mode: latex
%%% TeX-master: "sycl-1.2.1"
%%% TeX-auto-untabify: t
%%% TeX-PDF-mode: t
%%% ispell-local-dictionary: "american"
%%% End:


%***********************************************************************************
% Queue class
%***********************************************************************************
\subsection{Queue class}
\label{sec:interface.queue.class}
% Copyright (c) 2011-2020 Khronos Group, Inc.
%
% This work is licensed under a Creative Commons Attribution 4.0
% International License.
% http://creativecommons.org/licenses/by/4.0/

% !TEX root = sycl-1.2.1.tex

%*******************************************************************************
% Queue class
%*******************************************************************************

The SYCL \codeinline{queue} class encapsulates a single SYCL queue which
schedules kernels on a SYCL device. A SYCL queue may be an OpenCL queue in
which case it must encapsulate at least one valid underlying OpenCL
\codeinline{cl_command_queue}, or it may be a SYCL host queue in which case it
must not. The underlying OpenCL \codeinline{cl_command_queue}(s) may execute 
either in-order or out-of-order, however the SYCL \codeinline{queue}
must behave as an out-of-order queue.

A SYCL \codeinline{queue} can be used to submit \glspl{command-group} to be
executed by the \gls{sycl-runtime} using the \codeinline{submit} member
function.

All member functions of the \codeinline{queue} class are synchronous and errors
are handled by throwing synchronous SYCL exceptions. The \codeinline{submit}
member function schedules \glspl{command-group} asynchronously, so any errors
in the submission of a \gls{command-group} are handled by throwing synchronous
SYCL exceptions. Any exceptions from the \gls{command-group} after it has
been submitted are handled by throwing asynchronous SYCL exceptions to an
\gls{async-handler} on calling \codeinline{throw_asynchronous} or \codeinline{
wait_and_throw}, or on destruction of the SYCL \codeinline{queue}, if one was
provided when the SYCL \codeinline{queue} was constructed.

A SYCL \codeinline{queue} can wait for all \glspl{command-group} that it has
submitted by calling \codeinline{wait} or \codeinline{wait_and_throw}.

The default constructor of the SYCL \codeinline{queue} class will construct a
queue based on the SYCL \codeinline{device} returned from the \codeinline{default_selector} (see Section~\ref{sec:device-selector}), therefore
the constructed SYCL \codeinline{queue} could be either a host queue or a device
queue. All other constructors construct a host or device queue, determined by the
parameters provided. All constructors will implicitly construct a SYCL
\codeinline{platform}, \codeinline{device} and \codeinline{context} in order to
facilitate the construction of the queue.

With the exception of the interoperability constructor, each constructor takes as the last
parameter an optional SYCL \codeinline{property_list} to provide properties to
the SYCL \codeinline{queue}.

The SYCL \codeinline{queue} class provides the common reference semantics
(see Section~\ref{sec:reference-semantics}).

%*******************************************************************************
% Queue interface
%*******************************************************************************
\subsubsection{Queue interface}

A synopsis of the SYCL \codeinline{queue} class is provided below. The
constructors and member functions of the SYCL \codeinline{queue} class are
listed in Tables~\ref{table.constructors.queue} and \ref{table.members.queue}
respectively. The additional common special member functions and common member
functions are listed in \ref{sec:reference-semantics} in
Tables~\ref{table.specialmembers.common.reference} and
\ref{table.hiddenfriends.common.reference}, respectively.

%Interface for class: queue
\lstinputlistingSkipLicense{headers/queue.h}

%-------------------------------------------------------------------------------
\startTable{Constructor}
  \addFootNotes{Constructors of the \codeinline{queue} class}
  {table.constructors.queue}
  \addRow
    { explicit queue(const property_list \&propList = \{\}) }
    {
      Constructs a SYCL \codeinline{queue} instance using the device returned by
      an instance of \codeinline{default_selector}. Zero or more properties can
      be provided to the constructed SYCL \codeinline{queue} via an instance of
      \codeinline{property_list}.
    }
  \addRowTwoL
    { explicit queue(const async_handler \&asyncHandler, }
    { const property_list \&propList = \{\}) }
    {
      Constructs a SYCL \codeinline{queue} instance with an \codeinline{
      async_handler} using the device returned by an instance of \codeinline{
      default_selector}. Zero or more properties can be provided to the
      constructed SYCL \codeinline{queue} via an instance of \codeinline{
      property_list}.
    }
  \addRowTwoL
    { explicit queue(const device_selector \&deviceSelector, }
    { const property_list \&propList = \{\}) }
    {
      Constructs a SYCL \codeinline{queue} instance using the device returned by
      the \codeinline{deviceSelector} provided. Zero or more properties can be
      provided to the constructed SYCL \codeinline{queue} via an instance of
      \codeinline{property_list}.
    }
  \addRowThreeL
    { queue(const device_selector \&deviceSelector, }
    { const async_handler \&asyncHandler, }
    { const property_list \&propList = \{\}) }
    {
      Constructs a SYCL \codeinline{queue} instance with an \codeinline{
      async_handler} using the device returned by the \codeinline{deviceSelector
      } provided. Zero or more properties can be provided to the constructed
      SYCL \codeinline{queue} via an instance of \codeinline{property_list}.
    }
  \addRowTwoL
    { explicit queue(const device \&syclDevice, }
    { const property_list \&propList = \{\}) }
    {
      Constructs a SYCL \codeinline{queue} instance using the \codeinline{
      syclDevice} provided. Zero or more properties can be provided to the
      constructed SYCL \codeinline{queue} via an instance of \codeinline{
      property_list}.
    }
  \addRowThreeL
    { explicit queue(const device \&syclDevice, }
    { const async_handler \&asyncHandler, }
    { const property_list \&propList = \{\}) }
    {
      Constructs a SYCL \codeinline{queue} instance with an \codeinline{
      async_handler} using the \codeinline{syclDevice} provided. Zero or more
      properties can be provided to the constructed SYCL \codeinline{queue}
      via an instance of \codeinline{property_list}.
    }
  \addRowThreeL
    { explicit queue(const context \&syclContext, }
    { const device_selector \&deviceSelector, }
    { const property_list \&propList = \{\}) }
    {
      Constructs a SYCL \codeinline{queue} instance that is associated with the
      \codeinline{syclContext} provided, using the device returned by the
      \codeinline{deviceSelector} provided. Must throw an
      \codeinline{invalid_object_error} SYCL exception if
      \codeinline{syclContext} does not encapsulate the SYCL
      \codeinline{device} returned by \codeinline{deviceSelector}. Zero or more
      properties can be provided to the constructed SYCL \codeinline{queue}
      via an instance of \codeinline{property_list}.
    }
  \addRowFourL
    { explicit queue(const context \&syclContext, }
    { const device_selector \&deviceSelector, }
    { const async_handler \&asyncHandler, }
    { const property_list \&propList = \{\}) }
    {
      Constructs a SYCL \codeinline{queue} instance with an \codeinline{
      async_handler} that is associated with the \codeinline{syclContext}
      provided, using the device returned by the \codeinline{deviceSelector}
      provided. Must throw an \codeinline{invalid_object_error} SYCL exception
      if \codeinline{syclContext} does not encapsulate the SYCL
      \codeinline{device} returned by \codeinline{deviceSelector}. Zero or more
      properties can be provided to the constructed SYCL \codeinline{queue} via
      an instance of \codeinline{property_list}.
    }
  \addRowThreeL
    { explicit queue(const context \&syclContext, }
    { const device \&syclDevice, }
    { const property_list \&propList = \{\}) }
    {
      Constructs a SYCL \codeinline{queue} instance using the \codeinline{
      syclDevice} provided, and associated with the
      \codeinline{syclContext} provided. Must throw an
      \codeinline{invalid_object_error} SYCL exception if
      \codeinline{syclContext} does not encapsulate the SYCL
      \codeinline{device} \codeinline{syclDevice}. Zero or more
      properties can be provided to the constructed SYCL \codeinline{queue}
      via an instance of \codeinline{property_list}.
    }
  \addRowFourL
    { explicit queue(const context \&syclContext, }
    { const device \&syclDevice, }
    { const async_handler \&asyncHandler, }
    { const property_list \&propList = \{\}) }
    {
      Constructs a SYCL \codeinline{queue} instance with an \codeinline{
      async_handler} using the \codeinline{syclDevice} provided, and
      associated with the \codeinline{syclContext} provided. Must throw an
      \codeinline{invalid_object_error} SYCL exception if
      \codeinline{syclContext} does not encapsulate the SYCL
      \codeinline{device} \codeinline{syclDevice}.  Zero or more
      properties can be provided to the constructed SYCL \codeinline{queue} via
      an instance of \codeinline{property_list}.
    }
  \addRowThreeL
    { explicit queue(cl_command_queue clQueue,}
    { const context \&syclContext, }
    { const async_handler \&asyncHandler = \{\}) }
    {
      Constructs a SYCL \codeinline{queue} instance with an optional
      \codeinline{async_handler} from an OpenCL \codeinline{cl_command_queue}
      in accordance with the requirements described
      in~\ref{sec:opencl-interoperability}.
    }
\completeInfoTable
%-------------------------------------------------------------------------------

%-------------------------------------------------------------------------------
\startTable{Member function}
  \addFootNotes{Member functions for class queue}{table.members.queue}
  \addRow
    {cl_command_queue get() const}
    {   
      Returns a valid \codeinline{cl_command_queue} instance in accordance with the requirements described in \ref{sec:opencl-interoperability}.    
    }
  \addRow
    {context get_context () const}
    {
      Returns the SYCL queue's context.
      Reports errors using SYCL exception classes.
      The value returned must be equal to that returned by \codeinline{get_info<info::queue::context>()}.
    }
  \addRow
    {device get_device () const}
    {
      Returns the SYCL device the queue is associated with.
      Reports errors using SYCL exception classes.
      The value returned must be equal to that returned by \codeinline{get_info<info::queue::devices>()}.
    }
  \addRow
   {bool is_host () const}
   {
      Returns whether the queue is executing on a SYCL host device.
   }
  \addRow
    {void wait() }
    {
      Performs a blocking wait for the completion of all enqueued tasks
      in the queue.  Synchronous errors will be reported through SYCL
      exceptions.
    }
  \addRow
    {void wait_and_throw () }
    {
      Performs a blocking wait for the completion of all enqueued tasks
      in the queue.  Synchronous errors will be reported via SYCL
      exceptions. Asynchronous errors will be passed to the
      \gls{async-handler} passed to the queue on
      construction. If no \codeinline{async_handler} was provided then
      asynchronous exceptions will be lost.
    }
  \addRow
    {void throw_asynchronous () }
    {
      Checks to see if any asynchronous errors have been produced by
      the queue and if so reports them by passing them to the
      \gls{async-handler} passed to the queue on
      construction. If no \codeinline{async_handler} was provided then
      asynchronous exceptions will be lost.
    }
  \addRowFourL
    { template <info::queue param> }
    {  typename info::param_traits}
    {  <info::queue, param>::return_type}
    {  get_info ()  const}
    {Queries the platform for \codeinline{cl_command_queue_info}}
  \addRowTwoL
    {template <typename T>}
    {event submit(T cgf)}
    {Submit a \gls{command-group-function-object} to the queue, in order to be scheduled
    for execution on the device.}
  \addRowThreeL
    {template <typename T>}
    {event submit(T cgf,}
    {             queue \& secondaryQueue)}
    {Submit a \gls{command-group-function-object} to the queue, in order to be scheduled
    for execution on the device. On a kernel error, this \gls{command-group-function-object},
    is then scheduled for execution on the secondary queue. Returns an
    event, which corresponds to the queue the \gls{command-group-function-object}
    is being enqueued on.}
\completeTable
%-------------------------------------------------------------------------------

\subsubsection{Queue information descriptors}

A SYCL \gls{queue} can be queried for all of the following information using the
\codeinline{get_info} function. All SYCL queues have valid queries for them,
including the SYCL host queue. The available information is in
Table~\ref{table.queue.info}. The interface of all available device descriptors is
in appendix~\ref{appendix.queue.descriptors}.

\startInfoTable{Queue Descriptors}
\addInfoFootNotes{Queue information descriptors}{table.queue.info}
\addInfoRow
{info::queue::context}
{context}
{
  Returns the SYCL \codeinline{context} associated with this SYCL \codeinline{queue}.
}
\addInfoRow
{info::queue::device}
{device}
{
  Returns the SYCL \gls{device} associated with this SYCL \codeinline{queue}.
}
\addInfoRow
{info::queue::reference_count}
{cl_uint}
{Return the command-queue reference count.}
\completeTable

\subsubsection{Queue Properties}
\label{sec:queue-properties}

The properties that can be provided when constructing the SYCL
\codeinline{queue} class are describe in
Table~\ref{table.properties.queue}.

%---------------------------------------------------------------------
\startTable{Property}
\addFootNotes{Properties supported by the SYCL \codeinline{queue} class} {table.properties.queue}

\addRow
  { property::queue::enable_profiling }
  {
    The \codeinline{enable_profiling} property adds the requirement
    that the \gls{sycl-runtime} must capture profiling information for
    the \glspl{command-group} that are submitted from this SYCL
    \codeinline{queue} and provide said information via the SYCL
    \codeinline{event} class \codeinline{get_profiling_info} member
    function, if the associated SYCL \codeinline{device} supports
    queue profiling (i.e. the \codeinline{
    info::device::queue_profiling} info parameter returns \codeinline{
    true}).
  }
\completeTable
%---------------------------------------------------------------------

The constructors of the queue property classes are listed in Table~\ref{table.constructors.properties.queue}.

%---------------------------------------------------------------------
\startTable{Constructor}
\addFootNotes{Constructors of the queue property classes}
{table.constructors.properties.queue}
\addRow
{property::queue::enable_profiling::enable_profiling()}
{
  Constructs a SYCL \codeinline{enable_profiling} property instance.
}
\completeTable
%---------------------------------------------------------------------

\subsubsection{Queue error handling}
\label{sec:interface.queue.errors}

Queue errors come in two forms:
\begin{itemize}

  \item
    \textbf{Synchronous Errors} are those that we would expect to be
    reported directly at the point of waiting on an event, and hence
    waiting for a queue to complete, as well as any immediate errors
    reported by enqueuing work onto a queue. Such errors are returned
    through exceptions.

  \item
    \textbf{Asynchronous errors} are those that are produced through
    callback functions only. These will be stored within the queue's
    context until they are dispatched to the context's asynchronous
    error handler, the \gls{async-handler}. If a queue is constructed 
    with a user-supplied context, then it is this context's asynchronous error
    handler to which asynchronous errors are reported. If a queue is constructed
    without a user-supplied context, then the queue's constructor
    can be supplied with a queue-specific asynchronous error handler
    which will be used to construct the queue's context.
    To ensure that such errors are
    processed predictably in a known host thread, these errors are only
    passed to the asynchronous error handler on request when either
    \codeinline{wait_and_throw} is called or when
    \codeinline{throw_asynchronous} is called. If no
    asynchronous error handler is passed to the queue or its context
    on construction, then such errors go unhandled, much as they would
    if no callback were passed to an OpenCL context.

\end{itemize}

Note that if there are exceptions to be processed when a queue
using an asynchronous handler is destructed, the handler is called and
this might delay or block the destruction, according to the behavior
of the handler.


%%% Local Variables:
%%% mode: latex
%%% TeX-master: "sycl-1.2.1"
%%% TeX-auto-untabify: t
%%% TeX-PDF-mode: t
%%% ispell-local-dictionary: "american"
%%% End:


%***********************************************************************************
% Event class
%***********************************************************************************
\subsection{Event class}

An \keyword{event} in SYCL abstracts the \codeinline{cl_event} object in OpenCL. In
OpenCL the events mechanism is comprised of low-level event objects that
the developer uses to synchronize memory transfers,
enqueues of kernels and signaling barriers.

In SYCL, events are an abstraction of the OpenCL event objects, but they
retain the features and functionality of the OpenCL event mechanism. They
accommodate synchronization between different contexts, devices and platforms.
It is the responsibility of the SYCL implementation to ensure that when SYCL
events are used in OpenCL queues, the correct synchronization points are
created to allow cross-platform or cross-device synchronization.

Since data management and storage is handled by the \gls{sycl-runtime}, the
\codeinline{event} class is used for providing the appropriate interface for
OpenCL/SYCL interoperability. In the case where SYCL objects contain
OpenCL memory objects created outside of the SYCL mechanism, then events
can be used to provide the \gls{sycl-runtime} with the initial events that it has
to synchronize against. However, the events mechanism does not provide
full interoperability with OpenCL during SYCL code execution.
Interoperability is achieved by using the synchronization rules with the
\codeinline{buffer} and \codeinline{image} classes.

A SYCL event can be constructed from an OpenCL event or can return an OpenCL
event.
A SYCL event can also be returned by the submission of a \gls{command-group}.
The dependencies of the event returned via the submission of the command group
are the implementation-defined actions associated with the \gls{command-group}
execution. 

The SYCL \codeinline{event} class provides the common reference semantics
(see Section~\ref{sec:reference-semantics}).

The constructors and member functions of the SYCL \codeinline{event} class are listed in Tables~\ref{table.constructors.event} and \ref{table.members.event}, respectively. The additional common special member functions and common member functions are listed in Tables~\ref{table.specialmembers.common.reference} and \ref{table.members.common.reference}, respectively.

%Interface for class: event.h
\lstinputlisting{headers/event.h}

%------------------------------------------------------------------------------------------------------
\startTable{Constructor}
\addFootNotes{Constructors of the \codeinline{event} class}
{table.constructors.event}
  \addRow
    {event ()}
    {  
      Constructs a ready SYCL \codeinline{event}. If the constructed SYCL \codeinline{event} is waited on it will complete immediately.
    }
  \addRow
    {event (cl_event clEvent, const context\& syclContext)}
    {  
      Constructs a SYCL \codeinline{event} instance from an OpenCL \codeinline{cl_event} in accordance with the requirements described in \ref{sec:opencl-interoperability}.    
      The \codeinline{syclContext} must match the OpenCL context associated
      with the clEvent.
    }
\completeTable
%------------------------------------------------------------------------------------------------------

%------------------------------------------------------------------------------------------------------
\startTable{Member function}
\addFootNotes{Member functions for the \codeinline{event} class}{table.members.event}
  \addRow
    {cl_event get()}
    {  
      Returns a valid \codeinline{cl_event} instance in accordance with the requirements described in \ref{sec:opencl-interoperability}.      
    }
  \addRow
    {bool is_host() const}
    {  
      Returns true if this SYCL \codeinline{event} is a host event.     
    }
  \addRow
    {vector_class<event> get_wait_list()}
    {
      Return the list of events that this event waits for in
      the dependence graph.  Only direct dependencies are returned,
      and not transitive dependencies that direct dependencies wait on.
      Whether already completed events are included in the returned list
      is implementation defined.
    }
  \addRow
    {void wait()}
    {
      Wait for the event and the command associated with
      it to complete.
    }
  \addRow
    {void wait_and_throw()}
    {
      Wait for the event and the command associated with
      it to complete.

      If any uncaught asynchronous errors
      occurred on the context (or contexts) that the event
      is waiting on executions from, then will also call
      that context's asynchronous error handler with those
      errors.
    }
  \addRowTwoL
    {static void wait(}
    {  const vector_class<event> \&eventList)}
    {
      Synchronously wait on a list of events.
    }
  \addRowTwoL
    {static void wait_and_throw(}
    {const vector_class<event> \&eventList)}
    {
      Synchronously wait on a list of events.
      If any uncaught asynchronous errors occurred on
      the context (or contexts) that the events are waiting
      on executions from, then will also call those contexts'
      asynchronous error handlers with those errors.
    }

  \addRowFourL
    {template <info::event param>}
    {  typename info::param_traits}
    {  <info::event, param>::return_type}
    {  get_info() const }
    {
      Queries this SYCL \codeinline{event} for information requested by the
      template parameter \codeinline{param}.
      Specializations of \codeinline{info::param_traits} must be defined in
      accordance with the info parameters in Table~\ref{table.event.info} to
      facilitate returning the type associated with the \codeinline{param}
      parameter.
    }
  \addRowFourL
    {template <info::event_profiling param>}
    {  typename info::param_traits}
    {  <info::event_profiling, param>::return_type}
    {  get_profiling_info () const }
    {
      Queries this SYCL \codeinline{event} for profiling information requested
      by the parameter \codeinline{param}. If the requested profiling information
      is unavailable when \codeinline{get_profiling_info} is called due to
      incompletion of \glspl{command-group} associated with the \codeinline{event},
      then the call to \codeinline{get_profiling_info} will block until
      the requested profiling information is available. An example is asking for
      \codeinline{info::event_profiling::command_end} when the associated
      \gls{command-group} has yet to finish execution.
      Calls to \codeinline{get_profiling_info} must throw an 
      \codeinline{invalid_object_error} SYCL exception if the SYCL 
      \codeinline{queue} which submitted the \gls{command-group} this 
      SYCL \codeinline{event} is associated with was not constructed with 
      the \codeinline{property::queue::enable_profiling} property.
      Specializations of \codeinline{info::param_traits} must be defined in
      accordance with the info parameters in
      Table~\ref{table.event.profilinginfo} to facilitate returning the type
      associated with the \codeinline{param} parameter.
    }
\completeTable
%-------------------------------------------------------------------------------

\subsubsection{Event information and profiling descriptors}
A SYCL event can be queried for all of the following information using the
\codeinline{get_info} function. The available information is in
Table~\ref{table.event.info}. Profiling information available is in
Table~\ref{table.event.profilinginfo}. The interface of all available event and
profiling descriptors is in appendix~\ref{appendix.event.descriptors}.

\fixme{info table consistency changes: add namespace and enum class to the
descriptor.}

%-------------------------------------------------------------------------------
\startInfoTable{Event Descriptors}
\addInfoFootNotes{Event class information descriptors}{table.event.info}
\addInfoRow
{info::event::command_execution_status} {info::event_command_status}
{
    Returns the event status of the command group associated with this SYCL \codeinline{event} or the event status of the underlying OpenCL event if this SYCL \codeinline{event} instance was constructed with the interoperability constructor. 
}

\addInfoRow
{info::event::reference_count}
{cl_uint}
{Return the reference count of the event.}

\completeInfoTable
%-------------------------------------------------------------------------------
\startInfoTable{Event information profiling descriptor}
\addInfoFootNotes{Profiling information descriptors for the SYCL
  \codeinline{event} class}
{table.event.profilinginfo}

\addInfoRow
{info::event_profiling::command_submit}
{cl_ulong}
{
    Returns an implementation defined 64-bit value describing the time in
    nanoseconds when the associated \gls{command-group} was submitted.
}

\addInfoRow
{info::event_profiling::command_start}
{cl_ulong}
{
    Returns an implementation defined 64-bit value describing the time in
    nanoseconds when the associated \gls{command-group} started executing.
}

\addInfoRow
{info::event_profiling::command_end}
{cl_ulong}
{
    Returns an implementation defined 64-bit value describing the time in
    nanoseconds when the associated \gls{command-group} finished executing.
}
\completeInfoTable
%-------------------------------------------------------------------------------

%***********************************************************************************
% Data access and storage in SYCL
%***********************************************************************************
\section{Data access and storage in SYCL}
\label{sec:data.access.and.storage}

In SYCL, data storage and access are handled by separate classes.
\Glspl{buffer} and \glspl{image} handle
storage and ownership of the data, whereas \glspl{accessor} handle access to
the data. Buffers and images in SYCL are different from OpenCL buffers and images
in that they can be bound to more than one device or context and they get
destroyed when they go out-of-scope. They also handle ownership of the
data, while allowing exception handling for blocking
and non-blocking data transfers. Accessors manage data transfers between the host
and all of the devices in the system, as well as tracking of data dependencies.

%***********************************************************************************
% Host allocation
%***********************************************************************************
\subsection{Host allocation}

A \gls{sycl-runtime} may need to allocate temporary objects on the host
to handle some operations (such as copying data from one context to 
another).
Allocation on the host is managed using an allocator object, following the
standard C++ allocator class definition.
The default allocator for memory objects is implementation defined,
but the user can supply their own allocator class.

\begin{code}
{
  buffer<int, 1, UserDefinedAllocator<int> > b(d);
}
\end{code}

When an allocator returns a nullptr, the runtime could not create data on the
host. Note that in this case the runtime will raise an error if it requires
host memory but it is not available (e.g when moving data across OpenCL
contexts).

The definition of allocators extends the current functionality of SYCL,
ensuring that users can define allocator functions for specific hardware or
certain complex shared memory mechanisms (e.g. NUMA), and improves
interoperability with STL-based libraries (e.g, Intel's TBB provides an
allocator).


%***********************************************************************************
% Default Allocators
%***********************************************************************************
\subsubsection{Default Allocators}
\label{subsec:default.allocators}
A default allocator is always defined by the implementation, and it is
guaranteed to return non-\codeinline{nullptr} and new memory positions every call.
The default allocator for const buffers will remove the const-ness of the
type (therefore, the default allocator for a buffer of type ''const int'' will
be an \codeinline{Allocator<int>}).
This implies that host \glspl{accessor} will not synchronize with the pointer given
by the user in the buffer/image constructor, but will use the memory
returned by the \codeinline{Allocator} itself for that purpose.
The user can implement an allocator that returns the same address as the
one passed in the buffer constructor, but it is the responsibility of the
user to handle the potential race conditions.

%------------------------------------------------------------------------------------------------------
\startTable{Allocators}
\addFootNotes{SYCL Default Allocators}{table.default.allocators}
\addRow{buffer_allocator}
{
It is the default buffer allocator used by the runtime, when no allocator is
defined by the user.
}
\addRow{image_allocator}
{
It is the default allocator used by the runtime for the SYCL \codeinline{image} class when no allocator is provided by the user.
The \codeinline{image_allocator} is required allocate in elements of \codeinline{byte}.
}
\completeTable
%------------------------------------------------------------------------------------------------------
See Section \ref{Mutex} for details on manual host-device synchronization.

%***********************************************************************************
% Buffers
%***********************************************************************************

\subsection{Buffers}
\label{subsec:buffers}

The \codeinline{buffer} class defines a shared array of one, two or three
dimensions that can be used by kernels in queues and has to be accessed using
\gls{accessor} classes. Buffers are templated on both the type of their data,
and the number of dimensions that the data is stored and accessed through.

A \codeinline{buffer} does not map to only one OpenCL buffer
object, and all OpenCL buffer memory objects may be temporary for use
within a command group on a specific device. The only exception to
this rule is when a buffer is constructed from a \codeinline{cl_mem}
object to interoperate with OpenCL. Use of an interoperability
buffer on a queue mapping to a context other than that in which the
\codeinline{cl_mem} was created is an error.

Note that if no source data is provided for a buffer, the buffer uses
uninitialized memory for performance reasons. So it is up to the
programmer to explicitly construct the objects in this case if
required.

More generally, since the value type of a buffer is required to be
trivially copyable, there is no constructor or destructor called in
any case.

A SYCL \codeinline{buffer} can construct an instance of a SYCL \codeinline{buffer}
that reinterprets the original SYCL \codeinline{buffer} with a different
type, dimensionality and range using the member function
\codeinline{reinterpret}. The reinterpreted SYCL \codeinline{buffer} that is
constructed must behave as though it were a copy of the SYCL \codeinline{buffer}
that constructed it (see sec \ref{sec:reference-semantics}) with the exception
that the type, dimensionality and range of the reinterpreted SYCL
\codeinline{buffer} must reflect the type, dimensionality and range specified
when calling the \codeinline{reinterpret} member function. By extension of this
the class member types \codeinline{value_type}, \codeinline{reference} and
\codeinline{const_reference}, and the member functions \codeinline{get_range}
and \codeinline{get_count} of the reinterpreted SYCL \codeinline{buffer} must
reflect the new type, dimensionality and range. The data that the original SYCL
\codeinline{buffer} and the reinterpreted SYCL \codeinline{buffer} manage
remains unaffected, though the representation of the data when accessed through
the reinterpreted SYCL \codeinline{buffer} may alter to reflect the new type,
dimensionality and range. It is important to note that a reinterpreted SYCL
\codeinline{buffer} is a copy of the original SYCL \codeinline{buffer} only,
and not a new SYCL \codeinline{buffer}. Constructing more than one SYCL
\codeinline{buffer} managing the same host pointer is still undefined behavior.

The SYCL \codeinline{buffer} class template provides the common reference
semantics (see Section~\ref{sec:reference-semantics}).

%***********************************************************************************
% Buffer Interface
%***********************************************************************************

\subsubsection{Buffer Interface}

The constructors and member functions of the SYCL \codeinline{buffer} class template are listed in Tables~\ref{table.constructors.buffer} and \ref{table.members.buffer}, respectively. The additional common special member functions and common member functions are listed in Tables~\ref{table.specialmembers.common.reference} and \ref{table.members.common.reference}, respectively.

Each constructor excluding the interoperability constructor takes as the last parameter an optional SYCL \codeinline{property_list} to provide properties to the SYCL \codeinline{buffer}.

The SYCL \codeinline{buffer} class template takes a template parameter \codeinline{AllocatorT} for specifying an allocator which is used by the \gls{sycl-runtime} when allocating temporary memory on the host. If no template argument is provided then the default allocator for the SYCL \codeinline{buffer} class \codeinline{buffer_allocator} will be used (see~\ref{subsec:default.allocators}).

%Interface for class: buffer

\lstinputlisting{headers/buffer.h}

%-------------------------------------------------------------------------------
\startTable{Constructor}
\addFootNotes{Constructors of the \codeinline{buffer} class}
{table.constructors.buffer}
  \addRowTwoSL
    {  buffer(const range<dimensions> \& bufferRange, }
    {  const property_list \&propList = \{\}) }
    {
       Construct a SYCL \codeinline{buffer} instance with uninitialized memory.      
       The constructed SYCL \codeinline{buffer} will use a default constructed \codeinline{AllocatorT} when allocating memory on the host.
        The range of the constructed SYCL \codeinline{buffer} is specified by the \codeinline{bufferRange} parameter provided.
       Unless the member function \codeinline{set_final_data()} is called with a valid non-null pointer there will be no write back on destruction.
       Zero or more properties can be provided to the constructed SYCL \codeinline{buffer} via an instance of \codeinline{property_list}.
    }
  \addRowThreeSL
    {  buffer(const range<dimensions> \& bufferRange, }
    {  AllocatorT allocator, }
    {  const property_list \&propList = \{\}) }
    {
       Construct a SYCL \codeinline{buffer} instance with uninitialized memory.      
       The constructed SYCL \codeinline{buffer} will use the \codeinline{allocator} parameter provided when allocating memory on the host.
       The range of the constructed SYCL \codeinline{buffer} is specified by the \codeinline{bufferRange} parameter provided.
       Unless the member function \codeinline{set_final_data()} is called with a valid non-null pointer there will be no write back on destruction.
       Zero or more properties can be provided to the constructed SYCL \codeinline{buffer} via an instance of \codeinline{property_list}.
     }
  \addRowThreeSL
   { buffer(T* hostData,}
   { const range<dimensions> \& bufferRange, }
   { const property_list \&propList = \{\}) }
   {
       Construct a SYCL \codeinline{buffer} instance with the \codeinline{hostData} parameter provided. The ownership of this memory is given to the constructed SYCL \codeinline{buffer} for the duration of its lifetime.
       The constructed SYCL \codeinline{buffer} will use a default constructed \codeinline{AllocatorT} when allocating memory on the host.
        The range of the constructed SYCL \codeinline{buffer} is specified by the \codeinline{bufferRange} parameter provided.
       Zero or more properties can be provided to the constructed SYCL \codeinline{buffer} via an instance of \codeinline{property_list}.
    }
  \addRowFourSL
   { buffer(T* hostData,}
   { const range<dimensions> \& bufferRange, }
   { AllocatorT allocator, }
   { const property_list \&propList = \{\}) }
   {
       Construct a SYCL \codeinline{buffer} instance with the \codeinline{hostData} parameter provided. The ownership of this memory is given to the constructed SYCL \codeinline{buffer} for the duration of its lifetime.
       The constructed SYCL \codeinline{buffer} will use the \codeinline{allocator} parameter provided when allocating memory on the host.
        The range of the constructed SYCL \codeinline{buffer} is specified by the \codeinline{bufferRange} parameter provided.
       Zero or more properties can be provided to the constructed SYCL \codeinline{buffer} via an instance of \codeinline{property_list}.
   }
  \addRowThreeSL
    {  buffer(const T* hostData, }
    {  const range<dimensions> \& bufferRange, }
    {  const property_list \&propList = \{\}) }
    {
       Construct a SYCL \codeinline{buffer} instance with the \codeinline{hostData} parameter provided. The ownership of this memory is given to the constructed SYCL \codeinline{buffer} for the duration of its lifetime.
       The constructed SYCL \codeinline{buffer} will use a default constructed \codeinline{AllocatorT} when allocating memory on the host.
      The host address is \codeinline{const T}, so the host accesses can be read-only. However, the \tf{typename T} is not const so the device accesses can be both read and write accesses. Since, the \tf{hostData} is const, this buffer is only initialized with this memory and there is no write after its destruction, unless there is another final data address given after construction of the buffer.
        The range of the constructed SYCL \codeinline{buffer} is specified by the \codeinline{bufferRange} parameter provided.
       Zero or more properties can be provided to the constructed SYCL \codeinline{buffer} via an instance of \codeinline{property_list}.
    }
  \addRowFourSL
    {  buffer(const T* hostData, }
    {  const range<dimensions> \& bufferRange,}
    {  AllocatorT allocator, }
    {  const property_list \&propList = \{\}) }
    {
       Construct a SYCL \codeinline{buffer} instance with the \codeinline{hostData} parameter provided. The ownership of this memory is given to the constructed SYCL \codeinline{buffer} for the duration of its lifetime.
       The constructed SYCL \codeinline{buffer} will use the \codeinline{allocator} parameter provided when allocating memory on the host.       
      The host address is \codeinline{const T}, so the host accesses can be read-only. However, the \tf{typename T} is not const so the device accesses can be both read and write accesses. Since, the \tf{hostData} is const, this buffer is only initialized with this memory and there is no write after its destruction, unless there is another final data address given after construction of the buffer.
        The range of the constructed SYCL \codeinline{buffer} is specified by the \codeinline{bufferRange} parameter provided.
       Zero or more properties can be provided to the constructed SYCL \codeinline{buffer} via an instance of \codeinline{property_list}.
    }
  \addRowThreeSL
   { buffer(const shared_ptr_class<T> \&hostData,}
   { const range<dimensions> \& bufferRange, }
   { const property_list \&propList = \{\}) }
   {
       Construct a SYCL \codeinline{buffer} instance with the \codeinline{hostData} parameter provided. The ownership of this memory is given to the constructed SYCL \codeinline{buffer} for the duration of its lifetime.
       The constructed SYCL \codeinline{buffer} will use a default constructed \codeinline{AllocatorT} when allocating memory on the host.
        The range of the constructed SYCL \codeinline{buffer} is specified by the \codeinline{bufferRange} parameter provided.
       Zero or more properties can be provided to the constructed SYCL \codeinline{buffer} via an instance of \codeinline{property_list}.
    }
  \addRowFourSL
   { buffer(const shared_ptr_class<void> \&hostData,}
   { const range<dimensions> \& bufferRange, }
   { AllocatorT allocator, }
   { const property_list \&propList = \{\}) }
   {
       Construct a SYCL \codeinline{buffer} instance with the \codeinline{hostData} parameter provided. The ownership of this memory is given to the constructed SYCL \codeinline{buffer} for the duration of its lifetime.
       The constructed SYCL \codeinline{buffer} will use the \codeinline{allocator} parameter provided when allocating memory on the host.
        The range of the constructed SYCL \codeinline{buffer} is specified by the \codeinline{bufferRange} parameter provided.
       Zero or more properties can be provided to the constructed SYCL \codeinline{buffer} via an instance of \codeinline{property_list}.
   }
  \addRowThreeSL
    { template <typename InputIterator> }
    { buffer(InputIterator first, InputIterator last, }
    { const property_list \&propList = \{\}) }
    {
      Create a new allocated 1D buffer initialized from the given elements
      ranging from \codeinline{first} up to one before \codeinline{last}.
      The data is copied to an intermediate memory position by the runtime.
      Data is written back to the same iterator set if the iterator is not
      a const iterator.
      The constructed SYCL \codeinline{buffer} will use a default constructed \codeinline{AllocatorT} when allocating memory on the host.
      Zero or more properties can be provided to the constructed SYCL \codeinline{buffer} via an instance of \codeinline{property_list}.
    }
  \addRowFourSL
    { template <typename InputIterator> }
    { buffer(InputIterator first, InputIterator last, }
    { AllocatorT allocator = \{\}, }
    { const property_list \&propList = \{\}) }
    {
      Create a new allocated 1D buffer initialized from the given elements
      ranging from \codeinline{first} up to one before \codeinline{last}.
      The data is copied to an intermediate memory position by the runtime.
      Data is written back to the same iterator set if the iterator is not
      a const iterator.
      The constructed SYCL \codeinline{buffer} will use the \codeinline{allocator} parameter provided when allocating memory on the host.
      Zero or more properties can be provided to the constructed SYCL \codeinline{buffer} via an instance of \codeinline{property_list}.
    }
  \addRowThreeSL
    { buffer(buffer<T, dimensions, AllocatorT> \&b, }
    { const id<dimensions> \& baseIndex, }
    { const range<dimensions> \& subRange) }
    {
      Create a new sub-buffer without allocation to have separate
      accessors later. \codeinline{b} is the buffer with the real data.
      \codeinline{baseIndex} specifies the origin of the sub-buffer inside the
      buffer \codeinline{b}. \codeinline{subRange} specifies the size of the sub-buffer.
      The offset and range specified by \codeinline{baseIndex} and \codeinline{subRange} together must represent a contiguous region of the original SYCL \codeinline{buffer}.
      The total size of the sub-\codeinline{buffer} being constructed must be a multiple of the memory base address alignment of each SYCL \codeinline{device} that is executed on, otherwise the \gls{sycl-runtime} must throw an asynchronous \codeinline{invalid_object_error} SYCL exception.
      This value is retrievable via the SYCL \codeinline{device} class info query \codeinline{info::device::mem_base_addr_align}.
    }
    \addRowThreeSL
      { buffer(cl_mem clMemObject, }
      { const context \&syclContext, }
      { event availableEvent = \{\}) }
      {
        Available only when: \codeinline{dimensions == 1}.
        \newline
        Constructs a SYCL \codeinline{buffer} instance from an OpenCL \codeinline{cl_mem} in accordance with the requirements described in \ref{sec:opencl-interoperability}.
        The instance of the SYCL \codeinline{buffer} class template being constructed must wait for the SYCL \codeinline{event} parameter, if one is provided, \codeinline{availableEvent} to signal that the \codeinline{cl_mem} instance is ready to be used.
        The SYCL \codeinline{context} parameter \codeinline{syclContext} is the context associated with the memory object.
      } 
\completeInfoTable
%-------------------------------------------------------------------------------

%-------------------------------------------------------------------------------
\startTable{Member function}
\addFootNotes{Member functions for the \codeinline{buffer} class}{table.members.buffer}
  \addRow
    {range<dimensions> get_range() const}
    {
      Return a range object representing the
      size of the buffer in terms of number
      of elements in each dimension as passed
      to the constructor.
    }
  \addRow
    {size_t get_count() const}
    {
      Returns the total number of elements in the buffer.
      Equal to \codeinline{get_range()[0] * ... * get_range()[dimensions-1]}.
    }
  \addRow
    {size_t get_size() const}
    {
      Returns the size of the buffer storage in bytes.
      Equal to \codeinline{get_count()*sizeof(T)}.
    }
  \addRow
    {AllocatorT get_allocator() const}
    {
      Returns the allocator provided to the buffer.
    }
  \addRowThreeSL
    {template<access::mode mode, access::target target = access::target::global_buffer>}
    {accessor<T, dimensions, mode, target> }
    {get_access(handler \&commandGroupHandler)}
    {
      Returns a valid accessor to the buffer with the specified
      access mode and target in the command group buffer.
      The value of target can be
      \mbox{\codeinline{access::target::global_buffer}} or
      \codeinline{access::constant_buffer}.
    }
  \addRowThreeSL
    {template<access::mode mode>}
    {accessor<T, dimensions, mode, access::target::host_buffer> }
    {get_access()}
    {
      Returns a valid host accessor to the buffer with the specified
      access mode and target.
    }
  \addRowThreeSL
    {template<access::mode mode, access::target target=access::target::global_buffer>}
    {accessor<T, dimensions, mode, target> }
    {get_access(handler \&commandGroupHandler, range<dimensions> accessRange,
       id<dimensions> accessOffset = \{\})}
    {
      Returns a valid accessor to the buffer with the specified
      access mode and target in the command group buffer.
      Only the values starting from the given offset and up to the given 
      range are guaranteed to be updated.
      The value of target can be
      \mbox{\codeinline{access::target::global_buffer}} or
      \codeinline{access::constant_buffer}.
    }
  \addRowThreeSL
    {template<access::mode mode>}
    {accessor<T, dimensions, mode, access::target::host_buffer> }
    {get_access(range<dimensions> accessRange, id<dimensions> accessOffset = \{\})}
    {
      Returns a valid host accessor to the buffer with the specified
      access mode and target. 
      Only the values starting from the given offset and up to the given 
      range are guaranteed to be updated.
      The value of target can only be
      \codeinline{access::target::host_buffer}.
    }

   \addRowTwoL
    {template <typename Destination = std::nullptr_t>}
    {void set_final_data(Destination finalData = std::nullptr)}
    {
      The \codeinline{finalData} points to where the outcome of all
      the buffer processing is going to be copied to at destruction
      time, if the buffer was involved with a write accessor.

      Destination can be either an output iterator or a
      \codeinline{weak_ptr_class<T>}.

      Note that a raw pointer is a special case of output iterator and
      thus defines the host memory to which the result is to be
      copied.

      In the case of a weak pointer, the output is not updated if the
      weak pointer has expired.

      If \codeinline{Destination} is \codeinline{std::nullptr_t}, then
      the copy back will not happen.
    }

   \addRow
    {void set_write_back(bool flag = true)}
    {
      This method allows dynamically forcing or canceling the
      write-back of the data of a buffer on destruction according to
      the value of \codeinline{flag}.

      Forcing the write-back is similar to what happens during a
      normal write-back as described in \S~\ref{sec:buf-sync-rules}
      and \ref{sec:sharing-host-memory-with-dm}.

      If there is nowhere to write-back, using this function does not
      have any effect.
    }
   \addRow
    {bool is_sub_buffer() const}
    {
      Returns true if this SYCL \codeinline{buffer} is a sub-buffer, otherwise
      returns false.
    }
   \addRowThreeL
     { template <typename ReinterpretT, int ReinterpretDim> }
     { buffer<ReinterpretT, ReinterpretDim, AllocatorT> }
     { reinterpret(range<ReinterpretDim> reinterpretRange) const }
     {
       Creates and returns a reinterpreted SYCL \codeinline{buffer} with the
       type specified by \codeinline{ReinterpretT}, dimensions specified by
       \codeinline{ReinterpretDim} and range specified by
       \codeinline{reinterpretRange}. Must throw an
       \codeinline{invalid_object_error} SYCL exception if the total size in
       bytes represented by the type and range of the reinterpreted SYCL
       \codeinline{buffer} does not equal the total size in bytes represented by
       the type and range of this SYCL \codeinline{buffer}.
     }
\completeTable
%------------------------------------------------------------------------------------------------------

\subsubsection{Buffer Properties}
\label{sec:buffer-properties}

The properties that can be provided when constructing the SYCL \codeinline{buffer} class are describe in Table~\ref{table.properties.buffer}.

%---------------------------------------------------------------------
\startTable{Property}
\addFootNotes{Properties supported by the SYCL \codeinline{buffer} class} {table.properties.buffer}

\addRow
  { property::buffer::use_host_ptr }
  {
    The \codeinline{use_host_ptr} property adds the requirement that the \gls{sycl-runtime} must not allocate any memory for the SYCL \codeinline{buffer} and instead uses the provided host pointer directly. This prevents the \gls{sycl-runtime} from allocating additional temporary storage on the host.
  }
\addRow
  { property::buffer::use_mutex }
  {
    The \codeinline{use_mutex} property is valid for the SYCL \codeinline{buffer} and \codeinline{image} classes. The property adds the requirement that the memory which is owned by the SYCL \codeinline{buffer} can be shared with the application via a \codeinline{mutex_class} provided to the property. The mutex \codeinline{m} is locked by the runtime whenever the data is in use and unlocked otherwise. Data is synchronized with \codeinline{hostData}, when the mutex is unlocked by the runtime.
  }
\addRow
  { property::buffer::context_bound }
  {
    The \codeinline{context_bound} property adds the requirement that the SYCL \codeinline{buffer} can only be associated with a single SYCL \codeinline{context} that is provided to the property.
  }
\completeTable
%---------------------------------------------------------------------

The constructors and special member functions of the buffer property
classes are listed in
Tables~\ref{table.constructors.properties.buffer} and
\ref{table.members.properties.buffer} respectively.

%---------------------------------------------------------------------
\startTable{Constructor}
\addFootNotes{Constructors of the buffer property classes}
{table.constructors.properties.buffer}
\addRow
{property::buffer::use_host_ptr::use_host_ptr()}
{
  Constructs a SYCL \codeinline{use_host_ptr} property instance.
}
\addRow
{property::buffer::use_mutex::use_mutex(mutex_class \&mutexRef)}
{
  Constructs a SYCL \codeinline{use_mutex} property instance with a reference to \codeinline{mutexRef} parameter provided.
}
\addRow
{property::buffer::context_bound::context_bound(context boundContext)}
{
  Constructs a SYCL \codeinline{context_bound} property instance with a copy of a SYCL \codeinline{context}.
}
\completeTable
%---------------------------------------------------------------------

%---------------------------------------------------------------------
\startTable{Member function}
\addFootNotes{Member functions of the buffer property classes}
{table.members.properties.buffer}
\addRow
{mutex_class *property::buffer::use_mutex::get_mutex_ptr() const}
{
  Returns the \codeinline{mutex_class} which was specified when
  constructing this SYCL \codeinline{use_mutex} property.
}
\addRow
{context property::buffer::context_bound::get_context() const}
{
  Returns the \codeinline{context} which was specified when
  constructing this SYCL \codeinline{context_bound} property.
}
\completeTable
%---------------------------------------------------------------------

%***********************************************************************************
% Buffer Synchronization Rules
%***********************************************************************************
\subsubsection{Buffer Synchronization Rules}
\label{sec:buf-sync-rules}

Buffers are reference-counted. When a buffer value is constructed
from another buffer, the two values reference the same buffer and a
reference count is incremented. When a buffer value is destroyed,
the reference count is decremented. Only when there are no more
buffer values that reference a specific buffer is the actual
buffer destroyed and the buffer destruction behavior defined
below is followed.

If any error occurs on buffer destruction, it is reported
via the associated queue's asynchronous error handling mechanism.

The basic rule for the blocking behavior of a buffer destructor is
that it blocks if there is some data to write back because a
write-accessor on it has been created, or if the buffer was constructed
with attached host memory and is still in use.

More precisely:
\begin{enumerate}
  \item
    A buffer can be constructed with just a size and using the default
    buffer allocator.  The memory management for this type of buffer
    is entirely handled by the SYCL system. The destructor for this
    type of buffer does not need to block, even if work on the buffer has not
    completed. Instead, the SYCL system frees any storage required
    for the buffer asynchronously when it is no longer in use in queues.
    The initial contents of the buffer are unspecified.

  \item
    A buffer can be constructed with associated host memory and a default
    buffer allocator. The buffer will use this host memory for its full lifetime, but the
    contents of this host memory are unspecified for the lifetime of the
    buffer. If the host memory is modified by the host, or mapped to
    another buffer or image during the lifetime of this buffer, then
    the results are undefined. The initial contents of the buffer will
    be the contents of the host memory at the time of construction.

    When the buffer is destroyed, the destructor will block until all
    work in queues on the buffer have completed, then copy the contents
    of the buffer back to the host memory (if required) and then
    return.

    \begin{enumerate}
    \item
    If the type of the host data is \codeinline{const}, then the buffer is
    read-only; only read accessors are allowed on the buffer and
    no-copy-back to host memory is performed (although the host memory
    must still be kept available for use by SYCL). When using the default
    buffer allocator, the const-ness of the type will be removed in order to
    allow host allocation of memory, which will allow temporary host copies
    of the data by the \gls{sycl-runtime}, for example for speeding up
    host accesses.

    When the buffer is destroyed, the destructor will block until all
    work in queues on the buffer have completed and then return, as there
    is no copy of data back to host.

    \item

      If the type of the host data is not \codeinline{const} but the
      pointer to host data is \codeinline{const}, then the read-only
      restriction applies only on host and not on device accesses.

    When the buffer is destroyed, the destructor will block until all
    work in queues on the buffer have completed.

  \end{enumerate}

  \item
    A buffer can be constructed using a \codeinline{shared_ptr} to host data.
    This pointer is shared between the SYCL application and the runtime. In order
    to allow synchronization between the application and the runtime a
    \codeinline{mutex} is used which will be locked by the runtime whenever the
    data is in use, and unlocked when it is no longer needed.

    The \codeinline{shared_ptr} reference counting is used in order to prevent
    destroying the buffer host data prematurely. If the
    \codeinline{shared_ptr} is deleted from the user application before buffer
    destruction, the buffer can continue securely because the pointer
    hasn't been destroyed yet.  It will not copy data back to the host before destruction, however,
    as the application side has already deleted its copy.

    Note that since there is an implicit conversion of a
    \codeinline{unique_ptr_class} to a \codeinline{std::shared_ptr}, a
    \codeinline{unique_ptr_class} can also be used to pass the
    ownership to the \gls{sycl-runtime}.

  \item
    A buffer can be constructed from a pair of iterator values. In
    this case, the buffer construction will copy the data from the
    data range defined by the iterator pair. The destructor will
    not copy back any data and does not need to block.
    
  \end{enumerate}

If \codeinline{set_final_data()} is used to change where to write the
data back to, then the destructor of the buffer will block if a
write-accessor on it has been created.

A sub-buffer object can be created which is a sub-range reference to a
base buffer. This sub-buffer can be used to create accessors to the
base buffer, which have access to the range specified at time
of construction of the sub-buffer.

%***********************************************************************************
% Images
%***********************************************************************************
\subsection{Images}
\label{subsec:images}

The class \tclass{image}{\tf{int} dimensions}
(Table~\ref{table.constructors.image}) defines shared image data of one,
two or three dimensions, that can be used by kernels in queues and has to
be accessed using \gls{accessor} classes with image accessor modes.

The constructors and member functions of the SYCL \codeinline{image} class template are listed in Tables~\ref{table.constructors.image} and \ref{table.members.image}, respectively. The additional common special member functions and common member functions are listed in Tables~\ref{table.specialmembers.common.reference} and \ref{table.members.common.reference}, respectively.

Where relevant, it is the
responsibility of the user to ensure that the format of the data
matches the format described by
\codeinline{order} and \codeinline{type}.

The allocator template parameter of the SYCL \codeinline{image} class can be any allocator type including a custom allocator, however it must allocate in units of \codeinline{byte}.

If an image object is constructed from a \codeinline{cl_mem} object,
then the image is created and initialized from the OpenCL memory
object. The SYCL system may copy the data to the host, but must copy
it back (if modified) at the point of destruction of the image.
The user must provide a \codeinline{queue} and
\codeinline{event}. The memory object is assumed to only be available to the
\gls{sycl-runtime} after the event has signaled and is assumed to be
currently resident on the context and device signified by the
\codeinline{queue}.

For any image that is constructed with the range $(r1, r2, r3)$ with a element
type size in bytes of $s$, the image row pitch and image slice pitch should be
calculated as follows:

\begin{equation}
\label{image-row-pitch}
 r1 \cdot s
\end{equation}

\begin{equation}
\label{image-slice-pitch}
 r1 \cdot r2 \cdot s
\end{equation}

The SYCL \codeinline{image} class template provides the common reference
semantics (see Section~\ref{sec:reference-semantics}).

%***********************************************************************************
% Image Interface
%***********************************************************************************
\subsubsection{Image Interface}

Each constructor excluding the interoperability constructor takes as the last parameter an optional SYCL \codeinline{property_list} to provide properties to the SYCL \codeinline{image}.

The SYCL \codeinline{image} class template takes a template parameter \codeinline{AllocatorT} for specifying an allocator which is used by the \gls{sycl-runtime} when allocating temporary memory on the host. If no template argument is provided the default allocator for the SYCL \codeinline{image} class \codeinline{image_allocator} will be used~\ref{subsec:default.allocators}.

%Interface for class: image
\lstinputlisting{headers/image.h}

%------------------------------------------------------------------------------------------------------
\startTable{Constructor}
\addFootNotes{Constructors of the \codeinline{image} class template}
{table.constructors.image}
    \addRowFourSL
       { image(image_channel_order order, }
       { image_channel_type type, }
       { const range<dimensions> \& range, }
       { const property_list \&propList = \{\}) }
       {
         Construct a SYCL \codeinline{image} instance with uninitialized memory.       
         The constructed SYCL \codeinline{image} will use a default constructed \codeinline{AllocatorT} when allocating memory on the host.
         The element size of the constructed SYCL \codeinline{image} will be derived from the \codeinline{order} and \codeinline{type} parameters.
         The range of the constructed SYCL \codeinline{image} is specified by the \codeinline{range} parameter provided.
         The pitch of the constructed SYCL \codeinline{image} will be the default size determined by the \gls{sycl-runtime}.
         Unless the member function \codeinline{set_final_data()} is called with a valid non-null pointer there will be no write back on destruction.
         Zero or more properties can be provided to the constructed SYCL \codeinline{image} via an instance of \codeinline{property_list}.
       }
    \addRowFiveSL
       { image(image_channel_order order, }
       { image_channel_type type, }
       { const range<dimensions> \& range, }
       { AllocatorT allocator, }
       { const property_list \&propList = \{\}) }
       {
         Construct a SYCL \codeinline{image} instance with uninitialized memory.       
         The constructed SYCL \codeinline{image} will use the \codeinline{allocator} parameter provided when allocating memory on the host.
         The element size of the constructed SYCL \codeinline{image} will be derived from the \codeinline{order} and \codeinline{type} parameters.
         The range of the constructed SYCL \codeinline{image} is specified by the \codeinline{range} parameter provided.
         The pitch of the constructed SYCL \codeinline{image} will be the default size determined by the \gls{sycl-runtime}.
         Unless the member function \codeinline{set_final_data()} is called with a valid non-null pointer there will be no write back on destruction.
         Zero or more properties can be provided to the constructed SYCL \codeinline{image} via an instance of \codeinline{property_list}.
       }       
     \addRowFiveSL
       {image<dimensions>(image_channel_order order, }
       { image_channel_type type, }
       { const range<dimensions> \& range, }
       { const range<dimensions-1> \&pitch, }
       { const property_list \&propList = \{\}) }
       {
          Available only when: \codeinline{dimensions > 1}.  
          \newline    
          Construct a SYCL \codeinline{image} instance with uninitialized memory.
         The constructed SYCL \codeinline{image} will use a default constructed \codeinline{AllocatorT} when allocating memory on the host.
         The element size of the constructed SYCL \codeinline{image} will be derived from the \codeinline{order} and \codeinline{type} parameters.
         The range of the constructed SYCL \codeinline{image} is specified by the \codeinline{range} parameter provided.
         The pitch of the constructed SYCL \codeinline{image} will be the \codeinline{pitch} parameter provided.
         Unless the member function \codeinline{set_final_data()} is called with a valid non-null pointer there will be no write back on destruction.
         Zero or more properties can be provided to the constructed SYCL \codeinline{image} via an instance of \codeinline{property_list}.
        }
     \addRowSixSL
       {image<dimensions>(image_channel_order order, }
       { image_channel_type type, }
       { const range<dimensions> \& range, }
       { const range<dimensions-1> \&pitch, }
       { AllocatorT allocator, }
       { const property_list \&propList = \{\}) }
       {
          Available only when: \codeinline{dimensions > 1}.  
          \newline    
          Construct a SYCL \codeinline{image} instance with uninitialized memory.
         The constructed SYCL \codeinline{image} will use the \codeinline{allocator} parameter provided when allocating memory on the host.
         The element size of the constructed SYCL \codeinline{image} will be derived from the \codeinline{order} and \codeinline{type} parameters.
         The range of the constructed SYCL \codeinline{image} is specified by the \codeinline{range} parameter provided.
         The pitch of the constructed SYCL \codeinline{image} will be the \codeinline{pitch} parameter provided.
         Unless the member function \codeinline{set_final_data()} is called with a valid non-null pointer there will be no write back on destruction.
         Zero or more properties can be provided to the constructed SYCL \codeinline{image} via an instance of \codeinline{property_list}.
       }
    \addRowFiveSL
      {image(void *hostPointer, }
      { image_channel_order order, }
      { image_channel_type type, }
      { const range<dimensions> \& range, }
      { const property_list \&propList = \{\}) }
      {
         Construct a SYCL \codeinline{image} instance with the \codeinline{hostPointer} parameter provided. The ownership of this memory is given to the constructed SYCL \codeinline{image} for the duration of its lifetime.
         The constructed SYCL \codeinline{image} will use a default constructed \codeinline{AllocatorT} when allocating memory on the host.
         The element size of the constructed SYCL \codeinline{image} will be derived from the \codeinline{order} and \codeinline{type} parameters.
         The range of the constructed SYCL \codeinline{image} is specified by the \codeinline{range} parameter provided.
         The pitch of the constructed SYCL \codeinline{image} will be the default size determined by the \gls{sycl-runtime}.
         Unless the member function \codeinline{set_final_data()} is called with a valid non-null pointer any memory allocated by the \gls{sycl-runtime} is written back to \codeinline{hostPointer}.
         Zero or more properties can be provided to the constructed SYCL \codeinline{image} via an instance of \codeinline{property_list}.
    }
    \addRowSixSL
      {image(void *hostPointer, }
      { image_channel_order order, }
      { image_channel_type type, }
      { const range<dimensions> \& range, }
      { AllocatorT allocator, }
      { const property_list \&propList = \{\}) }
      {
         Construct a SYCL \codeinline{image} instance with the \codeinline{hostPointer} parameter provided. The ownership of this memory is given to the constructed SYCL \codeinline{image} for the duration of its lifetime.       
         The constructed SYCL \codeinline{image} will use the \codeinline{allocator} parameter provided when allocating memory on the host.
         The host address is \codeinline{const T}, so the host accesses can be read-only. However, the device accesses can be both read and write accesses. Since, the \tf{hostPointer} is const, this image is only initialized with this memory and there is no write after its destruction, unless there is another final data address given after construction of the image.
         The element size of the constructed SYCL \codeinline{image} will be derived from the \codeinline{order} and \codeinline{type} parameters.
         The range of the constructed SYCL \codeinline{image} is specified by the \codeinline{range} parameter provided.
         The pitch of the constructed SYCL \codeinline{image} will be the default size determined by the \gls{sycl-runtime}.
         Unless the member function \codeinline{set_final_data()} is called with a valid non-null pointer any memory allocated by the \gls{sycl-runtime} is written back to \codeinline{hostPointer}.
         Zero or more properties can be provided to the constructed SYCL \codeinline{image} via an instance of \codeinline{property_list}.
      }
    \addRowFiveSL
      {image(const void *hostPointer, }
      { image_channel_order order, }
      { image_channel_type type, }
      { const range<dimensions> \& range, }
      { const property_list \&propList = \{\}) }
      {
         Construct a SYCL \codeinline{image} instance with the \codeinline{hostPointer} parameter provided. The ownership of this memory is given to the constructed SYCL \codeinline{image} for the duration of its lifetime.
         The constructed SYCL \codeinline{image} will use a default constructed \codeinline{AllocatorT} when allocating memory on the host.
         The host address is \codeinline{const T}, so the host accesses can be read-only. However, the device accesses can be both read and write accesses. Since, the \tf{hostPointer} is const, this image is only initialized with this memory and there is no write after its destruction, unless there is another final data address given after construction of the image.
         The element size of the constructed SYCL \codeinline{image} will be derived from the \codeinline{order} and \codeinline{type} parameters.
         The range of the constructed SYCL \codeinline{image} is specified by the \codeinline{range} parameter provided.
         The pitch of the constructed SYCL \codeinline{image} will be the default size determined by the \gls{sycl-runtime}.
         Unless the member function \codeinline{set_final_data()} is called with a valid non-null pointer any memory allocated by the \gls{sycl-runtime} is written back to \codeinline{hostPointer}.
         Zero or more properties can be provided to the constructed SYCL \codeinline{image} via an instance of \codeinline{property_list}.
    }
    \addRowSixSL
      {image(const void *hostPointer, }
      { image_channel_order order, }
      { image_channel_type type, }
      { const range<dimensions> \& range, }
      { AllocatorT allocator, }
      { const property_list \&propList = \{\}) }
      {
         Construct a SYCL \codeinline{image} instance with the \codeinline{hostPointer} parameter provided. The ownership of this memory is given to the constructed SYCL \codeinline{image} for the duration of its lifetime.       
         The constructed SYCL \codeinline{image} will use the \codeinline{allocator} parameter provided when allocating memory on the host.
         The element size of the constructed SYCL \codeinline{image} will be derived from the \codeinline{order} and \codeinline{type} parameters.
         The range of the constructed SYCL \codeinline{image} is specified by the \codeinline{range} parameter provided.
         The pitch of the constructed SYCL \codeinline{image} will be the default size determined by the \gls{sycl-runtime}.
         Unless the member function \codeinline{set_final_data()} is called with a valid non-null pointer any memory allocated by the \gls{sycl-runtime} is written back to \codeinline{hostPointer}.
         Zero or more properties can be provided to the constructed SYCL \codeinline{image} via an instance of \codeinline{property_list}.
      }
  \addRowSixSL
    {image(void *hostPointer, }
    { image_channel_order order, }
    { image_channel_type type, }
    { const range<dimensions> \& range, }
    { const range<dimensions-1> \&pitch, }
    { const property_list \&propList = \{\}) }
    {
         Available only when: \codeinline{dimensions > 1}.  
         \newline    
          Construct a SYCL \codeinline{image} instance with the \codeinline{hostPointer} parameter provided. The ownership of this memory is given to the constructed SYCL \codeinline{image} for the duration of its lifetime.
         The constructed SYCL \codeinline{image} will use a default constructed \codeinline{AllocatorT} when allocating memory on the host.         
         The element size of the constructed SYCL \codeinline{image} will be derived from the \codeinline{order} and \codeinline{type} parameters.
         The range of the constructed SYCL \codeinline{image} is specified by the \codeinline{range} parameter provided.
         The pitch of the constructed SYCL \codeinline{image} will be the \codeinline{pitch} parameter provided.
         Unless the member function \codeinline{set_final_data()} is called with a valid non-null pointer any memory allocated by the \gls{sycl-runtime} is written back to \codeinline{hostPointer}.
         Zero or more properties can be provided to the constructed SYCL \codeinline{image} via an instance of \codeinline{property_list}.
    }
  \addRowSevenSL
    {image(void *hostPointer, }
    { image_channel_order order, }
    { image_channel_type type, }
    { const range<dimensions> \& range, }
    { const range<dimensions-1> \&pitch, }
    { AllocatorT allocator, }
    { const property_list \&propList = \{\}) }
    {
         Available only when: \codeinline{dimensions > 1}.  
         \newline    
         Construct a SYCL \codeinline{image} instance with the \codeinline{hostPointer} parameter provided. The ownership of this memory is given to the constructed SYCL \codeinline{image} for the duration of its lifetime.
         The constructed SYCL \codeinline{image} will use the \codeinline{allocator} parameter provided when allocating memory on the host.
         The element size of the constructed SYCL \codeinline{image} will be derived from the \codeinline{order} and \codeinline{type} parameters.
         The range of the constructed SYCL \codeinline{image} is specified by the \codeinline{range} parameter provided.
         The pitch of the constructed SYCL \codeinline{image} will be the \codeinline{pitch} parameter provided.
         Unless the member function \codeinline{set_final_data()} is called with a valid non-null pointer any memory allocated by the \gls{sycl-runtime} is written back to \codeinline{hostPointer}.
         Zero or more properties can be provided to the constructed SYCL \codeinline{image} via an instance of \codeinline{property_list}.
    }
    \addRowFiveSL
    {image(shared_ptr_class<void>\& hostPointer, }
    { image_channel_order order, }
    { image_channel_type type, }
    { const range<dimensions> \& range, }
    { const property_list \&propList = \{\}) }
    {
         Construct a SYCL \codeinline{image} instance with the \codeinline{hostPointer} parameter provided. The ownership of this memory is given to the constructed SYCL \codeinline{image} for the duration of its lifetime.
         The constructed SYCL \codeinline{image} will use a default constructed \codeinline{AllocatorT} when allocating memory on the host.
         The element size of the constructed SYCL \codeinline{image} will be derived from the \codeinline{order} and \codeinline{type} parameters.
         The range of the constructed SYCL \codeinline{image} is specified by the \codeinline{range} parameter provided.
         The pitch of the constructed SYCL \codeinline{image} will be the default size determined by the \gls{sycl-runtime}.
         Unless the member function \codeinline{set_final_data()} is called with a valid non-null pointer any memory allocated by the \gls{sycl-runtime} is written back to \codeinline{hostPointer}.
         Zero or more properties can be provided to the constructed SYCL \codeinline{image} via an instance of \codeinline{property_list}.
    }
    \addRowSixSL
    {image(shared_ptr_class<void>\& hostPointer, }
    { image_channel_order order, }
    { image_channel_type type, }
    { const range<dimensions> \& range, }
    { AllocatorT allocator, }
    { const property_list \&propList = \{\}) }
    {
         Construct a SYCL \codeinline{image} instance with the \codeinline{hostPointer} parameter provided. The ownership of this memory is given to the constructed SYCL \codeinline{image} for the duration of its lifetime.
         The constructed SYCL \codeinline{image} will use the \codeinline{allocator} parameter provided when allocating memory on the host.
         The element size of the constructed SYCL \codeinline{image} will be derived from the \codeinline{order} and \codeinline{type} parameters.
         The range of the constructed SYCL \codeinline{image} is specified by the \codeinline{range} parameter provided.
         The pitch of the constructed SYCL \codeinline{image} will be the default size determined by the \gls{sycl-runtime}.
         Unless the member function \codeinline{set_final_data()} is called with a valid non-null pointer any memory allocated by the \gls{sycl-runtime} is written back to \codeinline{hostPointer}.
         Zero or more properties can be provided to the constructed SYCL \codeinline{image} via an instance of \codeinline{property_list}.
    }    
  \addRowSixSL
    {image(shared_ptr_class<void>\& hostPointer, }
    { image_channel_order order, }
    { image_channel_type type, }
    { const range<dimensions> \& range, }
    { const range<dimensions-1> \& pitch, }
    { const property_list \&propList = \{\}) }
    {
         Construct a SYCL \codeinline{image} instance with the \codeinline{hostPointer} parameter provided. The ownership of this memory is given to the constructed SYCL \codeinline{image} for the duration of its lifetime.
         The constructed SYCL \codeinline{image} will use a default constructed \codeinline{AllocatorT} when allocating memory on the host.
         The element size of the constructed SYCL \codeinline{image} will be derived from the \codeinline{order} and \codeinline{type} parameters.
         The range of the constructed SYCL \codeinline{image} is specified by the \codeinline{range} parameter provided.
         The pitch of the constructed SYCL \codeinline{image} will be the \codeinline{pitch} parameter provided.
         Unless the member function \codeinline{set_final_data()} is called with a valid non-null pointer any memory allocated by the \gls{sycl-runtime} is written back to \codeinline{hostPointer}.
         Zero or more properties can be provided to the constructed SYCL \codeinline{image} via an instance of \codeinline{property_list}.
    }
  \addRowSevenSL
    {image(shared_ptr_class<void>\& hostPointer, }
    { image_channel_order order, }
    { image_channel_type type, }
    { const range<dimensions> \& range, }
    { const range<dimensions-1> \& pitch, }
    { AllocatorT allocator, }
    { const property_list \&propList = \{\}) }
    {
         Construct a SYCL \codeinline{image} instance with the \codeinline{hostPointer} parameter provided. The ownership of this memory is given to the constructed SYCL \codeinline{image} for the duration of its lifetime.
         The constructed SYCL \codeinline{image} will use the \codeinline{allocator} parameter provided when allocating memory on the host.
         The element size of the constructed SYCL \codeinline{image} will be derived from the \codeinline{order} and \codeinline{type} parameters.
         The range of the constructed SYCL \codeinline{image} is specified by the \codeinline{range} parameter provided.
         The pitch of the constructed SYCL \codeinline{image} will be the \codeinline{pitch} parameter provided.
         Unless the member function \codeinline{set_final_data()} is called with a valid non-null pointer any memory allocated by the \gls{sycl-runtime} is written back to \codeinline{hostPointer}.
         Zero or more properties can be provided to the constructed SYCL \codeinline{image} via an instance of \codeinline{property_list}.
    }
    \addRowThreeSL
      { image(cl_mem clMemObject, }
      { const context \&syclContext, }
      { event availableEvent = \{\}) }
      {   
        Constructs a SYCL \codeinline{image} instance from an OpenCL \codeinline{cl_mem} in accordance with the requirements described in \ref{sec:opencl-interoperability}.
        The instance of the SYCL \codeinline{image} class template being constructed must wait for the SYCL \codeinline{event} parameter, if one is provided, \codeinline{availableEvent} to signal that the \codeinline{cl_mem} instance is ready to be used.
        The SYCL \codeinline{context} parameter \codeinline{syclContext} is the context associated with the memory object.
      }       
\completeTable

%------------------------------------------------------------------------------------------------------

%------------------------------------------------------------------------------------------------------
\startTable{Member function}
\addFootNotes{Member functions of the \codeinline{image} class
  template}
{table.members.image}
  \addRow
    {range<dimensions> get_range() const}
    {
      Return a range object representing the
      size of the image in terms of the number
      of elements in each dimension as passed
      to the constructor.
    }

  \addRow
    {range<dimensions-1> get_pitch() const}
    {
      Available only when: \codeinline{dimensions > 1}.
      \newline
      Return a range object representing the
      pitch of the image in bytes.
    }

  \addRow
    {size_t get_count() const}
    {
      Returns the total number of elements in the image.
      Equal to \codeinline{get_range()[0] * ... * get_range()[dimensions-1]}.
    }

  \addRow
    {size_t get_size() const}
    {
      Returns the size of the image storage in bytes.  The number of
      bytes may be greater than \codeinline{get_count()*element size}
      due to padding of elements, rows and slices of the image for
      efficient access.
    }

  \addRow
    {AllocatorT get_allocator() const}
    {
      Returns the allocator provided to the image.
    }

  \addRowThreeSL
    {template<typename dataT, access::mode accessMode>}
    {accessor<dataT, dimensions, accessMode, access::target::image> }
    {get_access(handler \& commandGroupHandler)}
    {
      Returns a valid accessor to the image with the specified
      access mode and target. The only valid types for \codeinline{dataT} are \codeinline{cl_int4}, \codeinline{cl_uint4}, \codeinline{cl_float4} and \codeinline{cl_half4}.
    }
     \addRowThreeSL
    {template<typename dataT, access::mode accessMode>}
    {accessor<dataT, dimensions, accessMode, access::target::host_image> }
    {get_access()}
    {
      Returns a valid accessor to the image with the specified
      access mode and target. The only valid types for \codeinline{dataT} are \codeinline{cl_int4}, \codeinline{cl_uint4}, \codeinline{cl_float4} and \codeinline{cl_half4}.
    }
   \addRowTwoL
    {template <typename Destination = std::nullptr_t>}
    {void set_final_data(Destination finalData = std::nullptr)}
    {

      The \codeinline{finalData} points to where the output of all
      the image processing is going to be copied to at destruction
      time, if the image was involved with a write accessor.

      Destination can be either an output iterator, a
      \codeinline{weak_ptr_class<T>}.

      Note that a raw pointer is a special case of output iterator and
      thus defines the host memory to which the result is to be
      copied.

      In the case of a weak pointer, the output is not copied if the
      weak pointer has expired.

      If \codeinline{Destination} is \codeinline{std::nullptr_t}, then
      the copy back will not happen.

    }

   \addRow
    {void set_write_back(bool flag = true)}
    {
      This method allows dynamically forcing or canceling the
      write-back of the data of an image on destruction according to
      the value of \codeinline{flag}.

      Forcing the write-back is similar to what happens during a
      normal write-back as described in \S~\ref{sec:image-sync-rules}
      and \ref{sec:sharing-host-memory-with-dm}.

      If there is nowhere to write-back, using this function does not
      have any effect.
    }
\completeInfoTable
%------------------------------------------------------------------------------------------------------

\subsubsection{Image Properties}
\label{sec:image-properties}

The properties that can be provided when constructing the SYCL \codeinline{image} class are describe in Table~\ref{table.properties.image}.

%---------------------------------------------------------------------
\startTable{Property}
\addFootNotes{Properties supported by the SYCL \codeinline{image} class} {table.properties.image}

\addRow
  { property::image::use_host_ptr }
  {
    The \codeinline{use_host_ptr} property adds the requirement that the \gls{sycl-runtime} must not allocate any memory for the \codeinline{image} and instead uses the provided host pointer directly. This prevents the \gls{sycl-runtime} from allocating additional temporary storage on the host.
  }
\addRow
  { property::image::use_mutex }
  {
    The \codeinline{use_mutex} property is valid for the SYCL \codeinline{image} and \codeinline{image} classes. The property adds the requirement that the memory which is owned by the SYCL \codeinline{image} can be shared with the application via a \codeinline{mutex_class} provided to the property. The \codeinline{mutex_class} \codeinline{m} is locked by the runtime whenever the data is in use and unlocked otherwise. Data is synchronized with \codeinline{hostData}, when the \codeinline{mutex_class} is unlocked by the runtime.
  }
\addRow
  { property::image::context_bound }
  {
    The \codeinline{context_bound} property adds the requirement that the SYCL \codeinline{image} can only be associated with a single SYCL \codeinline{context} that is provided to the property.
  }
\completeTable
%---------------------------------------------------------------------

The constructors and member functions of the image property classes
are listed in Tables~\ref{table.constructors.properties.image} and
\ref{table.members.properties.image}

%---------------------------------------------------------------------
\startTable{Constructor}
\addFootNotes{Constructors of the image property classes}
{table.constructors.properties.image}
\addRow
{property::image::use_host_ptr::use_host_ptr()}
{
  Constructs a SYCL \codeinline{use_host_ptr} property instance.
}
\addRow
{property::image::use_mutex::use_mutex(mutex_class \&mutexRef)}
{
  Constructs a SYCL \codeinline{use_mutex} property instance with a reference to \codeinline{mutexRef} parameter provided.
}
\addRow
{property::image::context_bound::context_bound(context boundContext)}
{
  Constructs a SYCL \codeinline{context_bound} property instance with a copy of a SYCL \codeinline{context}.
}
\completeTable
%---------------------------------------------------------------------

%---------------------------------------------------------------------
\startTable{Member function}
\addFootNotes{Member functions of the image property classes}
{table.members.properties.image}
\addRow
{mutex_class *property::image::use_mutex::get_mutex_ptr() const}
{
  Returns the \codeinline{mutex_class} which was specified when
  constructing this SYCL \codeinline{use_mutex} property.
}
\addRow
{context property::image::context_bound::get_context() const}
{
  Returns the \codeinline{context} which was specified when
  constructing this SYCL \codeinline{context_bound} property.
}
\completeTable
%---------------------------------------------------------------------

%***********************************************************************************
% Image Synchronization Rules
%***********************************************************************************

\subsubsection{Image Synchronization Rules}
\label{sec:image-sync-rules}

The rules are similar to those described in
\S~\ref{sec:buf-sync-rules}.

For the lifetime of the image object, the associated host memory must
be left available to the \gls{sycl-runtime} and the contents of the associated
host memory is unspecified until the image object is destroyed. If an
image object value is copied, then only a reference to the underlying
image object is copied. The underlying image object is reference-counted.
Only after all image value references to the underlying image object
have been destroyed is the actual image object itself destroyed.

If an image object is constructed with associated host memory, then
its destructor blocks until all operations in all SYCL queues on
that image object have completed. Any modifications to the image data
will be copied back, if necessary, to the associated host memory.
Any errors occurring during destruction are reported to any associated
context's asynchronous error handler. If an image object is constructed
with a storage object, then the storage object defines what
synchronization or copying behavior occurs on image object destruction.


%***********************************************************************************
% Sharing Host Memory With The SYCL Data Management Classes
%***********************************************************************************

\subsection{Sharing Host Memory With The SYCL Data Management Classes}
\label{sec:sharing-host-memory-with-dm}

In order to allow the \gls{sycl-runtime} to do memory management and allow
for data dependencies, there are two classes defined, buffer and image. The
default behavior for them is that a ``raw'' pointer is given during the
construction of the data management class, with full ownership to use it until
the destruction of the SYCL object.

In this section we go in greater detail on  sharing or explicitly not
sharing host memory with the SYCL data classes, and we will use the buffer
class as an example. The same rules will apply to images as well.

%***********************************************************************************
% Default Behavior
%***********************************************************************************

\subsubsection{Default behavior}

When using a SYCL buffer, the ownership of the pointer passed to the constructor
of the class is, by default, passed to \gls{sycl-runtime}, and that pointer cannot be used
on the host side until the buffer or image is destroyed.
A SYCL application can use memory managed by a SYCL buffer within the buffer scope
by using a \codeinline{host accessor} as defined in~\ref{subsec:accessors}.
However, there is no guarantee that the host accessor synchronizes with the 
original host address used in its constructor.

The pointer passed in is the one used to copy data back to the host, if needed,
before buffer destruction.  The memory pointed by \gls{host-pointer}
will not be de-allocated by the runtime,
and the data is copied back from the device if there is
a need for it.

%***********************************************************************************
% SYCL ownership of the host memory
%***********************************************************************************

\subsubsection{SYCL ownership of the host memory}

In the case where there is host memory to be used for initialization of data
but there is no intention of using that host memory after the buffer is
destroyed, then the buffer can take full ownership of that host memory.

When a buffer owns the \gls{host-pointer} there is no copy back, by default.
In this situation the SYCL application may pass a unique pointer to the host data,
which will be then used by the runtime internally to initialize the data in the device.

If the pointer contained in the \codeinline{unique_ptr} is null, the pointer is initialized
internally in the runtime but no data is copied in.
This will be the generic case of a buffer constructor that takes no host
pointer.

For example, the following could be used:
\begin{code}
{
  cl::sycl::unique_ptr_class<int> ptr { data };
  buffer<int, 1> b { std::move(ptr) };
  // ptr is not valid anymore
  // There is nowhere to copy data back
}
\end{code}

However, optionally the \codeinline{buffer::set_final_data()} can be
set to a \codeinline{weak_ptr_class} to enable copying data
back, to another host memory address that is going to be valid after
buffer construction.

\begin{code}
{
  cl::sycl::unique_ptr_class<int> ptr { data };
  buffer<int, 1> b { std::move(ptr) };
  // ptr is not valid anymore
  // There is nowhere to copy data back
  // To get copy back, a location can be specified:
  b.set_final_data(weak_ptr_class<int> { .... })
}
\end{code}


%***********************************************************************************
% Shared SYCL ownership of the host memory
%***********************************************************************************

\subsubsection{Shared SYCL ownership of the host memory}

When a \codeinline{shared_ptr} is passed to the buffer constructor, then the buffer object and the developer's application share the memory region. If the shared pointer is still used on the application's
side then the data will be copied back from the buffer or image and will be available
to the application after the buffer or image is destroyed.

If the memory pointed to by the shared object is initialized to some data, then
that data is used to initialize the buffer.
If the shared pointer is null, the pointer is initialized by the runtime
internally (and, therefore, the user can use it afterwards in the host).

When the buffer is destroyed and the data have potentially been updated, if the number of copies of the shared pointer
outside the runtime is 0, there is no user-side shared pointer to read the data.
 Therefore the data is not copied out, and the buffer destructor does not need
to wait for the data processes to be finished from OpenCL, as the outcome is not needed
on the application's side.

This behavior can be overiden using the \codeinline{set_final_data()} method of
the buffer class, which will by any means force the buffer destructor to
wait until the data is copied to wherever the \codeinline{set_final_data()} method has
put the data (or not wait nor copy if set final data is
\codeinline{std::nullptr}).

\begin{code}
{
  cl::sycl::shared_ptr_class<int> ptr { data };
  {
    buffer<int, 1> b { ptr, range<2>{ 10, 10 } };
    // update the data
    [...]
  } // Data is copied back because there is an user side shared_ptr
}
\end{code}

\begin{code}
{
  cl::sycl::shared_ptr_class<int> ptr { data };
  {
    buffer<int, 1> b { ptr, range<2>{ 10, 10 } };
    // update the data
    [...]
    ptr.reset();
  } // Data is not copied back, there is no user side shared_ptr.
}
\end{code}


%***********************************************************************************
% Synchronization Primitives
%***********************************************************************************

\subsection{Synchronization Primitives}
\label{Mutex}

When the user wants to use the buffer simultaneously in the \gls{sycl-runtime}
and their own code (e.g. a multi-threaded mechanism) and want to use manual
synchonization without host \glspl{accessor}, a pointer to a \codeinline{mutex_class} can be
passed to the buffer constructor.

The runtime promises to lock the mutex whenever the data is in use and
unlock it when it no longer needs it.

\begin{code}
{
  cl::sycl::mutex_class m;
  auto shD = std::make_shared<int> { 42 }
  {
    buffer<int, 1> b { shD, m };

    std::lock_guard<mutex_class> lck { m };
    // User accesses the data
    do_something(shD);
    /* m is unlock when lck goes out of scope, by normal end of this
       block but also if an exception is thrown for example */
  }
}
\end{code}

When the runtime releases the mutex the user is guaranteed that the data was
copied back on the shared pointer --- unless the final data destination has been
changed using the member function \codeinline{set_final_data()}.

%***********************************************************************************
% Accessors
%***********************************************************************************
\subsection{Accessors}
\label{subsec:accessors}

%***********************************************************************************
% Accessors
%***********************************************************************************

An \gls{accessor} is defined by the SYCL \codeinline{accessor} class template.
An \codeinline{accessor} provides access to the data managed by a \gls{buffer}
or \gls{image}, or to shared \gls{local-memory} allocated by the runtime. An
\gls{accessor} allows users to define \textbf{requirements} to memory objects
(see Section~\ref{sub.section.memmodel.app}).

The SYCL \codeinline{accessor} class template takes five template parameters:

\begin{itemize}
\item A typename specifying the data type that the \codeinline{accessor} is
providing access to.
\item An integer specifying the dimensionality of the accessor.
\item A value of \codeinline{access::mode} specifying the mode of access the
\codeinline{accessor} is providing.
\item A value of \codeinline{access::target} specifying the target of access
the \codeinline{accessor} is providing.
\item A value of \codeinline{access::placeholder} specifying whether the
\codeinline{accessor} is a placeholder accessor.
\end{itemize}

The parameters described above determine the data an \codeinline{accessor}
provides access to and the way in which that access is provided. This separation
allows a \gls{sycl-runtime} implementation to choose an efficient way to provide
access to the data within an execution schedule.

Because of this the interface of the \codeinline{accessor} will
be different depending on the possible combinations of those parameters. There
are three main categories of accessor; buffer accesors (see Section%
~\ref{sub.section.accessors.buffer}), local accessors (see Section%
~\ref{sub.section.accessors.local}) and image accessors (see Section%
~\ref{sub.section.accessors.image}).

%*******************************************************************************
% Access targets
%*******************************************************************************

\subsubsection{Access targets}
\label{sub.section.access.targets}

The access target of an \codeinline{accessor} specifies what the accessor is
providing access to.

The \codeinline{access::target} enumeration, shown in
Table~\ref{interfaces.accesstarget.description}, describes the potential targets
of an \codeinline{accessor}.

\lstinputlisting{headers/accessTarget.h}

%-------------------------------------------------------------------------------
\startTable{access::target}
\addFootNotes{Enumeration of access modes available to accessors}
{interfaces.accesstarget.description}
\addRow
    {access::target::global_buffer}{Access \codeinline{buffer} via \gls{global-memory}.}
  \addRow
    {access::target::constant_buffer}{Access \codeinline{buffer} via \gls{constant-memory}.}
  \addRow
    {access::target::local}{Access work-group \gls{local-memory}.}
  \addRow
    {access::target::image}{Access an \codeinline{image}.}
  \addRow
    {access::target::host_buffer}{Access a \codeinline{buffer} immediately in host code.}
  \addRow
    {access::target::host_image}{Access an \codeinline{image} immediately in host code.}
  \addRow
    {access::target::image_array}{Access an array of \codeinline{image}s on a device.}
\completeTable
%-------------------------------------------------------------------------------

%*******************************************************************************
% Access modes
%*******************************************************************************

\subsubsection{Access modes}
\label{sub.section.access.mode}

The access mode of an \codeinline{accessor} specifies the kind of access that is
being provided. This information is used by the runtime to ensure that any
data dependencies are resolved by enqueuing any data transfers before
or after the execution of a kernel. If a command group contains only
\keyword{discard write mode} accesses to a buffer, then the previous contents
of the buffer (or sub-range of the buffer, if provided) are not
preserved. If a user wants to modify only certain parts of a buffer,
preserving other parts of the buffer, then the user should specify the
exact sub-range of modification of the buffer.
Atomic access is only valid to \codeinline{local}, \codeinline{global_buffer}
and \codeinline{host_buffer} targets (see next section).

The \codeinline{access::mode} enumeration, shown in Table~\ref{interfaces.accessmode.description}, describes the potential modes of an \codeinline{accessor}.

\lstinputlisting{headers/accessMode.h}

%-------------------------------------------------------------------------------
\startTable{access::mode}
\addFootNotes{Enumeration of access modes available to accessors}
{interfaces.accessmode.description}
 \addRow
    {access::mode::read}
    {%
      Read-only access.
    }
  \addRow
    {access::mode::write}
    {%
     Write-only access.
     Previous contents not discarded.
    }
  \addRow
    {access::mode::read_write}
    {%
      Read and write access.
    }
  \addRow
    {access::mode::discard_write}
    {%
      Write-only access.
      Previous contents discarded.
    }
  \addRow
    {access::mode::discard_read_write}
    {%
      Read and write access.
      Previous contents discarded.
    }
  \addRow
    {access::mode::atomic}
    {%
      Read and write atomic access.
    }
\completeTable
%-------------------------------------------------------------------------------

%*******************************************************************************
% Device and host accessors
%*******************************************************************************

\subsubsection{Device and host accessors}

A SYCL \codeinline{accessor} can be a device accessor in which case it provides
access to data within a SYCL kernel function, or a host accessor in which case
it provides immediate access on the host.

If an \codeinline{accessor} has the access target \codeinline{
access::target::global_buffer}, \codeinline{access::target::constant_buffer},
\codeinline{access::target::local}, \codeinline{access::target::image} or
\codeinline{access::target::image_array} then it is considered a device
accessor, and therefore can only be used within a SYCL kernel function and must
be associated with a \gls{command-group}. Creating a device accessor is a
non-blocking operation which defines a requirement on the device and adds the
requirement to the queue.

If an \codeinline{accessor} has the access target \codeinline{
access::target::host_buffer} or \codeinline{access::target::host_image} then it
is considered a host accessor and can only be used on the \gls{host}. Creating a
host accessor is a blocking operation which defines a requirement on the host
and blocks the caller until the requirement is satisfied.

A host accessor provides immediate access and continues to provide access until
it is destroyed.

%*******************************************************************************
% Placeholder accessor
%*******************************************************************************

\subsubsection{Placeholder accessor}
\label{sub.section.access.placeholder}

A placeholder accessor can be constructed outside of a command group and then
later bound to a command group. A SYCL \codeinline{accessor} is considered a
placeholder accessor if it has the access placeholder \codeinline{
access::placeholder::true_t}.

Accessors can optionally be defined as \keyword{placeholder} accessors.
A \keyword{placeholder} accessor defines an accessor instance that is
not bound to a specific \gls{command-group}. The accessor defines only
the type of the accessor (target memory, access mode, base type, \ldots).
When associated with a a command group using the appropriate
handler interface, it defines a \textbf{requirement} for the command group.
The same placeholder accessor can be required by multiple command groups.

The \codeinline{access::placeholder} enumeration, shown in
Table~\ref{interfaces.placeholder.description}, describes the potential
placeholder values of an \codeinline{accessor}.

%-------------------------------------------------------------------------------
\startTable{placeholder::mode}
 \addFootNotes{Enumeration of placeholder values available to accessors}
 {interfaces.placeholder.description}
 \addRow
    {access::placeholder::false_t}
    {
      Non-placeholder accessor.
    }
  \addRow
    {access::placeholder::true_t}
    {
      Placeholder accessor.
    }
\completeTable
%-------------------------------------------------------------------------------

%*******************************************************************************
% Buffer accessor
%*******************************************************************************

\subsubsection{Buffer accessor}
\label{sub.section.accessors.buffer}

A buffer accessor provides access to a SYCL \codeinline{buffer} instance. A SYCL \codeinline{accessor} is considered a buffer accessor if it has the access
target \codeinline{access::target::global_buffer}, \codeinline{
access::target::constant_buffer} or \codeinline{access::target::host_buffer}.

A buffer accessor can provide access to memory managed by a SYCL \codeinline{
buffer} class via either \gls{global-memory} or \gls{constant-memory},
corresponding to the access targets \codeinline{access::target::global_buffer}
and \codeinline{access::target::constant_buffer} respectively. A buffer accessor
accessing a SYCL \codeinline{buffer} via \gls{constant-memory} is restricted by
the available \gls{constant-memory} available on the SYCL \codeinline{device}
being executed on.

Alternatively a buffer accessor can provide access to memory managed by a SYCL
\codeinline{buffer} immediately on the \gls{host}, using the access target
\codeinline{access::target::host_buffer}. If the SYCL \codeinline{buffer} this
SYCL \codeinline{accessor} is accessing was constructed with the property 
\codeinline{property::buffer::use_host_ptr} the address of the memory accessed
on the \gls{host} must be the address the SYCL \codeinline{buffer} was
constructed with, otherwise the \gls{sycl-runtime} is free to allocate temporary
memory to provide access on the \gls{host}.

The data type of a buffer accessor must match that of the SYCL \codeinline{
buffer} which it is accessing.

The dimensionality of a buffer accessor must match that of the SYCL \codeinline{
buffer} which it is accessing, with the exception of \codeinline{0} in which
case the dimensionality of the SYCL \codeinline{buffer} must be \codeinline{1}.

There are three ways a SYCL \codeinline{accessor} can provide access to the
elements of a SYCL \codeinline{buffer}. Firstly by passing a SYCL \codeinline{
id} instance of the same dimensionality as the SYCL \codeinline{accessor}
subscript operator. Secondly by passing a single \codeinline{size_t} value to
multiple consecutive subscript operators (one for each dimension of the SYCL
\codeinline{accessor}, for example \codeinline{acc[id0][id1][id2]}). Finally, in the
case of the SYCL \codeinline{accessor} being \codeinline{0} dimensions, by
triggering the implicit conversion operator. Whenever a multi-dimensional index
is passed to a SYCL \codeinline{accessor} the linear index is calculated based
on the index \codeinline{\{id0, id1, id2\}} provided and the range of the SYCL
\codeinline{accessor} \codeinline{\{r0, r1, r2\}} according to row-major
ordering as follows:

\begin{equation}
\label{row-major-equation-buffer}
 id2 + (id1 \cdot r2) + (id0 \cdot r2 \cdot r1)
\end{equation}

A buffer accessor can optionally provide access to a sub range of a SYCL
\codeinline{buffer} by providing a range and offset on construction. In this
case the \gls{sycl-runtime} will only guarantee the latest copy of the data is
available in that given range and any modifications outside that range are
considered undefined behavior. This allows the \gls{sycl-runtime} to perform
optimizations such as reducing copies between devices. The indexing performed
when a SYCL \codeinline{accessor} provides access to the elements of a SYCL
\codeinline{buffer} is unaffected, i.e, the accessor will continue to index
from \codeinline{\{0,0,0\}}. This allows the offset to be provided either
manually or via the \codeinline{parallel_for} as in
\ref{listing.accessors.range}.

\begin{lstlisting}[label=listing.accessors.range]
    myQueue.submit([&](handler &cgh) {
      auto singleRange = range<3>(8, 16, 16);
      auto offset = id<3>(8, 0, 0);
      // We define the subset of the accessor we require for the kernel
      accessor<int, 1, access::mode::read_write, access::target::global_buffer>
          ptr(syclBuffer, cgh, singleRange, offset);
      // We offset the kernel by the same value to match indexes
      cgh.parallel_for<kernel>(singleRange, offset, [=](item<3> itemID) {
        ptr[itemID.get_linear_id()] = 2;
      });
    });
\end{lstlisting}

A buffer accessor with access target \codeinline{access::target::global_buffer}
can optionally provide atomic access to a SYCL \codeinline{buffer}, using the
access mode \codeinline{access::mode::atomic}, in which case all operators which
return an element of the SYCL \codeinline{buffer} return an instance of the SYCL
\codeinline{atomic} class.

The full list of capabilities that buffer accessors can support is described
in~\ref{table.accessors.buffer.capabilities}.

%-------------------------------------------------------------------------------
\begin{table}[!h]
    \setlength{\extrarowheight}{5pt}\scriptsize
    \begin{tabular}{| p{0.8 in} || p{0.4 in} | p{1.0 in} | p{1.4 in} | p{0.9 in} | p{0.6 in} |}
      \hline
        \cellcolor{lightgray} \textbf{Access target}
        & \cellcolor{lightgray} \textbf{Accessor type}
        & \cellcolor{lightgray} \textbf{Access modes}
        & \cellcolor{lightgray} \textbf{Data types}
        & \cellcolor{lightgray} \textbf{Dimensionalities}        
        & \cellcolor{lightgray} \textbf{Placeholder modes} \\
      \hline
        \tf{global_buffer}
        & device
        & \nlineVI{\tf{read}}{\tf{write}}{\tf{read_write}}{\tf{discard_write}} {\tf{discard_read_write}}{\tf{atomic}}
        & The data type of the SYCL buffer being accessed.
        & Between \tf{0} and \tf{3} (inclusive).
        & \nlineII{\tf{false_t}}{\tf{true_t}} \\
      \hline
        \tf{constant_buffer}
        & device
        & \tf{read}
        & The data type of the SYCL buffer being accessed.
        & Between \tf{0} and \tf{3} (inclusive).
        & \nlineII{\tf{false_t}}{\tf{true_t}} \\
      \hline
        \tf{host_buffer}
        & host
        & \nlineV{\tf{read}}{\tf{write}}{\tf{read_write}}{\tf{discard_write}} {\tf{discard_read_write}}
        & The data type of the SYCL buffer being accessed.
        & Between \tf{0} and \tf{3} (inclusive).
        & \nline{\tf{false_t}} \\
      \hline
    \end{tabular}
    \caption{Description of all the buffer accessor capabilities}
    \label{table.accessors.buffer.capabilities}
\end{table}
%-------------------------------------------------------------------------------

\subsubsection{Buffer accessor interface}

A synopsis of the SYCL \codeinline{accessor} class template buffer
specialization is provided below. The constructors and member functions of the
SYCL \codeinline{accessor} class template buffer specialization are listed in
Tables~\ref{table.constructors.accessor.buffer} and
\ref{table.members.accessor.buffer} respectively. The additional common special
member functions and common member functions are listed in
\ref{sec:reference-semantics} in
Tables~\ref{table.specialmembers.common.reference} and
\ref{table.members.common.reference}, respectively.

%Interface for class: accessor
\lstinputlisting[captionpos=b,caption=Accessor class for buffers, label=accessor.buffer.interface]
{headers/accessorBuffer.h}
\lstset{captionpos=b}

%------------------------------------------------------------------------------
\startTable{Constructor}
\addFootNotes{Constructors of the \codeinline{accessor} class template buffer
  specialization}
{table.constructors.accessor.buffer}
  \addRow
    { accessor(buffer<dataT, 1, AllocatorT> \&bufferRef) }
    {
      Available only when: \codeinline{((isPlaceholder ==
      access::placeholder::false_t \&\& accessTarget ==
      access::target::host_buffer) || (isPlaceholder ==
      access::placeholder::true_t \&\& (accessTarget ==
      access::target::global_buffer || accessTarget ==
      access::target::constant_buffer))) \&\& dimensions == 0}.
      \newline
      If \codeinline{isPlaceholder == access::placeholder::false_t}, constructs
      a SYCL \codeinline{accessor} instance for accessing a single element of a
      SYCL \codeinline{buffer} immediately on the host. If \codeinline{
      isPlaceholder == access::placeholder::true_t}, constructs a SYCL
      placeholder \codeinline{accessor}.
    }
  \addRowTwoL
    { accessor(buffer<dataT, 1, AllocatorT> \&bufferRef, }
    { handler \&commandGroupHandlerRef) }
    {
      Available only when: \codeinline{(isPlaceholder ==
      access::placeholder::false_t \&\& (accessTarget ==
      access::target::global_buffer || accessTarget ==
      access::target::constant_buffer)) \&\& dimensions == 0}.
      \newline
      Constructs a SYCL \codeinline{accessor} instance for accessing a single
      element of a SYCL \codeinline{buffer} within a SYCL kernel function on
      the SYCL \codeinline{queue} associated with \codeinline{
      commandGroupHandlerRef}.
    }
  \addRow
    { accessor(buffer<dataT, dimensions, AllocatorT> \&bufferRef) }
    {
      Available only when: \codeinline{((isPlaceholder ==
      access::placeholder::false_t \&\& accessTarget ==
      access::target::host_buffer) || (isPlaceholder ==
      access::placeholder::true_t \&\& (accessTarget ==
      access::target::global_buffer || accessTarget ==
      access::target::constant_buffer))) \&\& dimensions > 0}.
      \newline
      If \codeinline{isPlaceholder == access::placeholder::false_t}, constructs
      a SYCL \codeinline{accessor} instance for accessing a SYCL \codeinline{
      buffer} immediately on the host. If \codeinline{isPlaceholder ==
      access::placeholder::true_t}, constructs a SYCL placeholder \codeinline{
      accessor}.
    }
  \addRowTwoL
    { accessor(buffer<dataT, dimensions, AllocatorT> \&bufferRef, }
    { handler \&commandGroupHandlerRef) }
    {
      Available only when: \codeinline{(isPlaceholder ==
      access::placeholder::false_t \&\& (accessTarget ==
      access::target::global_buffer || accessTarget ==
      access::target::constant_buffer)) \&\& dimensions > 0}.
      \newline
      Constructs a SYCL \codeinline{accessor} instance for accessing a SYCL
      \codeinline{buffer} within a SYCL kernel function on the SYCL
      \codeinline{queue} associated with \codeinline{commandGroupHandlerRef}.
    }
  \addRowThreeL
    { accessor( buffer<dataT, dimensions, AllocatorT> \&bufferRef, }
    { range<dimensions> accessRange, id<dimensions> }
    { accessOffset = \{\}) }
    {
      Available only when: \codeinline{(isPlaceholder ==
      access::placeholder::false_t \&\& accessTarget ==
      access::target::host_buffer) ||  (isPlaceholder ==
      access::placeholder::true_t \&\& (accessTarget ==
      access::target::global_buffer || accessTarget ==
      access::target::constant_buffer)) \&\& dimensions > 0}.
      \newline
      If \codeinline{isPlaceholder == access::placeholder::false_t}, constructs
      a SYCL \codeinline{accessor} instance for accessing a range of a SYCL
      \codeinline{buffer} immediately on the host. If \codeinline{
      isPlaceholder == access::placeholder::true_t}, constructs a SYCL
      placeholder \codeinline{accessor}.
    }
  \addRowFourL
    { accessor( buffer<dataT, dimensions, AllocatorT> \&bufferRef, }
    { handler \&commandGroupHandlerRef, }
    { range<dimensions> accessRange, }
    { id<dimensions> accessOffset = \{\}) }
    {
      Available only when: \codeinline{(isPlaceholder ==
      access::placeholder::false_t \&\& (accessTarget ==
      access::target::global_buffer || accessTarget ==
      access::target::constant_buffer)) \&\& dimensions > 0}.
      \newline
      Constructs a SYCL \codeinline{accessor} instance for accessing a range of
      SYCL \codeinline{buffer} within a SYCL kernel function on the SYCL
      \codeinline{queue} associated with \codeinline{commandGroupHandlerRef},
      specified by \codeinline{accessRange} and \codeinline{accessOffset}.
    }
\completeTable
%-------------------------------------------------------------------------------

%-------------------------------------------------------------------------------
\startTable{Member function}
\addFootNotes{Member functions of the \codeinline{accessor} class template
  buffer specialization}
{table.members.accessor.buffer}
  \addRow
    { constexpr bool is_placeholder() const }
    {
      Returns \codeinline{true} if \codeinline{isPlaceholder ==
      access::placeholder::true_t} otherwise returns \codeinline{false}.
    }
  \addRow
    { size_t get_size() const }
    {
      Returns the size in bytes of the SYCL \codeinline{buffer} this SYCL
      \codeinline{accessor} is accessing.
    }
  \addRow
    { size_t get_count() const }
    {
      Returns the number of elements of the SYCL \codeinline{buffer} this SYCL \codeinline{accessor} is accessing.
    }
  \addRow
    { range<dimensions> get_range() const }
    {
      Available only when: \codeinline{dimensions > 0}.
      \newline
      Returns the range of this SYCL \codeinline{accessor}.
    }
  \addRow
    { id<dimensions> get_offset() const }
    {
      Available only when: \codeinline{dimensions > 0}.
      \newline
      Returns the offset of this SYCL \codeinline{accessor}.
    }
  \addRow
    { operator dataT \&() const }
    {
      Available only when: \codeinline{(accessMode == access::mode::write ||
      accessMode == access::mode::read_write || accessMode ==
      access::mode::discard_write || accessMode ==
      access::mode::discard_read_write) \&\& dimensions == 0)}.
      \newline
      Returns a reference to the element stored within the SYCL \codeinline{
      buffer} this SYCL \codeinline{accessor} is accessing.
    }
  \addRow
    { dataT \&operator[](id<dimensions> index) const }
    {
      Available only when: \codeinline{(accessMode == access::mode::write ||
      accessMode == access::mode::read_write || accessMode ==
      access::mode::discard_write || accessMode ==
      access::mode::discard_read_write) \&\& dimensions > 0)}.
      \newline
      Returns a reference to the element stored within the SYCL \codeinline{
      buffer} this SYCL \codeinline{accessor} is accessing at the index
      specified by \codeinline{index}.
    }
  \addRow
    { dataT \&operator[](size_t index) const }
    {
      Available only when: \codeinline{(accessMode == access::mode::write ||
      accessMode == access::mode::read_write || accessMode ==
      access::mode::discard_write || accessMode ==
      access::mode::discard_read_write) \&\& dimensions == 1)}.
      \newline
      Returns a reference to the element stored within the SYCL \codeinline{
      buffer} this SYCL \codeinline{accessor} is accessing at the index
      specified by \codeinline{index}.
    }
  \addRow
    { operator dataT() const }
    {
      Available only when: \codeinline{accessMode == access::mode::read \&\&
      dimensions == 0)}.
      \newline
      Returns the value of the element stored within the SYCL \codeinline{
      buffer} this SYCL \codeinline{accessor} is accessing.
    }
  \addRow
    { dataT operator[](id<dimensions> index) const }
    {
      Available only when: \codeinline{accessMode == access::mode::read \&\&
      dimensions > 0)}.
      \newline
      Returns the value of the element stored within the SYCL \codeinline{
      buffer} this SYCL \codeinline{accessor} is accessing at the index
      specified by \codeinline{index}.
    }
  \addRow
    { dataT operator[](size_t index) const }
    {
      Available only when: \codeinline{accessMode == access::mode::read \&\&
      dimensions == 1)}.
      \newline
      Returns the value of the element stored within the SYCL \codeinline{
      buffer} this SYCL \codeinline{accessor} is accessing at the index
      specified by \codeinline{index}.
    }    
  \addRowTwoL
    { operator atomic<dataT, }
    { access::address_space::global_space> () const }
    {
      Available only when: \codeinline{accessMode == access::mode::atomic \&\&
      dimensions == 0)}.
      \newline
      Returns an instance of SYCL \codeinline{atomic} of type \codeinline{dataT}
      providing atomic access to the element stored within the SYCL \codeinline{
      buffer} this SYCL \codeinline{accessor} is accessing.
    }
  \addRowTwoL
    { atomic<dataT, access::address_space::global_space> }
    { operator[](id<dimensions> index) const }
    {
      Available only when: \codeinline{accessMode == access::mode::atomic \&\&
      dimensions > 0)}.
      \newline
      Returns an instance of SYCL \codeinline{atomic} of type \codeinline{dataT}
      providing atomic access to the element stored within the SYCL \codeinline{
      buffer} this SYCL \codeinline{accessor} is accessing at the index
      specified by \codeinline{index}.
    }
  \addRowTwoL
    { atomic<dataT, access::address_space::global_space> }
    { operator[](size_t index) const }
    {
      Available only when: \codeinline{accessMode == access::mode::atomic \&\&
      dimensions == 1)}.
      \newline
      Returns an instance of SYCL \codeinline{atomic} of type \codeinline{dataT}
      providing atomic access to the element stored within the SYCL \codeinline{
      buffer} this SYCL \codeinline{accessor} is accessing at the index
      specified by \codeinline{index}.
    }
  \addRow
    { \__unspecified__ \&operator[](size_t index) const }
    {
      Available only when: \codeinline{dimensions > 1}.
      \newline
      Returns an instance of an undefined intermediate type representing a
      a SYCL \codeinline{accessor} of the same type as this SYCL \codeinline{
      accessor}, with the dimensionality \codeinline{dimensions-1} and
      containing an implicit SYCL \codeinline{id} with index \codeinline{
      dimensions} set to \codeinline{index}. The intermediate type returned
      must provide all available subscript operators which take a \codeinline{
      size_t} parameter defined by the SYCL \codeinline{accessor} class that
      are appropriate for the type it represents (including this subscript
      operator).
    }
  \addRow
    { dataT *get_pointer() const}
    {
      Available only when: \codeinline{accessTarget ==
      access::target::host_buffer}.
      Returns a pointer to the memory this SYCL \codeinline{accessor} memory is
      accessing.
    }
  \addRow
    { global_ptr<dataT> get_pointer() const}
    {
      Available only when: \codeinline{accessTarget ==
      access::target::global_buffer}.
      Returns a pointer to the memory this SYCL \codeinline{accessor} memory is
      accessing.
    }
  \addRow
    { constant_ptr<dataT> get_pointer() const}
    {
      Available only when: \codeinline{accessTarget ==
      access::target::global_buffer}.
      Returns a pointer to the memory this SYCL \codeinline{accessor} memory is
      accessing.
    }
\completeTable
%-------------------------------------------------------------------------------

%*******************************************************************************
% Local accessor
%*******************************************************************************

\subsubsection{Local accessor}
\label{sub.section.accessors.local}

A local accessor provides access to \gls{sycl-runtime} allocated shared memory via
\gls{local-memory}. A SYCL \codeinline{accessor} is considered a local accessor
if it has the access target \codeinline{access::target::local}.
The memory allocated by a local accessor is non-initialised
so it is the user's responsibility to construct and destroy objects explicitly
if required. The \gls{local-memory} that is allocated is shared between
all \glspl{work-item} of a \gls{work-group}.

A local accessor does not provide access on the \gls{host} and the
memory can not be copied back to the \gls{host}.

The data type of a local accessor can be any valid SYCL kernel argument (see
Section~\ref{sec:kernel.parameter.passing}.

The size of memory allocated by the \gls{sycl-runtime} is specified by a SYCL
\codeinline{range} provided on construction. The dimensionality of the SYCL
\codeinline{range} provided must match the SYCL \codeinline{accessor}, with the
exception of \codeinline{0} in which case the dimensionality of the SYCL
\codeinline{range} must be \codeinline{0}.

There are three ways that a SYCL \codeinline{accessor} can provide access to the
elements of the allocated memory. Firstly by passing a SYCL \codeinline{
id} instance of the same dimensionality as the SYCL \codeinline{accessor}
subscript operator. Secondly by passing a single \codeinline{size_t} value to
multiple consecutive subscript operators (one for each dimension of the SYCL
\codeinline{accessor}, for example \codeinline{acc[z][y][x]}). Finally, in the
case of the SYCL \codeinline{accessor} having \codeinline{0} dimensions, by
triggering the implicit conversion operator. Whenever a multi-dimensional index
is passed to a SYCL \codeinline{accessor}, the linear index is calculated based
on the index \codeinline{\{id0, id1, id2\}} provided and the range of the SYCL 
\codeinline{accessor} \codeinline{\{r0, r1, r2\}} according to row-major
ordering as follows:

\begin{equation}
\label{row-major-equation-local}
 id2 + (id1 \cdot r2) + (id0 \cdot r2 \cdot r1)
\end{equation}

A local accessor can optionally provide atomic access to allocated memory,
using the access mode \codeinline{access::mode::atomic}, in which case all
operators which return an element of the allocated memory return an instance of
the SYCL \codeinline{atomic} class.

Local accessors are not valid in the \codeinline{single_task} or basic 
\codeinline{parallel_for} SYCL kernel function invocations, due the fact that
local \glspl{work-group} are implicitly created, and the implementation is free
to choose any size.

The full list of capabilities that local accessors can support is described
in~\ref{table.accessors.local.capabilities}.

%-------------------------------------------------------------------------------
\begin{table}[!h]
    \setlength{\extrarowheight}{5pt}\scriptsize
    \begin{tabular}{| p{0.8 in} || p{0.4 in} | p{1.0 in} | p{1.4 in} | p{0.9 in} | p{0.6 in} |}
      \hline
        \cellcolor{lightgray} \textbf{Access target}
        & \cellcolor{lightgray} \textbf{Accessor type}
        & \cellcolor{lightgray} \textbf{Access modes}
        & \cellcolor{lightgray} \textbf{Data types}
        & \cellcolor{lightgray} \textbf{Dimensionalities}        
        & \cellcolor{lightgray} \textbf{Placeholder modes} \\
      \hline
        \tf{local}
        & device
        & \nlineII{\tf{read_write}}{\tf{atomic}}
        & All available data types supported in a SYCL kernel function.
        & Between \tf{0} and \tf{3} (inclusive).
        & \tf{false_t} \\
      \hline
    \end{tabular}
    \caption{Description of all the local accessor capabilities}
    \label{table.accessors.local.capabilities}
\end{table}
%-------------------------------------------------------------------------------

\subsubsection{Local accessor interface}

A synopsis of the SYCL \codeinline{accessor} class template local
specialization is provided below. The constructors and member functions of the
SYCL \codeinline{accessor} class template local specialization are listed in
Tables~\ref{table.constructors.accessor.local} and
\ref{table.members.accessor.local} respectively. The additional common special
member functions and common member functions are listed in
\ref{sec:reference-semantics} in Tables~\ref{table.specialmembers.common.reference} and
\ref{table.members.common.reference}, respectively.

%Interface for class: accessor
\lstinputlisting[captionpos=b,caption=Accessor class for locals, label=accessor.local.interface]
{headers/accessorLocal.h}
\lstset{captionpos=b}

%-------------------------------------------------------------------------------
\startTable{Constructor}
\addFootNotes {Constructors of the \codeinline{accessor} class template local
  specialization}
{table.constructors.accessor.local}
  \addRow
    { accessor(handler \&commandGroupHandlerRef) }
    {
      Available only when: \codeinline{dimensions == 0}.
      \newline
      Constructs a SYCL \codeinline{accessor} instance for accessing runtime
      allocated shared local memory of a single element (of type \codeinline{dataT}) within
      a SYCL kernel function on the SYCL \codeinline{queue} associated with \codeinline{
      commandGroupHandlerRef}.  The allocation is per work-group.
    }
  \addRowTwoL
    { accessor(range<dimensions> allocationSize, }
    { handler \&commandGroupHandlerRef) }
    {
      Available only when: \codeinline{dimensions > 0}.
      \newline      
      Constructs a SYCL \codeinline{accessor} instance for accessing runtime
      allocated shared local memory of size specified by \codeinline{
      allocationSize} within a SYCL kernel function on the SYCL \codeinline{
      queue} associated with \codeinline{commandGroupHandlerRef}.  \codeinline{allocationSize}
      defines the number of elements of type \codeinline{dataT} to be allocated.  The allocation
      is per work-group, and if multiple work-groups execute simultaneously in an
      implementation, each work-group will receive its own functionally independent
      allocation of size \codeinline{allocationSize} elements of type \codeinline{dataT}.
    }
\completeTable
%-------------------------------------------------------------------------------

%-------------------------------------------------------------------------------
\startTable{Member function}
\addFootNotes{Member functions of the \codeinline{accessor} class template
  local specialization}
{table.members.accessor.local}
  \addRow
    { size_t get_size() const }
    {
      Returns the size in bytes of the local memory allocation, per work-group,
      that this SYCL \codeinline{accessor} is accessing.
    }
  \addRow
    { size_t get_count() const }
    {
      Returns the number of \codeinline{dataT} elements in the local memory allocation, per work-group,
      that this SYCL \codeinline{accessor} is accessing.
    }
  \addRow
    { operator dataT \&() const }
    {
      Available only when: \codeinline{accessMode == access::mode::read_write
      \&\& dimensions == 0)}.
      \newline
      Returns a reference to the single element stored within the work-group's local memory
      allocation that this \codeinline{accessor} is accessing.
    }
  \addRow
    { dataT \&operator[](id<dimensions> index) const }
    {
      Available only when: \codeinline{accessMode == access::mode::read_write
      \&\& dimensions > 0)}.
      \newline
      Returns a reference to the element stored within the work-group's local memory
      allocation that this SYCL \codeinline{accessor} is accessing, at the index
      specified by \codeinline{index}.
    }
  \addRow
    { dataT \&operator[](size_t index) const }
    {
      Available only when: \codeinline{accessMode == access::mode::read_write
      \&\& dimensions == 1)}.
      \newline
      Returns a reference to the element stored within the work-group's local memory
      allocation that this SYCL \codeinline{accessor} is accessing at the index
      specified by \codeinline{index}.
    }
  \addRowTwoL
    { operator atomic<dataT, }
    { access::address_space::local_space> \&() const }
    {
      Available only when: \codeinline{accessMode == access::mode::atomic \&\&
      dimensions == 0)}.
      \newline
      Returns a reference to an instance of SYCL \codeinline{atomic} of type
      \codeinline{dataT} providing atomic access to the element stored within
      the work-group's local memory allocation that this SYCL \codeinline{accessor}
      is accessing.
    }
  \addRowTwoL
    { atomic<dataT, access::address_space::local_space> \& }
    { operator[](id<dimensions> index) const }
    {
      Available only when: \codeinline{accessMode == access::mode::atomic \&\&
      dimensions > 0)}.
      \newline
      Returns a reference to an instance of SYCL \codeinline{atomic} of type
      \codeinline{dataT} providing atomic access to the element stored within
      the work-group's local memory allocation that this SYCL \codeinline{accessor}
      is accessing, at the index specified by \codeinline{index}.
    }
  \addRowTwoL
    { atomic<dataT, access::address_space::local_space> \& }
    { operator[](size_t index) const }
    {
      Available only when: \codeinline{accessMode == access::mode::atomic \&\&
      dimensions == 1)}.
      \newline
      Returns a reference to an instance of SYCL \codeinline{atomic} of type
      \codeinline{dataT} providing atomic access to the element stored within
      the work-group's local memory allocation that this SYCL \codeinline{accessor}
      is accessing, at the index specified by \codeinline{index}.
    }
  \addRow
    { \__unspecified__ \&operator[](size_t index) const }
    {
      Available only when: \codeinline{dimensions > 1}.
      \newline
      Returns an instance of an undefined intermediate type representing a
      a SYCL \codeinline{accessor} of the same type as this SYCL \codeinline{
      accessor}, with the dimensionality \codeinline{dimensions-1} and
      containing an implicit SYCL \codeinline{id} with index \codeinline{
      dimensions} set to \codeinline{index}. The intermediate type returned
      must provide all available subscript operators which take a \codeinline{
      size_t} parameter defined by the SYCL \codeinline{accessor} class that
      are appropriate for the type it represents (including this subscript
      operator).
    }
  \addRow
    { local_ptr<dataT> get_pointer() const}
    {
      Available only when: \codeinline{accessTarget ==
      access::target::local}.
      Returns a pointer to the work-group's local memory allocation that this
      SYCL \codeinline{accessor} is accessing.
    }
\completeTable
%------------------------------------------------------------------------------

%******************************************************************************
% Image accessor
%******************************************************************************

\subsubsection{Image accessor}
\label{sub.section.accessors.image}

An image accessor provides access to a SYCL \codeinline{image} instance. A SYCL \codeinline{accessor} is considered an image accessor if it has the access
target \codeinline{access::target::image}, \codeinline{
access::target::image_array} or \codeinline{access::target::host_image}.
      
An image accessor can provide access to memory managed by a SYCL \codeinline{
image} class, using the access target \codeinline{access::target::image} or
\codeinline{access::target::image_array}.

Alternatively an image accessor can provide access to memory managed by a SYCL
\codeinline{image} immediately on the \gls{host}, using the access target
\codeinline{access::target::host_image}. If the SYCL \codeinline{image} this
SYCL \codeinline{accessor} is accessing was constructed with the property 
\codeinline{property::image::use_host_ptr} the address of the memory accessed
on the \gls{host} must be the address the SYCL \codeinline{image} was
constructed with, otherwise the \gls{sycl-runtime} is free to allocate
temporary memory to provide access on the \gls{host}.

The data type of an image accessor must be either \codeinline{cl_int4}, \codeinline{
cl_uint4}, \codeinline{cl_float4} or \codeinline{cl_half4}.

The dimensionality of an image accessor must match that of the SYCL
\codeinline{image} which it is providing access to, with the exception of when
the access target is \codeinline{access::target::image_array}, in which case
the dimensionality of the SYCL \codeinline{accessor} must be \codeinline{1}
less.

An image accessor with the access target \codeinline{access::target::image} or
\codeinline{access::target::host_image} can provide access to the elements of a
SYCL \codeinline{image} by passing a SYCL \codeinline{cl_int4} or \codeinline{
cl_float4} instance to the \codeinline{read} or \codeinline{write} member
functions. The \codeinline{read} member function optionally takes a SYCL
\codeinline{sampler} instance to perform a sampled read of the image. For
example \codeinline{acc.read(coords, sampler)}.

An image accessor with the access target \codeinline{
access::target::image_array} can provide access to a slice of an image array by
passing a \codeinline{size_t} value to the subscript operator. This returns an
instance of \codeinline{\__image_array_slice__}, an unspecified type providing
the interface of \codeinline{accessor<dataT, dimensions, mode,
access::target::image, access::placeholder::false_t>} which will provide access
to a slice of the image array specified by \codeinline{index}. The
\codeinline{\__image_array_slice__} returned can then provide access via the
\codeinline{read} or \codeinline{write} member functions as described above. For
example \codeinline{acc[arrayIndex].read(coords, sampler)}.

The full list of capabilities that image accessors can support is described
in~\ref{table.accessors.image.capabilities}.

%-------------------------------------------------------------------------------
\begin{table}[!h]
    \setlength{\extrarowheight}{5pt}\scriptsize
    \begin{tabular}{| p{0.8 in} || p{0.4 in} | p{1.0 in} | p{1.4 in} | p{0.9 in} | p{0.6 in} |}
      \hline
        \cellcolor{lightgray} \textbf{Access target}
        & \cellcolor{lightgray} \textbf{Accessor type}
        & \cellcolor{lightgray} \textbf{Access modes}
        & \cellcolor{lightgray} \textbf{Data types}
        & \cellcolor{lightgray} \textbf{Dimensionalities}        
        & \cellcolor{lightgray} \textbf{Placeholder modes} \\
      \hline
        \tf{image}
        & device
        & \tf{\nlineIII{read}{write}{discard_write}}
        & \nlineIV{\tf{cl_int4}}{\tf{cl_uint4}}{\tf{cl_float4}}{\tf{cl_half4}}
        & Between \tf{1} and \tf{3} (inclusive).
        & \tf{false_t} \\
      \hline
        \tf{image_array}
        & device
        & \tf{\nlineIII{read}{write}{discard_write}}
        & \nlineIV{\tf{cl_int4}}{\tf{cl_uint4}}{\tf{cl_float4}}{\tf{cl_half4}}
        & Between \tf{1} and \tf{2} (inclusive).
        & \tf{false_t} \\
      \hline
        \tf{host_image}
        & host
        & \tf{\nlineIV{read}{write}{read_write}{discard_write}}
        & \nlineIV{\tf{cl_int4}}{\tf{cl_uint4}}{\tf{cl_float4}}{\tf{cl_half4}}
        & Between \tf{1} and \tf{3} (inclusive).
        & \tf{false_t} \\
      \hline
    \end{tabular}
    \caption{Description of all the image accessor capabilities}
    \label{table.accessors.image.capabilities}
\end{table}
%-------------------------------------------------------------------------------

\subsubsection{Image accessor interface}

A synopsis of the SYCL \codeinline{accessor} class template image
specialization is provided below. The constructors and member functions of the
SYCL \codeinline{accessor} class template image specialization are listed in
Tables~\ref{table.constructors.accessor.image} and
\ref{table.members.accessor.image} respectively. The additional common special
member functions and common member functions are listed in
\ref{sec:reference-semantics} in
Tables~\ref{table.specialmembers.common.reference} and
\ref{table.members.common.reference}, respectively.

\lstinputlisting[caption=Accessor interface for images,label=images.listing1]{headers/accessorImage.h}

%-------------------------------------------------------------------------------
\startTable{Constructor}
\addFootNotes {Constructors of the \codeinline{accessor} class template
  image specialization}
{table.constructors.accessor.image}
  \addRowThreeL
    { template <typename AllocatorT> }
    { accessor(image<dimensions, AllocatorT> }
    { \&imageRef) }
    {
      Available only when: \codeinline{accessTarget ==
      access::target::host_image}.
      \newline
      Constructs a SYCL \codeinline{accessor} instance for accessing a SYCL
      \codeinline{image} immediately on the host.
    }
  \addRowThreeL
    { template <typename AllocatorT> }
    { accessor(image<dimensions, AllocatorT> }
    { \&imageRef, handler \&commandGroupHandlerRef) }
    {
      Available only when: \codeinline{accessTarget ==
      access::target::image}.
      \newline
      Constructs a SYCL \codeinline{accessor} instance for accessing a SYCL
      \codeinline{image} within a SYCL kernel function on the SYCL
      \codeinline{queue} associated with \codeinline{commandGroupHandlerRef}.
    }
  \addRowThreeL
    { template <typename AllocatorT> }
    { accessor(image<dimensions + 1, AllocatorT> }
    { \&imageRef, handler \&commandGroupHandlerRef) }
    {
      Available only when: \codeinline{accessTarget ==
      access::target::image_array \&\& dimensions < 3}.
      \newline
      Constructs a SYCL \codeinline{accessor} instance for accessing a SYCL
      \codeinline{image} as an array of images, within a SYCL kernel function on
      the SYCL \codeinline{queue} associated with \codeinline{
      commandGroupHandlerRef}.
    }
\completeTable
%-------------------------------------------------------------------------------

%-------------------------------------------------------------------------------
\startTable{Member function}
  \addFootNotes{Member functions of the \codeinline{accessor} class template
    image specialization}
  {table.members.accessor.image}
  \addRow
    { size_t get_size() const }
    {
      Returns the size in bytes of the SYCL \codeinline{image} this SYCL
      \codeinline{accessor} is accessing.
    }
  \addRow
    { size_t get_count() const }
    {
      Returns the number of elements of the SYCL \codeinline{image} this SYCL \codeinline{accessor} is accessing.
    }
  \addRowTwoSL
    { template <typename coordT> }
    { dataT read(const coordT \&coords) const }
    {
      Available only when: \codeinline{(accessTarget == access::target::image \&\&
      accessMode == access::mode::read) || (accessTarget == access::target::host_image \&\&
      (accessMode == access::mode::read || accessMode == access::mode::read_write))}.
      \newline
      Reads and returns an element of the image at the coordinates specified by \codeinline{coords}. Permitted types for \codeinline{coordT} are \codeinline{
      cl_int} when \codeinline{dimensions == 1}, \codeinline{cl_int2} when \codeinline{dimensions ==
      2} and \codeinline{cl_int4} when \codeinline{dimensions == 3}.
    }
  \addRowTwoSL
    { template <typename coordT> }
    { dataT read(const coordT \&coords, const sampler \&smpl) const }
    {
      Available only when: \codeinline{(accessTarget == access::target::image \&\&
      accessMode == access::mode::read) || (accessTarget == access::target::host_image \&\&
      (accessMode == access::mode::read || accessMode == access::mode::read_write))}.
      \newline
      Reads and returns a sampled element of the image at the coordinates
      specified by \codeinline{coords} using the sampler specified by
      \codeinline{smpl}. Permitted types for \codeinline{coordT} are
      \codeinline{cl_int} and \codeinline{cl_float} when \codeinline{dimensions
      == 1}, \codeinline{cl_int2} and \codeinline{cl_float2} when \codeinline{
      dimensions == 2} and \codeinline{cl_int4} and \codeinline{cl_float4} when
      \codeinline{dimensions == 3}.
    }
  \addRowTwoSL
    { template <typename coordT> }    
    { void write(const coordT \&coords, const dataT \&color) const }
    {
      Available only when: \codeinline{(accessTarget == access::target::image \&\&
      (accessMode == access::mode::write || accessMode == access::mode::discard_write)) || 
      (accessTarget == access::target::host_image \&\& (accessMode ==
      access::mode::write || accessMode == access::mode::discard_write ||
      accessMode == access::mode::read_write))}.
      \newline
      Writes the value specified by \codeinline{color} to the element of the
      image at the coordinates specified by \codeinline{coords}. Permitted types
      for \codeinline{coordT} are \codeinline{cl_int} when \codeinline{
      dimensions == 1}, \codeinline{cl_int2} when \codeinline{dimensions == 2}
      and \codeinline{cl_int4} when \codeinline{dimensions == 3}.
    }
  \addRowTwoL
    { \__image_array_slice__ }
    { operator[](size_t index) const }
    {
      Available only when: \codeinline{accessTarget ==
      access::target::image_array \&\& dimensions < 3}.
      \newline
      Returns an instance of \codeinline{\__image_array_slice__}, an
      unspecified type which provides the interface of \codeinline{accessor<
      dataT, dimensions, mode, access::target::image, access::placeholder::
      false_>} which will provide access to a slice of the image array specified
      by \codeinline{index}.
    }
\completeTable

\clearpage


%%% Local Variables:
%%% mode: latex
%%% TeX-master: "sycl-1.2.1"
%%% TeX-auto-untabify: t
%%% TeX-PDF-mode: t
%%% ispell-local-dictionary: "american"
%%% End:


%***********************************************************************************
% Address space classes
%***********************************************************************************

\subsection{Address space classes}

In OpenCL, there are four different address spaces. These are: global,
local, constant and private.  In OpenCL C, these address spaces are
manually specified using OpenCL-specific keywords. In SYCL, the device
compiler is expected to auto-deduce the address space for pointers in
common situations of pointer usage. However, there are situations
where auto-deduction is not possible. Here are the most common
situations:
\begin{itemize}
  \item
    When linking SYCL kernels with OpenCL C functions. In this case,
    it is necessary to specify the address space for any pointer
    parameters when declaring an \codeinline{extern \"C\"} function.

  \item
    When declaring data structures with pointers inside, it is not
    possible for the SYCL compiler to deduce at the time of
    declaration of the data structure what address space pointer
    values assigned to members of the structure will be. So, in this
    case, the address spaces will have to be explicitly declared by
    the developer.

  \item
    When a pointer is declared as a variable, but not initialized,
    then address space deduction is not automatic and so an explicit
    pointer class should be used, or the pointer should be initialized
    at declaration.
\end{itemize}

Direct declaration of pointers with address spaces is discouraged as the
definition is implementation defined. Users must rely on the
\codeinline{multi_ptr} class to handle address space boundaries and
interoperability.

%-------------------------------------------------------------------------------
%*******************************************************************************
% Multi-pointer class
%*******************************************************************************

\subsubsection{Multi-pointer class}
\label{sec:multiptr}
The multi-pointer class is the common interface for the explicit pointer
classes, defined in \ref{sec:pointerclasses}.

There are situations where a user may want to template a data structure by an
address space. Or, a user may want to write templates that adapt to the address
space of a pointer. An example might be wrapping a pointer inside a class, where
a user may need to template the class according to the address space of the
pointer the class is initialized with. In this case, the \codeinline{multi_ptr}
class enables users to do this.

In order to facilitate SYCL/OpenCL C
interoperability, the \codeinline{pointer} type is provided. It is an
implementation defined type which corresponds to the underlying OpenCL C
pointer type and can be used in \codeinline{extern "C"} function
declarations for OpenCL functions used in SYCL kernels.
\codeinline{multi_ptr} class defines a \codeinline{get} member function that returns the underlying OpenCL C pointer.

The \codeinline{multi_ptr} class provides constructors for address space qualified and non address space qualified pointers to allow interoperability between plain C++ and OpenCL C.
Implementations should reject programs that try assign a pointer with an address space not consistent with the address space represented by the \codeinline{multi_ptr} specialization.

It is possible to use the \codeinline{void} type for the \codeinline{multi_ptr}
class, but in that case some functionality is disabled.
\codeinline{multi_ptr<void>} does not provide the \codeinline{reference} or
\codeinline{const_reference} types, the access operators
(\codeinline{operator*()}, \codeinline{operator->()}), the arithmetic
operators or \codeinline{prefetch} member function.
Conversions from \codeinline{multi_ptr} to \codeinline{multi_ptr<void>} of the
same address space are allowed, and will occur implicitly.
Conversions from \codeinline{multi_ptr<void>} to any other
\codeinline{multi_ptr} type of the same address space
are allowed, but must be explicit.
The same rules apply to \codeinline{multi_ptr<const void>}.

An overview of the interface provided for the \codeinline{multi_ptr} class
follows.

\lstinputlisting{headers/multipointer.h}

%-------------------------------------------------------------------------------
\startTable{Constructor}
\addFootNotes{Constructors of the SYCL \codeinline{multi_ptr} class template}
{table.constructors.multiptr}
\addRowTwoL
{template <typename ElementType, access::address_space Space>}
{multi_ptr()}
{Default constructor.}
\addRowTwoL
{template <typename ElementType, access::address_space Space>}
{multi_ptr(const multi_ptr \&)}
{Copy constructor.}
\addRowTwoL
{template <typename ElementType, access::address_space Space>}
{multi_ptr(multi_ptr\&\&)}
{Move constructor.}
\addRowTwoL
{template <typename ElementType, access::address_space Space>}
{multi_ptr(pointer_t)}
{Constructor that takes as an argument a pointer of type \codeinline{ElementType}.}
\addRowTwoL
{template <typename ElementType, access::address_space Space>}
{multi_ptr(ElementType*)}
{Constructor that takes as an argument a pointer of type \codeinline{ElementType}.
An implementation should reject an argument if the deduced address space is not compatible with \codeinline{Space}.}
\addRowTwoL
{template <typename ElementType, access::address_space Space>}
{multi_ptr(std::nullptr_t)}
{Constructor from a \codeinline{nullptr}.}
\addRowFourL
{template <typename ElementType,
access::address_space Space = access::address_space::global_space>}
{template <int dimensions, access::mode Mode>}
{multi_ptr(}
{accessor<ElementType, dimensions, Mode, access::target::global_buffer>)}
{Constructs a
\codeinline{multi_ptr<ElementType, access::address_space::global_space>}
from an accessor of \codeinline{access::target::global_buffer}.}
\addRowFourL
{template <typename ElementType,
access::address_space Space = access::address_space::local_space>}
{template <int dimensions, access::mode Mode>}
{multi_ptr(}
{accessor<ElementType, dimensions, Mode, access::target::local>)}
{Constructs a
\codeinline{multi_ptr<ElementType, access::address_space::local_space>}
from an accessor of \codeinline{access::target::local}.}
\addRowFourL
{template <typename ElementType,
access::address_space Space = access::address_space::constant_space>}
{template <int dimensions, access::mode Mode>}
{multi_ptr(}
{accessor<ElementType, dimensions, Mode, access::target::constant_buffer>)}
{Constructs a
\codeinline{multi_ptr<ElementType, access::address_space::constant_space>}
from an accessor of \codeinline{access::target::constant_buffer}.}
\addRowTwoL
{template <typename ElementType, access::address_space Space>}
{multi_ptr<ElementType, Space> make_ptr(ElementType*)}
{
Global function to create a \codeinline{multi_ptr} instance depending
on the address space of the \codeinline{pointer} type.
An implementation must reject an argument if the deduced address space is not compatible with \codeinline{Space}.}
\addRowTwoL
{template <typename ElementType, access::address_space Space>}
{multi_ptr<ElementType, Space> make_ptr(multi_ptr<ElementType, Space>::pointer_t)}
{
Global function to create a \codeinline{multi_ptr} instance from
an OpenCL C \codeinline{pointer}.}
\completeTable
%-------------------------------------------------------------------------------

%-------------------------------------------------------------------------------
\startTable{Member function}
\addFootNotes{Member functions of \codeinline{multi_ptr} class}{table.multiptr.members}
\addRowTwoL
{template <typename ElementType, access::address_space Space>}
{multi_ptr \&operator=(const multi_ptr\&)}
{Copy assignment operator.}
\addRowTwoL
{template <typename ElementType, access::address_space Space>}
{multi_ptr \&operator=(multi_ptr\&\&)}
{Move assignment operator.}
\addRowTwoL
{template <typename ElementType, access::address_space Space>}
{multi_ptr \&operator=(pointer_t)}
{Assigns a pointer of \codeinline{ElementType} to the \codeinline{multi_ptr}.}
\addRowTwoL
{template <typename ElementType, access::address_space Space>}
{multi_ptr \&operator=(ElementType*)}
{Assigns a pointer of type \codeinline{ElementType}.
An implementation should reject an argument if the deduced address space is not compatible with \codeinline{Space}.}
\addRowTwoL
{template <typename ElementType, access::address_space Space>}
{multi_ptr \&operator=(std::nullptr_t)}
{Assigns \codeinline{nullptr} to the \codeinline{multi_ptr}.}
\addRowTwoL
{template <typename ElementType, access::address_space Space>}
{ElementType\& operator*() const}
{Available only when: \codeinline{!std::is_void<ElementType>::value}.
\newline
Operator that returns a reference to the \codeinline{ElementType}
of the \codeinline{multi_ptr} class.}
\addRowTwoL
{template <typename ElementType, access::address_space Space>}
{ElementType* operator->() const}
{Available only when: \codeinline{!std::is_void<ElementType>::value}.
\newline
Returns the underlying pointer.}
\addRowTwoL
{template <typename ElementType, access::address_space Space>}
{pointer_t get() const}
{Returns the underlying OpenCL C pointer.}
\addRowTwoL
{template <typename ElementType, access::address_space Space>}
{operator ElementType*() const}
{Implicit conversion to the underlying pointer type.}
\addRowTwoL
{template <typename ElementType, access::address_space Space>}
{operator multi_ptr<void, Space>() const}
{Available only when:
\codeinline{!std::is_void<ElementType>::value \&\& !std::is_const<ElementType>::value}.
\newline
Implicit conversion to a \codeinline{multi_ptr} of type \codeinline{void}.}
\addRowTwoL
{template <typename ElementType, access::address_space Space>}
{operator multi_ptr<const void, Space>() const}
{Available only when:
\codeinline{!std::is_void<ElementType>::value \&\& std::is_const<ElementType>::value}.
\newline
Implicit conversion to a \codeinline{multi_ptr} of type \codeinline{const void}.}
\addRowTwoL
{template <access::address_space Space>}
{operator multi_ptr<const ElementType, Space>() const}
{Implicit conversion to a \codeinline{multi_ptr}
of type \codeinline{const ElementType}.}
\addRowThreeL
{template <access::address_space Space>}
{template <typename ElementType>}
{explicit operator multi_ptr<ElementType, Space>() const}
{Available only for the \codeinline{multi_ptr<void>}
and \codeinline{multi_ptr<const void>} specializations.
\newline
Explicit conversion of a \codeinline{multi_ptr<void>}
or \codeinline{multi_ptr<const void>} pointer object to a
\codeinline{multi_ptr} of type \codeinline{ElementType}.}
The conversion must retain the \codeinline{const} qualifier.
\addRowTwoL
{template <typename ElementType, access::address_space Space>}
{multi_ptr\& operator++()}
{Available only when: \codeinline{!std::is_void<ElementType>::value}.
\newline
Increments the pointer by 1.}
\addRowTwoL
{template <typename ElementType, access::address_space Space>}
{multi_ptr operator++(int)}
{Available only when: \codeinline{!std::is_void<ElementType>::value}.
\newline
Increments the pointer by 1 and returns a new \codeinline{multi_ptr}
with the value of the previous pointer.}
\addRowTwoL
{template <typename ElementType, access::address_space Space>}
{multi_ptr\& operator--()}
{Available only when: \codeinline{!std::is_void<ElementType>::value}.
\newline
Decrements the pointer by 1.}
\addRowTwoL
{template <typename ElementType, access::address_space Space>}
{multi_ptr operator--(int)}
{Available only when: \codeinline{!std::is_void<ElementType>::value}.
\newline
Decrements the pointer by 1 and returns a new \codeinline{multi_ptr}
with the value of the previous pointer.}
\addRowTwoL
{template <typename ElementType, access::address_space Space>}
{multi_ptr\& operator+=(difference_type r)}
{Available only when: \codeinline{!std::is_void<ElementType>::value}.
\newline
Moves the pointer forward by \codeinline{r}.}
\addRowTwoL
{template <typename ElementType, access::address_space Space>}
{multi_ptr\& operator-=(difference_type r)}
{Available only when: \codeinline{!std::is_void<ElementType>::value}.
\newline
Moves the pointer backward by \codeinline{r}.}
\addRowTwoL
{template <typename ElementType, access::address_space Space>}
{multi_ptr operator+(difference_type r) const}
{Available only when: \codeinline{!std::is_void<ElementType>::value}.
\newline
Creates a new \codeinline{multi_ptr} that points \codeinline{r} forward
compared to \codeinline{*this}.}
\addRowTwoL
{template <typename ElementType, access::address_space Space>}
{multi_ptr operator-(difference_type r) const}
{Available only when: \codeinline{!std::is_void<ElementType>::value}.
\newline
Creates a new \codeinline{multi_ptr} that points \codeinline{r} backward
compared to \codeinline{*this}.}
  \addRow
    { void prefetch(size_t numElements) const }
    {
      Available only when: \codeinline{Space ==
      access::address_space::global_space}.
      \newline
      Prefetches a number of elements specified by \codeinline{numElements} into
      the \gls{global-memory} cache. This operation is an implementation defined
      optimization and does not effect the functional  behavior of the SYCL
      kernel function.
    }
\completeTable
%-------------------------------------------------------------------------------

%-------------------------------------------------------------------------------
\startTable{Non-member function}
\addFootNotes{Non-member functions of the \codeinline{multi_ptr} class}
{table.multipointer.nonmemberfunctions}
\addRowThreeL
{template <typename ElementType, access::address_space Space>}
{bool operator==(const multi_ptr<ElementType, Space>\& lhs,}
{                const multi_ptr<ElementType, Space>\& rhs)}
{Comparison operator \codeinline{==} for \codeinline{multi_ptr} class.}
\addRowThreeL
{template <typename ElementType, access::address_space Space>}
{bool operator!=(const multi_ptr<ElementType, Space>\& lhs,}
{                const multi_ptr<ElementType, Space>\& rhs)}
{Comparison operator \codeinline{!=} for \codeinline{multi_ptr} class.}
\addRowThreeL
{template <typename ElementType, access::address_space Space>}
{bool operator<(const multi_ptr<ElementType, Space>\& lhs,}
{               const multi_ptr<ElementType, Space>\& rhs)}
{Comparison operator \codeinline{<} for \codeinline{multi_ptr} class.}
\addRowThreeL
{template <typename ElementType, access::address_space Space>}
{bool operator>(const multi_ptr<ElementType, Space>\& lhs,}
               {const multi_ptr<ElementType, Space>\& rhs)}
{Comparison operator \codeinline{>} for \codeinline{multi_ptr} class.}
\addRowThreeL
{template <typename ElementType, access::address_space Space>}
{bool operator<=(const multi_ptr<ElementType, Space>\& lhs,}
{                const multi_ptr<ElementType, Space>\& rhs)}
{Comparison operator \codeinline{<=} for \codeinline{multi_ptr} class.}
\addRowThreeL
{template <typename ElementType, access::address_space Space>}
{bool operator>=(const multi_ptr<ElementType, Space>\& lhs,}
                {const multi_ptr<ElementType, Space>\& rhs)}
{Comparison operator \codeinline{>=} for \codeinline{multi_ptr} class.}
\addRowTwoL
{template <typename ElementType, access::address_space Space>}
{bool operator!=(const multi_ptr<ElementType, Space>\& lhs, std::nullptr_t rhs)}
{Comparison operator \codeinline{!=} for \codeinline{multi_ptr} class with a
\codeinline{std::nullptr_t}.}

\addRowTwoL
{template <typename ElementType, access::address_space Space>}
{bool operator!=(std::nullptr_t lhs, const multi_ptr<ElementType, Space>\& rhs)}
{Comparison operator \codeinline{!=} for \codeinline{multi_ptr} class with a
\codeinline{std::nullptr_t}.}

\addRowTwoL
{template <typename ElementType, access::address_space Space>}
{bool operator==(const multi_ptr<ElementType, Space>\& lhs, std::nullptr_t rhs)}
{Comparison operator \codeinline{==} for \codeinline{multi_ptr} class with a
\codeinline{std::nullptr_t}.}

\addRowTwoL
{template <typename ElementType, access::address_space Space>}
{bool operator==(std::nullptr_t lhs, const multi_ptr<ElementType, Space>\& rhs)}
{Comparison operator \codeinline{==} for \codeinline{multi_ptr} class with a
\codeinline{std::nullptr_t}.}

\addRowTwoL
{template <typename ElementType, access::address_space Space>}
{bool operator>(const multi_ptr<ElementType, Space>\& lhs, std::nullptr_t rhs)}
{Comparison operator \codeinline{>} for \codeinline{multi_ptr} class with a
\codeinline{std::nullptr_t}.}

\addRowTwoL
{template <typename ElementType, access::address_space Space>}
{bool operator>(std::nullptr_t lhs, const multi_ptr<ElementType, Space>\& rhs)}
{Comparison operator \codeinline{>} for \codeinline{multi_ptr} class with a
\codeinline{std::nullptr_t}.}

\addRowTwoL
{template <typename ElementType, access::address_space Space>}
{bool operator<(const multi_ptr<ElementType, Space>\& lhs, std::nullptr_t rhs)}
{Comparison operator \codeinline{<} for \codeinline{multi_ptr} class with a
\codeinline{std::nullptr_t}.}

\addRowTwoL
{template <typename ElementType, access::address_space Space>}
{bool operator<(std::nullptr_t lhs, const multi_ptr<ElementType, Space>\& rhs)}
{Comparison operator \codeinline{<} for \codeinline{multi_ptr} class with a
\codeinline{std::nullptr_t}.}

\addRowTwoL
{template <typename ElementType, access::address_space Space>}
{bool operator>=(const multi_ptr<ElementType, Space>\& lhs, std::nullptr_t rhs)}
{Comparison operator \codeinline{>=} for \codeinline{multi_ptr} class with a
\codeinline{std::nullptr_t}.}

\addRowTwoL
{template <typename ElementType, access::address_space Space>}
{bool operator>=(std::nullptr_t lhs, const multi_ptr<ElementType, Space>\& rhs)}
{Comparison operator \codeinline{>=} for \codeinline{multi_ptr} class with a
\codeinline{std::nullptr_t}.}

\addRowTwoL
{template <typename ElementType, access::address_space Space>}
{bool operator<=(const multi_ptr<ElementType, Space>\& lhs, std::nullptr_t rhs)}
{Comparison operator \codeinline{<=} for \codeinline{multi_ptr} class with a
\codeinline{std::nullptr_t}.}

\addRowTwoL
{template <typename ElementType, access::address_space Space>}
{bool operator<=(std::nullptr_t lhs, const multi_ptr<ElementType, Space>\& rhs)}
{Comparison operator \codeinline{<=} for \codeinline{multi_ptr} class with a
\codeinline{std::nullptr_t}.}

\completeTable
%*******************************************************************************
% Explicit pointer classes
%*******************************************************************************
\subsubsection{Explicit pointer aliases}
\label{sec:pointerclasses}

SYCL provides aliases to the \codeinline{multi_ptr} class template (see Section
\ref{sec:multiptr}) for each specialization of \codeinline{
access::address_space}.

A synopsis of the SYCL \codeinline{multi_ptr} class template 
aliases is provided below.

%Interface of the explicit pointer classes
\lstinputlisting{headers/pointer.h}

%*******************************************************************************
% Samplers
%*******************************************************************************
\subsection{Samplers}
\label{subsec:samplers}

The SYCL \codeinline{sampler} class encapsulates a configuration for sampling an image \codeinline{accessor}. A SYCL \codeinline{sampler} may be an OpenCL sampler, in which case it must encapsulate a valid underlying OpenCL \codeinline{cl_sampler}, or it may be a host sampler, in which case it must not.

The constructors and member functions of the SYCL \codeinline{sampler} class are listed in Tables~\ref{table.constructors.sampler} and \ref{table.members.sampler}, respectively. The additional common special member functions and common member functions are listed in Tables~\ref{table.specialmembers.common.reference} and \ref{table.members.common.reference}, respectively.

The members of the \codeinline{sampler} class that provide information on the sampler (\codeinline{get_addressing_mode()}, \codeinline{get_filtering_mode()}, \codeinline{get_coordinate_normalization_mode()}) are callable from host code.  Invoking these queries within device kernel code produces undefined results.

% Interface of the sampler class
\lstinputlisting{headers/sampler.h}

%------------------------------------------------------------------------------------------------------
\startTable{addressing_mode}
  \addFootNotes{Addressing modes description}{table.addressing.mode.sampler}

  \addRow
  {mirrored_repeat}
  {
    Out of range coordinates will be flipped at every integer junction. This addressing mode
    can only be used with normalized coordinates. If normalized coordinates are not used, this
    addressing mode may generate image coordinates that are undefined.
  }

  \addRow
  {repeat}
  {
    Out of range image coordinates are wrapped to the valid range. This addressing mode can only
    be used with normalized coordinates. If normalized coordinates are not used, this
    addressing mode may generate image coordinates that are undefined.
  }

  \addRow
  {clamp_to_edge}
  {
    Out of range image coordinates are clamped to the extent.
  }

  \addRow
  {clamp}
  {
    Out of range image coordinates will return a border color.
  }

  \addRow
  {none}
  {
    For this addressing mode the programmer guarantees that the image coordinates used to
    sample elements of the image refer to a location inside the image; otherwise the results are
    undefined.
  }
\completeTable
%------------------------------------------------------------------------------------------------------

%------------------------------------------------------------------------------------------------------
\startTable{filtering_mode}
  \addFootNotes{Filtering modes description}{table.filtering.mode.sampler}
  \addRow
  {nearest}
  {
    Chooses a color of nearest pixel.
  }
  \addRow
  {linear}
  {
    Performs a linear sampling of adjacent pixels.
  }
\completeTable
%------------------------------------------------------------------------------------------------------

%------------------------------------------------------------------------------------------------------
\startTable{coordinate_normalization_mode}
  \addFootNotes{Coordinate normalization modes description}{table.normalization.mode.sampler}
  \addRow
  {normalized}
  {
    Normalizes image coordinates.
  }
  \addRow
  {unnormalized}
  {
    Does not normalize image coordinates.
  }
\completeTable
%------------------------------------------------------------------------------------------------------

%------------------------------------------------------------------------------------------------------
\startTable{Constructor}
\addFootNotes {Constructors the \codeinline{sampler} class}
{table.constructors.sampler}
  \addRowFourL
    {sampler(}
    {  coordinate_normalization_mode normalizationMode,}
    {  addressing_mode addressingMode,}
    {  filtering_mode filteringMode)}
    {
      Constructs a SYCL \codeinline{sampler} instance with address mode, filtering mode and coordinate normalization mode specified by the respective parameters.
      It is not valid to construct a SYCL \codeinline{sampler} within a SYCL kernel function.
    }
  \addRowTwoL
    { sampler(cl_sampler clSampler, }
    { const context \&syclContext) }
    {  
      Constructs a SYCL \codeinline{sampler} instance from an OpenCL \codeinline{cl_sampler} in accordance with the requirements described in \ref{sec:opencl-interoperability}.
    }
\completeTable
%------------------------------------------------------------------------------------------------------

%------------------------------------------------------------------------------------------------------
\startTable{Member function}
\addFootNotes{Member functions for the \codeinline{sampler} class}
{table.members.sampler}
  \addRow
    {addressing_mode get_addressing_mode() const}
    {
       Return the addressing mode used to construct this SYCL \codeinline{sampler}.
    }
  \addRow
    {filtering_mode get_filtering_mode() const}
    {
        Return the filtering mode used to construct this SYCL \codeinline{sampler}.
    }
  \addRow
    {coordinate_normalization_mode get_coordinate_normalization_mode() const}
    {
       Return the coordinate normalization mode used to construct this SYCL \codeinline{sampler}.
    }
\completeTable
%------------------------------------------------------------------------------------------------------

%***********************************************************************************
% Expressing parallelism through kernels
%***********************************************************************************
\section{Expressing parallelism through kernels}
\label{sec:expr-parall-thro}
%!TEX root = sycl-1.2.1.tex
%***********************************************************************************
% Expressing parallelism through kernels
%***********************************************************************************
%\section{Expressing parallelism through kernels}

%***********************************************************************************
% Ranges and index space identifiers
%***********************************************************************************

\subsection{Ranges and index space identifiers}
\label{ranges-identifiers}

The data parallelism of the OpenCL execution model and its exposure
through SYCL requires instantiation of a parallel execution over a
range of iteration space coordinates. To achieve this we expose types
to define the range of execution and to identify a given execution
instance's point in the iteration space.

To achieve this we expose seven types: \codeinline{range},
\codeinline{nd_range}, \codeinline{id}, \codeinline{item}, \codeinline{h_item},
\codeinline{nd_item} and \codeinline{group}.

When constructing ids or ranges from integers, the elements are
written in row-major format.

%-------------------------------------------------------------------------------
\startTable{Type}
\addFootNotes{Summary of types used to identify points in an index space, and ranges over which those points can vary}
{table.id.summary}
  \addRow {id} {A point within a range}
  \addRow {range} {Bounds over which an \codeinline{id} may vary}
  \addRow {item} {Pairing of an \codeinline{id} (specific point) and the \codeinline{range} that it is bounded by}
  \addRow {nd_range} {Encapsules both global and local (work-group size) \mbox{\codeinline{range}s} over which work-item \mbox{\codeinline{id}s} will vary}
  \addRow {nd_item} {Encapsulates two \mbox{\codeinline{item}s}, one for global \codeinline{id} and \codeinline{range}, and one for local \codeinline{id} and \codeinline{range}}
  \addRow {h_item} {Index point queries within hierarchical parallelism (\codeinline{parallel_for_work_item}).  Encapsulates physical global and local \mbox{\codeinline{id}s} and \mbox{\codeinline{range}s}, as well as a logical local \codeinline{id} and \codeinline{range} defined by hierarchical parallelism}
  \addRow {group} {Work-group queries within hierarchical parallelism (\codeinline{parallel_for_work_group}), and exposes the \codeinline{parallel_for_work_item} construct that identifies code to be executed by each work-item.  Encapsulates work-group \mbox{\codeinline{id}s} and \mbox{\codeinline{range}s}}
\completeTable
%-------------------------------------------------------------------------------

%*******************************************************************************
% Range class
%*******************************************************************************

\subsubsection{\texttt{range} class}
\label{range-class}

\tclass{range}{\tf{int} dimensions} is a 1D, 2D or 3D vector that defines
the iteration domain of either a single work-group in a parallel
dispatch, or the overall dimensions of the dispatch. It can be
constructed from integers.

The SYCL \codeinline{range} class template provides the common by-value
semantics (see Section~\ref{sec:byval-semantics}).

A synopsis of the SYCL \codeinline{range} class is provided below. The constructors, member functions and non-member functions of the SYCL \codeinline{range} class are listed in Tables~\ref{table.constructors.id}, \ref{table.members.range} and \ref{table.functions.range} respectively. The additional common special member functions and common member functions are listed in \ref{sec:byval-semantics} in Tables~\ref{table.specialmembers.common.byval} and \ref{table.members.common.byval} respectively.

\lstinputlisting{headers/range.h}

%-------------------------------------------------------------------------------
\startTable{Constructor}
\addFootNotes{Constructors of the \codeinline{range} class template}
{table.constructors.range}
  \addRow
    {range(size_t dim0)}
    {
        Construct a 1D range with value dim0.
        Only valid when the template parameter \codeinline{dimensions} is equal
          to 1.
    }
  \addRow
    {range(size_t dim0, size_t dim1)}
    {
        Construct a 2D range with values dim0 and dim1.
        Only valid when the template parameter \codeinline{dimensions} is equal
        to 2.
    }
  \addRow
    {range(size_t dim0, size_t dim1, size_t dim2)}
    {
        Construct a 3D range with values dim0, dim1 and dim2.
        Only valid when the template parameter \codeinline{dimensions} is equal
  to 3.
    }
\completeTable
%-------------------------------------------------------------------------------

%-------------------------------------------------------------------------------
\startTable{Member function}
\addFootNotes{Member functions of the \codeinline{range} class
  template}
{table.members.range}
  \addRow
    {size_t get(int dimension) const}
    {
        Return the value of the specified dimension of the
        \codeinline{range}.
    }
  \addRow
    {size_t \&operator[](int dimension)}
    {
        Return the l-value of the specified dimension of the
        \codeinline{range}.
    }
  \addRow
    {size_t operator[](int dimension) const}
    {
        Return the value of the specified dimension of the
        \codeinline{range}.
    }
  \addRow
    {size_t size() const}
    {
        Return the size of the range computed as dimension0*...*dimensionN.
    }
  \addRow
    { range<dimensions> operatorOP(const range<dimensions> \&rhs) const }
    {
      Where \codeinline{OP} is: \codeinline{+}, \codeinline{-}, \codeinline{*},
      \codeinline{/}, \codeinline{\%}, \codeinline{<<}, \codeinline{>>},
      \codeinline{\&}, \codeinline{|}, \codeinline{^}, \codeinline{\&\&},
      \codeinline{||}, \codeinline{<}, \codeinline{>}, \codeinline{<=},
      \codeinline{>=}.
      \newline
      Constructs and returns a new instance of the SYCL \codeinline{range} class
      template with the same dimensionality as this SYCL \codeinline{range},
      where each element of the new SYCL \codeinline{range} instance is the
      result of an element-wise \codeinline{OP} operator between each element of
      this SYCL \codeinline{range} and each element of the \codeinline{rhs}
      \codeinline{range}. If the operator returns a \codeinline{bool} the result
      is the cast to \codeinline{size_t}.
    }
  \addRow
    { range<dimensions> operatorOP(const size_t \&rhs) const }
    {
      Where \codeinline{OP} is: \codeinline{+}, \codeinline{-}, \codeinline{*},
      \codeinline{/}, \codeinline{\%}, \codeinline{<<}, \codeinline{>>},
      \codeinline{\&}, \codeinline{|}, \codeinline{^}, \codeinline{\&\&},
      \codeinline{||}, \codeinline{<}, \codeinline{>}, \codeinline{<=},
      \codeinline{>=}.
      \newline
      Constructs and returns a new instance of the SYCL \codeinline{range} class
      template with the same dimensionality as this SYCL \codeinline{range},
      where each element of the new SYCL \codeinline{range} instance is the result
      of an element-wise \codeinline{OP} operator between each element of this
      SYCL \codeinline{range} and the \codeinline{rhs} \codeinline{size_t}. If
      the operator returns a \codeinline{bool} the result is the cast to
      \codeinline{size_t}.
    }
  \addRow
    { range<dimensions> \&operatorOP(const range<dimensions> \&rhs) }
    {
      Where \codeinline{OP} is: \codeinline{+=}, \codeinline{-=},\codeinline{
      *=}, \codeinline{/=}, \codeinline{\%=}, \codeinline{<<=}, \codeinline{
      >>=}, \codeinline{\&=}, \codeinline{|=}, \codeinline{^=}.
      \newline
      Assigns each element of this SYCL \codeinline{range} instance with the
      result of an element-wise \codeinline{OP} operator between each element
      of this SYCL \codeinline{range} and each element of the \codeinline{rhs}
      \codeinline{range} and returns a reference to this SYCL \codeinline{
      range}. If the operator returns a \codeinline{bool} the result is the cast
      to \codeinline{size_t}.
    }
  \addRow
    { range<dimensions> \&operatorOP(const size_t \&rhs) }
    {
      Where \codeinline{OP} is: \codeinline{+=}, \codeinline{-=},\codeinline{
      *=}, \codeinline{/=}, \codeinline{\%=}, \codeinline{<<=}, \codeinline{
      >>=}, \codeinline{\&=}, \codeinline{|=}, \codeinline{^=}.
      \newline
      Assigns each element of this SYCL \codeinline{range} instance with the
      result of an element-wise \codeinline{OP} operator between each element
      of this SYCL \codeinline{range} and the \codeinline{rhs} \codeinline{
      size_t} and returns a reference to this SYCL \codeinline{range}. If the
      operator returns a \codeinline{bool} the result is the cast to
      \codeinline{size_t}.
    }
\completeTable
%-------------------------------------------------------------------------------

%-------------------------------------------------------------------------------
\startTable{Non-member function}
\addFootNotes {Non-member functions of the SYCL \codeinline{range} class
  template}{table.functions.range}
  \addRowThreeL
    { template <int dimensions> }
    { range<dimensions> operatorOP(const size_t \&lhs, }
    { const range<dimensions> \&rhs) }
    {
      Where \codeinline{OP} is: \codeinline{+}, \codeinline{-}, \codeinline{*},
      \codeinline{/}, \codeinline{\%}, \codeinline{<<}, \codeinline{>>},
      \codeinline{\&}, \codeinline{|}, \codeinline{^}.
      \newline
      Constructs and returns a new instance of the SYCL \codeinline{range} class
      template with the same dimensionality as the \codeinline{rhs} SYCL
      \codeinline{range}, where each element of the new SYCL \codeinline{range}
      instance is the result of an element-wise \codeinline{OP} operator between
      the \codeinline{lhs} \codeinline{size_t} and each element of the
      \codeinline{rhs} SYCL \codeinline{range}. If the operator returns a
      \codeinline{bool} the result is the cast to \codeinline{size_t}.
    }
\completeTable
%-------------------------------------------------------------------------------


%*******************************************************************************
% nd_range class
%*******************************************************************************
\subsubsection{\texttt{nd_range} class}
\label{subsubsec:nd-range-class}

%Interface for class: nd_range
\lstinputlisting{headers/ndRange.h}

\tclass{nd_range}{\tf{int} dimensions} defines the iteration domain of both
the work-groups and the overall dispatch. To define this the
\codeinline{nd_range} comprises two ranges: the whole range over which
the kernel is to be executed, and the range of each work
group.

The SYCL \codeinline{nd_range} class template provides the common by-value
semantics (see Section~\ref{sec:byval-semantics}).

A synopsis of the SYCL \codeinline{nd_range} class is provided below. The constructors and member functions of the SYCL \codeinline{nd_range} class are listed in Tables~\ref{table.constructors.ndrange} and \ref{table.members.ndrange} respectively. The additional common special member functions and common member functions are listed in \ref{sec:byval-semantics} in Tables~\ref{table.specialmembers.common.byval} and \ref{table.members.common.byval} respectively.

\fixme{UPDATE: change parameter naming to match interface}
%------------------------------------------------------------------------------------------------------
\startTable{Constructor}
\addFootNotes{Constructors of the \codeinline{nd_range} class}{table.constructors.ndrange}
  \addRowFourL
    {nd_range<dimensions>(}
    {  range<dimensions> globalSize,}
    {  range<dimensions> localSize)}
    {  id<dimensions> offset = id<dimensions>())}
    {
        Construct an \codeinline{nd_range} from the local and global
        constituent ranges as well as an optional offset. If the
        offset is not provided it will default to no offset.
    }
  \completeTable
%------------------------------------------------------------------------------------------------------

%------------------------------------------------------------------------------------------------------
\startTable{Member function}
  \addFootNotes{Member functions for the \codeinline{nd_range} class}
  {table.members.ndrange}

  \addRow
    {range<dimensions> get_global_range() const}
    {
        Return the constituent global range.
    }
  \addRow
    {range<dimensions> get_local_range() const}
    {
        Return the constituent local range.
    }
  \addRow
    {range<dimensions> get_group_range() const}
    {
        Return a range representing the number of groups in each
        dimension.  This range would result from
        \codeinline{globalSize/localSize} as provided on construction.
    }
  \addRow
    {id<dimensions> get_offset() const}
    {
        Return the constituent offset.
    }
 \completeTable
%-------------------------------------------------------------------------------

%*******************************************************************************
% id class
%*******************************************************************************
\subsubsection{\texttt{id} class}
\label{id-class}

\tclass{id}{\tf{int} dimensions} is a vector of dimensions that is used to
represent an \gls{id} into a global or local
\codeinline{range}. It can be used as an index in an accessor of the
same rank. The \tf{[n]} operator returns the component \tf{n} as an
\tf{size_t}.

The SYCL \codeinline{id} class template provides the common by-value semantics
(see Section~\ref{sec:byval-semantics}).

A synopsis of the SYCL \codeinline{id} class is provided below. The constructors, member functions and non-member functions of the SYCL \codeinline{id} class are listed in Tables~\ref{table.constructors.id}, \ref{table.members.id} and \ref{table.functions.id} respectively. The additional common special member functions and common member functions are listed in \ref{sec:byval-semantics} in Tables~\ref{table.specialmembers.common.byval} and \ref{table.members.common.byval} respectively.

\lstinputlisting{headers/id.h}

\fixme{UPDATE: remove un-needed templates in the table, for consistency with
tables of other classes. No actual interface change.}
%-------------------------------------------------------------------------------
\startTable{Constructor}
\addFootNotes{Constructors of the \codeinline{id} class template}
{table.constructors.id}
  \addRow
    {id()}
    {
        Construct a SYCL \codeinline{id} with the value \codeinline{0} for each dimension.
    }
  \addRow
    {id(size_t dim0)}
    {
        Construct a 1D \codeinline{id} with value dim0.
        Only valid when the template parameter \codeinline{dimensions} is equal
        to 1.
    }
  \addRow
    {id(size_t dim0, size_t dim1)}
    {
        Construct a 2D \codeinline{id} with values dim0, dim1.
        Only valid when the template parameter \codeinline{dimensions} is equal
  to 2.
    }
  \addRow
    {id(size_t dim0, size_t dim1, size_t dim2)}
    {
        Construct a 3D \codeinline{id} with values dim0, dim1, dim2. 
        Only valid when the template parameter \codeinline{dimensions} is equal
  to 3.
    }
  \addRow
    {id(const range<dimensions> \&range)}
    {
        Construct an \codeinline{id} from the dimensions of \codeinline{r}.
    }
  \addRow
    {id(const item<dimensions> \&item)}
    {
        Construct an \codeinline{id} from \codeinline{item.get_id()}.
    }
\completeTable
%-------------------------------------------------------------------------------

%-------------------------------------------------------------------------------
\startTable{Member function}
\addFootNotes{Member functions of the \codeinline{id} class template}
{table.members.id}
  \addRow
    {size_t get(int dimension) const}
    {
        Return the value of the \codeinline{id} for dimension
        \codeinline{dimension}.
    }
  \addRow
    {size_t \&operator[](int dimension)}
    {
        Return a reference to the requested dimension of the \codeinline{id}
        object.
    }
  \addRow
    {size_t operator[](int dimension) const}
    {
        Return the value of the requested dimension of the \codeinline{id}
        object.
    }
  \addRow
    { id<dimensions> operatorOP(const id<dimensions> \&rhs) const }
    {
      Where \codeinline{OP} is: \codeinline{+}, \codeinline{-}, \codeinline{*},
      \codeinline{/}, \codeinline{\%}, \codeinline{<<}, \codeinline{>>},
      \codeinline{\&}, \codeinline{|}, \codeinline{^}, \codeinline{\&\&},
      \codeinline{||}, \codeinline{<}, \codeinline{>}, \codeinline{<=},
      \codeinline{>=}.
      \newline
      Constructs and returns a new instance of the SYCL \codeinline{id} class
      template with the same dimensionality as this SYCL \codeinline{id}, where
      each element of the new SYCL \codeinline{id} instance is the result of an
      element-wise \codeinline{OP} operator between each element of this SYCL
      \codeinline{id} and each element of the \codeinline{rhs} \codeinline{id}.
      If the operator returns a \codeinline{bool} the result is the cast to
      \codeinline{size_t}.
    }
  \addRow
    { id<dimensions> operatorOP(const size_t \&rhs) const }
    {
      Where \codeinline{OP} is: \codeinline{+}, \codeinline{-}, \codeinline{*},
      \codeinline{/}, \codeinline{\%}, \codeinline{<<}, \codeinline{>>},
      \codeinline{\&}, \codeinline{|}, \codeinline{^}, \codeinline{\&\&},
      \codeinline{||}, \codeinline{<}, \codeinline{>}, \codeinline{<=},
      \codeinline{>=}.
      \newline
      Constructs and returns a new instance of the SYCL \codeinline{id} class
      template with the same dimensionality as this SYCL \codeinline{id}, where
      each element of the new SYCL \codeinline{id} instance is the result of an
      element-wise \codeinline{OP} operator between each element of this SYCL
      \codeinline{id} and the \codeinline{rhs} \codeinline{size_t}. If the
      operator returns a \codeinline{bool} the result is the cast to
      \codeinline{size_t}.
    }
  \addRow
    { id<dimensions> \&operatorOP(const id<dimensions> \&rhs) }
    {
      Where \codeinline{OP} is: \codeinline{+=}, \codeinline{-=},\codeinline{
      *=}, \codeinline{/=}, \codeinline{\%=}, \codeinline{<<=}, \codeinline{
      >>=}, \codeinline{\&=}, \codeinline{|=}, \codeinline{^=}.
      \newline
      Assigns each element of this SYCL \codeinline{id} instance with the
      result of an element-wise \codeinline{OP} operator between each element
      of this SYCL \codeinline{id} and each element of the \codeinline{rhs}
      \codeinline{id} and returns a reference to this SYCL \codeinline{id}. If
      the operator returns a \codeinline{bool} the result is the cast to
      \codeinline{size_t}.
    }
  \addRow
    { id<dimensions> \&operatorOP(const size_t \&rhs) }
    {
      Where \codeinline{OP} is: \codeinline{+=}, \codeinline{-=},\codeinline{
      *=}, \codeinline{/=}, \codeinline{\%=}, \codeinline{<<=}, \codeinline{
      >>=}, \codeinline{\&=}, \codeinline{|=}, \codeinline{^=}.
      \newline
      Assigns each element of this SYCL \codeinline{id} instance with the
      result of an element-wise \codeinline{OP} operator between each element
      of this SYCL \codeinline{id} and the \codeinline{rhs} \codeinline{size_t}
      and returns a reference to this SYCL \codeinline{id}. If the operator
      returns a \codeinline{bool} the result is the cast to \codeinline{size_t}.
    }
\completeTable
%-------------------------------------------------------------------------------

%-------------------------------------------------------------------------------
\startTable{Non-member function}
\addFootNotes{Non-member functions of the \codeinline{id} class template}
{table.functions.id}
  \addRowThreeL
    { template <int dimensions> }
    { id<dimensions> operatorOP(const size_t \&lhs, }
    { const id<dimensions> \&rhs }
    {
      Where \codeinline{OP} is: \codeinline{+}, \codeinline{-}, \codeinline{*},
      \codeinline{/}, \codeinline{\%}, \codeinline{<<}, \codeinline{>>},
      \codeinline{\&}, \codeinline{|}, \codeinline{^}.
      \newline
      Constructs and returns a new instance of the SYCL \codeinline{id} class
      template with the same dimensionality as the \codeinline{rhs} SYCL
      \codeinline{id}, where each element of the new SYCL \codeinline{id}
      instance is the result of an element-wise \codeinline{OP} operator between
      the \codeinline{lhs} \codeinline{size_t} and each element of the
      \codeinline{rhs} SYCL \codeinline{id}. If the operator returns a
      \codeinline{bool} the result is the cast to \codeinline{size_t}.
    }
\completeTable
%-------------------------------------------------------------------------------

%*******************************************************************************
% item class
%*******************************************************************************
\subsubsection{\texttt{item} class}
\label{subsec:item.class}

\gls{item} identifies an instance of the function object
executing at each point in a \codeinline{range}. It is passed to a
\codeinline{parallel_for} call or returned by member functions of \codeinline{h_item}.
It encapsulates enough information to identify the work-item's range
of possible values and its ID in that range. It can optionally carry the offset of the
range if provided to the \codeinline{parallel_for}.
Instances of the \codeinline{item} class are
not user-constructible and are passed by the runtime to each instance
of the function object.

The SYCL \codeinline{item} class template provides the common by-value semantics
(see Section~\ref{sec:byval-semantics}).

\subsubsection{Item interface}

A synopsis of the SYCL \codeinline{item} class is provided below. The member functions of the SYCL \codeinline{item} class are listed in Table~\ref{table.members.id}. The additional common special member functions and common member functions are listed in \ref{sec:byval-semantics} in Tables~\ref{table.specialmembers.common.byval} and \ref{table.members.common.byval} respectively.

%Interface for class: item
\lstinputlisting{headers/item.h}

%-------------------------------------------------------------------------------
\startTable{Member function}
\addFootNotes{Member functions for the \codeinline{item} class}
{table.members.item}

  \addRow
    {id<dimensions> get_id() const}
    {
        Return the constituent \codeinline{id}
        representing the work-item's position in the iteration space.
    }
  \addRow
    {size_t get_id(int dimension) const}
    {
      Return the requested dimension of the constituent \codeinline{id}
      representing the work-item's position in the iteration space.
    }
  \addRow
    {size_t operator[](int dimension) const}
    {
      Return the constituent \codeinline{id} value representing the
      work-item's position in the iteration space in the given
      \codeinline{dimension}.
    }
  \addRow
    {range<dimensions> get_range() const}
    {
      Returns a \codeinline{range} representing the dimensions of the
      range of possible values of the \codeinline{item}.
    }
  \addRow
    {size_t get_range(int dimension) const}
    {
      Return the same value as get_range().get(dimension)
    }
  \addRow
    {id<dimensions> get_offset() const}
    {
      Returns an \codeinline{id} representing the $n$-dimensional offset
      provided to the \codeinline{parallel_for} and that is added by
      the runtime to the global-ID of each work-item, if this item
      represents a global range. For an item converted from an item with
      no offset this will always return an \codeinline{id} of all 0 values.

      This member function is only available if \codeinline{with_offset} is \codeinline{true}.
    }
  \addRow
    {operator item<dimensions, true>() const}
    {
      Returns an \codeinline{item} representing the same information as the object holds
      but also includes the offset set to 0. This conversion allow users to seamlessly
      write code that assumes an offset and still provides an offset-less \codeinline{item}.

      This member function is only available if \codeinline{with_offset} is \codeinline{false}.
    }
  \addRow
    {size_t get_linear_id() const}
    {
        Return the id as a linear index value. Calculating a linear
        address from the multi-dimensional index follow the equation 
  \ref{row-major-equation-buffer}.
    }
 \completeTable
%-------------------------------------------------------------------------------

%*******************************************************************************
% nd_item class
%*******************************************************************************

\subsubsection{\texttt{nd_item} class}
\label{nditem-class}

\codeinline{nd_item<int dimensions>} identifies an instance of the function object
executing at each point in an \codeinline{nd_range<int dimensions>} passed to a
\codeinline{parallel_for} call. It encapsulates enough
information to identify the \gls{work-item}'s local and global \glspl{id}, the
\gls{work-group-id} and also provides the \codeinline{barrier} and \codeinline{mem_fence} member functions, for performing \gls{work-group-barrier} and \gls{work-group-mem-fence} operations respectively. Instances of the \codeinline{nd_item<int dimensions>} class are not
user-constructible and are passed by the runtime to each instance of the
function object.

The SYCL \codeinline{nd_item} class template provides the common by-value
semantics (see Section~\ref{sec:byval-semantics}).

A synopsis of the SYCL \codeinline{nd_item} class is provided below. The member functions of the SYCL \codeinline{nd_item} class are listed in Table~\ref{table.members.nditem}. The additional common special member functions and common member functions are listed in \ref{sec:byval-semantics} in Tables~\ref{table.specialmembers.common.byval} and \ref{table.members.common.byval} respectively.

%interface for nd_item class
\lstinputlisting{headers/nditem.h}

%------------------------------------------------------------------------------------------------------
\startTable{Member function}
\addFootNotes{Member functions for the \codeinline{nd_item} class}
{table.members.nditem}
  \addRow
    {id<dimensions> get_global_id() const}
    {
        Return the constituent \gls{global-id} representing the
        work-item's position in the global iteration space.
    }
  \addRow
    {size_t get_global_id(int dimension) const}
    {
        Return the constituent element of the \gls{global-id}
        representing the work-item's position in the \gls{nd-range}
        in the given \codeinline{dimension}.
    }
  \addRow
   {size_t get_global_linear_id() const}
    {
      Return the flattened \gls{id} of the current work-item after subtracting the offset.
      Calculating a linear \gls{id}
      from a multi-dimensional index follows the equation \ref{row-major-equation-buffer}.
    }
  \addRow
    {id<dimensions> get_local_id() const}
    {
        Return the constituent \gls{local-id} representing the
        work-item's position within the current \gls{work-group}.
    }
  \addRow
    {size_t get_local_id(int dimension) const}
    {
      Return the constituent element of the \gls{local-id} representing the
      work-item's position within the current \gls{work-group} in the given
      \codeinline{dimension}.
    }
   \addRow
    {size_t get_local_linear_id() const}
   {
      Return the flattened \gls{id} of the current work-item within the current
      \gls{work-group}.
      Calculating a linear address
      from a multi-dimensional index follows the equation \ref{row-major-equation-buffer}.
   }
  \addRow
    {group<dimensions> get_group() const}
    {
        Return the constituent \gls{work-group}, \codeinline{group}
          representing the \gls{work-group}'s position within the overall
          \gls{nd-range}.
    }
   \addRow
    {size_t get_group(int dimension) const}
    {
        Return the constituent element of the group \codeinline{id} representing
        the work-group's position within the overall \codeinline{nd_range} in the
        given \codeinline{dimension}.
    }
    \addRow
   {size_t get_group_linear_id() const}
    {
        Return the group id as a linear index value. Calculating a linear address
      from a multi-dimensional index follows the equation \ref{row-major-equation-buffer}.
    }
  \addRow
    {range<dimensions> get_group_range() const}
    {Returns the number of \glspl{work-group} in the iteration space.}
  \addRow
    {size_t get_group_range(int dimension) const}
    {
    Return the number of \glspl{work-group} for \codeinline{dimension} in the
    iteration space.
    }
  \addRow
    {range<dimensions> get_global_range() const}
    {
        Returns a \codeinline{range} representing the dimensions of the
        global iteration space.
    }
  \addRow
    {size_t get_global_range(int dimension) const}
    {
      Return the same value as get_global_range().get(dimension)
    }
  \addRow
    {range<dimensions> get_local_range() const}
    {
        Returns a \codeinline{range} representing the dimensions of the current
        work-group.
    }
  \addRow
    {size_t get_local_range(int dimension) const}
    {
      Return the same value as get_local_range().get(dimension)
    }
  \addRow
    {id<dimensions> get_offset() const}
    {
        Returns an \gls{id} representing the n-dimensional offset
        provided to the constructor of the \codeinline{nd_range} and that
        is added by the runtime to the \gls{global-id} of each \gls{work-item}.
    }
  \addRow
    {nd_range<dimensions> get_nd_range() const}
    {
        Returns the \codeinline{nd_range} of the current execution.
    }
  \addRowThreeL
    {void barrier(}
    {access::fence_space accessSpace =}
    {access::fence_space::global_and_local) const}
    {
      Executes a \gls{work-group-barrier} with memory ordering on the local address
      space, global address space or both based on the value of
      \codeinline{accessSpace}. The current work-item will wait at the
      barrier until all work-items in the current work-group have
      reached the barrier. In addition the barrier performs a fence
      operation ensuring that all memory accesses in the specified
      address space issued before the barrier complete before those
      issued after the barrier.
    }
  \addRowFourL
    { template <access::mode accessMode = }
    { access::mode::read_write> }
    { void mem_fence(access::fence_space accessSpace = }
    { access::fence_space::global_and_local) const }
    {
      Available only when: \codeinline{accessMode == access::mode::read_write ||
      accessMode == access::mode::read || accessMode == access::mode::write}.
      \newline
      Executes a \gls{work-group-mem-fence} with memory ordering on the local
      address space, global address space or both based on the value of
      \codeinline{accessSpace}. If \codeinline{accessMode ==
      access::mode::read_write} the current work-item will ensure that all load
      and store memory accesses in the specified address space issued before the
      mem-fence complete before those issued after the mem-fence. If
      \codeinline{accessMode == access::mode::read} the current work-item will
      ensure that all load memory accesses in the specified address space issued
      before the mem-fence complete before those issued after the mem-fence. If
      \codeinline{accessMode == access::mode::write} the current work-item will
      ensure that all store memory accesses in the specified address space
      issued before the mem-fence complete before those issued after the mem-
      fence.
    }
  \addRowFourL
    { template <typename dataT> }
    { device_event async_work_group_copy( }
    { local_ptr<dataT> dest, global_ptr<dataT> src, }
    { size_t numElements) const }
    {
      Permitted types for \codeinline{dataT} are all scalar and vector types.
      Asynchronously copies a number of elements specified by \codeinline{
      numElements} from the source pointer \codeinline{src} to destination
      pointer \codeinline{dest} and returns a SYCL \codeinline{device_event}
      which can be used to wait on the completion of the copy. 
    }
  \addRowFourL
    { template <typename dataT> }
    { device_event async_work_group_copy( }
    { global_ptr<dataT> dest, local_ptr<dataT> src, }
    { size_t numElements) const }
    {
      Permitted types for \codeinline{dataT} are all scalar and vector types.
      Asynchronously copies a number of elements specified by \codeinline{
      numElements} from the source pointer \codeinline{src} to destination
      pointer \codeinline{dest} and returns a SYCL \codeinline{device_event}
      which can be used to wait on the completion of the copy. 
    }
  \addRowFourL
    { template <typename dataT> }
    { device_event async_work_group_copy( }
    { local_ptr<dataT> dest, global_ptr<dataT> src, }
    { size_t numElements, size_t srcStride) const }
    {
      Permitted types for \codeinline{dataT} are all scalar and vector types.
      Asynchronously copies a number of elements specified by \codeinline{
      numElements} from the source pointer \codeinline{src} to destination 
      pointer \codeinline{dest} with a source stride specified by
      \codeinline{srcStride} and returns a SYCL \codeinline{device_event}
      which can be used to wait on the completion of the copy. 
    }
  \addRowFourL
    { template <typename dataT> }
    { device_event async_work_group_copy( }
    { global_ptr<dataT> dest, local_ptr<dataT> src, }
    { size_t numElements, size_t destStride) const }
    {
      Permitted types for \codeinline{dataT} are all scalar and vector types.
      Asynchronously copies a number of elements specified by \codeinline{
      numElements} from the source pointer \codeinline{src} to destination
      pointer \codeinline{dest} with a destination stride specified by
      \codeinline{destStride} and returns a SYCL \codeinline{device_event}
      which can be used to wait on the completion of the copy. 
    }
  \addRowTwoL
    { template <typename... eventTN> }
    { void wait_for(eventTN... events) const }
    {
      Permitted type for \codeinline{eventTN} is \codeinline{device_event}.
      Waits for the asynchronous operations associated with each \codeinline{
      device_event} to complete.
    }
\completeTable
%-------------------------------------------------------------------------------

%*******************************************************************************
% h_item class
%*******************************************************************************

\subsubsection{\texttt{h_item} class}
\label{hitem-class}

\codeinline{h_item<int dimensions>} identifies an instance of a \codeinline{group::parallel_for_work_item} function object
executing at each point in a local \codeinline{range<int dimensions>} passed to a \codeinline{parallel_for_work_item}
call or to the corresponding \codeinline{parallel_for_work_group} call if no \codeinline{range}
is passed to the \codeinline{parallel_for_work_item} call.
It encapsulates enough information to identify the \gls{work-item}'s local and global \glspl{item}
according to the information given to \codeinline{parallel_for_work_group} (physical ids)
as well as the \gls{work-item}'s logical local \glspl{item} in the logical local range.
All returned \glspl{item} objects are offset-less.
Instances of the \codeinline{h_item<int dimensions>} class are not
user-constructible and are passed by the runtime to each instance of the
function object.

The SYCL \codeinline{h_item} class template provides the common by-value
semantics (see Section~\ref{sec:byval-semantics}).

A synopsis of the SYCL \codeinline{h_item} class is provided below. The member functions of the SYCL \codeinline{h_item} class are listed in Table~\ref{table.members.hitem}. The additional common special member functions and common member functions are listed in \ref{sec:byval-semantics} in Tables~\ref{table.specialmembers.common.byval} and \ref{table.members.common.byval} respectively.

\lstinputlisting{headers/hitem.h}

%------------------------------------------------------------------------------------------------------
\startTable{Member function}
\addFootNotes{Member functions for the \codeinline{h_item} class}
{table.members.hitem}
  \addRow
    {item<dimensions, false> get_global() const}
    {
        Return the constituent global \gls{item} representing the
        work-item's position in the global iteration space as provided upon kernel invocation.
    }
  \addRow
    {item<dimensions, false> get_local() const}
    {
        Return the same value as \codeinline{get_logical_local()}.
    }
  \addRow
    {item<dimensions, false> get_logical_local() const}
    {
        Return the constituent element of the logical local \gls{item}
        work-item's position in the local iteration space as provided upon the invocation of the
        \codeinline{group::parallel_for_work_item}.

        If the \codeinline{group::parallel_for_work_item} was called without any logical local range
        then the member function returns the physical local \gls{item}.

        A physical id can be computed from a logical id by getting the remainder of the integer division
        of the logical id and the physical range:
        \codeinline{get_logical_local().get() \% get_physical_local.get_range() == get_physical_local().get()}.
    }
  \addRow
    {item<dimensions, false> get_physical_local() const}
    {
        Return the constituent element of the physical local \gls{item}
        work-item's position in the local iteration space as provided (by the user or the runtime)
        upon the kernel invocation.
    }
  \addRow
    {range<dimensions> get_global_range() const}
    {
      Return the same value as \codeinline{get_global().get_range()}
    }
  \addRow
    {size_t get_global_range(int dimension) const}
    {
      Return the same value as \codeinline{get_global().get_range(dimension)}
    }
  \addRow
    {id<dimensions> get_global_id() const}
    {
      Return the same value as \codeinline{get_global().get_id()}
    }
  \addRow
    {size_t get_global_id(int dimension) const}
    {
      Return the same value as \codeinline{get_global().get_id(dimension)}
    }

  \addRow
    {range<dimensions> get_local_range() const}
    {
      Return the same value as \codeinline{get_local().get_range()}
    }
  \addRow
    {size_t get_local_range(int dimension) const}
    {
      Return the same value as \codeinline{get_local().get_range(dimension)}
    }
  \addRow
    {id<dimensions> get_local_id() const}
    {
      Return the same value as \codeinline{get_local().get_id()}
    }
  \addRow
    {size_t get_local_id(int dimension) const}
    {
      Return the same value as \codeinline{get_local().get_id(dimension)}
    }

  \addRow
    {range<dimensions> get_logical_local_range() const}
    {
      Return the same value as \codeinline{get_logical_local().get_range()}
    }
  \addRow
    {size_t get_logical_local_range(int dimension) const}
    {
      Return the same value as \codeinline{get_logical_local().get_range(dimension)}
    }
  \addRow
    {id<dimensions> get_logical_local_id() const}
    {
      Return the same value as \codeinline{get_logical_local().get_id()}
    }
  \addRow
    {size_t get_logical_local_id(int dimension) const}
    {
      Return the same value as \codeinline{get_logical_local().get_id(dimension)}
    }

  \addRow
    {range<dimensions> get_physical_local_range() const}
    {
      Return the same value as \codeinline{get_physical_local().get_range()}
    }
  \addRow
    {size_t get_physical_local_range(int dimension) const}
    {
      Return the same value as \codeinline{get_physical_local().get_range(dimension)}
    }
  \addRow
    {id<dimensions> get_physical_local_id() const}
    {
      Return the same value as \codeinline{get_physical_local().get_id()}
    }
  \addRow
    {size_t get_physical_local_id(int dimension) const}
    {
      Return the same value as \codeinline{get_physical_local().get_id(dimension)}
    }

 \completeTable
%-------------------------------------------------------------------------------

%*******************************************************************************
% group class
%*******************************************************************************
\subsubsection{\texttt{group} class}
\label{group-class}

The \codeinline{group<int dimensions>} encapsulates all functionality
required to represent a particular \gls{work-group} within a
parallel execution. It is not user-constructable.

The local range stored in the group class is provided either by
the programmer, when it is passed as an optional parameter to
\codeinline{parallel_for_work_group}, or by the runtime system when it
selects the optimal work-group size. This allows the developer to
always know how many concurrent work-items are active in each
executing work-group, even through the abstracted iteration range of the
\codeinline{parallel_for_work_item} loops.

The SYCL \codeinline{group} class template provides the common by-value
semantics (see Section~\ref{sec:byval-semantics}).

A synopsis of the SYCL \codeinline{group} class is provided below. The member functions of the SYCL \codeinline{group} class are listed in Table~\ref{table.members.group}. The additional common special member functions and common member functions are listed in \ref{sec:byval-semantics} in Tables~\ref{table.specialmembers.common.byval} and \ref{table.members.common.byval} respectively.

The \codeinline{group} class also provides the \codeinline{
mem_fence} member function for performing a \gls{work-group-mem-fence}
operation.

%Interface for class: group
\lstinputlisting{headers/group.h}

%-------------------------------------------------------------------------------
\startTable{Member function}
\addFootNotes{Member functions for the \codeinline{group} class}
{table.members.group}
  \addRow
    {id<dimensions> get_id() const}
    {
        Return an \gls{id} representing the index of the work-group
        within the \gls{nd-range} for every dimension.
    }
  \addRow
    {size_t get_id(int dimension) const}
    {
        Return the index of the work-group in the given \codeinline{dimension}.
    }
  \addRow
    {range<dimensions> get_global_range() const}
    {
        Return a SYCL \codeinline{range} representing all dimensions of the global range.
    }
  \addRow
    {size_t get_global_range(int dimension) const}
    {
        Return the dimension of the global range specified by the \codeinline{dimension} parameter.
    }
  \addRow
    {range<dimensions> get_local_range() const}
    {
        Return a SYCL \codeinline{range} representing all dimensions of the local range.
        This local range may have been provided by the programmer, or chosen by the \gls{sycl-runtime}.
    }
  \addRow
    {size_t get_local_range(int dimension) const}
    {
        Return the dimension of the local range specified by the \codeinline{dimension} parameter.
    }
  \addRow
    {range<dimensions> get_group_range() const}
    {
        Return a \codeinline{range} representing the number of \glspl{work-group} in the \codeinline{nd_range}.
    }
  \addRow
    {size_t get_group_range(int dimension) const}
    {
        Return element \codeinline{dimension} from the constituent group range.
    }
  \addRow
    {size_t operator[](int dimension) const}
    {
        Return the index of the group in the given \codeinline{dimension}
        within the \codeinline{nd_range}.
    }
  \addRow
    {size_t get_linear_id() const}
    {
        Get a linearized version of the \gls{work-group-id}.
        Calculating a linear \gls{work-group-id}
        from a multi-dimensional index follows the equation \ref{row-major-equation-buffer}.
    }
  \addRowTwoL
    {template <typename workItemFunctionT>}
    {void parallel_for_work_item(workItemFunctionT func) const}
    {
      Launch the work-items for this work-group.

      \codeinline{func} is a function object type with a public member function
      \codeinline{void F::operator()(h_item<dimensions>)}
      representing the work-item computation.

      This member function can only be invoked within a
      \codeinline{parallel_for_work_group} context.  It is
      undefined behavior for this member function to be invoked
      from within the \codeinline{parallel_for_work_group} form that
      does not define work-group size, because then the number of
      work-items that should execute the code is not defined.  It is
      expected that this form of \codeinline{parallel_for_work_item}
      is invoked within the \codeinline{parallel_for_work_group} form
      that specififies the size of a work-group.
    }
  \addRowThreeL
    {template <typename workItemFunctionT>}
    {void parallel_for_work_item(range<dimensions> }
    { logicalRange, workItemFunctionT func) const}
    {
      Launch the work-items for this work-group using a logical local range.
      The function object \codeinline{func} is executed as if the kernel where
      invoked with \codeinline{logicalRange} as the local range. This new local
      range is emulated and may not map one-to-one with the physical range.
      
      \codeinline{logicalRange} is the new local range to be used.
      This range can be smaller or larger than the one used to invoke the kernel.
      \codeinline{func} is a function object type with a public member function
      \codeinline{void F::operator()(h_item<dimensions>)}
      representing the work-item computation.

      Note that the logical range does not need to be uniform
      across all work-groups in a kernel.  For example the logical range may depend on
      a work-group varying query (e.g. \codeinline{group::get_linear_id}),
      such that different work-groups in the same kernel invocation execute
      different logical range sizes.

      This member function can only be invoked within a
      \codeinline{parallel_for_work_group} context.
    }
  \addRowFourL
    { template <access::mode accessMode = }
    { access::mode::read_write> }
    { void mem_fence(access::fence_space accessSpace = }
    { access::fence_space::global_and_local) const }
    {
      Available only when: \codeinline{accessMode == access::mode::read_write ||
      accessMode == access::mode::read || accessMode == access::mode::write}.
      \newline
      Executes a \gls{work-group-mem-fence} with memory ordering on the local
      address space, global address space or both based on the value of
      \codeinline{accessSpace}. If \codeinline{accessMode ==
      access::mode::read_write} the current work-item will ensure that all load
      and store memory accesses in the specified address space issued before the
      mem-fence complete before those issued after the mem-fence. If
      \codeinline{accessMode == access::mode::read} the current work-item will
      ensure that all load memory accesses in the specified address space issued
      before the mem-fence complete before those issued after the mem-fence. If
      \codeinline{accessMode == access::mode::write} the current work-item will
      ensure that all store memory accesses in the specified address space
      issued before the mem-fence complete before those issued after the mem-
      fence.
    }
  \addRowFourL
    { template <typename dataT> }
    { device_event async_work_group_copy( }
    { local_ptr<dataT> dest, global_ptr<dataT> src, }
    { size_t numElements) const }
    {
      Permitted types for \codeinline{dataT} are all scalar and vector types.
      Asynchronously copies a number of elements specified by \codeinline{
      numElements} from the source pointer \codeinline{src} to destination
      pointer \codeinline{dest} and returns a SYCL \codeinline{device_event}
      which can be used to wait on the completion of the copy. 
    }
  \addRowFourL
    { template <typename dataT> }
    { device_event async_work_group_copy( }
    { global_ptr<dataT> dest, local_ptr<dataT> src, }
    { size_t numElements) const }
    {
      Permitted types for \codeinline{dataT} are all scalar and vector types.
      Asynchronously copies a number of elements specified by \codeinline{
      numElements} from the source pointer \codeinline{src} to destination
      pointer \codeinline{dest} and returns a SYCL \codeinline{device_event}
      which can be used to wait on the completion of the copy. 
    }
  \addRowFourL
    { template <typename dataT> }
    { device_event async_work_group_copy( }
    { local_ptr<dataT> dest, global_ptr<dataT> src, }
    { size_t numElements, size_t srcStride) const }
    {
      Permitted types for \codeinline{dataT} are all scalar and vector types.
      Asynchronously copies a number of elements specified by \codeinline{
      numElements} from the source pointer \codeinline{src} to destination 
      pointer \codeinline{dest} with a source stride specified by
      \codeinline{srcStride} and returns a SYCL \codeinline{device_event}
      which can be used to wait on the completion of the copy. 
    }
  \addRowFourL
    { template <typename dataT> }
    { device_event async_work_group_copy( }
    { global_ptr<dataT> dest, local_ptr<dataT> src, }
    { size_t numElements, size_t destStride) const }
    {
      Permitted types for \codeinline{dataT} are all scalar and vector types.
      Asynchronously copies a number of elements specified by \codeinline{
      numElements} from the source pointer \codeinline{src} to destination
      pointer \codeinline{dest} with a destination stride specified by
      \codeinline{destStride} and returns a SYCL \codeinline{device_event}
      which can be used to wait on the completion of the copy. 
    }
  \addRowTwoL
    { template <typename... eventTN> }
    { void wait_for(eventTN... events) const }
    {
      Permitted type for \codeinline{eventTN} is \codeinline{device_event}.
      Waits for the asynchronous operations associated with each \codeinline{
      device_event} to complete.
    }
\completeTable
%-------------------------------------------------------------------------------

%*******************************************************************************
% device_event class
%*******************************************************************************
\subsubsection{\texttt{device_event} class}
\label{device-event-class}

The SYCL \codeinline{device_event} class encapsulates a single SYCL device event
which is available only within SYCL kernel functions and can be used to wait for
asynchronous operations within a SYCL kernel function to complete. A SYCL device
event may be an OpenCL device event, in which case it must encapsulate a valid
underlying OpenCL \codeinline{event_t}, or it may be a SYCL host device event,
in which case it must not.

All member functions of the \codeinline{device_event} class must not throw a
SYCL exception.

\subsubsection{Device event interface}

A synopsis of the SYCL \codeinline{device_event} class is provided below. The
constructors and member functions of the SYCL \codeinline{device_event} class
are listed in Table~\ref{table.constructors.device-event} and
\ref{table.members.device-event} respectively.

%Interface of device event class
\lstinputlisting{headers/deviceEvent.h}

%-------------------------------------------------------------------------------
\startTable{Member function}
\addFootNotes{Member functions of the SYCL \codeinline{device_event} class}{table.members.device-event}
  \addRow
    { void wait() }
    {
      Waits for the asynchronous operation associated with this SYCL
      \codeinline{device_event} to complete.
    }
\completeTable
%-------------------------------------------------------------------------------

%-------------------------------------------------------------------------------
\startTable{Constructor}
\addFootNotes{Constructors of the \codeinline{device_event} class}
{table.constructors.device-event}
  \addRow
    {device_event(___unspecified___)}
    {
      Unspecified implementation defined constructor.
    }
\completeTable
%-------------------------------------------------------------------------------

%***********************************************************************************
% Command group scope
%***********************************************************************************
\subsection{Command group scope}
\label{sec:command.group.scope}

A \gls{command-group-scope} in SYCL, as it is defined in 
Section~\ref{sec:executionmodel}, consists of a single kernel or explicit memory
operation (\codeinline{handler} methods such as \codeinline{copy}, \codeinline{update_host},
\codeinline{fill}), together with its \textbf{requirements}.
The commands that enqueue a kernel or explicit memory operation and the requirements
for its execution form the \gls{command-group-function-object}.
The command group
function object takes as a parameter an instance of the \gls{handler} class which 
encapsulates all the member functions executed in the command group scope. 
The methods and objects defined in this scope will define the requirements for the
kernel execution or explicit memory operation, and will be used by the \gls{sycl-runtime}
to evaluate if the operation is ready for execution.
Host code within a \gls{command-group-function-object} (typically setting up
requirements) is executed once, before the command group submit call returns.
This abstraction of the kernel
execution unifies the data with its processing, and consequently allows more
abstraction and flexibility in the parallel programming models that can be
implemented on top of SYCL.

The \gls{command-group-function-object} and the \codeinline{handler} class
serve as an interface for the encapsulation of \gls{command-group-scope}.
A \gls{sycl-kernel-function} is defined as a function object. All the device data accesses are
defined inside this group and any transfers are managed by the \gls{sycl-runtime}. The
rules for the data transfers regarding device and
host data accesses are better described in the data management section
(\ref{sec:data.access.and.storage}), where buffers (\ref{subsec:buffers}) and
accessor (\ref{subsec:accessors}) classes are described.
The overall memory model of the SYCL application is described in
Section~\ref{sub.section.memmodel.app}.

It is possible to obtain events for the start of the \gls{command-group-function-object},
the kernel starting, and the command group completing.
These events are most useful for
profiling, because safe synchronization in SYCL requires synchronization on
buffer availability, not on kernel completion. This is because
the memory that data is stored in upon kernel
completion is not rigidly specified. The events are provided at the submission of the
\gls{command-group-function-object} to the queue to be executed on.

It is possible for a \gls{command-group-function-object} to fail to enqueue to a queue,
or for it to fail to execute correctly. A user can therefore supply a secondary
queue when submitting a command group to the primary queue. If the \gls{sycl-runtime}
fails to enqueue or execute a command group on a primary queue, it can attempt
to run the command group on the secondary queue. The circumstances in which it
is, or is not, possible for a \gls{sycl-runtime} to fall-back from primary to
secondary queue are unspecified in the specification.  Even if a command group
is run on the secondary queue, the requirement that host code within the command group
is executed exactly once remains, regardless of whether the fallback queue is used for
execution.

The command group \codeinline{handler} class provides the interface
for all of the member functions that are able to be executed inside the command group
scope, and it is also provided as a scoped object to all of the data access
requests. The \gls{handler} class provides the interface
in which every command in the command group scope will be submitted to a queue.

%***********************************************************************************
% Command group handler class
%***********************************************************************************

\subsection{Command group \texttt{handler} class}
\label{sec:handlerClass}

A \gls{handler} object can only be constructed by the SYCL
runtime. All of the accessors defined in \gls{command-group-scope} take as a
parameter an instance of the \gls{handler}, and all the
kernel invocation functions are member functions of this class.

The constructors of the SYCL \codeinline{handler} class are described in Table~\ref{table.constructors.handler}.

It is disallowed for an instance of the SYCL \codeinline{handler} class to be moved or copied.

%Interface for class: handler
\lstinputlisting{headers/commandGroupHandler.h}

%-------------------------------------------------------------------------------
\startTable{Constructor}
\addFootNotes{Constructors of the \codeinline{handler} class}
{table.constructors.handler}
  \addRow
    {handler(___unspecified___)}
    {
      Unspecified implementation defined constructor.
    }
\completeTable
%-------------------------------------------------------------------------------

%***********************************************************************************
% SYCL API for definition requisites
%***********************************************************************************
\subsection{SYCL functions for adding requirements}
\label{sub.section.requirement}

Requirements for execution of SYCL kernels can be specified directly using
handler methods.

\startTable{Member function}
\addFootNotes{Member functions of the \codeinline{handler} class}
{table.members.handler.requirements}
  \addRowFourSL
    { template <typename dataT, int dimensions, }
    { access::mode accessMode, access::target accessTarget> }
    { void require(accessor<dataT, dimensions, accessMode, accessTarget, }
    { placeholder::true_t> acc) }
    {
      Requires access to the memory object associated with the placeholder
      accessor. 
      The \gls{command-group} now has a \textbf{requirement} to gain access 
      to the given memory object before executing the kernel.
    }
\completeTable

%***********************************************************************************
% SYCL functions for invoking kernels
%***********************************************************************************

\subsection{SYCL functions for invoking kernels}
\label{subsec:invokingkernels}

\Glspl{kernel} can be invoked as \keyword{single tasks}, basic
\keyword{data-parallel} \glspl{kernel}, OpenCL-style \gls{nd-range} in
\glspl{work-group}, or SYCL \keyword{hierarchical parallelism}.

Each function takes a kernel name template parameter. The \gls{kernel-name}
must be a datatype that is unique for each kernel invocation. If a
kernel is a named function object, and its type is globally visible,  then the
kernel's function object type will be automatically used as the kernel name and so the user does
not need to supply a name. If the kernel function is a C++11 lambda function,
then the user must manually provide a kernel name to enable linking
between host and device code to occur.

All the functions for invoking kernels are member functions of the command group
\codeinline{handler} class~\ref{sec:handlerClass}, which 
is used to encapsulate all the member functions provided in a command group scope.
Table~\ref{table.members.handler.kernel} lists all the members of the 
\codeinline{handler} class related to the kernel invocation.

%-------------------------------------------------------------------------------
\startTable{Member function}
\addFootNotes{Member functions of the \codeinline{handler} class}
{table.members.handler.kernel}
  \addRowTwoL
    {template <typename T>}
    {void set_arg(int argIndex, T \&\&arg)}
    {
      Set a kernel argument for an OpenCL kernel through the
      SYCL/OpenCL interoperability interface. The \codeinline{index}
      value specifies which parameter of the OpenCL kernel is being
      set and \codeinline{arg} specifies the kernel argument.

      Index 0 is the first parameter.

      The argument can be either a SYCL accessor, a SYCL sampler or a
      trivially copyable and standard-layout C++ type.
     }

  \addRowTwoL
    {template <typename... Ts>}
    {void set_args(Ts \&\&... args)}
    {
      Set all the given kernel \codeinline{args} arguments for an
      OpenCL kernel, as if \codeinline{set_arg()} was used with each
      of them in the same order and increasing index always starting
      at 0.
    }

   \addRowTwoL
    {template <typename KernelName, typename KernelType>}
    {void single_task(KernelType kernelFunc)}
    {
      Defines and invokes a \gls{sycl-kernel-function} as a lambda function
      or a named function object type.
      If it is a named function object and the function object type is globally
      visible there is no need for the developer
      to provide a \gls{kernel-name} (\codeinline{typename KernelName}) for it,
      as described in~\ref{subsec:invokingkernels}.
    }
   \addRowThreeL
    {template <typename KernelName, typename KernelType, int dimensions>}
    {void parallel_for(}
    {range<dimensions> numWorkItems, KernelType kernelFunc)}
    {
      Defines and invokes a \gls{sycl-kernel-function} as a lambda function
      or a named function object type,
      for the specified range and given an id or item for indexing in the
      indexing space defined by range.
      If it is a named function object and the function object type is globally
      visible there is no need for the developer
      to provide a \gls{kernel-name} (\codeinline{typename KernelName}) for it,
      as described in~\ref{subsec:invokingkernels}.
    }
    \addRowFourL
    {template <typename KernelName, typename KernelType, int dimensions>}
    {void parallel_for(}
    { range<dimensions> numWorkItems,}
    { id<dimensions> workItemOffset, KernelType kernelFunc)}
    {
      Defines and invokes a \gls{sycl-kernel-function} as a lambda function
      or a named function object type,
      for the specified range and offset and given an id or item for indexing in
      the indexing space defined by range.
      If it is a named function object and the function object type is globally
      visible there is no need for the developer
      to provide a \gls{kernel-name} (\codeinline{typename KernelName}) for it,
      as described in~\ref{subsec:invokingkernels}.
    }
  \addRowThreeL
    {template <typename KernelName, typename KernelType, int dimensions>}
    {void parallel_for(}
    {nd_range<dimensions> executionRange, KernelType kernelFunc)}
    {
      Defines and invokes a \gls{sycl-kernel-function} as a lambda function
      or a named function object type,
      for the specified \gls{nd-range} and given an \gls{nd-item}
      for indexing in the indexing space defined by the \gls{nd-range}.
      If it is a named function object and the function object type is globally
      visible there is no need for the developer
      to provide a \gls{kernel-name} (\codeinline{typename KernelName}) for it,
      as described in~\ref{subsec:invokingkernels}.
    }
  \addRowFourL
    {template <typename KernelName, typename WorkgroupFunctionType, int dimensions>}
    {void parallel_for_work_group(}
    { range<dimensions> numWorkGroups,}
    { WorkgroupFunctionType kernelFunc)}
    {
      Hierarchical kernel invocation method of a kernel defined as a
      lambda encoding the body of each work-group to launch. May
      contain multiple calls to
      \codeinline{parallel_for_work_item(..)} methods representing the
      execution on each work-item. Launches
      \codeinline{num_work_groups} work-groups of runtime-defined
      size. Described in detail in~\ref{subsec:invokingkernels}.
    }
\addRowFiveL
  {template <typename KernelName, typename WorkgroupFunctionType, int dimensions>}
  {void parallel_for_work_group(}
  { range<dimensions> numWorkGroups,}
  { range<dimensions> workGroupSize,}
  { WorkgroupFunctionType kernelFunc)}
  {
    Hierarchical kernel invocation method of a kernel defined as a
    lambda encoding the body of each work-group to launch. May
    contain multiple calls to
    \codeinline{parallel_for_work_item} methods representing the
    execution on each work-item. Launches
    \codeinline{num_work_groups} work-groups of
    \codeinline{work_roup_size} work-items each. Described in
    detail in~\ref{subsec:invokingkernels}.
  }
  \addRow
    {void single_task(kernel syclKernel)}
    {
      Defines and invokes a \gls{sycl-kernel-function} as a lambda function
      or a named function object type, executes exactly once.
    }
  \addRowThreeL
    { template <int dimensions> void parallel_for(}
    { range<dimensions> numWorkItems,}
    { kernel syclKernel)}
    {
      Kernel invocation method of a pre-compiled \gls{kernel} defined by SYCL
      \codeinline{sycl-kernel-function} instance,
      for the specified range and given an id or item for indexing in the
      indexing space defined by range,
      described in detail in~\ref{subsec:invokingkernels}
    }
  \addRowThreeL
    { template <int dimensions> void parallel_for(}
    { range<dimensions> numWorkItems, }
    { id<dimensions> workItemOffset, kernel syclKernel) }
    {
      Kernel invocation method of a pre-compiled \gls{kernel} defined by SYCL
      \codeinline{sycl-kernel-function} instance,
      for the specified range and offset and given an id or item for indexing in the
      indexing space defined by range,
      described in detail in~\ref{subsec:invokingkernels}
    }
   \addRowThreeL
    { template <int dimensions> void parallel_for(}
    { nd_range<dimensions> ndRange,}
    { kernel syclKernel)}
    {
      Kernel invocation method of a pre-compiled \gls{kernel} defined by SYCL \codeinline{kernel} instance,
      for the specified \codeinline{ndrange} and given an \codeinline{nd_item}
      for indexing in the indexing space defined by the \codeinline{nd_range},
      described in detail in~\ref{subsec:invokingkernels}
    }
\completeTable

%Interface for apis
%\lstinputlisting{headers/parallelFor.h}

%***********************************************************************************
% single_task invoke
%***********************************************************************************

\subsubsection{\texttt{single_task} invoke}

SYCL provides a simple interface to enqueue a kernel that will be
sequentially executed on an OpenCL device. Only one instance of the
kernel will be executed. This interface is useful as a primitive for more
complicated parallel algorithms, as it can easily create a chain of
sequential tasks on an OpenCL device with each of them managing its
own data transfers.

This function can only be called inside a command group using the
\codeinline{handler} object created by the runtime.
Any accessors that are used in a kernel should be defined inside the
same command group.

Local accessors are disallowed for single task invocations.

\lstinputlisting{code/singletask.cpp}

For single tasks, the kernel method takes no parameters, as there
is no need for \gls{index-space-classes} in a unary index space.

%***********************************************************************************
% parallel_for invoke
%***********************************************************************************
\subsubsection{\texttt{parallel_for} invoke}

The \codeinline{parallel_for} member function of the SYCL \codeinline{handler} class provides
an interface to define and invoke a SYCL kernel function in a command group, to execute in
parallel execution over a 3 dimensional index space.
There are three overloads of the \codeinline{parallel_for} member function which provide
variations of this interface, each with a different level of complexity and providing a
different set of features.

For the simplest case, users need only provide the global range (the total number of work-items in the index space) via a SYCL \codeinline{range} parameter, and the \gls{sycl-runtime} will select a local range (the number of work-items in each work-group).
The local range chosen by the \gls{sycl-runtime} is entirely implementation defined.
In this case the function object that represents the SYCL kernel function must take either a single SYCL \codeinline{id} parameter, or a single SYCL \codeinline{item} parameter, representing the currently executing work-item within the range specified by the \codeinline{range} parameter.

The execution of the kernel function is the same whether the parameter to the SYCL kernel function is
a SYCL \codeinline{id} or a SYCL \codeinline{item}.
What differs is the functionality that is available to the SYCL kernel function via the respective interfaces.

Below is an example of invoking a SYCL kernel function with \codeinline{parallel_for} using a lambda function,
and passing a SYCL \codeinline{id} parameter. In this case only the global id is available. This variant of
\codeinline{parallel_for} is designed for when it is not necessary to query the global range of the index space
being executed across, or the local (work-group) size chosen by the implementation.

\lstinputlisting{code/basicparallelfor.cpp}

Below is an example of invoking a SYCL kernel function with \codeinline{parallel_for} using a lambda function
and passing a SYCL \codeinline{item} parameter. In this case both the global id and global range are queryable.
This variant of \codeinline{parallel_for} is designed for when it is necessary to query the global range within
which the global id will vary.  No information is queryable on the local (work-group) size chosen by the
implementation.

\lstinputlisting{code/basicParallelForItem.cpp}

For SYCL kernel functions invoked via the above described overload of the \codeinline{parallel_for} member function,
it is disallowed to use local accessors or to use a \gls{work-group-barrier} or \gls{work-group-mem-fence} operation.

The following two examples show how a kernel function object can be launched
over a 3D grid, with 3 elements in each dimension. In the first case
work-item ids range from 0 to 2 inclusive, and in the second case
work-item ids run from 1 to 3.

\lstinputlisting{code/parallelfor.cpp}

The last case of a parallel_for invocation enables
low-level functionality of work-items and work-groups. This becomes
valuable when an execution requires groups of work-items that communicate and
synchronize. These are exposed in SYCL through 
\codeinline{parallel_for (nd_range,...)}
and the \codeinline{nd_item} class, 
which provides all the functionality of OpenCL for an nd-range.
In this case, the developer needs to define the
 \codeinline{nd_range} that the kernel will execute on in order to
 have fine grained  control of the enqueing of the kernel. 
 This variation of parallel_for expects an
 \codeinline{nd_range}, specifying both local and global ranges,
 defining the global
 number of work-items and the number in each cooperating work-group.
 The resulting function object is passed an \codeinline{nd_item}
 instance making all the information available, as well as
 \gls{work-group-barrier} and \gls{work-group-mem-fence} operations
 to synchronize or guarantee memory consistency between the \gls{work-item}s in the \gls{work-group}.

The following example shows how sixty-four work-items may be launched
in a three-dimensional grid with four in each dimension, and divided
into eight work-groups. Each group of work-items synchronizes with a
\gls{work-group-barrier}.

\lstinputlisting{code/parallelforbarrier.cpp}

Optionally, in any of these variations of parallel_for invocations,
the developer may also pass an offset. An offset is an
instance of the \codeinline{id} class added to the identifier 
for each point in the range.

In all of these cases the underlying \gls{nd-range} will be created
and the kernel defined as a function object will be created and enqueued
as part of the command group scope.

%***********************************************************************************
% Parallel For hierarchical invoke invoke
%***********************************************************************************

\subsubsection{Parallel For hierarchical invoke}

The hierarchical parallel kernel execution interface provides the same
functionality as is available from the \gls{nd-range} interface, but
exposed differently. To execute the same sixty-four work-items in
sixteen work-groups that we saw in the previous example, we execute an
outer \codeinline{parallel_for_work_group} call to create the
groups. The member function
\codeinline{handler::parallel_for_work_group} is parameterized by the
number of work-groups, such that the size of each group is chosen by
the runtime, or by the number of work-groups and number of work-items
for users who need more control.

The body of the outer \codeinline{parallel_for_work_group} call
consists of a lambda function or function object. The body of this
function object contains code that is executed only once for the
entire work-group. If the code has no side-effects and the compiler
heuristic suggests that it is more efficient to do so, this code will be
executed for each work-item.

Within this region any variable declared will have the semantics of
\gls{local-memory}, shared between all \glspl{work-item} in the 
\gls{work-group}. If the
device compiler can prove that an array of such variables is accessed only by
a single work-item throughout the lifetime of the work-group, for
example if access is derived from the id of the work-item with no
transformation, then it can allocate the data in private memory or
registers instead.

To guarantee use of private per-work-item memory, the
\codeinline{private_memory} class can be used to wrap the data.
This class very simply constructs private data for a given group across the
entire group. The id of the current work-item is passed to any access
to grab the correct data.

The \codeinline{private_memory} class has the following interface:
\label{paragraph.private.memory}
\lstinputlisting{headers/priv.h}
%-------------------------------------------------------------------------------
\startTable{Constructor}
\addFootNotes{Constructor of the \codeinline{private_memory} class}
{table.constructors.private.memory}
  \addRow
    {private_memory(const group<Dimensions> \&)}
    {
      Place an object of type \codeinline{T} in the underlying private memory of each \glspl{work-item}.
      The type \codeinline{T} must be default constructible.
      The underlying constructor will be called for each \gls{work-item}.
    }
\completeTable
%-------------------------------------------------------------------------------
\startTable{Member functions}
\addFootNotes{Member functions of the \codeinline{private_memory} class}
{table.members.private.memory}
  \addRow
    {T \&operator()(const h_item<Dimensions> \&id)}
    {
      Retrieve a reference to the object for the \glspl{work-item}.
    }
\completeTable
%-------------------------------------------------------------------------------

\Gls{private-memory} is allocated per underlying \gls{work-item}, not per
iteration of the \codeinline{parallel_for_work_item} loop. The number
of instances of a private memory object is only under direct control
if a work-group size is passed to the
\codeinline{parallel_for_work_group} call. If the underlying
work-group size is chosen by the runtime, the number of private memory
instances is opaque to the program. Explicit private memory
declarations should therefore be used with care and with a full
understanding of which instances of a
\codeinline{parallel_for_work_item} loop will share the same
underlying variable.

Also within the lambda body can be a sequence of calls to
\codeinline{parallel_for_work_item}. At the edges of these inner
parallel executions the work-group synchronizes. As a result the pair
of \codeinline{parallel_for_work_item} calls in the code below is
equivalent to the parallel execution with a \gls{work-group-barrier} in the earlier example.

\lstinputlisting{code/parallelforworkgroup.cpp}

It is valid to use more flexible dimensions of the work-item loops. In
the following example we issue 8 work-groups but let the runtime
choose their size, by not passing a work-group size to the
\codeinline{parallel_for_work_group} call. The
\codeinline{parallel_for_work_item} loops may also vary in size, with
their execution ranges unrelated to the dimensions of the work-group,
and the compiler generating an appropriate iteration space to fill the
gap. In this case, the \codeinline{h_item} provides access to local ids and
ranges that reflect both kernel and \codeinline{parallel_for_work_item} invocation ranges.

\lstinputlisting{code/parallelforworkgroup2.cpp}

This interface offers a more intuitive way for tiling parallel
programming paradigms. In summary, the hierarchical model allows a
developer to distinguish the execution at work-group level and at
work-item level using the \codeinline{parallel_for_work_group} and the nested
\tf{parallel_for_work_item} functions. It also provides this visibility
to the compiler without the need for difficult loop fission such that
host execution may be more efficient.


%***********************************************************************************
% SYCL functions for explicit memory operations
%***********************************************************************************
\subsection{SYCL functions for explicit memory operations}
\label{subsec:explicitmemory}

In addition to \glspl{kernel}, \gls{command-group} objects can also be used to
perform manual operations on host and device memory by using the 
\keyword{copy} API of the \gls{handler}.
Manual copy operations can be seen as specialized kernels executing on the 
device, except that typically this operations will be implemented using the
OpenCL host API (e.g, enqueue copy operations).

The SYCL memory objects involved in a copy operation are specified using
accessors.
Explicit copy operations have a source and a destination.
When an accessor is the \textit{source} of the operation, the destination can be 
a host pointer or another accessor.
The \textit{source} accessor can have either \codeinline{read} or
\codeinline{read_write} access mode.

When an accessor is the \textit{destination} of the explicit copy operation,
the source can be a host pointer or another accessor.
The \textit{destination} accessor can have either
\codeinline{write}, \codeinline{read_write}, \codeinline{discard_write},
\codeinline{discard_read_write} access modes.

When accessors are both the origin and the destination,
the operation is executed on objects controlled by the SYCL runtime.
The SYCL runtime is allowed to not perfom an explicit in-copy operation
if a different path to update the data is available according to
the SYCL Application Memory Model.

The most recent copy of the memory object may reside on any context controlled
by the SYCL runtime, or on the host in a pointer controlled by the
SYCL runtime.  The SYCL runtime will ensure that data is copied to the destination
once the \gls{command-group} has completed execution.

Whenever a host pointer is used as either the host or the destination of these
explicit memory operations, it is the responsibility
of the user for that pointer to have at least as much memory allocated as
the accessor is giving access to, e.g: if an accessor accesses a range
of 10 elements of \codeinline{int} type, the host pointer must at least have 
\codeinline{10 * sizeof(int)} bytes of memory allocated.

A special case is the \codeinline{update_host} method. 
This method only requires an accessor, and instructs the runtime to update
the internal copy of the data in the host, if any. This is particularly
useful when users use manual synchronization with host pointers, e.g.\ 
via mutex objects on the \codeinline{buffer} constructors.

Table~\ref{table.members.handler.copy} describes the interface for the 
explicit copy operations.

%-------------------------------------------------------------------------------
\startTable{Member function}
\addFootNotes{Member functions of the \codeinline{handler} class}
{table.members.handler.copy}
  \addRowThreeL
    {template <typename T, int dim, access::mode mode, access::target tgt>}
    {void copy(accessor<T, dim, mode, tgt> src,}
    { shared_ptr_class<T> dest)}
    { Copies the contents of the memory pointed to by \codeinline{src} into the
      memory object accessed by \codeinline{dest}.
      \codeinline{src} must have at least as many bytes as the 
      range accessed by \codeinline{dest}.}
  \addRowThreeL
    {template <typename T, int dim, access::mode mode, access::target tgt>}
    {void copy(shared_ptr_class<T> src}
    { accessor<T, dim, mode, tgt> dest)}
    { Copies the contents of the memory object accessed via \codeinline{src} 
      into the memory pointed to by \codeinline{dest}.
      \codeinline{dest} must have at least as many bytes as the 
      range accessed by \codeinline{src}.}
  \addRowThreeL
    {template <typename T, int dim, access::mode mode, access::target tgt>}
    {void copy(accessor<T, dim, mode, tgt> src,}
    { T * dest)}
    { Copies the contents of the memory pointed to by \codeinline{src} into the
      memory object accessed by \codeinline{dest}.
      \codeinline{src} must have at least as many bytes as the 
      range accessed by \codeinline{dest}.}
  \addRowThreeL
    {template <typename T, int dim, access::mode mode, access::target tgt>}
    {void copy(const T * src}
    { accessor<T, dim, mode, tgt> dest)}
    { Copies the contents of the memory object accessed via \codeinline{src} 
      into the memory pointed to by \codeinline{dest}.
      \codeinline{dest} must have at least as many bytes as the 
      range accessed by \codeinline{src}.}
  \addRowThreeL
    {template <typename T, int dim, access::mode mode, access::target tgt>}
    {void copy(accessor<T, dim, mode, tgt> src}
    { accessor<T, dim, mode, tgt> dest)}
    { Copies the contents of the memory object accessed by \codeinline{src} 
      into the memory object accessed by \codeinline{dest}.
      \codeinline{src} must have at least as many bytes as the 
      range accessed by \codeinline{dest}.}
  \addRowTwoL
    {template <typename T, int dim, access::mode mode, access::target tgt>}
    {void update_host(accessor<T, dim, mode, tgt> acc)}
    { The contents of the memory object accessed via \codeinline{acc} 
      on the host are guaranteed to be up-to-date after this 
      \gls{command-group} object execution is complete.}
  \addRowThreeL
    {template <typename T, int dim, access::mode mode, access::target tgt>}
    {void fill(accessor<T, dim, mode, tgt> dest,}
    {          const T\& src)}
    {Replicates the value of \codeinline{src} into the
      memory object accessed by \codeinline{dest}.
      T must be an integral scalar value or a SYCL vector type.
    }

\completeTable

The listing below illustrates how to use explicit copy 
operations in SYCL. The example copies half of the contents of 
a \codeinline{vector_class} into the device, leaving the rest of the 
contents of the buffer on the device unchanged.

\lstinputlisting{code/explicitcopy.cpp}


%%% Local Variables:
%%% mode: latex
%%% TeX-master: "sycl-1.2.1"
%%% TeX-auto-untabify: t
%%% TeX-PDF-mode: t
%%% ispell-local-dictionary: "american"
%%% End:


%*******************************************************************************
% Kernel class
%*******************************************************************************
\subsection{Kernel class}
\label{subsec:kernel.class}

The \codeinline{kernel} class is an abstraction of a \gls{kernel} object
in SYCL. In the
most common case the kernel object will contain the compiled version of a kernel
invoked inside a command group using one of the parallel interface functions as
described in~\ref{subsec:invokingkernels}. The \gls{sycl-runtime} will create
a kernel object, when it needs to enqueue the kernel on a command queue.

In the case where a developer would like to pre-compile a kernel or compile and
link it with an existing program, then the kernel object will be created and
contain that kernel using the program class, as defined
in~\ref{sec:interfaces.program.class}. In both of the above cases, the developer
cannot instantiate a kernel object but can instantiate a named function object type
that they could use, or create a function object from a kernel method using C++11 features.
The kernel class object needs a \codeinline{parallel_for(...)} invocation or an
explicitly built SYCL \codeinline{kernel} instance, for this compilation of the kernel to be triggered.

The SYCL \codeinline{kernel} class provides the common reference semantics
(see Section~\ref{sec:reference-semantics}).

The kernel class also provides the interface for getting information from a
kernel object. The kernel information descriptor interface is described in%
~\ref{appendix.kernel.descriptors} and the description is in the
Table~\ref{table.kernel.info}.

The constructors and member functions of the SYCL \codeinline{kernel} class are listed in Tables~\ref{table.constructors.kernel} and \ref{table.members.kernel}, respectively. The additional common special member functions and common member functions are listed in Tables~\ref{table.specialmembers.common.reference} and \ref{table.members.common.reference}, respectively.

%Interface for class: kernel
\lstinputlisting{headers/kernelWIP.h}

%-------------------------------------------------------------------------------
\startTable{Constructor}
\addFootNotes{Constructors of the SYCL \codeinline{kernel} class}
{table.constructors.kernel}
  \addRow{kernel (cl_kernel clKernel, const context\& syclContext)}
  {
    Constructs a SYCL \codeinline{kernel} instance from an OpenCL \codeinline{cl_kernel} in accordance with the requirements described in \ref{sec:opencl-interoperability}.
    The SYCL \codeinline{context} must represent the same underlying OpenCL context associated with the OpenCL kernel object.
  }
\completeTable
%-------------------------------------------------------------------------------
%-------------------------------------------------------------------------------
\startTable{Member functions}
\addFootNotes{Member functions of the \codeinline{kernel} class}
{table.members.kernel}
  \addRow
    {cl_kernel get() const}
    {   
      Returns a valid \codeinline{cl_kernel} instance in accordance with the requirements described in \ref{sec:opencl-interoperability}.
     }
  \addRow
  {bool is_host() const}
  {
    Returns true if this SYCL \codeinline{kernel} is a host kernel.
  }
  \addRow
    {context get_context() const}
    {
        Return the context that this kernel is defined for.
        The value returned must be equal to that returned by \codeinline{get_info<info::kernel::context>()}.
    }

  \addRow
    {program get_program() const}
    {
        Return the program that this kernel is part of.
        The value returned must be equal to that returned by \codeinline{get_info<info::kernel::program>()}.
    }

  \addRowFourL
    {template <info::kernel param>}
    {typename info::param_traits<}
    {info::kernel, param>::return_type}
    {get_info() const}
    {
        Query information from the kernel object using the
        \codeinline{info::kernel_info} descriptor.
    }


  \addRowFourL
    {template <info::kernel_work_group param>}
    {typename info::param_traits<}
    {info::kernel_work_group, param>::return_type}
    {get_work_group_info(const device \&dev) const}
    {
        Query information from the work-group from a kernel using the
        \codeinline{info::kernel_work_group} descriptor for a specific
        device
    }

\completeTable
%-------------------------------------------------------------------------------
\fixme{info table consistency changes: add namespace and enum class to the
descriptor.}

%-------------------------------------------------------------------------------
\startInfoTable{Kernel Descriptors}
\addInfoFootNotes{Kernel class information descriptors}
{table.kernel.info}

\addInfoRow
{info::kernel::function_name}
{string_class}
{Return the kernel function name.}

\addInfoRow
{info::kernel::num_args}
{cl_uint}
{Return the number of arguments to the extracted OpenCL C kernel.}

\addInfoRow
{info::kernel::context}
{context}
{Return the SYCL \codeinline{context} associated with this SYCL \codeinline{kernel}.}

\addInfoRow
{info::kernel::program}
{program}
{Return the SYCL \codeinline{program} associated with this SYCL \codeinline{kernel}.}

\addInfoRow
{info::kernel::reference_count}
{cl_uint}
{
  Returns the reference count of the encapsulated SYCL \codeinline{cl_kernel},
  if this SYCL \codeinline{kernel} is an OpenCL program. Must thrown a \codeinline{invalid_object_error} SYCL exception if this SYCL \codeinline{kernel} is a host kernel.
}

\addInfoRow
{info::kernel::attributes}
{string_class}
{
  % Curious in this case we need to escape this __
  Return any attributes specified using the \codeinline{\__attribute__} qualifier
  with the kernel function declaration in the program source.
}

\completeInfoTable

\startInfoTable{Kernel Work-group Information Descriptors}
\addInfoFootNotes{Kernel work-group information descriptors}
{table.kernel.workgroupinfo}

\addInfoRow
{info::kernel_work_group::global_work_size}
{range<3>}
{Returns the maximum global work size. Only valid if device is of device_type 
 custom or the kernel is a built-in OpenCL kernel.}

\addInfoRow
{info::kernel_work_group::work_group_size}
{size_t}
{Returns the maximum work-group size that can be used to execute 
  a kernel on a specific device. }

\addInfoRow
{info::kernel_work_group::compile_work_group_size}
{range<3>}
{Returns the work-group size specified by the device compiler if applicable, 
  otherwise returns $(0,0,0)$}

\addInfoRow
{info::kernel_work_group::preferred_work_group_size_multiple}
{size_t}
{
  Returns a value, of which work-group size is preferred to be a multiple,
  for executing a kernel on a particular device.  This is a performance
  hint.  The value must be less than or equal to that returned by
  \codeinline{info::kernel_work_group::work_group_size}.
}

\addInfoRow
{info::kernel_work_group::private_mem_size}
{cl_ulong}
{
  Returns the minimum amount of private memory, in bytes, used by each work-item
  in the kernel. This value may include any private memory needed by an 
  implementation to execute the kernel, including that used by the language
  built-ins and variables declared inside the kernel in the private address
  space.
}

\completeInfoTable

%-------------------------------------------------------------------------------

%*******************************************************************************
% Program class
%*******************************************************************************
\subsection{Program class}
\label{sec:interfaces.program.class}

The SYCL \codeinline{program} class encapsulates a single SYCL program. A SYCL program may be an OpenCL program, in which case it must encapsulate a valid underlying OpenCL \codeinline{cl_program}, depending on it's state, or it may be a SYCL host program, in which case it must not.

A SYCL \codeinline{program} can be used to compile and link both SYCL programs and OpenCL programs.

A SYCL \codeinline{program} instance can be in one of three states defined by \codeinline{program_state}: 
\begin{itemize}
\item A SYCL \codeinline{program} in the \codeinline{program_state::none} state must have no encapsulated \codeinline{cl_program}.
\item A SYCL \codeinline{program} in the \codeinline{program_state::compiled} state must encapsulate a \codeinline{cl_program} that has been compiled but not yet linked, if that SYCL \codeinline{program} is an OpenCL program. It must have no encapsulated \codeinline{cl_program} if that SYCL \codeinline{program} is a host program.
\item A SYCL \codeinline{program} in the \codeinline{program_state::linked} state must encapsulate a \codeinline{cl_program} that has been either compiled and linked or built, if that SYCL \codeinline{program} is an OpenCL program. It must have no encapsulated \codeinline{cl_program} if that SYCL \codeinline{program} is a host program.
\end{itemize}

A SYCL \codeinline{program} host program must follow the same state changes as an OpenCL program, however the transitions are implementation defined.

All member functions of the \codeinline{program} class are synchronous and errors are handled by throwing synchronous SYCL exceptions.

There is no default constructor for the SYCL \codeinline{program} as all constructors require a SYCL \codeinline{context} instance to be provided. 
The only exception is a constructor taking a \codeinline{vector_class} containing SYCL \codeinline{program} instances.
This constructor links them together into a new SYCL \codeinline{program}.

The encapsulated \codeinline{cl_program} of an OpenCL program can contain either SYCL kernel functions or OpenCL C kernel functions. When a \codeinline{program} instance is constructed using a non-OpenCL interoperability constructor, it is in the \codeinline{program_state::none} state and should then be compiled or built by specifying the SYCL kernel name (either the type of the function object or the explicit kernel name type specified when defining the SYCL kernel function). When a \codeinline{program} instance is constructed using an OpenCL interoperability constructor, it can be in either the \codeinline{program_state::compiled} or \codeinline{program_state::linked} state  and should not be compiled or built, only linked.

The compiler options that can be provided are described in the OpenCL specification \cite[p.~145, \S~5.6.4]{opencl12} and the linker options that can be provided are described in \cite[p.~148,\S~5.6.5]{opencl12}.

The SYCL \codeinline{program} class provides the common reference semantics
(see Section~\ref{sec:reference-semantics}).

\subsubsection{Program interface}

A synopsis of the SYCL \codeinline{program} class is provided below. The constructors and member functions of the SYCL \codeinline{program} class are listed in Tables~\ref{table.constructors.program} and \ref{table.members.program} respectively. The additional common special member functions and common member functions are listed in
\ref{sec:reference-semantics} in Tables~\ref{table.specialmembers.common.reference} and \ref{table.members.common.reference}, respectively.

%Interface for class: program
\lstinputlisting{headers/programWIP.h}

%------------------------------------------------------------------------------------------------------
\startTable{Constructor}
\addFootNotes{Constructors of the SYCL \codeinline{program} class}
{table.constructors.program}
  \addRow
    {program () = delete}
    {
      Default constructor is deleted.
    }
  \addRowTwoL
    {explicit program (}
    {const context \&context)}
    {
      Constructs an instance of SYCL \codeinline{program} in the \codeinline{program_state::none} state, associated with the \codeinline{context} provides and the SYCL \codeinline{device}s that are associated with the \codeinline{context}.
    }
  \addRowThreeL
    {program (}
    {const context \&context,}
    {vector_class<device> deviceList)}
    {
      Constructs an instance of SYCL \codeinline{program} in the \codeinline{program_state::none} state, associated with the \codeinline{context} provides and \codeinline{deviceList}.
    }
  \addRowThreeL
    {program (}
    {vector_class<program> programList,}
    {string_class linkOptions = "")}
    {
      Constructs an instance of SYCL \codeinline{program} in the \codeinline{program_state::linked} by linking together each SYCL \codeinline{program} instance in \codeinline{programList}. Each SYCL \codeinline{program} in \codeinline{programList} must be in the \codeinline{program_state::compiled} state and must be associated with the same SYCL \codeinline{context}. Otherwise must throw an \codeinline{invalid_object_error} SYCL exception.  Must throw a \codeinline{feature_not_supported} SYCL exception if any device that the program is to be linked for returns \codeinline{false} for the device information query \codeinline{info::device::is_linker_available}.

    }
  \addRowThreeL
    {program (}
    {const context \&context,}
    {cl_program clProgram)}
    {
      Constructs a SYCL \codeinline{program} instance from an OpenCL \codeinline{cl_program} in accordance with the requirements described in \ref{sec:opencl-interoperability}. The state of the constructed SYCL \codeinline{program} can be either \codeinline{program_state::compiled} or \codeinline{program_state::linked}, depending on the state of the \codeinline{clProgram}. Otherwise must throw an \codeinline{invalid_object_error} SYCL exception.
    }
\completeTable
%-------------------------------------------------------------------------------

%-------------------------------------------------------------------------------
\startTable{Member functions}
\addFootNotes{Member functions of the SYCL \codeinline{program} class}{table.members.program}
  \addRow
    {cl_program get() const}
    {
      Returns a valid \codeinline{cl_program} instance in accordance with the requirements described in \ref{sec:opencl-interoperability}. Must throw an \codeinline{invalid_object_error} SYCL exception if this \codeinline{program} is in the \codeinline{program_state::none} state.
    }
  \addRow
    {bool is_host() const}
    {
      Returns true if this SYCL \codeinline{program} is a host program.
    }
  \addRowThreeL
    {template<typename kernelT>}
    {void compile_with_kernel_type(}
    {string_class compileOptions = "")}
    {
      Compiles the SYCL kernel function defined by the type \codeinline{kernelT} into the encapsulated \codeinline{cl_program} with the compiler options specified by \codeinline{compileOptions}, if this SYCL \codeinline{program} is an OpenCL program. Sets the state of this SYCL \codeinline{program} to \codeinline{program_state::compiled}. Must throw an \codeinline{invalid_object_error} SYCL exception if this \codeinline{program} was not in the \codeinline{program_state::none} state when called. Must throw a  \codeinline{compile_program_error} SYCL exception if the compilation fails.  Must throw a \codeinline{feature_not_supported} SYCL exception if any device that the program is to be compiled for returns \codeinline{false} for the device information query \codeinline{info::device::is_compiler_available}.
    }
  \addRowTwoL
    {void compile_with_source(string_class kernelSource, }
    {string_class compileOptions = "")}
    {
      Compiles the OpenCL C kernel function defined by \codeinline{kernelSource} into the encapsulated \codeinline{cl_program} with the compiler options specified by \codeinline{compileOptions}, if this SYCL \codeinline{program} is an OpenCL program. Sets the state of this SYCL \codeinline{program} to \codeinline{program_state::compiled}. Must throw an \codeinline{invalid_object_error} SYCL exception if this \codeinline{program} was not in the \codeinline{program_state::none} state when called. Must throw a  \codeinline{compile_program_error} SYCL exception if the compilation fails.  Must throw a \codeinline{feature_not_supported} SYCL exception if any device that the program is to be compiled for returns \codeinline{false} for the device information query \codeinline{info::device::is_compiler_available}.
    }
  \addRowThreeL
    {template<typename kernelT>}
    {void build_with_kernel_type(}
    {string_class buildOptions = "")}
    {
      Builds the SYCL kernel function defined by the type \codeinline{kernelT} into the encapsulated \codeinline{cl_program} with the compiler options specified by \codeinline{buildOptions}, if this SYCL \codeinline{program} is an OpenCL program. Sets the state of this SYCL \codeinline{program} to \codeinline{program_state::linked}. Must throw an \codeinline{invalid_object_error} SYCL exception if this \codeinline{program} was not in the \codeinline{program_state::none} state when called. Must throw a  \codeinline{compile_program_error} SYCL exception if the compilation fails.  Must throw a \codeinline{feature_not_supported} SYCL exception if any device that the program is to be built for returns \codeinline{false} for the device information queries \codeinline{info::device::is_compiler_available} or \codeinline{info::device::is_linker_available}.

    }
  \addRowTwoL
    {void build_with_source(string_class kernelSource, }
    {string_class buildOptions = "")}
    {
      Builds the OpenCL C kernel function defined by \codeinline{kernelSource} into the encapsulated \codeinline{cl_program} with the compiler options specified by \codeinline{buildOptions}, if this SYCL \codeinline{program} is an OpenCL program. Sets the state of this SYCL \codeinline{program} to \codeinline{program_state::linked}. Must throw an \codeinline{invalid_object_error} SYCL exception if this \codeinline{program} was not in the \codeinline{program_state::none} state when called. Must throw a  \codeinline{compile_program_error} SYCL exception if the compilation fails.  Must throw a \codeinline{feature_not_supported} SYCL exception if any device that the program is to be built for returns \codeinline{false} for the device information queries \codeinline{info::device::is_compiler_available} or \codeinline{info::device::is_linker_available}.
    }
  \addRow
    {void link(string_class linkOptions = "")}
    {
      Links the encapsulated \codeinline{cl_program} with the compiler options specified by \codeinline{linkOptions}, if this SYCL \codeinline{program} is an OpenCL program. Sets the state of this SYCL \codeinline{program} to \codeinline{program_state::linked}. Must throw an \codeinline{invalid_object_error} SYCL exception if this \codeinline{program} was not in the \codeinline{program_state::compiled} state when called. Must throw a \codeinline{compile_program_error} SYCL exception if the linking fails.  Must throw a \codeinline{feature_not_supported} SYCL exception if any device that the program is to be linked for returns \codeinline{false} for the device information query \codeinline{info::device::is_linker_available}.
    }
  \addRowTwoL
    {template <typename kernelT>}
    {bool has_kernel<kernelT>() const}
    {
      Returns true if the SYCL kernel function defined by the type \codeinline{kernelT} is an available kernel, either within the the encapsulated \codeinline{cl_program}, if this SYCL \codeinline{program} is an OpenCL program, or on the host if this SYCL \codeinline{program} is a host program, otherwise returns false. Must throw an \codeinline{invalid_object_error} SYCL exception if this SYCL \codeinline{program} is in the \codeinline{program_state::none} state.
    }
  \addRow
    {bool has_kernel(string_class kernelName) const}
    {
      Returns true if the OpenCL C kernel function defined by the \codeinline{string_class} \codeinline{kernelName} is an available kernel within the encapsulated \codeinline{cl_program} and this SYCL \codeinline{program} is not a host program, otherwise returns false. Must throw an \codeinline{invalid_object_error} SYCL exception if this SYCL \codeinline{program} is in the \codeinline{program_state::none} state.
    }
  \addRowTwoL
    {template <typename kernelT>}
    {kernel get_kernel<kernelT>() const}
    {
      Returns a SYCL \codeinline{kernel} OpenCL kernel instance encapsulating a \codeinline{cl_kernel} for the SYCL kernel function defined by the type \codeinline{kernelT}, if this SYCL \codeinline{program} is an OpenCL program. Returns a SYCL \codeinline{kernel} host kernel if this SYCL \codeinline{program} is a host program. Must throw an \codeinline{invalid_object_error} SYCL exception if this SYCL \codeinline{program} is in the \codeinline{program_state::none} state or if the SYCL kernel function specified by \codeinline{kernelT} is not available in this SYCL \codeinline{program}.
    }
  \addRow
    {kernel get_kernel(string_class kernelName) const}
    {
      Returns a SYCL \codeinline{kernel} OpenCL kernel instance encapsulating a \codeinline{cl_kernel} for the OpenCL C kernel function defined by the \codeinline{string_class} \codeinline{kernelName}, if this SYCL \codeinline{program} is an OpenCL program. Must throw an \codeinline{invalid_object_error} SYCL exception if this SYCL \codeinline{program} is a host program, this SYCL \codeinline{program} is in the \codeinline{program_state::none} state or the \codeinline{cl_program} encapsulated by this SYCL \codeinline{program} does not contain the OpenCL C kernel function specified by \codeinline{kernelName}. Returns a SYCL \codeinline{kernel} host kernel if this SYCL \codeinline{program} is a host program.
    }
  \addRowFourL
    {template<info::program param>}
    {typename info::param_traits<}
    {info::program, param>::return_type}
    { get_info() const}
    {
      Queries this SYCL \codeinline{program} for information requested by the template parameter \codeinline{param}.
      Specializations of \codeinline{info::param_traits} must be defined in accordance with the info parameters in Table~\ref{table.program.info} to facilitate returning the type associated with the \codeinline{param} parameter.
    }
    \addRow
    {vector_class<vector_class<char>> get_binaries() const}
    {
      Returns a \codeinline{vector_class} of \codeinline{vector_class<char>} representing the compiled binaries for each associated SYCL \codeinline{device}. Must throw an \codeinline{invalid_object_error} SYCL exception if this \codeinline{program} was not in the \codeinline{program_state::compiled} or \codeinline{program_state::linked} states when called.
    }
  \addRow
    {context get_context() const}
    {
        Returns the SYCL \codeinline{context} that this SYCL \codeinline{program} was constructed with.
        The value returned must be equal to that returned by \codeinline{get_info<info::program::context>()}.
        }
  \addRow
    {vector_class<device> get_devices() const}
    {
        Returns a \codeinline{vector_class} containing all SYCL \codeinline{device}s that are associated with this SYCL \codeinline{program}.
        The value returned must be equal to that returned by \codeinline{get_info<info::program::devices>()}.
    }
  \addRow
    {string_class get_compile_options() const}
    {
       Returns the compile options that were provided when the encapsulated \codeinline{cl_program} was explicitly compiled.  If the program was built instead of explicitly compiled, if the program has not yet been compiled, or if the program has been compiled for only the host device (which does not have an underlying \codeinline{cl_program}), then an empty string is returned.  If the program was constructed from a \codeinline{cl_program}, then an empty string is returned unless the \codeinline{cl_program} was explicitly compiled, in which case the compile options used in the explicit compile are returned.
    }
  \addRow
    {string_class get_link_options() const}
    {
       Returns the link options that were provided to the most recent invocation of \codeinline{program::link}.  If the program has not been explicitly linked using \codeinline{program::link}, constructed with an explicitly linking constructor, or if the program has been linked for only the host device, then an empty string is returned.  If the program was constructed from a \codeinline{cl_program}, then an empty string is returned unless the \codeinline{cl_program} was explicitly linked, in which case the link options used in that explicit link are returned.  If the program object was constructed using a contructor form that links a vector of programs (and leaves the program in \codeinline{program_state::linked}), then the link options passed to this constructor are returned.
    }
  \addRow
    {string_class get_build_options() const}
    {
       Returns the compile, link, or build options, from whichever of those operations was performed most recently on the encapsulated \codeinline{cl_program}.  If no compile, link, or build operations have been performed on this SYCL \codeinline{program} object, or if the \codeinline{program} only includes the host device in its \codeinline{deviceList}, then an empty string is returned.
    }
  \addRow
    {program_state get_state() const}
    {
       Returns the current state of this SYCL \codeinline{program}.
    }
\completeTable
%-------------------------------------------------------------------------------

\subsubsection{Program information descriptors}

A SYCL \codeinline{program} can be queried for all of the following information using the \codeinline{get_info} member function. All SYCL \codeinline{program}s must have valid values for every query, including a host program. The information that can be queried is described in Table~\ref{table.program.info}. The interface for all information types and enumerations are described in appendix~\ref{appendix.program.descriptors}.

%-------------------------------------------------------------------------------
\startInfoTable{Program Descriptor}
\addInfoFootNotes{Program class information descriptors}
{table.program.info}
\addInfoRow
    {info::program::reference_count}
    {cl_uint}
    {
      Returns the reference count of the encapsulated SYCL \codeinline{cl_program},
  if this SYCL \codeinline{program} is an OpenCL program. Must throw an \codeinline{invalid_object_error} SYCL exception if this SYCL \codeinline{program} is a host program.
    }
\addInfoRow
    {info::program::context}
    {context}
    {
      Returns the SYCL \codeinline{context} associated with this \codeinline{program}.
    }
  \addInfoRow
    {info::program::devices}
    {vector_class<device>}
    {
      Returns a \codeinline{vector_class} containing the SYCL \codeinline{device}s that this \codeinline{program} has been compiled for.
    }
\completeInfoTable

%*******************************************************************************
% Defining kernels
%*******************************************************************************
\subsection{Defining kernels}
\fixme{updated text for kernel functions.}
In SYCL functions that are executed in parallel on a SYCL device are referred to
as \gls{sycl-kernel-function}. A \gls{kernel} containing such a
\gls{sycl-kernel-function} is enqueued on a device queue in order to
be executed on
that particular device. The return type of the \gls{sycl-kernel-function} is
\tf{void}, and all kernel accesses between host and device are defined using the
accessor class~\ref{subsec:accessors}.

There are three ways of defining kernels, defining them as named function objects,
lambda functions or as OpenCL \tf{cl_kernel} objects. However, in the case of
OpenCL kernels, the developer is expected to have created the kernel and set the
kernel arguments.

\subsubsection{Defining kernels as named function objects}
\label{sec:interfaces.kernels.as.function-objects}

A kernel can be defined as a named function object type. These function objects
provide the same functionality as any C++ function object, with the
restriction that they need to follow C++11 standard layout rules.
The kernel function can be templated via templating the kernel
function object type. The \tf{operator()} function may take different
parameters depending on the data accesses defined for the
specific kernel. For details on restrictions for kernel naming,
please refer to~\ref{sec:naming.kernels}.

The following example defines a \gls{sycl-kernel-function}, \textit{RandomFiller},
which initializes a buffer with a random number. 
The random number is generated during the construction of the function 
object while processing the command group.
The \codeinline{operator()} member function of the function object receives an \codeinline{item} object.
This method will be called for each work item of the execution range.
The value of the random number will be assigned to each element of the
buffer.
In this case, the accessor and the scalar random number are members of the
function object and therefore will be parameters to the device kernel.
Usual restrictions of passing parameters to kernels apply.


\lstinputlisting{code/myfunctor.cpp}

%*******************************************************************************
% Defining kernels as lambda functions
%*******************************************************************************
\subsubsection{Defining kernels as lambda functions}
\label{sec:interfaces.kernels.as.lambdas}

In C++11, function objects can be defined using lambda functions. We allow lambda
functions to define kernels in SYCL, but we have an extra requirement to
name \keyword{lambda functions} in order to enable the linking of the SYCL
device kernels with the host code to invoke them. The name of a lambda function
in SYCL is a C++ class. If the lambda function relies on template arguments,
then the name of the lambda function must contain those template arguments. The
class used for the name of a lambda function is only used for naming purposes
and is not required to be defined. For details on restrictions for kernel
naming, please refer to~\ref{sec:naming.kernels}.

To invoke a C++11 lambda, the kernel name must be included explicitly by the
user as a template parameter to the kernel invoke function.

The kernel function for the lambda function is the lambda function itself.
The kernel lambda must use copy for all of its captures (i.e.\ \tf{[=]}).

\lstinputlisting{code/mykernel.cpp}

%*******************************************************************************
% Defining kernels using program objects
%*******************************************************************************
\subsubsection{Defining kernels using program objects}

In case the developer needs to specify compiler flags or special linkage options
for a kernel, then a kernel object can be used, as described in
\ref{sec:interfaces.program.class}. The \gls{sycl-kernel-function} is defined
as a named function object
\ref{sec:interfaces.kernels.as.function-objects} or lambda function
\ref{sec:interfaces.kernels.as.lambdas}. The user can obtain a program
object for the kernel with the \codeinline{get_kernel} method. This
method is templated by the \gls{kernel-name}, so that the user
can specify the kernel whose associated kernel they wish to obtain.

In the following example, the kernel is defined as a lambda function.
The example obtains the program object for the lambda function kernel
and then passes it to the \codeinline{parallel_for}.

\lstinputlisting{code/myprogram.cpp}

In the above example, the \gls{sycl-kernel-function} is defined in the
\codeinline{parallel_for} invocation as part of a lambda function which is named
using the type of the forward declared class ``myKernel''. The type of the
function object and the program object enable the compilation and linking of the kernel
in the program class, \emph{a priori} of its actual invocation as a kernel
object. For more details on the SYCL device compiler please refer to
chapter~\ref{chapter.device.compiler}.

In the next example, a SYCL kernel is linked with an existing pre-compiled
OpenCL C program object to create a combined program object, which is then
called in a \codeinline{parallel_for}.

\lstinputlisting{code/myprogramlink.cpp}

%*******************************************************************************
% Defining kernels using OpenCL C kernel objects
%*******************************************************************************
\subsubsection{Defining kernels using OpenCL C kernel objects}
\label{sec:interfaces.kernels.opencl.objects}

In OpenCL C \cite{opencl12} program and kernel objects can be created
using the OpenCL C API, which is available in the SYCL
system. Interoperability of OpenCL C kernels and the SYCL system is
achieved by allowing the creation of a SYCL \codeinline{kernel} object from
an OpenCL \gls{kernel} object.

The constructor using kernel objects from \ref{table.constructors.kernel}:
\begin{lstlisting}[style=nonumbers]
kernel::kernel(cl_kernel kernel, const context& syclContext)
\end{lstlisting}

creates a \codeinline{kernel} which can be enqueued using
all of the \tf{parallel_for} functions that can enqueue a kernel object.
This way of defining kernels assumes that
the developer is using OpenCL C to create the kernel and to set the
kernel arguments. The system assumes that the developer has already
set kernel arguments when they are trying to enqueue the
kernel. Buffers do give ownership to their accessors on specific
contexts and the developer can enqueue OpenCL kernels in the same way
as enqueuing SYCL kernels.  However, the system is not responsible for
data management at this point.  Note that like all constructors from
OpenCL API objects, constructing a \codeinline{kernel} from
a \codeinline{cl_kernel} will retain a reference to the kernel and the
user code should call \codeinline{clReleaseKernel} if the
\codeinline{cl_kernel} is no longer needed in the calling context.


%*******************************************************************************
% OpenCL Kernel Conventions and SYCL
%*******************************************************************************
\subsection{OpenCL Kernel Conventions and SYCL}
\label{sec:opencl_kern_conventions_and_sycl}

OpenCL and SYCL use opposite conventions for the unit stride dimension.  SYCL
aligns with C++ conventions, which is important to understand from a performance
perspective when porting code to SYCL.  The unit stride dimension, at least for data, is implicit in the
linearization equations in SYCL (Equation~\ref{row-major-equation-buffer}) and OpenCL.  SYCL aligns with C++ array subscript ordering
\codeinline{arr[a][b][c]}, in that range constructor dimension ordering used to launch a kernel
(e.g. \codeinline{range<3> R\{a,b,c\}}) and range and ID queries within a kernel, are ordered in
the same way as the C++ multi-dimensional subscript operators (unit stride on the right).

When specifying a \codeinline{range} as the global or local size
in a \codeinline{parallel_for} that invokes an OpenCL interop kernel (through
\codeinline{cl_kernel} interop or \codeinline{compile_with_source}/
\codeinline{build_with_source}),
the highest dimension of the range in SYCL will map to the
lowest dimension within the OpenCL kernel.  That statement applies to both
an underlying enqueue operation such as \codeinline{clEnqueueNDRangeKernel}
in OpenCL, and also ID and size queries within the OpenCL kernel.
For example, a 3D global range specified in SYCL as:

\begin{lstlisting}[style=nonumbers]
range<3> R{r0,r1,r2};
\end{lstlisting}

maps to an \codeinline{clEnqueueNDRangeKernel} \codeinline{global_work_size} argument
of:

\begin{lstlisting}[style=nonumbers]
size_t cl_interop_range[3] = {r2,r1,r0};
\end{lstlisting}

Likewise, a 2D global range specified in SYCL as:

\begin{lstlisting}[style=nonumbers]
range<2> R{r0,r1};
\end{lstlisting}

maps to an \codeinline{clEnqueueNDRangeKernel} \codeinline{global_work_size} argument
of:

\begin{lstlisting}[style=nonumbers]
size_t cl_interop_range[2] = {r1,r0};
\end{lstlisting}

The mapping of highest dimension in SYCL to lowest dimension in OpenCL applies to all
operations where a multi-dimensional construct must be mapped, such as when mapping SYCL
explicit memory operations to OpenCL APIs like \codeinline{clEnqueueCopyBufferRect}.

Work-item and work-group ID and range queries have the same reversed convention for unit
stride dimension between SYCL and OpenCL.  For example, with three, two, or one dimensional SYCL
global ranges, OpenCL and SYCL kernel code queries relate to the range as shown in Table~\ref{table.syclOpenCL.mapping}.
The ``SYCL kernel query'' column applies for SYCL-defined kernels, and the ``OpenCL kernel query'' column
applies for kernels defined through OpenCL interop.

\startGenericThreeColTable{8cm}{3cm}{3cm}{SYCL kernel query}{OpenCL kernel query}{Returned Value}
\multicolumn{3}{|c|}{\cellcolor{Gray}With enqueued 3D SYCL global \codeinline{range} of \codeinline{range<3> R\{r0,r1,r2\}}} \\ \hline
\genericThreeColRow {nd_item::get_global_range(0) / item::get_range(0)}{get_global_size(2)}{\codeinline{r0}}
\genericThreeColRow {nd_item::get_global_range(1) / item::get_range(1)}{get_global_size(1)}{\codeinline{r1}}
\genericThreeColRow {nd_item::get_global_range(2) / item::get_range(2)}{get_global_size(0)}{\codeinline{r2}}
\genericThreeColRow {nd_item::get_global_id(0) / item::get_id(0)}{get_global_id(2)}{Value in range 0..(\codeinline{r0}-1)}
\genericThreeColRow {nd_item::get_global_id(1) / item::get_id(1)}{get_global_id(1)}{Value in range 0..(\codeinline{r1}-1)}
\genericThreeColRow {nd_item::get_global_id(2) / item::get_id(2)}{get_global_id(0)}{Value in range 0..(\codeinline{r2}-1)}
\multicolumn{3}{|c|}{\cellcolor{Gray}With enqueued 2D SYCL global \codeinline{range} of \codeinline{range<2> R\{r0,r1\}}} \\ \hline
\genericThreeColRow {nd_item::get_global_range(0) / item::get_range(0)}{get_global_size(1)}{\codeinline{r0}}
\genericThreeColRow {nd_item::get_global_range(1) / item::get_range(1)}{get_global_size(0)}{\codeinline{r1}}
\genericThreeColRow {nd_item::get_global_id(0) / item::get_id(0)}{get_global_id(1)}{Value in range 0..(\codeinline{r0}-1)}
\genericThreeColRow {nd_item::get_global_id(1) / item::get_id(1)}{get_global_id(0)}{Value in range 0..(\codeinline{r1}-1)}
\multicolumn{3}{|c|}{\cellcolor{Gray}With enqueued 1D SYCL global \codeinline{range} of \codeinline{range<1> R\{r0\}}} \\ \hline
\genericThreeColRow {nd_item::get_global_range(0) / item::get_range(0)}{get_global_size(0)}{\codeinline{r0}}
\genericThreeColRow {nd_item::get_global_id(0) / item::get_id(0)}{get_global_id(0)}{Value in range 0..(\codeinline{r0}-1)}
\completeGenericTabular
\captionGenericTable{Example range mapping from SYCL enqueued three dimensional global \codeinline{range} to OpenCL and SYCL queries}
\label{table.syclOpenCL.mapping}
\completeGenericTable


%***********************************************************************************
% Rules for parameter passing to kernels
%***********************************************************************************
\subsection{Rules for parameter passing to kernels}
\label{sec:kernel.parameter.passing}
In a case where a kernel is a named function object or a lambda function, any
member variables encapsulated within the function object or variables captured by
the lambda function must be treated according to the following rules:

\begin{itemize}
  \item
    Any accessor must be passed as an argument to the device kernel in
    a form that allows the device kernel to access the data in the
    specified way. For OpenCL 1.0--1.2 class devices, this means that
    the argument must be passed via \tf{clSetKernelArg} and be
    compiled as a kernel parameter of the valid reference type. For
    global shared data access, the parameter must be an OpenCL
    \tf{global} pointer. For an accessor that specifies OpenCL
    \tf{constant} access, the parameter must be an OpenCL
    \tf{constant} pointer. For images, the accessor must be passed as
    an \tf{image_t} and/or sampler.

  \item
    The \gls{sycl-runtime} and compiler(s) must produce the necessary
    conversions to enable accessor arguments from the host to be
    converted to the correct type of parameter on the device.

  \item
    A local accessor provides access to work-group-local memory. The
    accessor is not constructed with any buffer, but instead
    constructed with a size and base data type. The runtime must
    ensure that the work-group-local memory is allocated per
    work-group and available to be used by the kernel via the local
    accessor.

  \item
    C++ standard layout values must be passed by value to the kernel.

  \item
    C++ non-standard layout values must not be passed as arguments to
    a kernel that is compiled for a device.

  \item
    It is illegal to pass a buffer or image (instead of an accessor
    class) as an argument to a kernel. Generation of a compiler error
    in this illegal case is optional.

  \item
    Sampler objects (\codeinline{sampler}) can be passed as
    parameters to kernels.

  \item
    It is illegal to pass a pointer or reference argument to a
    kernel. Generation of a compiler error in this illegal case is
    optional.  \item Any aggregate types such as structs or classes
    should follow the rules above recursively. It is not necessary to
    separate struct or class members into separate OpenCL kernel
    parameters if all members of the aggregate type are unaffected by
    the rules above.

\end{itemize}

%%% Local Variables:
%%% mode: latex
%%% TeX-master: "sycl-1.2.1"
%%% TeX-auto-untabify: t
%%% TeX-PDF-mode: t
%%% ispell-local-dictionary: "american"
%%% End:


%************************
% Error Handling
%************************

\section{Error handling}
\label{error-handling}

\subsection{Error Handling Rules}

Error handling in SYCL uses exceptions. If an error occurs, it can
be propagated at the point of a function call. An exception will
be thrown and may be caught by the user using standard C++ exception handling
mechanisms. For example, any exception which is triggered from code executed on the
host is able to be propagated at the call site and it will follow the standard
C++ exception handling mechanisms.

SYCL applications are asynchronous in the sense that host and device code executions
are executed asynchronously. As a result of this, the errors that occur on a
device cannot be propagated directly from the call site, and they will not be
detected until the error-causing task executes or tries to execute.
We refer to those errors as asynchronous errors.
A good example of an asynchronous error is an out-of-bounds access error.
In this case, if the kernel is enqueued on a SYCL OpenCL device, then the
out-of-bounds error is asynchronous with respect to the SYCL host application,
because it is executed on the device. The standard exception mechanisms
will not be available as this is an asynchronous error.

SYCL queues are by default asynchronous, as they schedule tasks on SYCL devices.
The queue constructor can optionally get an asynchronous handler object
\codeinline{async_handler}, which is a function class instance. If waiting and
exception handling member functions are used on queues, the \gls{async-handler} receives
a list of C++ exception objects.

If an asynchronous error occurs in a queue that has no user-supplied
asynchronous error handler object \codeinline{async_handler}, then no exception
is thrown and the error is not available to the user in any specified way.
Implementations may provide extra debugging information to users to trap and handle
asynchronous errors. If a synchronous error occurs in a SYCL application and it
is not handled, the application will exit abnormally.

If an error occurs when running or enqueuing a command group which has
a secondary queue specified, then the command group may be enqueued
to the secondary queue instead of the primary queue. The error handling in this
case is also configured using the \gls{async-handler} provided for both
queues. If there is no \gls{async-handler} given on any of the queues,
then no asynchronous error reporting is done and no exceptions are thrown. If
the primary queue fails and there is an \gls{async-handler} given at
this queue's construction, which populates the \codeinline{exception_list}
parameter, then any errors will be added and can be thrown whenever the user
chooses to handle those exceptions. Since there were errors on the primary
queue and a secondary queue was given, then the execution of the kernel is
re-scheduled to the secondary queue and any error reporting for the kernel
execution on that queue is done through that queue, in the same way as
described above. The secondary queue may fail as well, and the errors will be
thrown if there is an \gls{async-handler} and either
\codeinline{wait_and_throw()} or \codeinline{throw()} are called on that queue.
The \gls{command-group-function-object} event returned by that function will be
relevant to the queue where the kernel has been enqueued.

\subsection{Exception Class Interface}
\label{subsec:exception.class}
\lstset{
       style=nonumbers,
       basicstyle=\ttfamily\small,
       backgroundcolor=\color{white},
       frame=none,
     }
\lstinputlisting{headers/exception.h}

The SYCL \codeinline{exception_ptr_class}
class is used to store SYCL \codeinline{exception} objects and allows
exception objects to be transferred between threads. It is equivalent to the
\codeinline{exception_ptr_class} class. The SYCL \codeinline{exception_list}
class is also available in order to provide a list of synchronous and
asynchronous exceptions.

There are two categories of errors, the \codeinline{runtime_error}
that refers to the scheduling errors that may happen during execution, and the
\codeinline{device_error} that refers to the execution errors on a SYCL device.

Errors can occur both in the SYCL library and SYCL host side, as well as the
OpenCL runtime and device side. The member functions on these exceptions provide the
corresponding information.
If there is an OpenCL error associated with the exception triggered, then the
OpenCL error code will be given by the method \codeinline{get_cl_code()}.
In the case where there is no OpenCL error associated with the exception
triggered, the OpenCL error code will be 0.

The asynchronous handler object \codeinline{async_handler} is a
\codeinline{function_class} with an \codeinline{exception_list} as a parameter.
The asynchronous handler is an optional parameter to a constructor of the
\codeinline{queue} class and it is the only way to handle asynchronous errors
occurring on a SYCL device. The asynchronous handler may be a named function object
type, a lambda function or a \codeinline{function_class},
that can be given to the queue and 
will be executed on error. The \codeinline{exception_list} object is constructed
from the \gls{sycl-runtime} and is populated with the errors caught during the
execution of all the kernels running on the same queue.


%---------------------------------------------------------------------
\startTable{Member function}
\addFootNotes{Member functions of the SYCL \codeinline{exception} class}
{table.members.exception}
  \addRow
    {const char *what() const}
    {Returns an implementation defined non-null constant C-style string that describes the error that triggered the exception.}
  \addRow
    {bool has_context() const}
    {Returns \codeinline{true} if this SYCL \codeinline{exception} has an associated SYCL \codeinline{context} and \codeinline{false} if it does not.}
  \addRow
    {context get_context() const}
    {Returns the SYCL \codeinline{context} that is associated with this SYCL \codeinline{exception} if one is available. Must throw an \codeinline{invalid_object_error} SYCL exception if this SYCL \codeinline{exception} does not have a SYCL \codeinline{context}.
    }
  \addRow
    {cl_int get_cl_code() const}
    {
      Returns the OpenCL error code if the exception was thrown as an OpenCL error, otherwise returns \codeinline{CL_SUCCESS}.
    }
\completeTable

%------------------------------------------------------------------------------------------------------
%------------------------------------------------------------------------------------------------------
\startTable{Member function}

\addFootNotes{Member functions of the \codeinline{exception_list}}
{table.members.exceptionlist}

\addRow
{ size_t size() const }
{ Returns the size of the list }
\addRow
 {iterator begin() const }
 {Returns an iterator to the beginning of the list of asynchronous exceptions.}
\addRow
 {iterator end() const}
 {Returns an iterator to the end of the list of asynchronous exceptions.}

\completeTable
%------------------------------------------------------------------------------------------------------
%------------------------------------------------------------------------------------------------------
\startTable{Runtime Error Exception Type}
\addFootNotes{Exceptions types that derive from the \codeinline{runtime_error} class}{table.runtime.error.types}
\addRow
{kernel_error}
{Error that occurred before or while enqueuing the SYCL kernel.}
\addRow
{nd_range_error}
{Error regarding the SYCL \codeinline{nd_range} specified for the SYCL kernel}
\addRow
{accessor_error}
{Error regarding the SYCL \codeinline{accessor} objects defined.}
\addRow
{event_error}
{Error regarding associated SYCL \codeinline{event} objects.}
\addRow
{invalid_parameter_error}
{Error regarding parameters to the SYCL kernel, it may apply to any captured parameters
to the kernel lambda.}
\completeTable
%------------------------------------------------------------------------------------------------------
%------------------------------------------------------------------------------------------------------
\startTable{Device Error Exception Type}
\addFootNotes{Exception types that derive from the SYCL \codeinline{device_error} class}{table.device.error.types}

\addRow
{compile_program_error}
{Error while compiling the SYCL kernel to a SYCL device.}

\addRow
{link_program_error}
{Error while linking the SYCL kernel to a SYCL device.}

\addRow
{invalid_object_error}
{Error regarding any memory objects being used inside the kernel}

\addRow
{memory_allocation_error}
{Error on memory allocation on the SYCL device for a SYCL kernel.}

\addRow
{platform_error}
{The SYCL platform will trigger this exception on error.}

\addRow
{profiling_error}
{The \gls{sycl-runtime} will trigger this error if there is an error when profiling info
is enabled.}

\addRow
{feature_not_supported}
{Exception thrown when an optional feature or extension is used in a kernel
but its not available on the device the SYCL kernel is being enqueued on.}
\completeTable
%------------------------------------------------------------------------------------------------------

%***********************************************************************************
% Data types
%***********************************************************************************
\section{Data types}

SYCL as a C++11 programming model supports the C++11 ISO standard data types,
and it also provides the ability for all SYCL applications to be executed on SYCL
compatible devices, OpenCL and host devices. The scalar and vector data types that
are supported by the SYCL system are defined below. More details about the SYCL
device compiler support for fundamental and OpenCL interoperability types are found
in~\ref{subsection:scalartypes}.

%***********************************************************************************
% Scalar data types
%***********************************************************************************
\subsection{Scalar data types}

The fundamental C++ data types which are supported in SYCL are described in
Table~\ref{table.types.fundamental}. Note these types are fundamental and therefore
do not exist within the \codeinline{cl::sycl} namespace.

Additional scalar data types which are supported by SYCL within the \codeinline{
cl::sycl} namespace are described in Table~\ref{table.types.additional}.

%-------------------------------------------------------------------------------
\startTable{Scalar data type}
\addFootNotes{Additional scalar data types supported by SYCL}{table.types.additional}
\addRow
{
  byte
}
{
  A signed or unsigned 8-bit integer, as defined by the C++11 ISO Standard.
}
\completeTable
%-------------------------------------------------------------------------------

The OpenCL C language standard~\cite[\S 6.11]{opencl12} defines its own built-in
scalar data types, and these have additional requirements in terms of size and
signedness on top of what is guaranteed by ISO C++. For the purpose of interoperability and portability, SYCL defines a set of aliases to C++ types
within the \codeinline{cl::sycl} namespace using the \codeinline{cl_} prefix.
These aliases are described in Table~\ref{table.types.aliases}

%-------------------------------------------------------------------------------
\startTable{Scalar data type alias}
\addFootNotes{Scalar data type aliases supported by SYCL}{table.types.aliases}
\addRow
{
  cl_bool
}
{
  Alias to a conditional data type which can be either true or false. The value
  true expands to the integer constant 1 and the value false expands to the
  integer constant 0.
}
\addRow
{
  cl_char
}
{
  Alias to a signed 8-bit integer, as defined by the C++11 ISO Standard.
}
\addRow
{
  cl_uchar
}
{
  Alias to an unsigned 8-bit integer, as defined by the C++11 ISO Standard.
}
\addRow
{
 cl_short
}
{
  Alias to a signed 16-bit integer, as defined by the C++11 ISO Standard.
}
\addRow
{
  cl_ushort
}
{
  Alias to an unsigned 16-bit integer, as defined by the C++11 ISO Standard.
}
\addRow
{
 cl_int
}
{
  Alias to a signed 32-bit integer, as defined by the C++11 ISO Standard.
}
\addRow
{
 cl_uint
}
{
  Alias to an unsigned 32-bit integer, as defined by the C++11 ISO Standard.
}
\addRow
{
 cl_long
}
{
  Alias to a signed 64-bit integer, as defined by the C++11 ISO Standard.
}
\addRow
{
 cl_ulong
}
{
  Alias to an unsigned 64-bit integer, as defined by the C++11 ISO Standard.
}
\addRow
{
  cl_float
}
{
  Alias to a 32-bit floating-point. The float data type must conform to the IEEE
  754 single precision storage format.
}
\addRow
{
 cl_double
}
{
  Alias to a 64-bit floating-point. The double data type must conform to the IEEE
  754 double precision storage format.
}
\addRow
{
  cl_half
}
{
  Alias to a 16-bit floating-point. The half data type must conform to the IEEE
  754-2008 half precision storage format. A SYCL \codeinline{feature_not_supported}
  exception must be thrown if the \codeinline{half} type is used in a SYCL kernel
  fucntion which executes on a SYCL \codeinline{device} that does not support the
  extension \codeinline{khr_fp16}.
}
\completeTable
%-----------------------------------------------------------------------------------

%***********************************************************************************
% Vector types
%***********************************************************************************
\subsection{Vector types}
\label{sec:vector.type}
% Copyright (c) 2012-2019 Khronos Group.
%
% This work is licensed under a Creative Commons Attribution 4.0
% International License.
% http://creativecommons.org/licenses/by/4.0/

% !TEX root = sycl-1.2.1.tex

%*******************************************************************************
% Vector types
%*******************************************************************************
%\subsection{Vector types}
%\label{sec:vector.type}

SYCL provides a cross-platform class template that works
efficiently on SYCL devices as well as in host C++ code. This type
allows sharing of vectors between the host and its SYCL devices. The
vector supports methods that allow construction of a new vector from a
swizzled set of component elements.

\tclass{vec}{\tf{typename} dataT, \tf{int} numElements} is a vector type that
compiles down to the OpenCL built-in vector types on OpenCL devices, where
possible, and provides compatible support on the host. The \codeinline{vec}
class is templated on its number of elements and its element type. The number of
elements parameter, \tf{numElements}, can be one of: 1, 2, 3, 4, 8 or 16. Any
other value should produce a compilation failure. The element type parameter,
\tf{dataT}, must be one of the basic scalar types supported in device code.

The SYCL \codeinline{vec} class template provides interoperability with the
underlying OpenCL vector type defined by \codeinline{vector_t} which is
available only when compiled for the device. The SYCL \codeinline{vec} class can
be constructed from an instance of \codeinline{vector_t} and can implicitly
convert to an instance of \codeinline{vector_t} in order to support
interoperability with OpenCL C functions from a SYCL kernel function.

An instance of the SYCL \codeinline{vec} class template can also be
implicitly converted to an instance of the data type when the number of
elements is \codeinline{1} in order to allow single element vectors and
scalars to be convertible with each other.

\subsubsection{Vec interface}

The constructors, member functions and non-member functions of the SYCL \codeinline{vec} class template are listed in Tables~\ref{table.constructors.vec}, \ref{table.members.vec} and \ref{table.functions.vec} respectively.

%Interface for class: vec
\lstinputlistingSkipLicense{headers/vec.h}

%-------------------------------------------------------------------------------
\startTable{Constructor}
\addFootNotes {Constructors of the SYCL \codeinline{vec} class template}
{table.constructors.vec}
  \addRow
    {vec()}
    {
      Default construct a vector with element type \codeinline{dataT} and
      with \codeinline{numElements} dimensions by default construction of
      each of its elements.
    }
  \addRow
    {explicit vec(const dataT \&arg)}
    {
      Construct a vector of element type \codeinline{dataT} and
      \codeinline{numElements} dimensions by setting each value to \codeinline{arg} by
      assignment.
    }
  \addRowTwoL
    {template <typename... argTN>}
    {vec(const argTN\&... args)}
    {
      Construct a SYCL \codeinline{vec} instance from any combination of scalar and SYCL \codeinline{vec} parameters of the same element type, providing the total number of elements for all parameters sum to \codeinline{numElements} of this \codeinline{vec} specialization.
    }    
  \addRow
    {vec(const vec<dataT, numElements> \&rhs)}
    {
      Construct a vector of element type \codeinline{dataT} and number of elements \codeinline{numElements} by copy from another similar vector.
    }
  \addRow
    {vec(vector_t openclVector)}
    {
      Available only when: compiled for the device.
      \newline
      Constructs a SYCL \codeinline{vec} instance from an instance of the underlying OpenCL vector type defined by \codeinline{vector_t}.
    }
\completeTable
%-------------------------------------------------------------------------------

%-------------------------------------------------------------------------------
\startTable{Member function}
\addFootNotes{Member functions for the SYCL \codeinline{vec} class template}
{table.members.vec}
  \addRow
  {operator vector_t() const}
  {
    Available only when: compiled for the device.
    \newline
    Converts this SYCL \codeinline{vec} instance to the underlying OpenCL vector type
    defined by \codeinline{vector_t}.
  }
  \addRow
  {operator dataT() const}
  {
    Available only when: \codeinline{numElements == 1}.
    Converts this SYCL \codeinline{vec} instance to an instance of \codeinline{dataT} with
    the value of the single element in this SYCL \codeinline{vec} instance.
  }
  \addRow
  {size_t get_count() const}
  {
    Returns the number of elements of this SYCL \codeinline{vec}.
  }
  \addRow
  {size_t get_size() const}
  {
    Returns the size of this SYCL \codeinline{vec} in bytes.
    \newline
    3-element vector size matches 4-element vector size to provide
    interoperability with OpenCL vector types. The same rule applies to vector
    alignment as described in \ref{memory-layout-and-alignment}.
  }
  \addRowTwoL
  {template<typename convertT, rounding_mode roundingMode = rounding_mode::automatic>}
  {vec<convertT, numElements> convert() const}
  {
    Converts this SYCL \codeinline{vec} to a SYCL \codeinline{vec} of a different element type specified by \codeinline{convertT} using the rounding mode specified by \codeinline{roundingMode}. The new SYCL \codeinline{vec} type must have the same number of elements as this SYCL \codeinline{vec}. The different rounding modes are described in Table~\ref{table.vec.roundingmodes}.
  }
  \addRowTwoL
  {template<typename asT>}
  {asT as() const}
  {
    Bitwise reinterprets this SYCL \codeinline{vec} as a SYCL \codeinline{vec} of a different element type and number of elements specified by \codeinline{asT}. The new SYCL \codeinline{vec} type must have the same storage size in bytes as this SYCL \codeinline{vec}.
  }
  \addRowTwoL
    { template<int... swizzleIndexes> }
    {\__swizzled_vec__ swizzle() const}
    {
      Return an instance of the implementation defined intermediate class template \codeinline{\__swizzled_vec__} representing an index sequence which can be used to apply the swizzle in a valid expression as described in \ref{swizzled-vec-class}.
    }
  \addRow
    {\__swizzled_vec__ XYZW_ACCESS() const}
    {
      Available only when \codeinline{numElements <= 4}.
      \newline \newline
      Returns an instance of the implementation defined intermediate class template \codeinline{\__swizzled_vec__} representing an index sequence which can be used to apply the swizzle in a valid expression as described in \ref{swizzled-vec-class}.
      \newline \newline
      Where \codeinline{XYZW_ACCESS} is: \codeinline{x} for \codeinline{numElements == 1}, \codeinline{x, y} for \codeinline{numElements == 2}, \codeinline{x, y, z} for \codeinline{numElements == 3} and \codeinline{x, y, z, w} for \codeinline{numElements == 4}.
    }
  \addRow
    {\__swizzled_vec__ RGBA_ACCESS() const}
    {
      Available only when \codeinline{numElements == 4}.
      \newline \newline
      Returns an instance of the implementation defined intermediate class template \codeinline{\__swizzled_vec__} representing an index sequence which can be used to apply the swizzle in a valid expression as described in \ref{swizzled-vec-class}.
      \newline \newline
      Where \codeinline{RGBA_ACCESS} is: \codeinline{r, g, b, a}.
    }
  \addRow
    {\__swizzled_vec__ INDEX_ACCESS() const}
    {
      Returns an instance of the implementation defined intermediate class template \codeinline{\__swizzled_vec__} representing an index sequence which can be used to apply the swizzle in a valid expression as described in~\ref{swizzled-vec-class}.
      \newline \newline
      Where \codeinline{INDEX_ACCESS} is: \codeinline{s0} for \codeinline{numElements == 1}, \codeinline{s0, s1} for \codeinline{numElements == 2}, \codeinline{s0, s1, s2} for \codeinline{numElements == 3}, \codeinline{s0, s1, s2, s3} for \codeinline{numElements == 4}, \codeinline{s0, s1, s2, s3, s4, s5, s6, s7, s8} for \codeinline{numElements == 8} and \codeinline{s0, s1, s2, s3, s4, s5, s6, s7, s8, s9, sA, sB, sC, sD, sE, sF} for \codeinline{numElements == 16}.
    }
  \addRow
    {\__swizzled_vec__ XYZW_SWIZZLE() const}
    {
      Available only when \codeinline{numElements <= 4}, and when the macro \codeinline{SYCL_SIMPLE_SWIZZLES} is defined before including \codeinline{cl/sycl.hpp}.
      \newline \newline
      Returns an instance of the implementation defined intermediate class template \codeinline{\__swizzled_vec__} representing an index sequence which can be used to apply the swizzle in a valid expression as described in~\ref{swizzled-vec-class}.
    \newline \newline
      Where XYZW\_SWIZZLE is all permutations with repetition, of any subset with length greater than \codeinline{1}, of \codeinline{x, y} for \codeinline{numElements == 2}, \codeinline{x, y, z} for \codeinline{numElements == 3} and \codeinline{x, y, z, w} for \codeinline{numElements == 4}. For example a four element \codeinline{vec} provides permutations including \codeinline{xzyw}, \codeinline{xyyy} and \codeinline{xz}.
    }
  \addRow
    {\__swizzled_vec__ RGBA_SWIZZLE() const}
    {
      Available only when \codeinline{numElements == 4}, and when the macro \codeinline{SYCL_SIMPLE_SWIZZLES} is defined before including \codeinline{cl/sycl.hpp}.
      \newline \newline
      Returns an instance of the implementation defined intermediate class template \codeinline{\__swizzled_vec__} representing an index sequence which can be used to apply the swizzle in a valid expression as described in \ref{swizzled-vec-class}.
    \newline \newline
    Where RGBA\_SWIZZLE is all permutations with repetition, of any subset with length greater than \codeinline{1}, of \codeinline{r, g, b, a}.
    For example a four element \codeinline{vec} provides permutations including \codeinline{rbga}, \codeinline{rggg} and \codeinline{rb}.
    }
  \addRow
    {\__swizzled_vec__ lo() const}
    {
      Available only when: \codeinline{numElements > 1}.
      Return an instance of the implementation defined intermediate class template \codeinline{\__swizzled_vec__} representing an index sequence made up of the lower half of this SYCL vec which can be used to apply the swizzle in a valid expression as described in \ref{swizzled-vec-class}. When \codeinline{numElements == 3} this SYCL \codeinline{vec} is treated as though \codeinline{numElements == 4} with the fourth element undefined.
    }
  \addRow
    {\__swizzled_vec__ hi() const}
    {
      Available only when: \codeinline{numElements > 1}.
      Return an instance of the implementation defined intermediate class template \codeinline{\__swizzled_vec__} representing an index sequence made up of the upper half of this SYCL vec which can be used to apply the swizzle in a valid expression as described in \ref{swizzled-vec-class}. When \codeinline{numElements == 3} this SYCL \codeinline{vec} is treated as though \codeinline{numElements == 4} with the fourth element undefined.
    }
  \addRow
    {\__swizzled_vec__ odd() const}
    {
      Available only when: \codeinline{numElements > 1}.
      Return an instance of the implementation defined intermediate class template \codeinline{\__swizzled_vec__} representing an index sequence made up of the odd indexes of this SYCL vec which can be used to apply the swizzle in a valid expression as described in \ref{swizzled-vec-class}. When \codeinline{numElements == 3} this SYCL \codeinline{vec} is treated as though \codeinline{numElements == 4} with the fourth element undefined.
    }    
  \addRow
    {\__swizzled_vec__ even() const}
    {
      Available only when: \codeinline{numElements > 1}.
      Return an instance of the implementation defined intermediate class template \codeinline{\__swizzled_vec__} representing an index sequence made up of the even indexes of this SYCL vec which can be used to apply the swizzle in a valid expression as described in \ref{swizzled-vec-class}. When \codeinline{numElements == 3} this SYCL \codeinline{vec} is treated as though \codeinline{numElements == 4} with the fourth element undefined.
    }
  \addRowTwoL
  {template <access::address_space addressSpace>}
  {void load(size_t offset, multi_ptr<const dataT, addressSpace> ptr)}
  {
    Loads the values at the address of \codeinline{ptr} offset in elements of type \codeinline{dataT} by \codeinline{numElements * offset}, into the components of this SYCL \codeinline{vec}.
  }
  \addRowTwoL
  {template <access::address_space addressSpace>}
  {void store(size_t offset, multi_ptr<dataT, addressSpace> ptr) const}
  {
    Stores the components of this SYCL \codeinline{vec} into the values at the address of \codeinline{ptr} offset in elements of type \codeinline{dataT} by \codeinline{numElements * offset}.
  }

  \addRowTwoL
    {vec<dataT, numElements> \&operator=(}
    {  const vec<dataT, numElements> \&rhs)}
    {
      Assign each element of the \codeinline{rhs} SYCL \codeinline{vec} to each element of this SYCL \codeinline{vec} and return a reference to this SYCL \codeinline{vec}.
    }
  \addRowTwoL
    {vec<dataT, numElements> \&operator=(}
    {  const dataT \&rhs)}
    {
      Assign each element of the \codeinline{rhs} scalar to each element of this SYCL \codeinline{vec} and return a reference to this SYCL \codeinline{vec}.
    }
 \completeTable
%-------------------------------------------------------------------------------

%-------------------------------------------------------------------------------
\startTable{Hidden friend function}
\addFootNotes{Hidden friend functions of the \codeinline{vec} class template}
{table.functions.vec}

  \addRow
  {vec operatorOP(const vec \&lhs, const vec \&rhs)}
  {
    When \codeinline{OP} is \codeinline{\%} available only when: \codeinline{dataT != cl_float \&\& dataT != cl_double \&\& dataT != cl_half}.
    \newline
    Construct a new instance of the SYCL \codeinline{vec} class template with the same template parameters as \codeinline{lhs} \codeinline{vec} with each element of the new SYCL \codeinline{vec} instance the result of an element-wise \codeinline{OP} arithmetic operation between each element of \codeinline{lhs} \codeinline{vec} and each element of the \codeinline{rhs} SYCL \codeinline{vec}.
    \newline \newline
    Where \codeinline{OP} is: \codeinline{+}, \codeinline{-}, \codeinline{*}, \codeinline{/}, \codeinline{\%}.
  }

  \addRow
  {vec operatorOP(const vec \&lhs, const dataT \&rhs)}
  {
    When \codeinline{OP} is \codeinline{\%} available only when: \codeinline{dataT != cl_float \&\& dataT != cl_double \&\& dataT != cl_half}.
    \newline
    Construct a new instance of the SYCL \codeinline{vec} class template with the same template parameters as \codeinline{lhs} \codeinline{vec} with each element of the new SYCL \codeinline{vec} instance the result of an element-wise \codeinline{OP} arithmetic operation between each element of \codeinline{lhs} \codeinline{vec} and the \codeinline{rhs} scalar.
    \newline \newline
    Where \codeinline{OP} is: \codeinline{+}, \codeinline{-}, \codeinline{*}, \codeinline{/}, \codeinline{\%}.
  }

  \addRow
  {vec \&operatorOP(vec \&lhs, const vec \&rhs)}
  {
    When \codeinline{OP} is \codeinline{\%=} available only when: \codeinline{dataT != cl_float \&\& dataT != cl_double \&\& dataT != cl_half}.
    \newline
    Perform an in-place element-wise \codeinline{OP} arithmetic operation between each element of \codeinline{lhs} \codeinline{vec} and each element of the \codeinline{rhs} SYCL \codeinline{vec} and return \codeinline{lhs} \codeinline{vec}.
    \newline \newline
    Where \codeinline{OP} is: \codeinline{+=}, \codeinline{-=}, \codeinline{*=}, \codeinline{/=}, \codeinline{\%=}.
  }

   \addRow
  {vec \&operatorOP(vec \&lhs, const dataT \&rhs)}
  {
    When \codeinline{OP} is \codeinline{\%=} available only when: \codeinline{dataT != cl_float \&\& dataT != cl_double \&\& dataT != cl_half}.
    \newline
    Perform an in-place element-wise \codeinline{OP} arithmetic operation between each element of \codeinline{lhs} \codeinline{vec} and \codeinline{rhs} scalar and return \codeinline{lhs} \codeinline{vec}.
    \newline \newline
    Where \codeinline{OP} is: \codeinline{+=}, \codeinline{-=}, \codeinline{*=}, \codeinline{/=}, \codeinline{\%=}.
  }

  \addRow
  {vec \&operatorOP(vec \&v)}
  {
    Perform an in-place element-wise \codeinline{OP} prefix arithmetic operation on each element of \codeinline{lhs} \codeinline{vec}, assigning the result of each element to the corresponding element of \codeinline{lhs} \codeinline{vec} and return \codeinline{lhs} \codeinline{vec}.
    \newline \newline
    Where \codeinline{OP} is: \codeinline{++}, \codeinline{--}. 
  }

  \addRow
  {vec operatorOP(vec \&v, int)}
  {
    Perform an in-place element-wise \codeinline{OP} post-fix arithmetic operation on each element of \codeinline{lhs} \codeinline{vec}, assigning the result of each element to the corresponding element of \codeinline{lhs} \codeinline{vec} and returns a copy of \codeinline{lhs} \codeinline{vec} before the operation is performed.
    \newline \newline
    Where \codeinline{OP} is: \codeinline{++}, \codeinline{--}.
  }
  
  \addRow
  {vec operatorOP(const vec \&lhs, const vec \&rhs)}
  {
    Available only when: \codeinline{dataT != cl_float \&\& dataT != cl_double \&\& dataT != cl_half}.
    \newline
    Construct a new instance of the SYCL \codeinline{vec} class template with the same template parameters as \codeinline{lhs} \codeinline{vec} with each element of the new SYCL \codeinline{vec} instance the result of an element-wise \codeinline{OP} bitwise operation between each element of \codeinline{lhs} \codeinline{vec} and each element of the \codeinline{rhs} SYCL \codeinline{vec}.
    \newline \newline
    Where \codeinline{OP} is: \codeinline{\&}, \codeinline{\|}, \codeinline{\^}.
  }

  \addRow
  {vec operatorOP(const vec \&lhs, const dataT \&rhs)}
  {
    Available only when: \codeinline{dataT != cl_float \&\& dataT != cl_double \&\& dataT != cl_half}.
    \newline
    Construct a new instance of the SYCL \codeinline{vec} class template with the same template parameters as \codeinline{lhs} \codeinline{vec} with each element of the new SYCL \codeinline{vec} instance the result of an element-wise \codeinline{OP} bitwise operation between each element of \codeinline{lhs} \codeinline{vec} and the \codeinline{rhs} scalar.
    \newline \newline
    Where \codeinline{OP} is: \codeinline{\&}, \codeinline{\|}, \codeinline{\^}.
  }

  \addRow
  {vec \&operatorOP(vec \&lhs, const vec \&rhs)}
  {
    Available only when: \codeinline{dataT != cl_float \&\& dataT != cl_double \&\& dataT != cl_half}.
    \newline
    Perform an in-place element-wise \codeinline{OP} bitwise operation between each element of \codeinline{lhs} \codeinline{vec} and the \codeinline{rhs} SYCL \codeinline{vec} and return \codeinline{lhs} \codeinline{vec}.
    \newline \newline
    Where \codeinline{OP} is: \codeinline{\&=}, \codeinline{\|=}, \codeinline{\^=}.
  }

  \addRow
  {vec \&operatorOP(vec \&lhs, const dataT \&rhs)}
  {
    Available only when: \codeinline{dataT != cl_float \&\& dataT != cl_double \&\& dataT != cl_half}.
    \newline
    Perform an in-place element-wise \codeinline{OP} bitwise operation between each element of \codeinline{lhs} \codeinline{vec} and the \codeinline{rhs} scalar and return a \codeinline{lhs} \codeinline{vec}.
    \newline \newline
    Where \codeinline{OP} is: \codeinline{\&=}, \codeinline{\|=}, \codeinline{\^=}. 
  }

  \addRow
  {vec<RET, numElements> operatorOP(const vec \&lhs, const vec \&rhs)}
  {
    Construct a new instance of the SYCL \codeinline{vec} class template with the same template parameters as \codeinline{lhs} \codeinline{vec} with each element of the new SYCL \codeinline{vec} instance the result of an element-wise \codeinline{OP} logical operation between each element of \codeinline{lhs} \codeinline{vec} and each element of the \codeinline{rhs} SYCL \codeinline{vec}.
    \newline \newline
    The \codeinline{dataT} template parameter of the constructed SYCL \codeinline{vec}, \codeinline{RET}, varies depending on the \codeinline{dataT} template parameter of this SYCL \codeinline{vec}. For a SYCL \codeinline{vec} with \codeinline{dataT} of type \codeinline{cl_char} or \codeinline{cl_uchar} \codeinline{RET} must be \codeinline{cl_char}. For a SYCL \codeinline{vec} with \codeinline{dataT} of type \codeinline{cl_short}, \codeinline{cl_ushort} or \codeinline{cl_half} \codeinline{RET} must be \codeinline{cl_short}. For a SYCL \codeinline{vec} with \codeinline{dataT} of type \codeinline{cl_int}, \codeinline{cl_uint} or \codeinline{cl_float} \codeinline{RET} must be \codeinline{cl_int}. For a SYCL \codeinline{vec} with \codeinline{dataT} of type \codeinline{cl_long}, \codeinline{cl_ulong} or \codeinline{cl_double} \codeinline{RET} must be \codeinline{cl_long}.
    \newline \newline
    Where \codeinline{OP} is: \codeinline{\&\&}, \codeinline{\|\|}.
  }

  \addRow
  {vec<RET, numElements> operatorOP(const vec \&lhs, const dataT \&rhs)}
  {
    Construct a new instance of the SYCL \codeinline{vec} class template with the same template parameters as this SYCL \codeinline{vec} with each element of the new SYCL \codeinline{vec} instance the result of an element-wise \codeinline{OP} logical operation between each element of \codeinline{lhs} \codeinline{vec} and the \codeinline{rhs} scalar.
    \newline \newline
    The \codeinline{dataT} template parameter of the constructed SYCL \codeinline{vec}, \codeinline{RET}, varies depending on the \codeinline{dataT} template parameter of this SYCL \codeinline{vec}. For a SYCL \codeinline{vec} with \codeinline{dataT} of type \codeinline{cl_char} or \codeinline{cl_uchar} \codeinline{RET} must be \codeinline{cl_char}. For a SYCL \codeinline{vec} with \codeinline{dataT} of type \codeinline{cl_short}, \codeinline{cl_ushort} or \codeinline{cl_half} \codeinline{RET} must be \codeinline{cl_short}. For a SYCL \codeinline{vec} with \codeinline{dataT} of type \codeinline{cl_int}, \codeinline{cl_uint} or \codeinline{cl_float} \codeinline{RET} must be \codeinline{cl_int}. For a SYCL \codeinline{vec} with \codeinline{dataT} of type \codeinline{cl_long}, \codeinline{cl_ulong} or \codeinline{cl_double} \codeinline{RET} must be \codeinline{cl_long}.
    \newline \newline
    Where \codeinline{OP} is: \codeinline{\&\&}, \codeinline{\|\|}.
  }

  \addRow
  {vec operatorOP(const vec \&lhs, const vec \&rhs)}
  {
    Available only when: \codeinline{dataT != cl_float \&\& dataT != cl_double \&\& dataT != cl_half}.
    \newline
    Construct a new instance of the SYCL \codeinline{vec} class template with the same template parameters as \codeinline{lhs} \codeinline{vec} with each element of the new SYCL \codeinline{vec} instance the result of an element-wise \codeinline{OP} bitshift operation between each element of \codeinline{lhs} \codeinline{vec} and each element of the \codeinline{rhs} SYCL \codeinline{vec}. If \codeinline{OP} is \codeinline{>>}, \codeinline{dataT} is a signed type and \codeinline{lhs} \codeinline{vec} has a negative value any vacated bits viewed as an unsigned integer must be assigned the value \codeinline{1}, otherwise any vacated bits viewed as an unsigned integer must be assigned the value \codeinline{0}.
    \newline \newline
    Where \codeinline{OP} is: \codeinline{<<}, \codeinline{>>}.
  }

   \addRow
  {vec operatorOP(const vec \&lhs, const dataT \&rhs)}
  {
    Available only when: \codeinline{dataT != cl_float \&\& dataT != cl_double \&\& dataT != cl_half}.
    \newline
    Construct a new instance of the SYCL \codeinline{vec} class template with the same template parameters as \codeinline{lhs} \codeinline{vec} with each element of the new SYCL \codeinline{vec} instance the result of an element-wise \codeinline{OP} bitshift operation between each element of \codeinline{lhs} \codeinline{vec} and the \codeinline{rhs} scalar. If \codeinline{OP} is \codeinline{>>}, \codeinline{dataT} is a signed type and \codeinline{lhs} \codeinline{vec} has a negative value any vacated bits viewed as an unsigned integer must be assigned the value \codeinline{1}, otherwise any vacated bits viewed as an unsigned integer must be assigned the value \codeinline{0}.
    \newline \newline
    Where \codeinline{OP} is: \codeinline{<<}, \codeinline{>>}.
  }
  
  \addRow
  {vec \&operatorOP(vec \&lhs, const vec \&rhs)}
  {
    Available only when: \codeinline{dataT != cl_float \&\& dataT != cl_double \&\& dataT != cl_half}.
    \newline
    Perform an in-place element-wise \codeinline{OP} bitshift operation between each element of \codeinline{lhs} \codeinline{vec} and the \codeinline{rhs} SYCL \codeinline{vec} and returns \codeinline{lhs} \codeinline{vec}. If \codeinline{OP} is \codeinline{>>\=}, \codeinline{dataT} is a signed type and \codeinline{lhs} \codeinline{vec} has a negative value any vacated bits viewed as an unsigned integer must be assigned the value \codeinline{1}, otherwise any vacated bits viewed as an unsigned integer must be assigned the value \codeinline{0}.
    \newline \newline
    Where \codeinline{OP} is: \codeinline{<<\=}, \codeinline{>>\=}.
  }

  \addRow
  {vec \&operatorOP(vec \&lhs, const dataT \&rhs)}
  {
    Available only when: \codeinline{dataT != cl_float \&\& dataT != cl_double \&\& dataT != cl_half}.
    \newline
    Perform an in-place element-wise \codeinline{OP} bitshift operation between each element of \codeinline{lhs} \codeinline{vec} and the \codeinline{rhs} scalar and returns a reference to this SYCL \codeinline{vec}. If \codeinline{OP} is \codeinline{>>\=}, \codeinline{dataT} is a signed type and \codeinline{lhs} \codeinline{vec} has a negative value any vacated bits viewed as an unsigned integer must be assigned the value \codeinline{1}, otherwise any vacated bits viewed as an unsigned integer must be assigned the value \codeinline{0}.
    \newline \newline
    Where \codeinline{OP} is: \codeinline{<<\=}, \codeinline{>>\=}.
  }

  \addRow
    {vec<RET, numElements> operatorOP(const vec\& lhs, const vec \&rhs)}
    {
      Construct a new instance of the SYCL \codeinline{vec} class template with the element type \codeinline{RET} with each element of the new SYCL \codeinline{vec} instance the result of an element-wise \codeinline{OP} relational operation between each element of \codeinline{lhs} \codeinline{vec} and each element of the \codeinline{rhs} SYCL \codeinline{vec}. Each element of the SYCL \codeinline{vec} that is returned must be \codeinline{-1} if the operation results in \codeinline{true} and \codeinline{0} if the operation results in \codeinline{false} or either this SYCL \codeinline{vec} or the \codeinline{rhs} SYCL \codeinline{vec} is a NaN.
      \newline \newline
      The \codeinline{dataT} template parameter of the constructed SYCL \codeinline{vec}, \codeinline{RET}, varies depending on the \codeinline{dataT} template parameter of this SYCL \codeinline{vec}. For a SYCL \codeinline{vec} with \codeinline{dataT} of type \codeinline{cl_char} or \codeinline{cl_uchar} \codeinline{RET} must be \codeinline{cl_char}. For a SYCL \codeinline{vec} with \codeinline{dataT} of type \codeinline{cl_short}, \codeinline{cl_ushort} or \codeinline{cl_half} \codeinline{RET} must be \codeinline{cl_short}. For a SYCL \codeinline{vec} with \codeinline{dataT} of type \codeinline{cl_int}, \codeinline{cl_uint} or \codeinline{cl_float} \codeinline{RET} must be \codeinline{cl_int}. For a SYCL \codeinline{vec} with \codeinline{dataT} of type \codeinline{cl_long}, \codeinline{cl_ulong} or \codeinline{cl_double} \codeinline{RET} must be \codeinline{cl_long}.
      \newline \newline
      Where \codeinline{OP} is: \codeinline{==}, \codeinline{!=}, \codeinline{<}, \codeinline{>}, \codeinline{<=}, \codeinline{>=}.
    }
    
  \addRow
    {vec<RET, numElements> operatorOP(const vec \&lhs, const dataT \&rhs)}
    {
      Construct a new instance of the SYCL \codeinline{vec} class template with the \codeinline{dataT} parameter of \codeinline{RET} with each element of the new SYCL \codeinline{vec} instance the result of an element-wise \codeinline{OP} relational operation between each element of \codeinline{lhs} \codeinline{vec} and the \codeinline{rhs} scalar. Each element of the SYCL \codeinline{vec} that is returned must be \codeinline{-1} if the operation results in \codeinline{true} and \codeinline{0} if the operation results in \codeinline{false} or either \codeinline{lhs} \codeinline{vec} or the \codeinline{rhs} SYCL \codeinline{vec} is a NaN.
      \newline \newline
      The \codeinline{dataT} template parameter of the constructed SYCL \codeinline{vec}, \codeinline{RET}, varies depending on the \codeinline{dataT} template parameter of this SYCL \codeinline{vec}. For a SYCL \codeinline{vec} with \codeinline{dataT} of type \codeinline{cl_char} or \codeinline{cl_uchar} \codeinline{RET} must be \codeinline{cl_char}. For a SYCL \codeinline{vec} with \codeinline{dataT} of type \codeinline{cl_short}, \codeinline{cl_ushort} or \codeinline{cl_half} \codeinline{RET} must be \codeinline{cl_short}. For a SYCL \codeinline{vec} with \codeinline{dataT} of type \codeinline{cl_int}, \codeinline{cl_uint} or \codeinline{cl_float} \codeinline{RET} must be \codeinline{cl_int}. For a SYCL \codeinline{vec} with \codeinline{dataT} of type \codeinline{cl_long}, \codeinline{cl_ulong} or \codeinline{cl_double} \codeinline{RET} must be \codeinline{cl_long}.
      \newline \newline
      Where \codeinline{OP} is: \codeinline{==}, \codeinline{!=}, \codeinline{<}, \codeinline{>}, \codeinline{<=}, \codeinline{>=}. 
    }

  \addRow
  { vec operatorOP(const dataT \&lhs, const vec \&rhs) }
  {
    When \codeinline{OP} is \codeinline{\%} available only when: \codeinline{
    dataT != cl_float \&\& dataT != cl_double \&\& dataT != cl_half}.
    \newline
    Construct a new instance of the SYCL \codeinline{vec} class template with
    the same template parameters as the \codeinline{rhs} SYCL \codeinline{vec}
    with each element of the new SYCL \codeinline{vec} instance the result of
    an element-wise \codeinline{OP} arithmetic operation between the
    \codeinline{lhs} scalar and each element of the \codeinline{rhs} SYCL
    \codeinline{vec}.
    \newline \newline
    Where \codeinline{OP} is: \codeinline{+}, \codeinline{-}, \codeinline{*},
    \codeinline{/}, \codeinline{\%}.
  }

  \addRow
  {vec operatorOP(const dataT \&lhs, const vec \&rhs)}
  {
    Available only when: \codeinline{
    dataT != cl_float \&\& dataT != cl_double \&\& dataT != cl_half}.
    \newline
    Construct a new instance of the SYCL \codeinline{vec} class template with
    the same template parameters as the \codeinline{rhs} SYCL \codeinline{vec}
    with each element of the new SYCL \codeinline{vec} instance the result of
    an element-wise \codeinline{OP} bitwise operation between the \codeinline{
    lhs} scalar and each element of the \codeinline{rhs} SYCL \codeinline{vec}.
    \newline \newline
    Where \codeinline{OP} is: \codeinline{\&}, \codeinline{|}, \codeinline{^}.
  }

   \addRow
  {vec<RET, numElements> operatorOP(const dataT \&lhs, const vec \&rhs)}
  {
    Available only when: \codeinline{
    dataT != cl_float \&\& dataT != cl_double \&\& dataT != cl_half}.
    \newline
    Construct a new instance of the SYCL \codeinline{vec} class template with
    the same template parameters as the \codeinline{rhs} SYCL \codeinline{vec}
    with each element of the new SYCL \codeinline{vec} instance the result of
    an element-wise \codeinline{OP} logical operation between the \codeinline{
    lhs} scalar and each element of the \codeinline{rhs} SYCL \codeinline{vec}.
    \newline \newline
    The \codeinline{dataT} template parameter of the constructed SYCL \codeinline{vec}, \codeinline{RET}, varies depending on the \codeinline{dataT} template parameter of this SYCL \codeinline{vec}. For a SYCL \codeinline{vec} with \codeinline{dataT} of type \codeinline{cl_char} or \codeinline{cl_uchar} \codeinline{RET} must be \codeinline{cl_char}. For a SYCL \codeinline{vec} with \codeinline{dataT} of type \codeinline{cl_short}, \codeinline{cl_ushort} or \codeinline{cl_half} \codeinline{RET} must be \codeinline{cl_short}. For a SYCL \codeinline{vec} with \codeinline{dataT} of type \codeinline{cl_int}, \codeinline{cl_uint} or \codeinline{cl_float} \codeinline{RET} must be \codeinline{cl_int}. For a SYCL \codeinline{vec} with \codeinline{dataT} of type \codeinline{cl_long}, \codeinline{cl_ulong} or \codeinline{cl_double} \codeinline{RET} must be \codeinline{cl_long}.
    \newline \newline
    Where \codeinline{OP} is: \codeinline{\&\&}, \codeinline{||}.
  }

  \addRow
  {vec operatorOP(const dataT \&lhs, const vec \&rhs)}
  {
    Construct a new instance of the SYCL \codeinline{vec} class template with
    the same template parameters as the \codeinline{rhs} SYCL \codeinline{vec}
    with each element of the new SYCL \codeinline{vec} instance the result of
    an element-wise \codeinline{OP} bitshift operation between the \codeinline{
    lhs} scalar and each element of the \codeinline{rhs} SYCL \codeinline{vec}.
    If \codeinline{OP} is \codeinline{>>}, \codeinline{dataT} is a signed type
    and this SYCL \codeinline{vec} has a negative value any vacated bits viewed
    as an unsigned integer must be assigned the value \codeinline{1}, otherwise
    any vacated bits viewed as an unsigned integer must be assigned the value
    \codeinline{0}.
    \newline \newline
    Where \codeinline{OP} is: \codeinline{<<}, \codeinline{>>}.
  }

  \addRow
    {vec<RET, numElements> operatorOP(const dataT \&lhs, const vec \&rhs)}
    {
      Available only when: \codeinline{
      dataT != cl_float \&\& dataT != cl_double \&\& dataT != cl_half}.
      \newline
      Construct a new instance of the SYCL \codeinline{vec} class template with
      the element type \codeinline{RET} with each element of the new SYCL
      \codeinline{vec} instance the result of an element-wise \codeinline{OP}
      relational operation between the \codeinline{lhs} scalar and each element
      of the \codeinline{rhs} SYCL \codeinline{vec}. Each element of the SYCL
      \codeinline{vec} that is returned must be \codeinline{-1} if the operation
      results in \codeinline{true} and \codeinline{0} if the operation results
      in \codeinline{false} or either this SYCL \codeinline{vec} or the
      \codeinline{rhs} SYCL \codeinline{vec} is a NaN.
      \newline \newline
      The \codeinline{dataT} template parameter of the constructed SYCL
      \codeinline{vec}, \codeinline{RET}, varies depending on the \codeinline{
      dataT} template parameter of this SYCL \codeinline{vec}. For a SYCL
      \codeinline{vec} with \codeinline{dataT} of type \codeinline{cl_char} or
      \codeinline{cl_uchar} \codeinline{RET} must be \codeinline{cl_char}. For a
      SYCL \codeinline{vec} with \codeinline{dataT} of type \codeinline{
      cl_short}, \codeinline{cl_ushort} or \codeinline{cl_half} \codeinline{RET}
      must be \codeinline{cl_short}. For a SYCL \codeinline{vec} with
      \codeinline{dataT} of type \codeinline{cl_int}, \codeinline{cl_uint} or
      \codeinline{cl_float} \codeinline{RET} must be \codeinline{cl_int}. For a
      SYCL \codeinline{vec} with \codeinline{dataT} of type \codeinline{
      cl_long}, \codeinline{cl_ulong} or \codeinline{cl_double} \codeinline{RET}
      must be \codeinline{cl_long}.
      \newline \newline
      Where \codeinline{OP} is: \codeinline{==}, \codeinline{!=}, \codeinline{<}, \codeinline{>}, \codeinline{<=}, \codeinline{>=}.
    }
    
     \addRow
  { vec \&operator~(const vec \&v)}
  {
    Available only when: \codeinline{dataT != cl_float \&\& dataT != cl_double \&\& dataT != cl_half}.
    \newline
    Construct a new instance of the SYCL \codeinline{vec} class template with the same template parameters as \codeinline{v} \codeinline{vec} with each element of the new SYCL \codeinline{vec} instance the result of an element-wise \codeinline{OP} bitwise operation on each element of \codeinline{v} \codeinline{vec}.
  }
  
  \addRow
  {vec<RET, numElements> operator!(const vec \&v)}
  {
    Construct a new instance of the SYCL \codeinline{vec} class template with the same template parameters as \codeinline{v} \codeinline{vec} with each element of the new SYCL \codeinline{vec} instance the result of an element-wise \codeinline{OP} logical operation on each element of \codeinline{v} \codeinline{vec}. Each element of the SYCL \codeinline{vec} that is returned must be \codeinline{-1} if the operation results in \codeinline{true} and \codeinline{0} if the operation results in \codeinline{false} or this SYCL \codeinline{vec} is a NaN.
      \newline \newline
      The \codeinline{dataT} template parameter of the constructed SYCL \codeinline{vec}, \codeinline{RET}, varies depending on the \codeinline{dataT} template parameter of this SYCL \codeinline{vec}. For a SYCL \codeinline{vec} with \codeinline{dataT} of type \codeinline{cl_char} or \codeinline{cl_uchar} \codeinline{RET} must be \codeinline{cl_char}. For a SYCL \codeinline{vec} with \codeinline{dataT} of type \codeinline{cl_short}, \codeinline{cl_ushort} or \codeinline{cl_half} \codeinline{RET} must be \codeinline{cl_short}. For a SYCL \codeinline{vec} with \codeinline{dataT} of type \codeinline{cl_int}, \codeinline{cl_uint} or \codeinline{cl_float} \codeinline{RET} must be \codeinline{cl_int}. For a SYCL \codeinline{vec} with \codeinline{dataT} of type \codeinline{cl_long}, \codeinline{cl_ulong} or \codeinline{cl_double} \codeinline{RET} must be \codeinline{cl_long}.
  }

\completeTable
%-------------------------------------------------------------------------------

\subsubsection{Aliases}

SYCL provides aliases for \codeinline{vec<dataT, numElements>} as \codeinline{
<dataT><numElements>} for the data types: \codeinline{char}, \codeinline{short},
\codeinline{int}, \codeinline{long}, \codeinline{float}, \codeinline{double},
\codeinline{half}, \codeinline{cl_char}, \codeinline{cl_uchar}, \codeinline{
cl_short}, \codeinline{cl_ushort}, \codeinline{cl_int}, \codeinline{cl_uint},
\codeinline{cl_long}, \codeinline{cl_ulong}, \codeinline{cl_float}, \codeinline{
cl_double} and \codeinline{cl_half} and the data types: \codeinline{signed char}
, \codeinline{unsigned char}, \codeinline{unsigned short},
\codeinline{unsigned int}, \codeinline{unsigned long}, \codeinline{long long}
and \codeinline{unsigned long long} represented with the short hand \codeinline{
schar}, \codeinline{uchar}, \codeinline{ushort}, \codeinline{uint}, \codeinline{
ulong}, \codeinline{longlong} and \codeinline{ulonglong} respectively, for
number of elements: \codeinline{2}, \codeinline{3}, \codeinline{4}, \codeinline{
8}, \codeinline{16}. For example \codeinline the alias to \codeinline{
vec<float, 4>} would be \codeinline{float4}.

\subsubsection{Swizzles}

Swizzle operations can be performed in two ways. Firstly by calling the \codeinline{swizzle} member function template, which takes a variadic number of integer template arguments between \codeinline{0} and \codeinline{numElements-1}, specifying swizzle indexes. Secondly by calling one of the simple swizzle member functions defined in \ref{table.members.vec} as \codeinline{XYZW_SWIZZLE} and \codeinline{RGBA_SWIZZLE}. Note that the simple swizzle functions are only available for up to 4 element vectors and are only available when the macro \codeinline{SYCL_SIMPLE_SWIZZLES} is defined before including \codeinline{CL/sycl.hpp}.

In both cases the return type is always an instance of \codeinline{\__swizzled_vec__}, an implementation defined temporary class representing a swizzle of the original SYCL \codeinline{vec} instance. Both kinds of swizzle member functions must not perform the swizzle operation themselves, instead the swizzle operation must be performed by the returned instance of \codeinline{\__swizzled_vec__} when used within an expression, meaning if the returned \codeinline{\__swizzled_vec__} is never used in an expression no swizzle operation is performed.

Both the \codeinline{swizzle} member function template and the simple swizzle member functions allow swizzle indexes to be repeated.

A series of static constexpr values are provided within the \codeinline{elem} struct to allow specifying named swizzle indexes when calling the \codeinline{swizzle} member function template.

\subsubsection{Swizzled vec class}
\label{swizzled-vec-class}

The \codeinline{\__swizzled_vec__} class must define an unspecified temporary which provides the entire interface of the SYCL \codeinline{vec} class template, including swizzled member functions, with the additions and alterations described below:

\begin{itemize}

\item The \codeinline{\__swizzled_vec__} class template must be readable as an r-value reference on the RHS of an expression. In this case the swizzle operation is performed on the RHS of the expression and then the result is applied to the LHS of the expression.

\item The \codeinline{\__swizzled_vec__} class template must be assignable as an l-value reference on the LHS of an expression. In this case the RHS of the expression is applied to the original SYCL \codeinline{vec} which the \codeinline{\__swizzled_vec__} represents via the swizzle operation. Note that a \codeinline{\__swizzled_vec__} that is used in an l-value expression may not contain any repeated element indexes. \newline For example: \codeinline{f4.xxxx() = fx.wzyx()} would not be valid.

\item The \codeinline{\__swizzled_vec__} class template must be convertible to an instance of SYCL \codeinline{vec} with the type \codeinline{dataT} and number of elements specified by the swizzle member function, if \codeinline{numElements > 1}, and must be convertible to an instance of type \codeinline{dataT}, if \codeinline{numElements == 1}.

\item The \codeinline{\__swizzled_vec__} class template must be non-copyable, non-moveable, non-user constructible and may not be bound to a l-value or escape the expression it was constructed in. For example \codeinline{auto x = f4.x()} would not be valid.

\item The \codeinline{\__swizzled_vec__} class template should return \codeinline{\__swizzled_vec__ \&} for each operator inhetired from the \codeinline{vec} class template interface which would return  \codeinline{vec<dataT, numElements> \&}.

\end{itemize}

\subsubsection{Rounding modes}

The various rounding modes that can be used in the \codeinline{as} member function template are described in Table~\ref{table.vec.roundingmodes}.

%-------------------------------------------------------------------------------
\startTable{Rounding mode}
\addFootNotes{Rounding modes for the SYCL \codeinline{vec} class template}
{table.vec.roundingmodes}
  \addRow
    {automatic}
    {
      Default rounding mode for the SYCL \codeinline{vec} class element type. \codeinline{rtz} (round toward zero) for integer types and \codeinline{rte} (round to nearest even) for floating-point types.
    }
  \addRow
    {rte}
    {
      Round to nearest even.
    }
  \addRow
    {rtz}
    {
      Round toward zero.
    }
  \addRow
    {rtp}
    {
      Round toward positive infinity.
    }
  \addRow
    {rtn}
    {
      Round toward negative infinity.
    }  
  \completeTable
%------------------------------------------------------------------------------------------

\subsubsection{Memory layout and alignment}
\label{memory-layout-and-alignment}

The elements of an instance of the SYCL \codeinline{vec} class template are stored in memory sequentially and contiguously and are aligned to the size of the element type in bytes multiplied by the number of elements:

\begin{equation}
\label{vec-memory-alignment}
\texttt{sizeof}(\texttt{dataT}) \cdot \texttt{numElements}
\end{equation}

The exception to this is when the number of element is three in which case the SYCL \codeinline{vec} is aligned to the size of the element type in bytes multiplied by four:

\begin{equation}
\label{vec3-memory-alignment}
\texttt{sizeof}(\texttt{dataT}) \cdot 4
\end{equation}

This is true for both host and device code in order to allow for instances of the \codeinline{vec} class template to be passed to SYCL kernel functions.

\subsubsection{Considerations for endianness}

As SYCL supports both big-endian and little-endian on OpenCL devices, users must
take care to ensure kernel arguments are processed correctly. This is
particularly true for SYCL \codeinline{vec} arguments as the order in which a
SYCL \codeinline{vec} is loaded differs between big-endian and little-endian.

Users should consult vendor documentation for guidance on how to handle kernel
arguments in these situations.


%%% Local Variables:
%%% mode: latex
%%% TeX-master: "sycl-1.2.1"
%%% TeX-auto-untabify: t
%%% TeX-PDF-mode: t
%%% ispell-local-dictionary: "american"
%%% End:


%***********************************************************************************
% Synchronization and atomics
%***********************************************************************************
\section{Synchronization and atomics}

\fixme{UPDATE: moved it in this part of the specification, as it seemed too early
to discuss atomics right at the start of the programming interface before any of
the interfaces were introduced.}

The SYCL specification offers the same set of synchronization
operations that are available to OpenCL C programs, for compatibility
and portability across OpenCL devices. The available features are:
\begin{itemize}
  \item
    Accessor classes: Accessor classes specify acquisition and release
    of buffer and image data structures to provide points at which
    underlying queue synchronization primitives must be generated.
  \item
    Atomic operations: OpenCL 1.2 devices only support the equivalent
    of relaxed C++ atomics and SYCL uses the C++11 library syntax to
    make this available. This is provided for forward compatibility
    with future SYCL versions.
  %\item
  %  Fences:
  \item
    Barriers: Barrier primitives are made available to synchronize
    sets of work-items within individual work-groups. They are exposed
    through the \codeinline{nd_item} class that abstracts the current
    point in the overall iteration space.
  \item
    Hierarchical parallel dispatch: In the hierarchical parallelism
    model of describing computations, synchronization within the
    work-group is made explicit through multiple instances of the
    \codeinline{parallel_for_work_item} function call, rather than
    through the use of explicit \gls{work-group-barrier} operations.
\end{itemize}

A \gls{work-group-barrier} or \gls{work-group-mem-fence} may provide ordering semantics over the local address space, global address space or both. All memory operations initiated before the \gls{work-group-barrier} or \gls{work-group-mem-fence} operation in the specified address space(s) will be completed before any memory operation after the operation. Address spaces are specified using the \codeinline{fence_space} enum class:

\lstinputlisting{headers/synchronization.h}

The SYCL specification provides atomic operations based on the C++11
library syntax. The only available ordering, due to constraints of the
OpenCL 1.2 memory model, is \codeinline{memory_order_relaxed}. No
default order is supported because a default order would imply
sequential consistency. The SYCL atomic library may map directly to
the underlying C++11 library in host code, and must interact safely
with the host C++11 atomic library when used in host code. The SYCL
library must be used in device code to ensure that only the limited
subset of functionality is available. SYCL 1.2.1 device compilers should
give a compilation error on use of the \codeinline{std::atomic}
classes and functions in device code.

The template parameter \codeinline{addressSpace} is permitted to be
\codeinline{access::address_space::global_space} or \codeinline{
access::address_space::local_space}.

The data type \codeinline{T} is permitted to be \codeinline{int}, \codeinline{
unsigned int}, \codeinline{long}, \codeinline{unsigned long}, \codeinline{
long long}, \codeinline{unsigned long long} and \codeinline{float}. Though
\codeinline{float} is only available for the \codeinline{store}, \codeinline{
load} and \codeinline{exchange} member functions. For any data type
\codeinline{T} which is 64bit, the member functions of the \codeinline{atomic}
class are required to compile however are only guaranteed to execute if the
64bit atomic extension \codeinline{cl_khr_int64_base_atomics} or \codeinline{
cl_khr_int64_extended_atomics} (depending on which extension provides support
for each given member function) is supported by the SYCL \codeinline{device}
which is executing the SYCL kernel function. If a member function is called with
a 64bit data type and the necessary extension is not supported by the SYCL
\codeinline{device} which is executing the SYCL kernel function, the
\gls{sycl-runtime} must throw a SYCL \codeinline{feature_not_supported}
exception. For more detail see Section~\ref{sec:extension.64bit-atomics}.

The atomic types are defined as follows, the constructors and member functions
for the SYCL \codeinline{atomic} class are listed in Tables~
\ref{table.atomics.constructors} and~\ref{table.atomics.members} respectively.

\lstinputlisting{headers/atomic.h}

As well as the member functions, a matching set of operations on atomic types
is provided by the SYCL library. As in the previous case, the only available
memory order is \codeinline{memory_order::relaxed}. The global functions are as
follows and described in Table~\ref{table.atomics.functions}.

\lstinputlisting{headers/atomicoperations.h}

The atomic operations and member functions behave as described in the C++11
specification, barring the restrictions discussed above. Note that care
must be taken when using \codeinline{compare_exchange_strong} to
perform many of the operations that would be expected of it in
standard CPU code due to the lack of forward progress guarantees
between work-items in SYCL. No work-item may be dependent on another
work-item to make progress if the code is to be portable.

%-------------------------------------------------------------------------------
\startTable{Constructor}
\addFootNotes{Constructors of the SYCL \codeinline{atomic} class
  template}
{table.atomics.constructors}
  \addRowTwoSL
    { template <typename pointerT> }
    { atomic(multi_ptr<pointerT, addressSpace> ptr) }
    {
      Permitted data types for \codeinline{pointerT} are any valid scalar data
      type which is the same size in bytes as \codeinline{T}. Constructs an
      instance of SYCL \codeinline{atomic} which is associated with the pointer
      \codeinline{ptr}, converted to a pointer of data type \codeinline{T}.
    }
\completeInfoTable
%-------------------------------------------------------------------------------

%-------------------------------------------------------------------------------
\startTable{Member function}
\addFootNotes{Member functions available on an object of type
  \codeinline{atomic<T>}}
{table.atomics.members}
  \addRowTwoL
    { void store(T operand, memory_order memoryOrder = }
    { memory_order::relaxed) }
    {
      Atomically stores the value \codeinline{operand} at the address of the
      \codeinline{multi_ptr} associated with this SYCL \codeinline{atomic}. The
      memory order of this atomic operation must be \codeinline{
      memory_order::relaxed}.
    }
  \addRowTwoL
    { T load(memory_order memoryOrder = }
    { memory_order::relaxed) const }
    {
      Atomically loads the value at the address of the \codeinline{multi_ptr}
      associated with this SYCL \codeinline{atomic}. Returns the value at the
      address of the \codeinline{multi_ptr} associated with this SYCL
      \codeinline{atomic} before the call. The memory order of this atomic
      operation must be \codeinline{memory_order::relaxed}.
    }
  \addRowTwoL
    { T exchange(T operand, memory_order memoryOrder = }
    { memory_order::relaxed) }
    {
      Atomically replaces the value at the address of the \codeinline{multi_ptr}
      associated with this SYCL \codeinline{atomic} with value \codeinline{
      operand} and returns the value at the address of the \codeinline{
      multi_ptr} associated with this SYCL \codeinline{atomic} before the call.
      The memory order of this atomic operation must be \codeinline{
      memory_order::relaxed}.
    }
  \addRowFiveL
    { bool compare_exchange_strong(T \&expected, T desired, }
    { memory_order successMemoryOrder = }
    { memory_order::relaxed, }
    { memory_order failMemoryOrder = }
    { memory_order::relaxed) }
    {
      Available only when: \codeinline{T != float}.
      \newline
      Atomically compares the value at the address of the \codeinline{multi_ptr}
      associated with this SYCL \codeinline{atomic} against the value of
      \codeinline{expected}. If the values are equal replaces value at address
      of the \codeinline{multi_ptr} associated with this SYCL \codeinline{
      atomic} with the value of \codeinline{desired}, otherwise assigns the
      original value at the address of the \codeinline{multi_ptr} associated
      with this SYCL \codeinline{atomic} to \codeinline{expected}. Returns
      \codeinline{true} if the comparison operation was successful. The memory
      order of this atomic operation must be \codeinline{
      memory_order::relaxed} for both success and fail.
    }
  \addRowTwoL
    { T fetch_add(T operand, memory_order memoryOrder = }
    { memory_order::relaxed) }
    {
      Available only when: \codeinline{T != float}.
      \newline
      Atomically adds the value \codeinline{operand} to the value at the address
      of the \codeinline{multi_ptr} associated with this SYCL \codeinline{
      atomic} and assigns the result to the value at the address of the
      \codeinline{multi_ptr} associated with this SYCL \codeinline{atomic}.
      Returns the value at the address of the \codeinline{multi_ptr} associated
      with this SYCL \codeinline{atomic} before the call. The memory order of
      this atomic operation must be \codeinline{memory_order::relaxed}.
    }
  \addRowTwoL
    { T fetch_sub(T operand, memory_order memoryOrder = }
    { memory_order::relaxed) }
    {
      Available only when: \codeinline{T != float}.
      \newline
      Atomically subtracts the value \codeinline{operand} to the value at the
      address of the \codeinline{multi_ptr} associated with this SYCL
      \codeinline{atomic} and assigns the result to the value at the address of
      the \codeinline{multi_ptr} associated with this SYCL \codeinline{atomic}.
      Returns the value at the address of the \codeinline{multi_ptr} associated
      with this SYCL \codeinline{atomic} before the call. The memory order of
      this atomic operation must be \codeinline{memory_order::relaxed}.
    }
  \addRowTwoL
    { T fetch_and(T operand, memory_order memoryOrder = }
    { memory_order::relaxed) }
    {
      Available only when: \codeinline{T != float}.
      \newline
      Atomically performs a bitwise AND between the value \codeinline{operand}
      and the value at the address of the \codeinline{multi_ptr} associated with
      this SYCL \codeinline{atomic} and assigns the result to the value at the
      address of the \codeinline{multi_ptr} associated with this SYCL
      \codeinline{atomic}. Returns the value at the address of the \codeinline{
      multi_ptr} associated with this SYCL \codeinline{atomic} before the call.
      The memory order of this atomic operation must be \codeinline{
      memory_order::relaxed}.
    }
  \addRowTwoL
    { T fetch_or(T operand, memory_order memoryOrder = }
    { memory_order::relaxed) }
    {
      Available only when: \codeinline{T != float}.
      \newline
      Atomically performs a bitwise OR between the value \codeinline{operand}
      and the value at the address of the \codeinline{multi_ptr} associated with
      this SYCL \codeinline{atomic} and assigns the result to the value at the
      address of the \codeinline{multi_ptr} associated with this SYCL
      \codeinline{atomic}. Returns the value at the address of the \codeinline{
      multi_ptr} associated with this SYCL \codeinline{atomic} before the call.
      The memory order of this atomic operation must be \codeinline{
      memory_order::relaxed}.
    }
  \addRowTwoL
    { T fetch_xor(T operand, memory_order memoryOrder = }
    { memory_order::relaxed) }
    {
      Available only when: \codeinline{T != float}.
      \newline
      Atomically performs a bitwise XOR between the value \codeinline{operand}
      and the value at the address of the \codeinline{multi_ptr} associated with
      this SYCL \codeinline{atomic} and assigns the result to the value at the
      address of the \codeinline{multi_ptr} associated with this SYCL
      \codeinline{atomic}. Returns the value at the address of the \codeinline{
      multi_ptr} associated with this SYCL \codeinline{atomic} before the call.
      The memory order of this atomic operation must be \codeinline{
      memory_order::relaxed}.
    }
  \addRowTwoL
    { T fetch_min(T operand, memory_order memoryOrder = }
    { memory_order::relaxed) }
    {
      Atomically computes the minimum of the value \codeinline{operand} and the
      value at the address of the \codeinline{multi_ptr} associated with this
      SYCL \codeinline{atomic} and assigns the result to the value at the address
      of the \codeinline{multi_ptr} associated with this SYCL \codeinline{
      atomic}. Returns the value at the address of the \codeinline{multi_ptr}
      associated with this SYCL \codeinline{atomic} before the call. The memory
      order of this atomic operation must be \codeinline{memory_order::relaxed}.
    }
  \addRowTwoL
    { T fetch_max(T operand, memory_order memoryOrder = }
    { memory_order::relaxed) }
    {
      Available only when: \codeinline{T != float}.
      \newline
      Atomically computes the maximum of the value \codeinline{operand} and the
      value at the address of the \codeinline{multi_ptr} associated with this
      SYCL \codeinline{atomic} and assigns the result to the value at the address
      of the \codeinline{multi_ptr} associated with this SYCL \codeinline{
      atomic}. Returns the value at the address of the \codeinline{multi_ptr}
      associated with this SYCL \codeinline{atomic} before the call. The memory
      order of this atomic operation must be \codeinline{memory_order::relaxed}.
    }
\completeInfoTable
%-------------------------------------------------------------------------------

%-------------------------------------------------------------------------------
\startTable{Functions}
\addFootNotes{Global functions available on atomic types}
{table.atomics.functions}
  \addRowThreeSL
    { template <typename T, access::address_space addressSpace> }
    { T atomic_load(atomic<T, addressSpace> object, }
    { memory_order memoryOrder = memory_order::relaxed) }
    {
      Equivalent to calling \codeinline{object.load(memoryOrder)}.
    }
  \addRowThreeSL
    { template <typename T, access::address_space addressSpace> }
    { void atomic_store(atomic<T, addressSpace> object, T operand, }
    { memory_order memoryOrder = memory_order::relaxed) }
    {
      Equivalent to calling \codeinline{object.store(operand, memoryOrder)}.
    }
  \addRowThreeSL
    { template <typename T, access::address_space addressSpace> }
    { T atomic_exchange(atomic<T, addressSpace> object, T operand, }
    { memory_order memoryOrder = memory_order::relaxed) }
    {
      Equivalent to calling \codeinline{object.exchange(operand, memoryOrder)}.
    }
  \addRowSevenSL
    { template <typename T, access::address_space addressSpace> }
    { bool atomic_compare_exchange_strong( }
    { atomic<T, addressSpace> object, T \&expected, T desired,}
    { memory_order successMemoryOrder = }
    { memory_order::relaxed }
    { memory_order failMemoryOrder = }
    { memory_order::relaxed) }
    {
      Equivalent to calling \codeinline{
      object.compare_exchange_strong(expected, desired, successMemoryOrder, failMemoryOrders)}.
    }
  \addRowThreeSL
    { template <typename T, access::address_space addressSpace> }
    { T atomic_fetch_add(atomic<T, addressSpace> object, T operand, }
    { memory_order memoryOrder = memory_order::relaxed) }
    {
      Equivalent to calling \codeinline{object.fetch_add(operand, memoryOrder)}.
    }
  \addRowThreeSL
    { template <typename T, access::address_space addressSpace> }
    { T atomic_fetch_sub(atomic<T, addressSpace> object, T operand, }
    { memory_order memoryOrder = memory_order::relaxed) }
    {
      Equivalent to calling \codeinline{object.fetch_sub(operand, memoryOrder)}.
    }
  \addRowThreeSL
    { template <typename T, access::address_space addressSpace> }
    { T atomic_fetch_and(atomic<T> operand, T object, }
    { memory_order memoryOrder = memory_order::relaxed) }
    {
      Equivalent to calling \codeinline{object.fetch_add(operand, memoryOrder)}.
    }
  \addRowThreeSL
    { template <typename T, access::address_space addressSpace> }
    { T atomic_fetch_or(atomic<T, addressSpace> object, T operand, }
    { memory_order memoryOrder = memory_order::relaxed) }
    {
      Equivalent to calling \codeinline{object.fetch_or(operand, memoryOrder)}.
    }
  \addRowThreeSL
    { template <typename T, access::address_space addressSpace> }
    { T atomic_fetch_xor(atomic<T, addressSpace> object, T operand, }
    { memory_order memoryOrder = memory_order::relaxed) }
    {
      Equivalent to calling \codeinline{object.fetch_xor(operand, memoryOrder)}.
    }
  \addRowThreeSL
    { template <typename T, access::address_space addressSpace> }
    { T atomic_fetch_min(atomic<T, addressSpace> object, T operand, }
    { memory_order memoryOrder = memory_order::relaxed) }
    {
      Equivalent to calling \codeinline{object.fetch_min(operand, memoryOrder)}.
    }
  \addRowThreeSL
    { template <typename T, access::address_space addressSpace> }
    { T atomic_fetch_max(atomic<T, addressSpace> object, T operand, }
    { memory_order memoryOrder = memory_order::relaxed) }
    {
      Equivalent to calling \codeinline{object.fetch_max(operand, memoryOrder)}.
    }
\completeTable
%-----------------------------------------------------------------------------------

%***********************************************************************************
% Stream class
%***********************************************************************************
\section{Stream class}
\label{subsection:stream}

The SYCL \codeinline{stream} class is a buffered output stream that allows outputting the values of built-in, vector and SYCL types to the console. The implementation of how values are streamed is left as an implementation detail.

The way in which values are output by an instance of the SYCL \codeinline{stream} class can also be altered using a range of manipulators.

An instance of the SYCL \codeinline{stream} class has a maximum buffer size that specifies maximum size of the character stream that can be output in bytes and a maximum statement size that specifies the maximum size of the character stream that can be output in a single statement in bytes.

All member functions of the \codeinline{stream} class are synchronous and errors are handled by throwing synchronous SYCL exceptions.

The SYCL \codeinline{stream} class provides the common reference semantics
(see Section~\ref{sec:reference-semantics}).

%***********************************************************************************
% Stream interface
%***********************************************************************************
\subsection{Stream class interface}

The constructors and member functions of the SYCL \codeinline{stream} class are listed in Tables~\ref{table.constructors.stream}, \ref{table.members.stream}, and \ref{table.globals.stream} respectively. The additional common special member functions and common member functions are listed in Tables~\ref{table.specialmembers.common.reference} and \ref{table.members.common.reference}, respectively.

The operand types that are supported by the SYCL \codeinline{stream} class \codeinline{operator<<()} operator are listed in Table~\ref{table.operands.stream}.

The manipulators that are supported by the SYCL \codeinline{stream} class \codeinline{operator<<()} operator are listed in Table~\ref{table.manipulators.stream}.

% Interface of the device class
\lstinputlisting{headers/stream.h}

%-------------------------------------------------------------------------------------------
\startTable{Stream operand type}
\addFootNotes{Operand types supported by the \codeinline{stream} class}{table.operands.stream}
\addRow
{
  char, signed char, unsigned char, int, unsigned int, short, unsigned short, long int, unsigned long int, long long int, unsigned long long int, cl_char, cl_uchar, cl_int, cl_uint, cl_short, cl_ushort, cl_long, cl_ulong, byte
}
{
  Outputs the value as a stream of characters.
}
\addRow
{
  float, double, half, cl_float, cl_double, cl_half
}
{
  Outputs the value according to the precision of the current statement as a stream of characters.
}
\addRow
{
  char *, const char *
}
{
  Outputs the string.
}
\addRow
{
  T *, const T *, multi_ptr
}
{
  Outputs the address of the pointer as a stream of characters.
}
\addRow
{
  vec
}
{
  Outputs the value of each component of the vector as a stream of characters.
}
\addRow
{
  id, range, item, nd_item, group, nd_range, h_item
}
{
  Outputs the value of each component of each id or range as a stream of characters.
}
\completeTable
%-------------------------------------------------------------------------------------------

%-------------------------------------------------------------------------------------------
\startTable{Stream manipulator}
\addFootNotes{Manipulators supported by the \codeinline{stream} class}{table.manipulators.stream}
\addRow
{
  endl
}
{
  Outputs a new-line character.
}
\addRow
{
  dec
}
{
  Outputs any subsequent values in the current statement in decimal base.
}
\addRow
{
  hex
}
{
  Outputs any subsequent values in the current statement in hexadecimal base.
}
\addRow
{
  oct
}
{
  Outputs any subsequent values in the current statement in octal base.
}
\addRow
{
  noshowbase
}
{
  Outputs any subsequent values without the base prefix.
}
\addRow
{
  showbase
}
{
  Outputs any subsequent values with the base prefix.
}
\addRow
{
  noshowpos
}
{
  Outputs any subsequent values without a plus sign if the value is positive.
}
\addRow
{
  showpos
}
{
  Outputs any subsequent values with a plus sign if the value is positive.
}
\addRow
{
  setw(int)
}
{
  Sets the field width of any subsequent values in the current statement.
}
\addRow
{
  setprecision(int)
}
{
  Sets the precision of any subsequent values in the current statement.
}
\addRow
{
  fixed
}
{
  Outputs any subsequent floating-point values in the current statement in fixed notation.
}
\addRow
{
  scientific
}
{
  Outputs any subsequent floating-point values in the current statement in scientific notation.
}
\addRow
{
  hexfloat
}
{
  Outputs any subsequent floating-point values in the current statement in hexadecimal notation.
}
\addRow
{
  defaultfloat
}
{
  Outputs any subsequent floating-point values in the current statement in the default notation.
}
\completeTable
%-------------------------------------------------------------------------------------------

%-------------------------------------------------------------------------------
\startTable{Constructor}
    \addFootNotes{Constructors of the \codeinline{stream} class}{table.constructors.stream}
  \addRow
    {stream(size_t bufferSize, size_t maxStatementSize, handler\& cgh)}
    {
      Constructs a SYCL \codeinline{stream} instance associated with the command group specified by \codeinline{cgh}, with a maximum buffer size specified by the parameter \codeinline{bufferSize} and a maximum statement size specified by the parameter \codeinline{maxStatementSize}.
    }
\completeTable
%-------------------------------------------------------------------------------

%-------------------------------------------------------------------------------
\startTable{Member function}
  \addFootNotes{Member functions of the \codeinline{stream} class}{table.members.stream}
  \addRow
    {size_t get_size() const}
    {
      Returns the maximum buffer size.
    }
  \addRow
    {size_t get_max_statement_size() const}
    {
      Returns the maximum statement size.
    }
\completeTable
%-------------------------------------------------------------------------------------------

%-------------------------------------------------------------------------------
\startTable{Global function}
\addFootNotes{Global functions of the \codeinline{stream} class}
{table.globals.stream}
  \addRow
    {template <typename T> const stream\& operator<<(const stream\& os, const T \&rhs)}
    {
      Outputs any valid values (see~\ref{table.operands.stream}) as a stream of characters and applies any valid manipulator (see~\ref{table.manipulators.stream}) to the current statements.
    }
\completeTable
%-------------------------------------------------------------------------------------------

\subsection{Synchronization}

An instance of the SYCL \codeinline{stream} class is required to synchronize with the host, and must output everything that is streamed to it via the \codeinline{operator<<()} operator within a SYCL kernel function by the time that the kernel function finishes execution. The point at which this synchronization occurs and the method by which this synchronization is performed are implementation defined. For example it is valid for an implementation to use \codeinline{printf()}.

In the case that an instance of the SYCL \codeinline{stream} class is used across multiple work items concurrently, there is no guarantee of ordering of outputs.

If an instance of the SYCL \codeinline{stream} class is used on a SYCL kernel function executed on a OpenCL context, there is no guarantee that statements are output in their entirety. If an instance of the SYCL \codeinline{stream} class is used on a SYCL kernel function executed on a host context, then the SYCL \codeinline{stream} class is required to output each statement in full without mixing with statements of other work items.

\subsection{Performance note}

The usage of the \codeinline{stream} class is designed for debugging purposes and is therefore not recommended for performance critical applications.

%***********************************************************************************
% SYCL built-in functions for SYCL host and device
%***********************************************************************************
\section{SYCL built-in functions for SYCL host and device}
\label{sycl:builtins}
% !TEX root = sycl-1.2.1.tex

%***********************************************************************************
% SYCL built-in functions for SYCL host and device
%***********************************************************************************
%\section{SYCL built-in functions for SYCL host and device}
%\label{sycl:builtins}

SYCL kernels may execute on any SYCL device, specifically an OpenCL device or SYCL host, which requires the functions used in the kernels to be compiled and linked for both device and host. In the SYCL system the OpenCL built-ins are available for the SYCL host and device within the \codeinline{cl::sycl} namespace, although, their semantics may be different. This section follows the OpenCL 1.2 specification document \cite[ch. 6.12]{opencl12} and describes the behavior of these functions for SYCL host and device.

The SYCL built-in functions are available throughout the SYCL application, and depending on where they execute, they are either implemented using their host implementation or the device implementation. The SYCL system guarantees that all of the built-in functions fulfill the same requirements for both host and device.

%***********************************************************************************
% Description of the built-in types available for SYCL host and device
%***********************************************************************************
\subsection{Description of the built-in types available for SYCL host and device  }
All of the OpenCL built-in types are available in the namespace \keyword{cl::sycl}. For the
purposes of this document we use generic type names for describing sets of valid SYCL types. The generic type names themselves are not valid SYCL types, but they represent a set of valid types, as defined in
Tables~\ref{table.gentypes}. Each generic type within a section is comprised of
a combination of scalar and/or SYCL \codeinline{vec} class specializations. Note
that any reference to the base type refers to the type of a scalar or the element
type of a SYCL \codeinline{vec} specialization.

In the OpenCL 1.2 specification document\cite[ch. 6.12.1]{opencl12} in Table 6.7 the work-item
functions are defined where they provide the size of the enqueued kernel NDRange. These
functions are
available in SYCL through the item and group classes see
sections~\ref{subsec:item.class}, \ref{nditem-class} and
\ref{group-class}.

---------------------------------------------------------------------------------

%------------------------------------------------------------------------------------------------------
\startTable{Generic type name } \addFootNotes{Generic type name
  description, which serves as a description for all valid types of
  parameters to kernel functions \cite{opencl12}}
{table.gentypes}

\addRow{floatn}
{\codeinline{float2, float3, float4, float8, float16}
}

\addRow{genfloatf}
{\codeinline{float, floatn}
}

\addRow{doublen}
{\codeinline{double2, double3, double4, double8, double16}
}

\addRow{genfloatd}
{\codeinline{double, doublen}
}

\addRow{halfn}
{\codeinline{half2, half3, half4, half8, half16}
}

\addRow{genfloath}
{\codeinline{half, halfn}
}

\addRow{genfloat}
{\codeinline{genfloatf, genfloatd, genfloath}
}

\addRow{sgenfloat}
{\codeinline{float, double, half}
}

\addRow{gengeofloat}
{\codeinline{float, float2, float3, float4}
}

\addRow{gengeodouble}
{\codeinline{double, double2, double3, double4}
}

\addRow {charn}
{\codeinline{char2, char3, char4, char8, char16}
}

\addRow {scharn}
{\codeinline{schar2, schar3, schar4, schar8, schar16}
}

\addRow{ucharn}
{\codeinline{uchar2, uchar3, uchar4, uchar8, uchar16}
}

\addRow{igenchar}
{\codeinline{signed char, scharn}
}

\addRow{ugenchar}
{\codeinline{unsigned char, ucharn}
}

\addRow{genchar}
{\codeinline{char, charn, igenchar, ugenchar}
}

\addRow{shortn}
{\codeinline{short2, short3, short4, short8, short16}
}

\addRow{genshort}
{\codeinline{short, shortn}
}

\addRow{ushortn}
{\codeinline{ushort2, ushort3, ushort4, ushort8, ushort16}
}

\addRow{ugenshort}
{\codeinline{unsigned short, ushortn}
}


\addRow{uintn}
{\codeinline{uint2, uint3, uint4, uint8, uint16}
}

\addRow{ugenint}
{\codeinline{unsigned int, uintn}
}

\addRow{intn}
{\codeinline{int2, int3, int4, int8, int16}
}

\addRow{genint}
{\codeinline{int, intn}
}

\addRow{ulongn}
{\codeinline{ulong2, ulong3, ulong4, ulong8, ulong16}
}

\addRow{ugenlong}
{\codeinline{unsigned long int, ulongn}
}

\addRow{longn}
{\codeinline{long2, long3, long4, long8, long16}
}

\addRow{genlong}
{\codeinline{long int, longn}
}

\addRow{ulonglongn}
{\codeinline{ulonglong2, ulonglong3, ulonglong4, ulonglong8, ulonglong16}
}

\addRow{ugenlonglong}
{\codeinline{unsigned long long int, ulonglongn}
}

\addRow{longlongn}
{\codeinline{longlong2, longlong3, longlong4, longlong8, longlong16}
}
\addRow{genlonglong}
{\codeinline{long long int, longlongn}
}

\addRow{igenlonginteger}
{\codeinline{genlong, genlonglong}
}

\addRow{ugenlonginteger}
{\codeinline{ugenlong, ugenlonglong}
}

\addRow{geninteger}
{\codeinline{genchar, genshort, ugenshort, genint, ugenint, igenlonginteger, ugenlonginteger}
}

\addRow{genintegerNbit}
{\codeinline{All types within geninteger whose base type are N bits in size, where N = 8, 16, 32, 64.}
}

\addRow{igeninteger}
{\codeinline{igenchar, genshort, genint, igenlonginteger}
}

\addRow{igenintegerNbit}
{\codeinline{All types within igeninteger whose base type are N bits in size, where N = 8, 16, 32, 64.}
}

\addRow{ugeninteger}
{\codeinline{ugenchar, ugenshort, ugenint, ugenlonginteger}
}

\addRow{ugenintegerNbit}
{\codeinline{All types within ugeninteger whose base type are N bits in size, where N = 8, 16, 32, 64.}
}

\addRow{sgeninteger}
{\codeinline{char, signed char, unsigned char, short, unsigned short, int, unsigned int, long int, unsigned long int, long long int, unsigned long long int}
}

\addRow{gentype}
{\codeinline{genfloat, geninteger}
}
\addRow{genfloatptr}
{
  All permutations of \codeinline{multi_ptr<dataT, addressSpace>} where \codeinline{dataT} is all types within \codeinline{genfloat} and \codeinline{addressSpace} is \codeinline{access::address_space::global_space}, \codeinline{access::address_space::local_space} and \codeinline{access::address_space::private_space}.
}
\addRow{genintptr}
{
    All permutations of \codeinline{multi_ptr<dataT, addressSpace>} where \codeinline{dataT} is all types within \codeinline{genint} and \codeinline{addressSpace} is \codeinline{access::address_space::global_space}, \codeinline{access::address_space::local_space} and \codeinline{access::address_space::private_space}.
}
\completeTable
%-------------------------------------------------------------------------------

%*******************************************************************************
% Work-item functions
%*******************************************************************************
\subsection{Work-item functions}

In the OpenCL 1.2 specification document\cite[ch. 6.12.1]{opencl12} in Table 6.7
the work-item functions are defined where they provide the size of the enqueued
kernel NDRange. These functions are available in SYCL through the
\codeinline{nd_item} and \codeinline{group} classes see
section~\ref{nditem-class} and \ref{group-class}.

%***********************************************************************************
% Math functions
%***********************************************************************************
\subsection{Math functions}

In SYCL the OpenCL math functions are available in the namespace
\keyword{cl::sycl} on host and device with the same precision guarantees as
defined in the OpenCL 1.2 specification document
\cite[ch. 7]{opencl12} for host and device. For a SYCL platform the numerical
requirements for host need to match the numerical requirements of the OpenCL
math built-in functions. The built-in functions can take as input float or
optionally double and their \keyword{vec} counterparts, for dimensions 1, 2, 3, 4,
8 and 16. On the host the vector types use the \keyword{vec}
class and on an OpenCL device use the corresponding OpenCL
vector types.

The built-in functions available for SYCL host and device, with the same
precision requirements for both host and device, are described in
Table~\ref{table.math.functions}.

%-------------------------------------------------------------------------------
\startTable{Math Function}
\addFootNotes{Math functions which work on SYCL Host and device. They correspond
to Table 6.7 of the OpenCL 1.2 specification\cite{opencl12}}
{table.math.functions}
\addRow{ genfloat acos (genfloat x)} {Inverse cosine function.}
\addRow{ genfloat acosh (genfloat x)} {Inverse hyperbolic cosine.}
\addRow{ genfloat acospi (genfloat x)} {Compute $acos{x}/\pi$}
\addRow{ genfloat asin (genfloat x)} {Inverse sine function.}
\addRow{ genfloat asinh (genfloat x)} {Inverse hyperbolic sine.}
\addRow{ genfloat asinpi (genfloat x)} {Compute $asin{x}/\pi$}
\addRow{ genfloat atan (genfloat y_over_x)} {Inverse tangent function.}
\addRow{ genfloat atan2 (genfloat y, genfloat x)} {Compute atan( y / x).}
\addRow{ genfloat atanh (genfloat x)} { Hyperbolic inverse tangent.}
\addRow{ genfloat atanpi (genfloat x)} {Compute atan (x) / $\pi$.}
\addRow{ genfloat atan2pi (genfloat y, genfloat x)} {Compute atan2 (y, x) / $\pi$.}
\addRow{ genfloat cbrt (genfloat x)} { Compute cube-root.}
\addRow{ genfloat ceil (genfloat x)} {Round to integral value using the round to positive infinity
rounding mode.}
\addRow{ genfloat copysign (genfloat x, genfloat y)} {Returns x with its sign changed to match
the sign of y.}

\addRow{ genfloat cos (genfloat x)} {Compute cosine.}
\addRow{ genfloat cosh (genfloat x)} {Compute hyperbolic cosine.}
\addRow{ genfloat cospi (genfloat x)} { Compute cos ($\pi  x$).}
\addRow{ genfloat erfc (genfloat x)} {Complementary error function.}
\addRow{ genfloat erf (genfloat x)} {Error function encountered in integrating the normal
distribution.}
\addRow{ genfloat exp (genfloat x )} { Compute the base-e exponential of x.}
\addRow{ genfloat exp2 (genfloat x)} {Exponential base 2 function.}
\addRow{ genfloat exp10 (genfloat x)} {Exponential base 10 function.}
\addRow{ genfloat expm1 (genfloat x)} { Compute $\exp{(x)}-1.0$.}
\addRow{ genfloat fabs (genfloat x)} {Compute absolute value of a floating-point number.}
\addRow{ genfloat fdim (genfloat x, genfloat y)} {$ x - y$  if $ x > y$,$ +0$ if x is less than or
equal to y.}
\addRow{ genfloat floor (genfloat x)} {Round to integral value using the round to negative infinity
rounding mode.}
\addRow{ genfloat fma (genfloat a, genfloat b, genfloat c)}
{
Returns the correctly rounded floating-point
representation of the sum of c with the infinitely
precise product of a and b. Rounding of
intermediate products shall not occur. Edge case
behavior is per the IEEE 754-2008 standard.
}
\addRowTwoSL
{  genfloat fmax (genfloat x, genfloat y)}
{  genfloat fmax (genfloat x, sgenfloat y)}
{
Returns y if $x < y$, otherwise it returns x. If one
argument is a NaN, fmax() returns the other
argument. If both arguments are NaNs, fmax()
returns a NaN.
}
\addRowTwoSL
{ genfloat fmin (genfloat x, genfloat y)}
{ genfloat fmin (genfloat x, sgenfloat y)}
{
Returns y if $y < x$, otherwise it returns x. If one
argument is a NaN, fmin() returns the other
argument. If both arguments are NaNs, fmin()
returns a NaN.
}
\addRow{
   genfloat fmod (genfloat x, genfloat y)
 }
 {
  Modulus. Returns $x – y * trunc (x/y)$.
  }
\addRow
{ genfloat fract (genfloat x, genfloatptr iptr)}
{
  Returns fmin( x $-$ floor (x), nextafter(genfloat(1.0), genfloat(0.0)) ). floor(x) is returned in iptr.
}
\addRow
{ genfloat frexp (genfloat x,  genintptr exp)}
{
Extract mantissa and exponent from x. For each
component the mantissa returned is a float with
magnitude in the interval [1/2, 1) or 0. Each
component of x equals mantissa returned * 2exp.
}
\addRow{
  genfloat hypot (genfloat x, genfloat y)
}
{
  Compute the value of the square root of x2+ y2 without undue overflow or underflow.
}
\addRow
{genint ilogb (genfloat x)}
{
  Return the exponent as an integer value.
}
\addRowTwoSL
{  genfloat ldexp (genfloat x, genint k) }
{  genfloat ldexp (genfloat x, int k) }
{
  Multiply x by 2 to the power k.
}

\addRow
{
  genfloat lgamma (genfloat x)
}
{
  Log gamma function. Returns the natural
logarithm of the absolute value of the gamma
function. The sign of the gamma function is
returned in the signp argument of \codeinline{lgamma_r}.
}
\addRow
{
  genfloat lgamma_r (genfloat x, genintptr signp)
}
{
  Log gamma function. Returns the natural
logarithm of the absolute value of the gamma
function. The sign of the gamma function is
returned in the signp argument of \codeinline{lgamma_r}.
}


\addRow{
genfloat log (genfloat x)
}
{
 Compute natural logarithm.
 }
\addRow{
genfloat log2 (genfloat x)
}
{
 Compute a base 2 logarithm.
 }
\addRow{
genfloat log10 (genfloat x)
}
{
Compute a base 10 logarithm.
}
\addRow{
genfloat log1p (genfloat x)
 }
 {
 Compute $loge(1.0 + x)$.
 }
\addRow{
genfloat logb (genfloat x)
}
{
Compute the exponent of x, which is the integral
part of logr ($| x |$).
}
\addRow{
genfloat mad (genfloat a,genfloat b, genfloat c)
}
{
mad approximates a * b + c. Whether or how the
product of a * b is rounded and how supernormal or
subnormal intermediate products are handled is not
defined. mad is intended to be used where speed is
preferred over accuracy.
}
\addRow{
 genfloat maxmag (genfloat x, genfloat y)
 }
{
Returns x if $| x |  >  | y |$, y if $| y | > | x |$, otherwise
fmax(x, y).
}
\addRow{genfloat minmag (genfloat x, genfloat y)}
{
Returns x if $| x | < | y |$, y if $| y | < | x |$, otherwise
fmin(x, y).
}
\addRow{
genfloat modf (genfloat x, genfloatptr iptr)
}
{
Decompose a floating-point number. The modf
function breaks the argument x into integral and
fractional parts, each of which has the same sign as
the argument. It stores the integral part in the object
pointed to by iptr.
}
\addRowTwoSL
{ genfloatf nan (ugenint nancode)}
{ genfloatd nan (ugenlonginteger nancode)}
{
Returns a quiet NaN. The nancode may be placed
in the significand of the resulting NaN.
}
\addRow{genfloat nextafter (genfloat x,
genfloat y)
}
{
Computes the next representable single-precision
floating-point value following x in the direction of
y. Thus, if y is less than x, nextafter() returns the
largest representable floating-point number less
than x.}

\addRow{
genfloat pow (genfloat x, genfloat y)
}
{
 Compute x to the power y.
}
\addRow
{
genfloat pown (genfloat x, genint y)
}
{
Compute x to the power y, where y is an integer.
}
\addRow
{
genfloat powr (genfloat x, genfloat y)
}
{
Compute x to the power y, where $ x > = 0 $.
}
\addRow{
genfloat remainder (genfloat x, genfloat y)
}
{
Compute the value r such that r = x - n*y, where n
is the integer nearest the exact value of x/y. If there
are two integers closest to x/y, n shall be the even
one. If r is zero, it is given the same sign as x.
}
\addRow{
genfloat remquo (genfloat x, genfloat y, genintptr quo)
}
{
The remquo function computes the value r such
that r = x - k*y, where k is the integer nearest the
exact value of x/y. If there are two integers closest
to x/y, k shall be the even one. If r is zero, it is
given the same sign as x. This is the same value
that is returned by the remainder function.
remquo also calculates the lower seven bits of the
integral quotient x/y, and gives that value the same
sign as x/y. It stores this signed value in the object
pointed to by quo.
}
\addRow{
genfloat rint (genfloat x)
}
{
Round to integral value (using round to nearest
even rounding mode) in floating-point format.
Refer to section 7.1 of the OpenCL 1.2 specification
document\cite{opencl12} for description of rounding
modes.
}
\addRow{
genfloat rootn (genfloat x, genint y)
}
{
Compute x to the power 1/y.
}
\addRow{
genfloat round (genfloat x)
}
{
 Return the integral value nearest to x rounding
halfway cases away from zero, regardless of the
current rounding direction.
}
\addRow{
genfloat rsqrt (genfloat x)
}
{
Compute inverse square root.
}
\addRow{
genfloat sin (genfloat x)
}
{
 Compute sine.
}
\addRow{
genfloat sincos (genfloat x, genfloatptr cosval)
}
{
Compute sine and cosine of x. The computed sine
is the return value and computed cosine is returned
in cosval.
}
\addRow{genfloat sinh (genfloat x)}{ Compute hyperbolic sine.}
\addRow{genfloat sinpi (genfloat x)}{ Compute sin ($\pi$ x).}
\addRow{genfloat sqrt (genfloat x)}{ Compute square root.}
\addRow{genfloat tan (genfloat x)}{ Compute tangent.}
\addRow{genfloat tanh (genfloat x)}{Compute hyperbolic tangent.}
\addRow{genfloat tanpi (genfloat x)}{ Compute tan ($\pi$ x).}
\addRow{genfloat tgamma (genfloat x)}{Compute the gamma function.}
\addRow{genfloat trunc (genfloat x)}{Round to integral value using the round to zero
rounding mode.}
\completeTable
%-------------------------------------------------------------------------------

In SYCL the implementation defined precision math functions  are defined in the
namespace \keyword{cl::sycl::native}. The functions that are available within
this namespace are specified in Tables~\ref{table.native.math.functions}.

%-------------------------------------------------------------------------------

\startTable{Native Math Function}
\addFootNotes{Native math functions}{table.native.math.functions}

\addRow
{
genfloatf cos (genfloatf x)
}
{
Compute cosine over an implementation-defined range.
The maximum error is implementation-defined.
}
\addRow
{
genfloatf divide (genfloatf x, genfloatf y)
}
{
Compute x / y over an implementation-defined range.
The maximum error is implementation-defined.
}
\addRow
{
genfloatf exp (genfloatf x)
}
{
 Compute the base- e exponential of x over an
implementation-defined range. The maximum error is
implementation-defined.
}
\addRow
{
genfloatf exp2 (genfloatf x)
}
{
Compute the base- 2 exponential of x over an
implementation-defined range. The maximum error is
implementation-defined.
}
\addRow
{
genfloatf exp10 (genfloatf x)
}
{
 Compute the base- 10 exponential of x over an
implementation-defined range. The maximum error is
implementation-defined.
}
\addRow
{
genfloatf log (genfloatf x)
}
{
 Compute natural logarithm over an implementation defined range.
 The maximum error is implementation-defined.
 }
\addRow
{
 genfloatf log2 (genfloatf x)
 }
 {
 Compute a base 2 logarithm over an implementation-defined
range. The maximum error is implementation-defined.
}
\addRow
{
genfloatf log10 (genfloatf x)
}
{
Compute a base 10 logarithm over an implementation-defined
range. The maximum error is implementation-defined.
}
\addRow
{
genfloatf powr (genfloatf x, genfloatf y)
}
{
Compute x to the power y, where $ x > = 0$. The range of
x and y are implementation-defined. The maximum error
is implementation-defined.
}
\addRow
{
genfloatf recip (genfloatf x)
}
{
Compute reciprocal over an implementation-defined
range. The maximum error is implementation-defined.
}

\addRow
{
genfloatf rsqrt (genfloatf x)
}
{
Compute inverse square root over an implementation-defined
range. The maximum error is implementation-defined.
}
\addRow
{
genfloatf sin (genfloatf x)
}
{
 Compute sine over an implementation-defined range.
The maximum error is implementation-defined.
}
\addRow
{
genfloatf sqrt (genfloatf x)
}
{
Compute square root over an implementation-defined
range. The maximum error is implementation-defined.
}
\addRow
{
genfloatf tan (genfloatf x)
}
{
Compute tangent over an implementation-defined range.
The maximum error is implementation-defined.
}
\completeTable
%-------------------------------------------------------------------------------

In SYCL the half precision math functions are defined in \keyword{cl::sycl::half_precision}. The functions that are available within this namespace are specified in Tables~\ref{table.half.math.functions}. These
functions are implemented with a minimum of 10-bits of accuracy i.e. an ULP value is less than or equal to 8192 ulp.

%-------------------------------------------------------------------------------
\startTable{Half Math function}
\addFootNotes{Half precision math functions}
{table.half.math.functions}

\addRow{
genfloatf cos (genfloatf x)
}
{
Compute cosine. x must be in the range -216 to +216.
}
\addRow{
genfloatf divide (genfloatf x, genfloatf y)
}
{
Compute x / y.
}
\addRow{genfloatf exp (genfloatf x) }{Compute the base- e exponential of x.}
\addRow{genfloatf exp2 (genfloatf x) }{Compute the base- 2 exponential of x.}
\addRow{genfloatf exp10 (genfloatf x) }{Compute the base- 10 exponential of x.}
\addRow{genfloatf log (genfloatf x) }{Compute natural logarithm.}
\addRow{genfloatf log2 (genfloatf x) }{Compute a base 2 logarithm.}
\addRow{genfloatf log10 (genfloatf x) }{Compute a base 10 logarithm.}
\addRow{genfloatf powr (genfloatf x, genfloatf y)}
{
Compute x to the power y, where $x >= 0$.
}
\addRow{genfloatf recip (genfloatf x)}{Compute reciprocal.}
\addRow{genfloatf rsqrt (genfloatf x)}{ Compute inverse square root.}
\addRow{genfloatf sin (genfloatf x) }{Compute sine. x must be in the range -216
to +216.}
\addRow{genfloatf sqrt (genfloatf x)}{ Compute square root.}
\addRow{genfloatf tan (genfloatf x) }{Compute tangent. x must be in the range
-216 to +216.}
\completeTable
%-------------------------------------------------------------------------------

%*******************************************************************************
% Integer functions
%*******************************************************************************
\subsection{Integer functions}
In SYCL the OpenCL integer math functions are available in the namespace
\keyword{cl::sycl} on host and device as defined in the OpenCL 1.2 specification
document\cite[par. 6.12.3] {opencl12}. The built-in functions can take as input
char, unsigned char, short, unsigned short, int, unsigned int, long long int,
unsigned long long int and their \keyword{vec} counterparts, for dimensions 2,
3, 4, 8 and 16. On the host the vector types use the
\keyword{vec} class and on an OpenCL device use the
corresponding OpenCL vector types. The supported integer math functions are
described in Table~\ref{table.integer.functions}.

%-------------------------------------------------------------------------------
\startTable{Integer Function}
\addFootNotes{Integer functions which work on SYCL Host and device, are
available in the \codeinline{cl::sycl} namespace. They correspond to Table 6.10 of
the OpenCL 1.2 specification \cite{opencl12}}{table.integer.functions}
\addRow
{
ugeninteger abs (geninteger x)
}
{
Returns $| x |$.
}
\addRow { ugeninteger abs_diff (geninteger x, geninteger y)}
{ Returns $| x - y |$  without modulo overflow.}

\addRow { geninteger add_sat (geninteger x, geninteger y)}
{ Returns $x + y$ and saturates the result.}

\addRow { geninteger hadd (geninteger x, geninteger y)}
{
 Returns $(x + y) >> 1$. The intermediate sum does
not modulo overflow.
}
\addRow { geninteger rhadd (geninteger x, geninteger y)}
 {
 Returns $(x + y + 1) >> 1$. The intermediate sum
does not modulo overflow.
}

\addRowTwoL
{geninteger clamp (geninteger x, geninteger minval, geninteger maxval)}
{geninteger clamp (geninteger x, sgeninteger minval, sgeninteger maxval)}
{
 Returns min(max(x, minval), maxval).
 Results are undefined if minval $>$ maxval.
}

\addRow { geninteger clz (geninteger x) }
{
 Returns the number of leading 0-bits in x, starting
 at the most significant bit position.
}

\addRowTwoL {  geninteger mad_hi (}
{geninteger a, geninteger b, geninteger c)}
 {Returns \codeinline{mul_hi(a, b)+c}. }

\addRowTwoL { geninteger mad_sat (geninteger a,}
{geninteger b, geninteger c)}
{Returns \codeinline{a * b + c} and saturates the result.}

\addRowTwoSL
{geninteger max (geninteger x, geninteger y)}
{geninteger max (geninteger x, sgeninteger y)}
{Returns y if $x < y$, otherwise it returns x.}


\addRowTwoSL
{geninteger min (geninteger x, geninteger y)}
{geninteger min (geninteger x, sgeninteger y)}
{Returns y if $y < x$, otherwise it returns x.}

\addRow {geninteger mul_hi (geninteger x, geninteger y) }
{
Computes \codeinline{x * y} and returns the high half of the
product of x and y.
}

\addRow {geninteger rotate (geninteger v, geninteger i) }
{
For each element in v, the bits are shifted left by
the number of bits given by the corresponding
element in i (subject to usual shift modulo rules
described in section 6.3). Bits shifted off the left
side of the element are shifted back in from the
right.
}

\addRow {geninteger sub_sat (geninteger x, geninteger y)}
{ Returns $x - y$ and saturates the result.}

\addRow{ ugeninteger16bit upsample (ugeninteger8bit hi, ugeninteger8bit lo)}
{\codeinline{result[i] = ((ushort)hi[i] << 8) | lo[i]}}

\addRow {igeninteger16bit upsample (igeninteger8bit hi, ugeninteger8bit lo)}
{\codeinline{result[i] = ((short)hi[i] << 8) | lo[i]}}

\addRow {ugeninteger32bit upsample (ugeninteger16bit hi, ugeninteger16bit lo)}
{\codeinline{result[i] = ((uint)hi[i] << 16) | lo[i]}}

\addRow {igeninteger32bit upsample (igeninteger16bit hi, ugeninteger16bit lo)}
{\codeinline{result[i] = ((int)hi[i] << 16) | lo[i]}}

\addRow {ugeninteger64bit upsample (ugeninteger32bit hi, ugeninteger32bit lo)}
{\codeinline{result[i] = ((ulonglong)hi[i] << 32) | lo[i]}}

\addRow {igeninteger64bit upsample (igeninteger32bit hi, ugeninteger32bit lo)}
{\codeinline{result[i] = ((longlong)hi[i] << 32) | lo[i]}}

\addRow {geninteger popcount (geninteger x)}
{ Returns the number of non-zero bits in x.}

\addRow
{geninteger32bit mad24 (geninteger32bit x, geninteger32bit y, geninteger32bit z)}
{
Multipy two 24-bit integer values x and y and add
the 32-bit integer result to the 32-bit integer z.
Refer to definition of mul24 to see how the 24-bit
integer multiplication is performed.
}
\addRow
{geninteger32bit mul24 (geninteger32bit x, geninteger32bit y)}
{
Multiply two 24-bit integer values x and y. x and y
are 32-bit integers but only the low 24-bits are used
to perform the multiplication. mul24 should only
be used when values in x and y are in the range [-
223, 223-1] if x and y are signed integers and in the
range [0, 224-1] if x and y are unsigned integers. If
x and y are not in this range, the multiplication
result is implementation-defined.
}
\completeTable
%-------------------------------------------------------------------------------

%*******************************************************************************
% Common functions
%*******************************************************************************
\subsection{Common functions}
In SYCL the OpenCL \keyword{common} functions are available in the namespace
\keyword{cl::sycl} on host and device as defined in the OpenCL 1.2 specification
document\cite[par. 6.12.4]{opencl12}. They are described here in
Table~\ref{table.common.functions}. The built-in functions can take as input
float or optionally double and their \keyword{vec} counterparts, for dimensions
2, 3, 4, 8 and 16. On the host the vector types use the
\keyword{vec} class and on an OpenCL device use the
corresponding OpenCL vector types.

%-------------------------------------------------------------------------------
\startTable{Common Function} \addFootNotes{Common functions which work
  on SYCL Host and device, are available in the \codeinline{cl::sycl}
  namespace. They correspond to Table 6.12 of the OpenCL 1.2
  specification \cite{opencl12}}{table.common.functions}
\addRowThreeSL
{genfloat clamp (genfloat x, genfloat minval, genfloat maxval)}
{genfloatf clamp (genfloatf x, float minval, float maxval)}
{genfloatd clamp (genfloatd x, double minval, double maxval)}
{
Returns fmin(fmax(x, minval), maxval).
Results are undefined if $minval > maxval$.
}

\addRow
{genfloat degrees (genfloat radians)}
{
 Converts radians to degrees,
i.e.$ (180 / \pi) *radians$.
}

\addRowThreeSL
 {genfloat max (genfloat x, genfloat y)}
 {genfloatf max (genfloatf x, float y)}
 {genfloatd max (genfloatd x, double y)}
{
Returns y if $x < y$, otherwise it returns x. If x or y
are infinite or NaN, the return values are undefined.
}

\addRowThreeSL
{genfloat min (genfloat x, genfloat y)}
{genfloatf min (genfloatf x, float y)}
{genfloatd min (genfloatd x, double y)}
{
Returns y if $y < x$, otherwise it returns x. If x or y
are infinite or NaN, the return values are undefined.
}

\addRowThreeSL
{genfloat mix (genfloat x, genfloat y, genfloat a)}
{genfloatf mix (genfloatf x, genfloatf y, float a)}
{genfloatd mix (genfloatd x, genfloatd y, double a)}
{
Returns the linear blend of $x \& y$ implemented as:
$x + (y - x) * a$.
\keyword{a} must be a value in the range 0.0 ... 1.0. If a is not
in the range 0.0 ... 1.0, the return values are
undefined.
}
\addRow
{genfloat radians (genfloat degrees) }
{
Converts degrees to radians, i.e. $(\pi / 180) * degrees$.
}
\addRowThreeSL
{genfloat step (genfloat edge, genfloat x)}
{genfloatf step (float edge, genfloatf x)}
{genfloatd step (double edge, genfloatd x)}
{
Returns 0.0 if $x < edge$, otherwise it returns 1.0.
}

\addRowThreeSL
{genfloat smoothstep (genfloat edge0,
genfloat edge1, genfloat x)}
{genfloatf smoothstep (float edge0,
float edge1, genfloatf x)}
{genfloatd smoothstep (double edge0,
double edge1, genfloatd x)}
{
Returns $0.0$ if $x <= edge0$ and $1.0$ if $x >= edge1$ and
performs smooth Hermite interpolation between 0
and 1 when $edge0 < x < edge1$. This is useful in
cases where you would want a threshold function
with a smooth transition.

This is equivalent to:\newline
\codeinline{gentype t;}\newline
\codeinline{t = clamp ((x <= edge0) / (edge1 >= edge0), 0, 1);}\newline
\codeinline{return t * t * (3 - 2 * t);}\newline

Results are undefined if $edge0 >= edge1$ or if x,
edge0 or edge1 is a NaN.
}
\addRow {genfloat sign (genfloat x) }
{
Returns $1.0$ if $x > 0$, $-0.0$ if $x = -0.0$, $+0.0$ if $x =+0.0$, or $-1.0$
if $x < 0$. Returns $0.0$ if x is a NaN.
}
\completeTable
%-------------------------------------------------------------------------------

%*******************************************************************************
% Geometric functions
%*******************************************************************************
\subsection{Geometric Functions}

In SYCL the OpenCL \keyword{geometric} functions are available in the namespace
\keyword{cl::sycl} on host and device as defined in the OpenCL 1.2 specification
document \cite[par. 6.12.5]{opencl12}. The built-in functions can take as input
float or optionally double and their \keyword{vec} counterparts, for dimensions
2, 3 and 4. On the host the vector types use the
\keyword{vec} class and on an OpenCL device use the
corresponding OpenCL vector types. All of the geometric functions use
round-to-nearest-even rounding mode. Table~\ref{table.geometric.functions}
contains the definitions of supported geometric functions.

%-------------------------------------------------------------------------------
\startTable{Geometric Function} \addFootNotes{Geometric functions
  which work on SYCL Host and device, are available in the
  \codeinline{cl::sycl} namespace. They correspond to Table 6.13 of
  the OpenCL 1.2 specification \cite{opencl12}}
{table.geometric.functions}

\addRowFourSL
{float4 cross (float4 p0, float4 p1)}
{float3 cross (float3 p0, float3 p1)}
{double4 cross (double4 p0, double4 p1)}
{double3 cross (double3 p0, double3 p1)}
{
Returns the cross product of p0.xyz and p1.xyz. The
$w$ component of float4 result returned will be 0.0.
}
\addRowTwoSL
{ float dot (gengeofloat p0, gengeofloat p1)}
{ double dot (gengeodouble p0, gengeodouble p1) }
{
Compute dot product.
}
\addRowTwoSL
{float distance (gengeofloat p0, gengeofloat p1)}
{double distance (gengeodouble p0, gengeodouble p1)}
{
Returns the distance between p0 and p1. This is
calculated as \codeinline{length(p0 - p1)}.
}
\addRowTwoSL
{float length (gengeofloat p)}
{double length (gengeodouble p)}
{
Return the length of vector p, i.e.,
$\sqrt{ p.x^2 + p.y^2 + ...}$
}
\addRowTwoSL
{gengeofloat normalize (gengeofloat p)}
{gengeodouble normalize (gengeodouble p)}
{
Returns a vector in the same direction as p but with a
length of 1.
}

\addRow{float fast_distance (gengeofloat p0, gengeofloat p1)}
{
Returns \codeinline{fast_length(p0 - p1)}.
}
\addRow{float fast_length (gengeofloat p) }
{
Returns the length of vector p computed as:
\codeinline{sqrt((half)(pow(p.x,2) + pow(p.y,2) + ...))}
}
\addRow {gengeofloat fast_normalize (gengeofloat p) }
{
Returns a vector in the same direction as p but with a
length of 1. fast_normalize is computed as:
\codeinline{p * rsqrt((half) (pow(p.x,2) + pow(p.y,2) + ... ))}\newline
The result shall be within 8192 ulps error from the
infinitely precise result of
\codeinline{if (all(p == 0.0f))}\newline
\codeinline{result = p;}\newline
\codeinline{else}\newline
\codeinline{result = p / sqrt (pow(p.x,2) + pow(p.y,2) + ... );}\newline
with the following exceptions:
\begin{enumerate}
\item If the sum of squares is greater than \codeinline{FLT_MAX}
then the value of the floating-point values in the
result vector are undefined.

\item If the sum of squares is less than \codeinline{FLT_MIN} then
the implementation may return back p.

\item If the device is in ``denorms are flushed to zero''
mode, individual operand elements with magnitude
less than \codeinline{sqrt(FLT_MIN)} may be flushed to zero
before proceeding with the calculation.
\end{enumerate}
}
\completeTable
%-------------------------------------------------------------------------------

%*******************************************************************************
% Relational functions
%*******************************************************************************
\subsection{Relational functions}

In SYCL the OpenCL \keyword{relational} functions are available in the namespace
\keyword{cl::sycl} on host and device as defined in the OpenCL 1.2 specification
document \cite[par. 6.12.6]{opencl12}. The built-in functions can take as input
char, unsigned char, short, unsigned short, int, unsigned int, long, unsigned
long, float or optionally double and their \keyword{vec} counterparts, for
dimensions 2,3,4,8, and 16. On the host the vector types use
the \keyword{vec} class and on an OpenCL device use the
corresponding OpenCL vector types.
The relational operators are available on both host and device.  The
relational functions are provided in addition to the the operators and will
return 0 if the conditional
is \keyword{false} and 1 otherwise.
The available built-in functions are described in
Tables~\ref{table.relational.functions}

\fixme{fixed typo originating from the OpenCL 1.2 spec: when those functions
take double as a parameter they should return longlong. }


%-------------------------------------------------------------------------------
\startTable{Relational Function}
\addFootNotes{ Relational functions which work on SYCL Host and device, are
available in the \codeinline{cl::sycl} namespace. They correspond to Table 6.14
of the OpenCL 1.2 specification \cite{opencl12}}{table.relational.functions}
\addRowTwoSL
{igeninteger32bit isequal (genfloatf x, genfloatf y)}
{igeninteger64bit isequal (genfloatd x, genfloatd y)}
{
Returns the component-wise compare of $x == y$.
}
\addRowTwoSL
{igeninteger32bit isnotequal (genfloatf x, genfloatf y)}
{igeninteger64bit isnotequal (genfloatd x, genfloatd y)}
{
Returns the component-wise compare of $x != y$.
}
\addRowTwoSL
{igeninteger32bit isgreater (genfloatf x, genfloatf y)}
{igeninteger64bit isgreater (genfloatd x, genfloatd y)}
{
Returns the component-wise compare of $x > y$.
}
\addRowTwoSL
{igeninteger32bit isgreaterequal (genfloatf x, genfloatf y)}
{igeninteger64bit isgreaterequal (genfloatd x, genfloatd y)}
{
Returns the component-wise compare of $x >= y$.
}
\addRowTwoSL
{igeninteger32bit isless (genfloatf x, genfloatf y)}
{igeninteger64bit isless (genfloatd x, genfloatd y)}
{
Returns the component-wise compare of $x < y$.
}

\addRowTwoSL
{igeninteger32bit islessequal (genfloatf x, genfloatf y)}
{igeninteger64bit islessequal (genfloatd x, genfloatd y)}
{
Returns the component-wise compare of $x <= y$.
}
\addRowTwoSL
{igeninteger32bit islessgreater (genfloatf x, genfloatf y)}
{igeninteger64bit islessgreater (genfloatd x, genfloatd y)}
{
Returns the component-wise compare of
$(x < y) || (x > y) $.
}

\addRowTwoSL
{igeninteger32bit isfinite (genfloatf x)}
{igeninteger64bit isfinite (genfloatd x)}
{
Test for finite value.
}
\addRowTwoSL
{igeninteger32bit isinf (genfloatf x)}
{igeninteger64bit isinf (genfloatd x)}
{
Test for infinity value (positive or negative) .
}
\addRowTwoSL
{igeninteger32bit isnan (genfloatf x)}
{igeninteger64bit isnan (genfloatd x)}
{
Test for a NaN.
}
\addRowTwoSL
{igeninteger32bit isnormal (genfloatf x)}
{igeninteger64bit isnormal (genfloatd x)}
{
Test for a normal value.
}
\addRowTwoSL
{igeninteger32bit isordered (genfloatf x, genfloatf y)}
{igeninteger64bit isordered (genfloatd x, genfloatd y)}
{
Test if arguments are ordered. isordered() takes
arguments x and y, and returns the result
isequal(x, x) $\&\&$ isequal(y, y).
}
\addRowTwoSL
{igeninteger32bit isunordered (genfloatf x, genfloatf y)}
{igeninteger64bit isunordered (genfloatd x, genfloatd y)}
{
Test if arguments are unordered. isunordered()
takes arguments x and y, returning non-zero if x or
y is NaN, and zero otherwise.
}

\addRowTwoSL
{igeninteger32bit signbit (genfloatf x)}
{igeninteger64bit signbit (genfloatd x)}
{
Test for sign bit. The scalar version of the
function returns a 1 if the sign bit in the float is set
else returns 0.

The vector version of the function
returns the following for each component in \keyword{floatn}:

-1 (i.e all bits set) if the sign bit in the float is set
else returns 0.
}

\addRow{int any (igeninteger x)}
{
Returns 1 if the most significant bit in any
component of x is set; otherwise returns 0.
}
\addRow {int all (igeninteger x) }
{
Returns 1 if the most significant bit in all
components of x is set; otherwise returns 0.
}
\addRow
{gentype bitselect (gentype a, gentype b, gentype c) }
{
Each bit of the result is the corresponding bit of a
if the corresponding bit of c is 0. Otherwise it is
the corresponding bit of b.
}
\addRowSixSL
{geninteger select (geninteger a, geninteger b, igeninteger c)}
{geninteger select (geninteger a, geninteger b, ugeninteger c)}
{genfloatf select (genfloatf a, genfloatf b, genint c)}
{genfloatf select (genfloatf a, genfloatf b, ugenint c)}
{genfloatd select (genfloatd a, genfloatd b, igeninteger64 c)}
{genfloatd select (genfloatd a, genfloatd b, ugeninteger64 c)}
{
For each component of a vector type:\newline
\codeinline{result[i] =  (MSB of c[i] is set) ? b[i] : a[i].}\newline
For a scalar type:\newline
\codeinline{result = c ? b : a}.\newline
\codeinline{geninteger} must have the same number
of elements and bits as \codeinline{gentype}.
}
\completeTable
%-------------------------------------------------------------------------------

%*******************************************************************************
% Vector data and store functions
%*******************************************************************************
\subsection{Vector data load and store functions}

The functionality from the OpenCL functions as defined in the OpenCL 1.2
specification document\cite[par. 6.12.7]{opencl12} is available in SYCL through
the \keyword{vec} class in section \ref{sec:vector.type}.

\subsection{Synchronization Functions}

In SYCL the OpenCL \keyword{synchronization} functions are available through the
\codeinline{nd_item} class~\ref{nditem-class}, as they are applied to work-items for local
or global address spaces. Please see~\ref{table.members.nditem}.

%*******************************************************************************
% printf function
%*******************************************************************************
\subsection{\texttt{printf} function}

The functionality of the \texttt{printf} function is covered by the
\codeinline{stream} class~\ref{subsection:stream}, which has the
capability to print to standard output all of the SYCL classes and primitives,
and covers the capabilities defined in the OpenCL 1.2 specification
document\cite[par. 6.12.13]{opencl12}.


%%% Local Variables:
%%% mode: latex
%%% TeX-master: "sycl-1.2.1"
%%% TeX-auto-untabify: t
%%% TeX-PDF-mode: t
%%% ispell-local-dictionary: "american"
%%% End:



%%% Local Variables:
%%% mode: latex
%%% TeX-master: "sycl-1.2.1"
%%% TeX-auto-untabify: t
%%% TeX-PDF-mode: t
%%% ispell-local-dictionary: "american"
%%% End:
