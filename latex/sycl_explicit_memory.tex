% Copyright (c) 2012-2019 Khronos Group.
%
% This work is licensed under a Creative Commons Attribution 4.0
% International License.
% http://creativecommons.org/licenses/by/4.0/

\subsection{SYCL functions for explicit memory operations}
\label{subsec:explicitmemory}

In addition to \glspl{kernel}, \gls{command-group} objects can also be used to
perform manual operations on host and device memory by using the 
\keyword{copy} API of the \gls{handler}.
Manual copy operations can be seen as specialized kernels executing on the 
device, except that typically this operations will be implemented using the
OpenCL host API (e.g, enqueue copy operations).

The SYCL memory objects involved in a copy operation are specified using
accessors.
Explicit copy operations have a source and a destination.
When an accessor is the \textit{source} of the operation, the destination can be 
a host pointer or another accessor.
The \textit{source} accessor can have either \codeinline{read} or
\codeinline{read_write} access mode.

When an accessor is the \textit{destination} of the explicit copy operation,
the source can be a host pointer or another accessor.
The \textit{destination} accessor can have either
\codeinline{write}, \codeinline{read_write}, \codeinline{discard_write},
\codeinline{discard_read_write} access modes.

When accessors are both the origin and the destination,
the operation is executed on objects controlled by the SYCL runtime.
The SYCL runtime is allowed to not perfom an explicit in-copy operation
if a different path to update the data is available according to
the SYCL Application Memory Model.

The most recent copy of the memory object may reside on any context controlled
by the SYCL runtime, or on the host in a pointer controlled by the
SYCL runtime.  The SYCL runtime will ensure that data is copied to the destination
once the \gls{command-group} has completed execution.

Whenever a host pointer is used as either the host or the destination of these
explicit memory operations, it is the responsibility
of the user for that pointer to have at least as much memory allocated as
the accessor is giving access to, e.g: if an accessor accesses a range
of 10 elements of \codeinline{int} type, the host pointer must at least have 
\codeinline{10 * sizeof(int)} bytes of memory allocated.

A special case is the \codeinline{update_host} method. 
This method only requires an accessor, and instructs the runtime to update
the internal copy of the data in the host, if any. This is particularly
useful when users use manual synchronization with host pointers, e.g.\ 
via mutex objects on the \codeinline{buffer} constructors.

Table~\ref{table.members.handler.copy} describes the interface for the
explicit copy operations.

%-------------------------------------------------------------------------------
\startTable{Member function}
\addFootNotes{Member functions of the \codeinline{handler} class}
{table.members.handler.copy}
  \addRowTwoL
    {template <typename T_src, int dim_src, access::mode mode_src, access::target tgt_src, typename T_dest, access::placeholder isPlaceholder>}
    {void copy(accessor<T_src, dim_src, mode_src, tgt_src, isPlaceholder> src, shared_ptr_class<T_dest> dest)}
    { Copies the contents of the memory object accessed by
      \codeinline{src} into the memory pointed to by \codeinline{dest}.
      \codeinline{dest} must have at least as many bytes as the
      range accessed by \codeinline{src}.}
  \addRowTwoL
    {template <typename T_src, typename T_dest, int dim_dest, access::mode mode_dest, access::target tgt_dest, access::placeholder isPlaceholder>}
    {void copy(shared_ptr_class<T_src> src, accessor<T_dest, dim_dest, mode_dest, tgt_dest, isPlaceholder> dest)}
    { Copies the contents of the memory pointed to by \codeinline{src}
      into the memory object accessed by \codeinline{dest}.
      \codeinline{src} must have at least as many bytes as the
      range accessed by \codeinline{dest}.}
  \addRowTwoL
    {template <typename T_src, int dim_src, access::mode mode_src, access::target tgt_src, typename T_dest, access::placeholder isPlaceholder>}
    {void copy(accessor<T_src, dim_src, mode_src, tgt_src, isPlaceholder> src, T_dest * dest)}
    { Copies the contents of the memory object accessed by
      \codeinline{src} into the memory pointed to by \codeinline{dest}.
      \codeinline{dest} must have at least as many bytes as the
      range accessed by \codeinline{src}.}
  \addRowTwoL
    {template <typename T_src, typename T_dest, int dim_dest, access::mode mode_dest, access::target tgt_dest, access::placeholder isPlaceholder>}
    {void copy(const T_src * src, accessor<T_dest, dim_dest, mode_dest, tgt_dest, isPlaceholder> dest)}
    { Copies the contents of the memory pointed to by \codeinline{src}
      into the memory object accessed by \codeinline{dest}.
      \codeinline{src} must have at least as many bytes as the
      range accessed by \codeinline{dest}.}
  \addRowTwoL
    {template <typename T_src, int dim_src, access::mode mode_src, access::target tgt_src, access::placeholder isPlaceholder_src, typename T_dest, int dim_dest, access::mode mode_dest, access::target tgt_dest, access::placeholder isPlaceholder_dest>}
    {void copy(accessor<T_src, dim_src, mode_src, tgt_src, isPlaceholder_src> src, accessor<T_dest, dim_dest, mode_dest, tgt_dest, isPlaceholder_dest> dest)}
    { Copies the contents of the memory object accessed by \codeinline{src}
      into the memory object accessed by \codeinline{dest}.
      \codeinline{src} must have at least as many bytes as the
      range accessed by \codeinline{dest}.}
  \addRowTwoL
    {template <typename T, int dim, access::mode mode, access::target tgt, access::placeholder isPlaceholder>}
    {void update_host(accessor<T, dim, mode, tgt, isPlaceholder> acc)}
    { The contents of the memory object accessed via \codeinline{acc}
      on the host are guaranteed to be up-to-date after this
      \gls{command-group} object execution is complete.}
  \addRowThreeL
    {template <typename T, int dim, access::mode mode, access::target tgt, access::placeholder isPlaceholder>}
    {void fill(accessor<T, dim, mode, tgt, isPlaceholder> dest,}
    {          const T\& src)}
    {Replicates the value of \codeinline{src} into the
      memory object accessed by \codeinline{dest}.
      T must be a scalar value or a SYCL vector type.
    }

\completeTable

The listing below illustrates how to use explicit copy
operations in SYCL. The example copies half of the contents of
a \codeinline{vector_class} into the device, leaving the rest of the
contents of the buffer on the device unchanged.

\lstinputlistingSkipLicense{code/explicitcopy.cpp}


%%% Local Variables:
%%% mode: latex
%%% TeX-master: "sycl-1.2.1"
%%% TeX-auto-untabify: t
%%% TeX-PDF-mode: t
%%% ispell-local-dictionary: "american"
%%% End:
